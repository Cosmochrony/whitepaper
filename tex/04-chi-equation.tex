\subsection{Parameter-Independent Relaxation}
  \label{subsec:parameter-independent-relaxation}

  To avoid the conceptual pitfalls associated with introducing a fundamental time coordinate, the evolution of the field
  $\chi$ is defined such that its physical content is invariant under any strictly monotonic reparameterization $t \rightarrow f(t)$.
  The dynamics is therefore not formulated in terms of an absolute time variable, but in terms of relations between
  successive configurations of the field and their spatial structure.

  The parameter $t$ appearing in the evolution equations serves only as an auxiliary ordering label for the relaxation
  process, ensuring a consistent causal ordering of field states through the monotonic progression of $\chi$ itself.
  Any monotonic reparameterization corresponds to a change of labeling without physical consequence.
  Observable quantities---such as relative relaxation rates, solitonic stability, or emergent curvature---depend only on
  ratios of field variations and spatial gradients, and are therefore independent of the specific choice of $t$.

  In this sense, the relaxation of $\chi$ defines the physical notion of time through its own
  internal progression, while $t$ itself plays the role of a gauge parameter rather than a fundamental observable.

\subsection{Minimal Evolution Equation}
  \label{subsec:minimal-evolution-equation}

  The dynamics of the $\chi$ field is governed by a minimal relaxation principle rather than by a conventional action-based
  variational formulation.
  The guiding assumption is that spacetime unfolds through a locally constrained, irreversible relaxation of $\chi$ toward
  larger values.

  We postulate the following first-order evolution equation:
  \begin{equation}
    \partial_t \chi = c \, \sqrt{1 - \frac{|\nabla \chi|^2}{c^2}} ,
    \label{eq:chi_dynamics}
  \end{equation}
  where $\nabla \chi$ denotes the spatial gradient of $\chi$.

  This equation ensures that the local rate of $\chi$-increase never exceeds the fundamental bound $c$ and reduces to
  $\partial_t \chi \approx c$ in homogeneous regions.

  This relaxation equation resembles a relativistic dispersion relation, where $c$ acts as a maximal
  ``speed of spacetime unfolding''.
  The square-root term ensures that local variations in $\chi$ cannot propagate faster than $c$, preserving causality while
  allowing cumulative effects (such as cosmic expansion) to exceed $c$ when integrated over large scales.

  The stability of this equation is demonstrated in~\ref{sec:stability_chi}, while explicit analytical solutions are
  derived in~\ref{sec:analytical_solutions_chi}.
  The coupling of the $\chi$ field with matter is further discussed in~\ref{sec:coupling_matter_chi}.

\subsection{Causality and Locality}
  \label{subsec:causality-and-locality}

  Equation~\eqref{eq:chi_dynamics} is explicitly local and causal.
  The evolution of $\chi$ at any spacetime point depends only on its immediate neighborhood through $\nabla \chi$.

  Importantly, no superluminal propagation occurs at the fundamental level.
  Apparent superluminal recession velocities in cosmology arise from integrating local $\chi$
  increments across extended regions, consistent with relativistic causality.

\subsection{Homogeneous Cosmological Limit}
  \label{subsec:homogeneous-cosmological-limit}

  In a spatially homogeneous and isotropic configuration, $\nabla \chi = 0$
  , and the evolution equation simplifies to:
  \begin{equation}
    \partial_t \chi = c .
  \end{equation}

  This implies a linear growth\cite{Friedmann1922,Lemaitre1927}:
  \begin{equation}
    \chi(t) = \chi_0 + c t ,
  \end{equation}

  where $\chi_0$ denotes the initial value of $\chi$.

  This simple relation already reproduces a Hubble-like expansion law when distances are identified with accumulated
  $\chi$ increments, as discussed in Section~\ref{sec:cosmology}.

\subsection{Influence of Local Structure}
  \label{subsec:influence-of-local-structure}

  In regions where $\nabla \chi \neq 0$, the effective rate of $\chi$-relaxation is reduced.
  This slowing plays a central role in the emergence of gravitational phenomena.

  Localized excitations---identified with particles---act as topological or dynamical constraints on $\chi$,
  increasing $|\nabla \chi|$ and thereby locally reducing $\partial_t \chi$.

  This mechanism leads naturally to time dilation and spatial curvature without invoking an independent
  gravitational field.

\subsection{Relation to Relativistic Kinematics}
  \label{subsec:relation-to-relativistic-kinematics}

  Spatial gradients of the $\chi$ field reduce the local relaxation rate according to Eq.~\eqref{eq:chi_dynamics}.
  This equation formally resembles a relativistic dispersion relation and ensures Lorentz-consistent behavior at small
  scales.
  In particular, the reduction of $\partial_t \chi$ in regions of large gradients mirrors the relativistic time dilation experienced
  near massive bodies or at high velocities.

  Unlike general relativity, however, this behavior arises from a scalar relaxation dynamics rather than from tensorial
  spacetime curvature.
  At a phenomenological level, the resulting distortions of clock rates and signal propagation can be summarized by
  introducing an \emph{effective} spacetime metric, which encodes how spatial inhomogeneities of $\chi$ modify causal
  structure and proper intervals.
  In this sense, the metric is not a fundamental dynamical field but a descriptive tool, introduced to capture the
  observable consequences of $\chi$-gradient-induced time dilation.

  At a phenomenological level, the effective metric may be approximated as:
  \begin{equation}
    g^{\text{eff}}_{\mu\nu}[\chi] = \eta_{\mu\nu} + \alpha \, \partial_\mu\chi \, \partial_\nu\chi,
  \end{equation}
  where $\eta_{\mu\nu}$ serves as a \emph{reference structure} (not an ontological background) and $\alpha$ is a coupling constant. This form is understood as:
  \begin{itemize}
    \item \textbf{Not fundamental:} $\eta_{\mu\nu}$ has no physical reality; it is a mathematical convention for defining $\partial_\mu$.
    \item \textbf{Lowest-order approximation:} Valid only for weak $\chi$ gradients; general forms require bootstrap determination (Appendix~\ref{sec:bootstrap}).
    \item \textbf{Effective descriptor:} $g^{\text{eff}}_{\mu\nu}[\chi]$ is a functional encoding observable causal structure.
  \end{itemize}
  This form is not unique and should be understood as a lowest-order approximation, valid in regimes where
  $\chi$ varies smoothly.

  More general constructions, including tensorial or nonlocal extensions, may be required beyond this
  approximation and are left for future work.

\subsection{Limitations and Scope}\label{subsec:limitations-and-scope}

  Equation~\eqref{eq:chi_dynamics} is intentionally minimal.
  It does not attempt to describe quantum fluctuations of $\chi$, nor does it incorporate backreaction effects beyond
  first order.

  Its purpose is to provide a unified kinematic backbone from which gravitational, quantum, and cosmological
  phenomena can be derived consistently.

  In the following sections, we apply this dynamical framework to particles, gravity, and entanglement.
