\subsection{Parameter-Independent Relaxation}
  \label{subsec:parameter-independent-relaxation}

  To avoid the conceptual pitfalls of a fundamental time coordinate, we define the dynamics not as a function of an
  absolute variable, but as a sequence of field configurations $(\chi_\lambda)$ where $\lambda$ is a strictly monotonic
  ordering parameter.
  At the fundamental level, the evolution of $\chi$ is defined by its relaxation flow:
  \begin{equation}
    \frac{d\chi}{d\lambda} = \mathcal{R}(\chi, \nabla\chi)
  \end{equation}
  where $\mathcal{R}$ represents the rate of geometric tension release toward equilibrium.
  The ``temporal'' derivative $\partial_t$ used in subsequent sections is then understood as a convenient
  reparametrization of this primary flow:
  \begin{equation}
    \partial_t \chi \equiv \frac{d\chi}{d\lambda} \frac{d\lambda}{dt}
  \end{equation}
  Since $t$ merely serves to label the relaxation ordering, all physical results are invariant under any monotonic
  reparameterization $t \rightarrow f(t)$ with $f'(t) > 0$.

  In this framework, the relaxation of $\chi$ is not occurring \textit{in} time; rather, the local rate of relaxation
  \textit{defines} the physical measure of time.
  What we perceive as duration is the cumulative displacement of the field toward its equilibrium state.
  Consequently, the ``speed of time'' is locally determined by the density of $\chi$-gradients, providing a direct
  link between field topology and temporal flow.

\subsection{Hamiltonian Derivation of the Evolution Equation}
  \label{subsec:hamiltonian-derivation}

  While the dynamics of $\chi$ can be viewed as a minimal relaxation principle, it can be more rigorously derived from a
  Hamiltonian constraint.
  We postulate that the dynamics of $\chi$ are governed by a Dirac-type kinematic constraint in phase space, analogous
  to the mass-shell condition for a massless relativistic particle:
  \begin{equation}
  (\partial_t \chi)
    ^2 + |\nabla \chi|^2 = c^2,
    \label{eq:hamiltonian_constraint}
  \end{equation}
  where $c$ is the fundamental velocity scale.
  Combined with the \textit{arrow of time} postulate ($\partial_t \chi \geq 0$), which reflects the irreversible
  relaxation of the Cosmochron, this leads uniquely to the first-order evolution equation:
  \begin{equation}
    \partial_t \chi = c \sqrt{1 - \frac{|\nabla \chi|^2}{c^2}}.
    \label{eq:chi_dynamics}
  \end{equation}

  This derivation grounds the ``minimal principle'' in the symplectic structure of the field's phase space, ensuring
  that $\chi$ acts as an intrinsic time coordinate.

  The mathematical stability of this equation is demonstrated in~\ref{subsec:stability_chi}, while explicit analytical
  solutions are derived in~\ref{subsec:analytical_solutions_chi}.
  The coupling of the $\chi$ field with matter, extending this kinematic backbone to a dynamical theory, is further
  discussed in~\ref{subsec:coupling_matter_chi} and Section~\ref{subsec:variational-formulation}.

\subsection{Variational Formulation and Born-Infeld Action}
  \label{subsec:variational-formulation}

  To extend this kinematic constraint to a full dynamical theory including matter, we propose an effective Lagrangian
  density of the Born-Infeld type:
  \begin{equation}
    \mathcal{L} = -c^2 \sqrt{1 - \frac{|\nabla \chi|^2}{c^2}} + \partial_t \chi - \frac{4\pi G}{c^2} \rho \chi ,
  \end{equation}
  where $\rho$ represents the matter density. The presence of the term $\partial_t \chi$ linear in the first-order
  temporal derivative is crucial: it ensures that the momentum conjugate to $\chi$, defined as
  $\Pi_\chi = \frac{\partial \mathcal{L}}{\partial (\partial_t \chi)}$, is a non-vanishing constant ($\Pi_\chi = 1$).

  In the Hamiltonian formalism, this constant momentum acts as a primary constraint that effectively enforces the
  unit-velocity evolution of the field.
  This structure ensures that the field dynamics remain locked onto the Hamiltonian
  constraint~\eqref{eq:hamiltonian_constraint} while the square-root term acts as a non-linear regularizer for spatial
  gradients.
  The variation with respect to $\chi$ yields a non-linear Poisson equation:
  \begin{equation}
    \nabla \cdot \left( \frac{\nabla \chi}{\sqrt{1 - |\nabla \chi|^2/c^2}} \right) = \frac{4\pi G}{c^2} \rho .
    \label{eq:nonlinear_poisson}
  \end{equation}
  This formulation naturally recovers the Newtonian limit for weak gradients ($|\nabla \chi| \ll c$) while preventing
  gravitational singularities as the gradient magnitude is bounded by $c$.

\subsection{Causality and Locality}
  \label{subsec:causality-and-locality}

  Equation~\eqref{eq:chi_dynamics} is explicitly local and causal.
  The evolution of $\chi$ at any spacetime point depends only on its immediate neighborhood through $\nabla \chi$.

  Importantly, no superluminal propagation occurs at the fundamental level.
  Apparent superluminal recession velocities in cosmology arise from integrating local $\chi$
  increments across extended regions, consistent with relativistic causality.

\subsection{Homogeneous Cosmological Limit}
  \label{subsec:homogeneous-cosmological-limit}

  In a spatially homogeneous and isotropic configuration, $\nabla \chi = 0$
  , and the evolution equation simplifies to:
  \begin{equation}
    \partial_t \chi = c .
  \end{equation}

  This implies a linear growth\cite{Friedmann1922,Lemaitre1927}:
  \begin{equation}
    \chi(t) = \chi_0 + c t ,
  \end{equation}

  where $\chi_0$ denotes the initial value of $\chi$.

  This simple relation already reproduces a Hubble-like expansion law when distances are identified with accumulated
  $\chi$ increments, as discussed in Section~\ref{sec:cosmology}.

  As shown in Appendix~\ref{sec:mond_derivation}, the requirement $\partial_t \chi \geq 0$ in an expanding background
  ($H_0$) implies a minimal residual gradient $\nabla \chi_{\min} \propto \sqrt{H_0}$.
  This ``acceleration floor'' provides a first-principles derivation for MOND-like phenomenology, explaining galactic
  rotation curves without invoking dark matter particles.

\subsection{Influence of Local Structure}
  \label{subsec:influence-of-local-structure}

  In regions where $\nabla \chi \neq 0$, the effective rate of $\chi$-relaxation is reduced.
  This slowing plays a central role in the emergence of gravitational phenomena.

  Localized excitations---identified with particles---act as topological or dynamical constraints on $\chi$,
  increasing $|\nabla \chi|$ and thereby locally reducing $\partial_t \chi$.

  This mechanism leads naturally to time dilation and spatial curvature without invoking an independent
  gravitational field.

\subsection{Unified Origin of Geometric and Field Effects}
  \label{subsec:unified-origin}

  The relationship between the $\chi$ field and the effective metric $g_{\mu\nu}$ is strictly hierarchical, reflecting
  the transition from fundamental relations to smooth geometry:

  \begin{enumerate}
    \item \textbf{Primacy of the Network:} The fundamental layer consists of the relational $\chi$-network where the connectivity matrix $K_{ij}$ dictates the flow of information and the local relaxation rate.
    \item \textbf{Geometric Emergence:} The metric $g_{\mu\nu}$ is an emergent statistical description of these connections. It represents the ``density of correlation'' within the field and acts as a coarse-grained summary
    for macroscopic observers.
    \item \textbf{The Hydrodynamic Analogy:} Just as pressure and temperature emerge from molecular collisions without acting back as independent ontological forces, the metric $g_{\mu\nu}$ encodes the state of the $\chi$ field. Gravitation is the manifestation of $\chi$-solitons (matter) locally modulating the network's connectivity.
  \end{enumerate}

  In this framework, the field equations are solved on the fundamental graph.
  The resulting configuration naturally defines the effective spacetime geometry, ensuring that gravitation and matter
  are two aspects of the same underlying $\chi$ dynamics.

\subsection{Limitations and Scope}
  \label{subsec:limitations-and-scope}

  Equation~\eqref{eq:chi_dynamics} is intentionally minimal.
  It does not attempt to describe quantum fluctuations of $\chi$, nor does it incorporate backreaction effects beyond
  first order.

  Its purpose is to provide a unified kinematic backbone from which gravitational, quantum, and cosmological
  phenomena can be derived consistently.

  In the following sections, we apply this dynamical framework to particles, gravity, and entanglement.
