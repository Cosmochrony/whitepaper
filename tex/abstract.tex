\abstract{
  We present Cosmochrony, a foundational pre-geometric framework where spacetime, inertia, mass,
  and interactions emerge from the irreversible relaxation of a single relational substrate $\chi$.
  By postulating a principle of ontological poverty at the origin,
  we derive the expansion of the admissible configuration space as the primary driver of cosmic evolution.

  A central result of the theory is the emergence of a Born–Infeld-like dynamics from the saturation of relaxation
  fluxes, providing a unified,
  non-singular resolution to both the Big Bang singularity and the self-energy of charged particles. We show that
  electric charge emerges naturally as a chiral-torsional invariant of the $\chi$-flux,
  eliminating the need for external gauge fields.

  Furthermore, numerical simulations indicate that Cosmochrony reproduces the flat rotation curves of galaxies and
  offers a structural explanation of the Hubble tension as emergent effects of the substrate’s relaxation,
  without invoking dark matter particles or a fundamental dark energy component.
  Finally, we provide a structural origin for quantum mechanics,
  where indeterminacy and entanglement arise from the non-injective projection of underlying $\chi$-configurations
  onto effective observables, and persist only within a critical projection regime that naturally accounts for Bell
  inequality violations and the emergence of classical behavior in massive systems.
}
