\abstract{
  We present Cosmochrony, a foundational pre-geometric framework in which spacetime,
  inertia, mass, and interactions emerge from the irreversible relaxation of a single
  relational substrate~$\chi$.
  By postulating a principle of ontological poverty at the origin, we derive the expansion
  of the admissible configuration space as the primary driver of cosmic evolution.

  A central result of the theory is the emergence of a Born--Infeld-like dynamics from the
  saturation of relational relaxation fluxes, providing a unified and non-singular
  resolution of both the initial cosmological singularity and the self-energy of charged
  particles.
  Within this framework, the speed of light and Planck's constant arise as complementary
  limits of projectability: $c$ bounds the maximal admissible propagation of relational
  flux, while $h$ sets the minimal resolvable granularity of the same underlying dynamics.

  We further show that matter and electric charge emerge naturally as stable saturation
  and chiral-torsional invariants of the $\chi$-flux, without invoking fundamental gauge
  fields.
  Numerical simulations indicate that Cosmochrony reproduces flat galactic rotation curves
  and provides a structural explanation of the Hubble tension as emergent effects of
  substrate relaxation, without dark matter particles or a fundamental dark energy
  component.

  Finally, quantum indeterminacy and entanglement are derived as consequences of the
  non-injective projection of underlying $\chi$-configurations onto effective observables.
  This mechanism naturally accounts for Bell inequality violations, the emergence of
  classical behavior in massive systems, and leads to concrete, falsifiable predictions,
  including precision spectral shifts in the Lamb regime, non-linear saturation in
  Schwinger pair-production, and signatures in ultra-high energy cosmic rays.
}
