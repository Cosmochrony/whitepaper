\subsection{Unified Origin of Geometric and Field Effects}
  \label{subsec:unified-origin}

  The relationship between the $\chi$ substrate and the effective spacetime metric
  $g_{\mu\nu}$ is strictly hierarchical, reflecting the transition from a fundamental
  pre-geometric relational structure to smooth geometric descriptions applicable at
  macroscopic scales.

  \begin{enumerate}
    \item \textbf{Primacy of the $\chi$ substrate:}
    At the fundamental level, physical reality is described solely in terms of the
    $\chi$ substrate and its intrinsic relational structure.
    The $\chi$ substrate is not defined on spacetime, does not possess numerical
    values, and is not governed by a dynamical field equation.
    Ordering, relaxation, and causal notions arise only at the level of admissible
    projected descriptions.

    \item \textbf{Emergent Geometry:}
    In regimes where projected $\chi$ configurations admit a stable, slowly varying
    description, geometric notions become meaningful.
    The spacetime metric $g_{\mu\nu}$ arises as an effective descriptor summarizing the
    correlations and relaxation ordering of projected $\chi$ configurations.
    It provides a coarse-grained geometric language suitable for macroscopic
    observers, without acquiring independent ontological status.

    \item \textbf{Unified Interpretation of Fields and Gravitation:}
    Within this effective geometric description, localized relaxation-resistant
    projected configurations—describable as solitonic structures—are identified with
    matter degrees of freedom.
    Gravitational phenomena correspond to the local modulation of effective relaxation
    ordering induced by such structures.
    The metric does not act as an independent dynamical agent, but encodes the
    collective geometric response associated with constrained projected descriptions.
  \end{enumerate}

  In this framework, no independent gravitational interaction or fundamental field is
  postulated.
  Matter, geometry, and gravitational phenomena emerge as complementary aspects of the
  same constrained ordering of projected $\chi$ configurations, ensuring a unified and
  internally consistent description across scales.
