\subsection{Parameter-Independent Relaxation}
  \label{subsec:parameter-independent-relaxation}

  To avoid the conceptual difficulties associated with a fundamental time coordinate,
  the Cosmochrony framework formulates physical evolution without reference to any
  external temporal parameter.
  At the fundamental level, the $\chi$ substrate does not evolve in time and admits no
  temporal parametrization.
  Physical evolution appears only at the effective level, as an ordered sequence of
  projected $\chi$ configurations, denoted
  $(\chi_{\mathrm{eff},\lambda})$, where $\lambda$ is a strictly monotonic ordering
  parameter labeling admissible stages of relaxation within projected descriptions.

  The ordering parameter $\lambda$ does not represent time, nor does it parametrize a
  trajectory of the fundamental $\chi$ substrate.
  It serves solely as an index that orders projected configurations once a macroscopic
  spacetime description becomes applicable.
  No fundamental dynamics unfolds with respect to $\lambda$ at the level of $\chi$
  itself.

  Accordingly, no fundamental evolution equation is postulated for $\chi$.
  Instead, admissible projected descriptions are constrained by an effective relaxation
  condition of the form
  \begin{equation}
    \mathcal{D}_{\lambda}\chi_{\mathrm{eff}} = \mathcal{R}[\chi_{\mathrm{eff}}] ,
  \end{equation}
  where $\mathcal{R}[\chi_{\mathrm{eff}}]$ denotes an effective relaxation functional
  characterizing the ordering of projected $\chi$ configurations.
  This relation should be understood as a consistency condition on admissible
  projections, not as a fundamental dynamical law.
  The functional $\mathcal{R}$ is defined only in regimes admitting a stable geometric
  interpretation and carries no meaning at the pre-geometric level.
  In particular, no spacetime derivative, metric structure, or geometric operator is
  assumed in its definition.

  Quantities commonly interpreted as temporal derivatives arise exclusively within
  effective descriptions.
  When projected $\chi$ configurations exhibit sufficient regularity, a coordinate
  time parameter may be introduced as a convenient label for the relaxation ordering.
  Such a parameter has no fundamental significance and does not alter the underlying
  structural constraints imposed by the $\chi$ substrate.

  Within this framework, relaxation does not occur \emph{in} time.
  Rather, the ordering of projected $\chi$ configurations defines what is operationally
  identified as physical duration.
  Local variations in the effective relaxation ordering give rise to the
  phenomenological notion of temporal flow, establishing a direct link between the
  structure of projected descriptions and the emergent concept of time.
