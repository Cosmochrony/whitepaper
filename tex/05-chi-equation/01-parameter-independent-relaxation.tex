\subsection{Parameter-Independent Relaxation}
  \label{subsec:parameter-independent-relaxation}

  To avoid the conceptual pitfalls associated with a fundamental time coordinate,
  the dynamics of the scalar quantity $\chi$ is formulated without reference to any
  external temporal parameter.
  Instead, physical evolution is described as an ordered sequence of $\chi$
  configurations, denoted $(\chi_\lambda)$, where $\lambda$ is a strictly monotonic
  ordering parameter labeling the relaxation process.

  At the fundamental level, the dynamics of $\chi$ is defined by an intrinsic
  relaxation flow:
  \begin{equation}
    \mathcal{D}_{\lambda}\chi = \mathcal{R}[\chi],
  \end{equation}
  where $\mathcal{R}[\chi]$ denotes the local rate of relaxation toward equilibrium,
  defined purely in terms of the relational structure of $\chi$ configurations.
  No spacetime derivative or geometric operator is assumed at this stage.

  Quantities commonly interpreted as temporal derivatives arise only at the level of
  effective descriptions, when $\chi$ configurations admit a quasi-stable geometric
  interpretation.
  In such regimes, a coordinate time parameter may be introduced as a convenient
  label of the relaxation ordering, but it carries no fundamental significance and
  does not affect the underlying dynamics.

  Within this framework, relaxation does not occur \emph{in} time.
  Rather, the cumulative relaxation of $\chi$ defines what is operationally
  identified as physical duration.
  Local variations in the relaxation rate provide the effective measure of temporal
  flow, establishing a direct connection between the structure of $\chi$
  configurations and the emergent notion of time.
