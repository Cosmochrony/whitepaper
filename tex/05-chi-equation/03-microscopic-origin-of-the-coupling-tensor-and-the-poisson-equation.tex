\subsection{Microscopic Origin of the Coupling Tensor and the Poisson Equation}
  \label{subsec:microscopic-origin-of-the-coupling-tensor-and-the-poisson-equation}

  For internal consistency, the effective coupling governing the ordering of projected
  $\chi$ configurations cannot be treated as a fixed universal constant.
  Instead, it must depend on the internal structural state of the projected description,
  reflecting how localized configurations either facilitate or resist further relaxation.
  In Cosmochrony, this dependence is captured through a constitutive relation linking
  effective coupling strength to variations of the effective scalar descriptor
  $\chi_{\mathrm{eff}}$, without invoking any underlying spatial substrate or fundamental
  interaction at the pre-geometric level.

  A convenient phenomenological parametrization of this dependence is given by
  \begin{equation}
    K_{\mathrm{eff}}
    = K_0 \exp\!\left(
                  -\frac{(\Delta \chi_{\mathrm{eff}})^2}{\chi_c^2}
    \right),
    \label{eq:effective_coupling_tensor}
  \end{equation}
  where $\Delta \chi_{\mathrm{eff}}$ denotes a measure of effective internal variation
  between correlated projected configurations,
  $K_0$ characterizes the maximal relaxation conductivity in a homogeneous effective
  background, and $\chi_c$ sets the characteristic scale beyond which structural
  inhomogeneities significantly suppress relaxation efficiency.
  This expression should be understood as a constitutive relation for admissible projected
  descriptions, not as a fundamental law governing the $\chi$ substrate.

  Projected configurations exhibiting strong internal variation—such as stable solitonic
  or bound structures—therefore reduce the effective coupling and locally slow the
  admissible relaxation ordering.
  This reduction does not correspond to the introduction of a new interaction.
  It reflects instead the intrinsic resistance of structured projected configurations to
  further relaxation.
  The resulting slowdown provides the microscopic origin of the emergent gravitational
  phenomenology discussed in the preceding sections.

  In regimes where a spacetime description becomes applicable, the local effective
  relaxation rate $\mathcal{D}_{\mathrm{loc}} \chi_{\mathrm{eff}}$ differs from its
  asymptotic value $\mathcal{D}_0$ far from localized structures.
  An effective gravitational potential $\Phi$ may then be introduced as a descriptive
  parameter through the relation
  \begin{equation}
    \frac{\mathcal{D}_{\mathrm{loc}} \chi_{\mathrm{eff}}}{\mathcal{D}_0}
    \simeq 1 + \frac{\Phi}{c^2},
    \label{eq:relaxation_potential_relation}
  \end{equation}
  which summarizes the relative slowdown of effective relaxation ordering in a form
  familiar from classical gravitational phenomenology.
  The potential $\Phi$ has no independent ontological status and serves only as a compact
  parametrization of relaxation inhomogeneities.

  In the weak-structure regime, where effective internal variations remain small compared
  to $\chi_c$, the spatial distribution of $\Phi$ admits a simplified elliptic
  description.
  At this coarse-grained level, the effective dynamics reduce to a Poisson-type relation,
  \begin{equation}
    \nabla^2 \Phi \simeq 4\pi G_{\mathrm{eff}} \rho,
    \label{eq:effective_poisson_equation}
  \end{equation}
  where $\rho$ denotes the effective density of localized, relaxation-resistant projected
  configurations and $G_{\mathrm{eff}}$ is an emergent coupling parameter encoding the
  collective response of the relaxation ordering to such structures.
  Both quantities are defined strictly within the effective geometric description.

  This Poisson equation is not fundamental.
  It represents the weak-field, macroscopic limit of the constrained ordering of
  projected $\chi$ configurations, expressed in a form adapted to effective spacetime
  description.
  Gravitation therefore appears not as an independent interaction, but as a descriptive
  manifestation of reduced relaxation conductivity induced by structured projected
  configurations.

  A fully relational formulation, consistent with but not required for the effective
  description adopted here, is provided in
  Appendix~\ref{app:relational_formulation}.

  \paragraph{Status of effective equations.}
    The effective equations introduced in this subsection—
