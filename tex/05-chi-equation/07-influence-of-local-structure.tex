\subsection{Influence of Local Structure}
  \label{subsec:influence-of-local-structure}

  In regions where projected $\chi$ configurations exhibit non-vanishing effective
  structural variations, the admissible local relaxation ordering is reduced.
  This slowdown plays a central role in the emergence of gravitational phenomena
  within the Cosmochrony framework.

  Localized relaxation-resistant configurations—describable in effective regimes as
  particle-like solitonic structures—act as constraints on the admissible ordering of
  projected $\chi$ configurations.
  By increasing the effective structural complexity of the projected description,
  they reduce the local effective relaxation rate without introducing any additional
  interaction or force.

  When a geometric description becomes applicable, this mechanism manifests
  phenomenologically as gravitational time dilation and spatial curvature.
  No independent gravitational field is postulated.
  Gravitation emerges instead as a collective consequence of locally constrained
  effective relaxation ordering, reflecting the presence of structured projected
  configurations.
