\subsection{Hamiltonian Derivation of the Evolution Equation}
  \label{subsec:hamiltonian-derivation}

  \subsubsection*{Local Constraint on Effective Relaxation Dynamics}

    At the fundamental level, the $\chi$ substrate does not evolve in time, admits no
    phase-space structure, and is not governed by a Hamiltonian dynamics.
    No spacetime coordinates, canonical variables, or variational principles are assumed
    in its definition.

    Nevertheless, when projected $\chi$ configurations admit a smooth and stable
    coarse-grained description, the admissible ordering of these configurations is subject
    to universal local constraints.
    These constraints may be summarized, within effective descriptions only, in a compact
    form resembling a Hamiltonian relation.
    This formulation is introduced solely as a descriptive parametrization of admissible
    local relaxation patterns and does not define a fundamental dynamics.

    Specifically, the effective local relaxation ordering associated with projected
    $\chi$ configurations is bounded by the invariant constant $c$.
    When expressed in effective relational variables, this constraint takes the form
    \begin{equation}
      \left(\mathcal{D}_{\mathrm{loc}} \chi_{\mathrm{eff}}\right)^2
      + \mathcal{V}_{\chi_{\mathrm{eff}}}^2
      = c^2 ,
      \label{eq:effective_hamiltonian_constraint}
    \end{equation}
    where $\mathcal{V}_{\chi_{\mathrm{eff}}}$ denotes an effective internal variation
    functional characterizing the resistance of projected configurations to further
    relaxation.
    No notion of spatial gradient, background geometry, or fundamental field variation is
    assumed at this level.

    Restricting to the monotonic ordering branch yields the effective evolution relation
    \begin{equation}
      \mathcal{D}_{\mathrm{loc}} \chi_{\mathrm{eff}}
      = c \sqrt{1 - \frac{\mathcal{V}_{\chi_{\mathrm{eff}}}^2}{c^2}} ,
      \label{eq:chi_dynamics}
    \end{equation}
    which encodes the universal slowdown of effective relaxation induced by localized
    structural constraints.
    This relation constrains admissible projected descriptions but does not represent an
    equation of motion for the fundamental $\chi$ substrate.

  \subsubsection*{Emergent Gravitational Description}

    Projected $\chi$ configurations exhibiting strong resistance to relaxation correspond,
    in effective descriptions, to localized regions of enhanced structural complexity.
    These regions locally reduce the admissible effective relaxation rate.

    When a spacetime interpretation becomes applicable, this reduction is conventionally
    described as gravitational time dilation.
    No independent gravitational field or interaction is postulated; the phenomenon arises
    as a direct consequence of the constrained ordering of projected $\chi$ configurations.

    In the weak-structure regime, where effective internal variation rates are small
    compared to $c$, the effective description admits a simplified relation governing the
    spatial distribution of relaxation slowdown:
    \begin{equation}
      \nabla \cdot \left(
                     \frac{\nabla \chi_{\mathrm{eff}}}
                     {\sqrt{1 - |\nabla \chi_{\mathrm{eff}}|^2 / c^2}}
      \right)
      \simeq \frac{4 \pi G_{\mathrm{eff}}}{c^2} \rho ,
      \label{eq:effective_poisson_relation}
    \end{equation}
    where $\rho$ denotes the effective density of localized relaxation-resistant
    configurations and $G_{\mathrm{eff}}$ is an emergent coupling parameter characterizing
    the collective response of the relaxation ordering to such structures.

    In the Newtonian limit, this relation reduces to an effective Poisson equation for a
    potential $\Phi$, defined operationally by
    \begin{equation}
      \Phi \equiv c^2 \ln \left(
                            \frac{\mathcal{D}_{\mathrm{loc}} \chi_{\mathrm{eff}}}{c}
      \right) .
      \label{eq:effective_gravitational_potential}
    \end{equation}
    This potential is not introduced as a fundamental field, but as a compact summary of
    how localized variations in effective relaxation ordering modulate physical clocks
    and rulers in regimes where classical gravitational phenomenology applies.

  \subsubsection*{Interpretational Status}

    The relations presented in this section do not define a fundamental Hamiltonian,
    action, or variational principle.
    They provide an effective local parametrization of the constraints governing admissible
    projected descriptions once a geometric interpretation becomes meaningful.

    The predictive content of Cosmochrony resides entirely in the structural properties of
    the pre-geometric $\chi$ substrate.
    Hamiltonian, geometric, and gravitational formalisms appear only as emergent descriptive
    tools, valid in restricted regimes and carrying no independent ontological status.
