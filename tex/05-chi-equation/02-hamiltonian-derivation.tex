\subsection{Hamiltonian Derivation of the Evolution Equation}
  \label{subsec:hamiltonian-derivation}

  \subsubsection*{Local Constraint on Relaxation Dynamics}

    The fundamental dynamics of the $\chi$ field is formulated without reference to
    spacetime coordinates or to a Hamiltonian structure.
    Nevertheless, its relaxation is subject to a universal local constraint reflecting
    the bounded character of admissible relaxation rates.

    In regimes where the relational structure of $\chi$ admits a smooth and stable
    coarse-grained description, this constraint can be summarized in a compact form
    resembling a Hamiltonian relation.
    This formulation is introduced solely as a descriptive parametrization of the
    admissible local relaxation configurations, and does not define a fundamental
    phase-space dynamics.

    Specifically, the local relaxation rate of $\chi$ is bounded by the invariant constant
    $c$.
    When expressed in effective relational variables, this bound takes the form
    \begin{equation}
    (\mathcal{D}_{\mathrm{loc}}\chi)^2
    + \mathcal{V}_{\chi}^2
    = c^2 ,
    \label{eq:effective_hamiltonian_constraint}
    \end{equation}
    where $\mathcal{V}_{\chi}$ denotes the effective internal variation rate of $\chi$,
    capturing how strongly local configurations resist relaxation.
    No notion of spatial gradient or background geometry is assumed at the fundamental
    level.

    Selecting the monotonic relaxation branch yields the effective evolution equation
    \begin{equation}
      \mathcal{D}_{\mathrm{loc}}\chi
      = c \sqrt{1 - \frac{\mathcal{V}_{\chi}^2}{c^2}} ,
      \label{eq:chi_dynamics}
    \end{equation}
    which encodes the universal slowdown of relaxation induced by structural variations
    of the $\chi$ field.

  \subsubsection*{Emergent Gravitational Description}

    Localized, relaxation-resistant configurations of $\chi$ generate regions of enhanced
    internal structure, which locally reduce the admissible relaxation rate.
    When a spacetime description becomes applicable, this effect is conventionally
    described as gravitational time dilation.
    No independent gravitational interaction or field is postulated; the effect arises
    directly from the constrained relaxation dynamics of $\chi$ itself.

    In the weak-structure regime, where internal variation rates are small compared to
    $c$, the effective description admits a simplified relation governing the spatial
    distribution of relaxation slowdown:
    \begin{equation}
      \nabla \cdot \left(
                     \frac{\nabla \chi}
                     {\sqrt{1 - |\nabla \chi|^2 / c^2}}
      \right)
      \simeq \frac{4 \pi G_{\mathrm{eff}}}{c^2} \rho ,
      \label{eq:effective_poisson_relation}
    \end{equation}
    where $\rho$ denotes the density of localized configurations and
    $G_{\mathrm{eff}}$ is an emergent coupling parameter characterizing the collective
    response of the relaxation flow to such structures.

    In the Newtonian limit, this relation reduces to an effective Poisson equation for a
    potential $\Phi$, defined operationally by
    \begin{equation}
      \Phi \equiv c^2 \ln \left(
                            \frac{\mathcal{D}_{\mathrm{loc}}\chi}{c}
      \right) .
      \label{eq:effective_gravitational_potential}
    \end{equation}
    This potential is not introduced as a fundamental field, but as a convenient summary
    of how localized variations of $\chi$ modulate relaxation in regimes where classical
    gravitational phenomenology applies.

  \subsubsection*{Interpretational Status}

    The relations presented in this section do not constitute a fundamental Hamiltonian
    or a variational principle.
    They provide an effective local parametrization of the constraints governing $\chi$
    relaxation when a geometric description becomes applicable.

    The predictive content of Cosmochrony resides entirely in the intrinsic,
    pre-geometric relaxation dynamics of the $\chi$ field.
    Hamiltonian, geometric, and gravitational structures appear only as emergent
    descriptive tools, valid in restricted regimes and carrying no independent
    ontological status.
