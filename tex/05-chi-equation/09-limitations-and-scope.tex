\subsection{Limitations and Scope}
  \label{subsec:limitations-and-scope}

  Equation~\eqref{eq:chi_dynamics} is intentionally minimal.
  It does not describe a fundamental dynamics of the $\chi$ substrate, nor does it aim
  to capture the full relational complexity of pre-geometric configurations.
  Its role is to constrain admissible projected descriptions in regimes where a stable
  geometric interpretation becomes applicable.

  In particular, this effective relation does not provide a first-principles account
  of quantum fluctuations, correlations, or probabilistic behavior at the level of the
  $\chi$ substrate itself.
  Such phenomena arise only at the level of projected descriptions, as consequences of
  non-injective projection, coarse-graining, and the limited representability of
  underlying relational structures within spacetime-based formalisms.
  Higher-order structural effects and strongly non-projectable regimes therefore lie
  outside the direct scope of the present formulation.

  Within these limitations, the constrained ordering relation nevertheless provides a
  unified kinematic backbone from which gravitational, quantum, and cosmological
  phenomena can be consistently \emph{described} and \emph{recovered} at the effective
  level.
  These phenomena are not derived from dynamical laws acting on $\chi$, but emerge as
  stable regularities within admissible projected descriptions once geometric
  interpretation and coarse-graining are introduced.

  More refined treatments of effective fluctuations, long-range correlations, and
  extended relational structures require a deeper analysis of the space of admissible
  projections and of the stability properties of the corresponding effective
  descriptions.
  Such developments fall beyond the present scope and are left for future work.

  In the following sections, the constrained descriptive framework introduced here is
  applied to particle-like excitations, gravitation, and quantum correlations, where
  its explanatory power can be directly assessed within its intended domain of
  validity.
