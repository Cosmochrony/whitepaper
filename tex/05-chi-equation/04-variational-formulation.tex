\subsection{Variational Formulation and Born--Infeld Action}
  \label{subsec:variational-formulation}

  In regimes where projected $\chi$ configurations admit a stable geometric
  interpretation, the effective relaxation constraints introduced above may be
  summarized in a compact variational form.
  This formulation is not fundamental and does not define a dynamics of the
  pre-geometric $\chi$ substrate.
  Rather, it provides an auxiliary and regularized representation of admissible
  projected descriptions in the presence of localized relaxation-resistant
  configurations.

  Motivated by Born--Infeld--type non-linear actions, originally introduced to control
  field singularities and enforce upper bounds on physical gradients~\cite{BornInfeld1934,DeserGibbons1998}, we
  consider the effective Lagrangian density
  \begin{equation}
    \mathcal{L}_{\mathrm{eff}}
    = -c^2 \sqrt{1 - \frac{|\nabla \chi_{\mathrm{eff}}|^2}{c^2}}
    + \mathcal{D}_{\mathrm{loc}}\chi_{\mathrm{eff}}
    - \frac{4\pi G_{\mathrm{eff}}}{c^2}\,\rho\,\chi_{\mathrm{eff}} ,
    \label{eq:leff_equation}
  \end{equation}
  where $\chi_{\mathrm{eff}}$ denotes the effective scalar descriptor introduced in
  Sec.~\ref{subsec:geometric-action},
  $\mathcal{D}_{\mathrm{loc}}\chi_{\mathrm{eff}}$ is the effective local relaxation
  ordering defined in Sec.~\ref{subsec:parameter-independent-relaxation}, and $\rho$
  represents the effective density of localized, relaxation-resistant projected
  configurations.
  All quantities appearing in this expression are defined strictly within the effective
  geometric description.

  The linear dependence on $\mathcal{D}_{\mathrm{loc}}\chi_{\mathrm{eff}}$ enforces the
  monotonicity of admissible projected descriptions without introducing additional
  propagating degrees of freedom.
  The square-root structure acts as a non-linear regulator, ensuring that effective
  spatial variations remain bounded by the universal constraint $c$.
  This role is directly analogous to that of Born--Infeld electrodynamics, where the
  action enforces a maximal field strength without postulating new microscopic
  dynamics.

  Within this auxiliary variational framework, the Euler--Lagrange equation associated
  with $\chi_{\mathrm{eff}}$ reproduces the non-linear elliptic relation governing the
  spatial distribution of relaxation slowdown:
  \begin{equation}
    \nabla \cdot
    \left(
      \frac{\nabla \chi_{\mathrm{eff}}}
      {\sqrt{1 - |\nabla \chi_{\mathrm{eff}}|^2 / c^2}}
    \right)
    = \frac{4\pi G_{\mathrm{eff}}}{c^2}\,\rho ,
    \label{eq:nonlinear_poisson}
  \end{equation}
  which coincides with the effective Poisson-type relation obtained in
  Sec.~\ref{subsec:microscopic-origin-of-the-coupling-tensor-and-the-poisson-equation}.
  No new physical content is introduced at this stage; the variational formulation
  merely provides a compact and internally consistent encoding of the previously
  established constraints.

  This Born--Infeld--like action should therefore be understood strictly as an
  \emph{auxiliary variational representation}.
  It does not constitute a fundamental action principle, nor does it define equations
  of motion for the $\chi$ substrate.
  Its purpose is to regularize the effective description, enforce universal structural
  bounds, and facilitate comparison with standard gravitational phenomenology in the
  appropriate weak-field regime.

  The physical interpretation of this effective action is discussed in
  Appendix~\ref{subsec:hydrodynamic-limit}, while its mathematical consistency with the
  underlying relational dynamics and discrete formulation is established in
  Appendix~\ref{sec:born-lagrangian_derivation}.

\subsection{Why This Is Not a Scalar--Tensor Theory}
  \label{subsec:not-scalar-tensor}

  The effective variational formulation introduced in the previous subsection may
  superficially resemble scalar--tensor or modified gravity theories, in which a
  scalar field couples to geometry or mediates gravitational interactions.
  It is therefore essential to clarify that Cosmochrony does \emph{not} belong to this
  class of frameworks, either conceptually or technically.

  First, the effective scalar descriptor $\chi_{\mathrm{eff}}$ is \emph{not} a
  fundamental dynamical field.
  It does not represent an independent physical degree of freedom propagating on
  spacetime, nor does it possess intrinsic values or conjugate momenta.
  By construction, $\chi_{\mathrm{eff}}$ is a projected, coarse-grained descriptor of
  relational features of the pre-geometric substrate $\chi$, defined only in regimes
  where a spacetime interpretation becomes applicable.
  There is therefore no scalar field at the fundamental level whose dynamics could be
  coupled to geometry.

  Second, no modification of the gravitational sector is postulated.
  The metric appearing in effective descriptions is not an independent dynamical field,
  nor is it sourced or altered by a scalar field equation.
  Instead, geometric quantities summarize correlations between projected
  $\chi_{\mathrm{eff}}$ configurations.
  The Born--Infeld--like variational form introduced earlier does not define a new
  gravitational theory; it merely encodes, in a compact and regularized manner, the
  constraints governing admissible projected descriptions.
  In particular, there is no scalar--tensor coupling term of the form
  $f(\chi_{\mathrm{eff}}) R$, nor any modification of the Einstein--Hilbert action.

  Third, the effective equations derived from the auxiliary variational formulation do
  not introduce additional propagating modes.
  In scalar--tensor theories, the scalar field typically carries its own dynamics,
  leading to extra degrees of freedom, modified wave propagation, or additional
  polarization states.
  In Cosmochrony, by contrast, $\chi_{\mathrm{eff}}$ does not propagate independently.
  All effective equations constrain admissible relaxation patterns of projected
  configurations and do not enlarge the physical phase space.

  Finally, the origin of gravitational phenomenology in Cosmochrony is fundamentally
  different.
  Gravitation does not arise from the exchange of a scalar mediator or from a dynamical
  coupling between scalar and tensor sectors.
  It emerges instead from the local inhibition of relaxation ordering induced by
  structured projected configurations.
  The appearance of Poisson-like equations or gravitational potentials reflects the
  weak-structure limit of this constrained ordering, not the presence of a scalar
  gravitational field.

  In summary, Cosmochrony should not be interpreted as a scalar--tensor or modified
  gravity theory.
  The effective scalar descriptor $\chi_{\mathrm{eff}}$ is neither fundamental nor
  dynamical, the variational formulation is auxiliary rather than postulatory, and no
  additional gravitational degrees of freedom are introduced.
  The framework therefore avoids the conceptual and phenomenological issues commonly
  associated with scalar--tensor theories, while reproducing standard gravitational
  behavior in the appropriate effective regimes.
