\subsection{Variational Formulation and Born--Infeld Action}
  \label{subsec:variational-formulation}

  In regimes where projected $\chi$ configurations admit a stable geometric
  interpretation, the effective relaxation constraints introduced above may be
  conveniently summarized using a variational formulation.
  This formulation is not fundamental and does not define a dynamics of the
  $\chi$ substrate itself.
  Rather, it provides a compact and regularized description of admissible projected
  descriptions in the presence of localized relaxation-resistant configurations.

  Motivated by Born--Infeld--type non-linear actions originally introduced to control
  field singularities~\cite{BornInfeld1934,DeserGibbons1998}, we consider the effective
  Lagrangian density
  \begin{equation}
    \mathcal{L}_{\mathrm{eff}}
    = -c^2 \sqrt{1 - \frac{|\nabla \chi_{\mathrm{eff}}|^2}{c^2}}
    + \mathcal{D}_{\mathrm{loc}}\chi_{\mathrm{eff}}
    - \frac{4\pi G_{\mathrm{eff}}}{c^2} \rho \,\chi_{\mathrm{eff}} ,
  \end{equation}
  where $\chi_{\mathrm{eff}}$ denotes the effective scalar descriptor introduced in
  Sec.~\ref{subsec:geometric-action},
  $\mathcal{D}_{\mathrm{loc}}\chi_{\mathrm{eff}}$ is the effective local relaxation
  ordering defined in Sec.~\ref{subsec:parameter-independent-relaxation}, and $\rho$
  represents the effective density of localized, relaxation-resistant projected
  configurations.
  All quantities appearing in this expression are defined exclusively within the
  effective geometric description.

  The linear dependence on $\mathcal{D}_{\mathrm{loc}}\chi_{\mathrm{eff}}$ ensures that
  admissible projected descriptions remain monotonic and constrained, without
  introducing additional propagating degrees of freedom.
  The square-root structure acts as a non-linear regulator enforcing the universal
  upper bound on effective spatial variations, in direct analogy with the original
  role of Born--Infeld electrodynamics.

  Within this effective variational framework, the Euler--Lagrange equation associated
  with $\chi_{\mathrm{eff}}$ reproduces the non-linear elliptic relation governing the
  spatial distribution of relaxation slowdown:
  \begin{equation}
    \nabla \cdot
    \left(
      \frac{\nabla \chi_{\mathrm{eff}}}
      {\sqrt{1 - |\nabla \chi_{\mathrm{eff}}|^2 / c^2}}
    \right)
    = \frac{4\pi G_{\mathrm{eff}}}{c^2} \rho ,
    \label{eq:nonlinear_poisson}
  \end{equation}
  which coincides with the effective Poisson-type relation obtained in
  Sec.~\ref{subsec:microscopic-origin-of-the-coupling-tensor-and-the-poisson-equation}.

  This variational formulation should be understood strictly as a compact and
  regularized representation of admissible projected descriptions.
  It does not constitute a fundamental action principle, nor does it define equations
  of motion for the $\chi$ substrate.
  Its sole purpose is to ensure internal consistency of the effective description and
  to provide a transparent link with standard gravitational phenomenology in the
  appropriate weak-field regime.
