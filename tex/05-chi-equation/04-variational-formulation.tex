\subsection{Variational Formulation and Born--Infeld Action}
  \label{subsec:variational-formulation}

  In regimes where $\chi$ admits a stable geometric interpretation, the effective
  relaxation constraint introduced above may be conveniently summarized using a
  variational formulation.
  This formulation is not fundamental, but provides a compact and regularized
  description of the coarse-grained dynamics of $\chi$ in the presence of localized
  excitations.

  Motivated by Born--Infeld--type non-linear actions originally introduced to control
  field singularities~\cite{BornInfeld1934,DeserGibbons1998}, we consider the effective
  Lagrangian density
  \begin{equation}
    \mathcal{L}_{\mathrm{eff}}
    = -c^2 \sqrt{1 - \frac{|\nabla \chi|^2}{c^2}}
    + \mathcal{D}_{\mathrm{loc}}\chi
    - \frac{4\pi G_{\mathrm{eff}}}{c^2} \rho \chi ,
  \end{equation}
  where $\mathcal{D}_{\mathrm{loc}}\chi$ denotes the effective local relaxation rate
  introduced in Sec.~\ref{subsec:parameter-independent-relaxation}, and $\rho$
  represents the density of localized excitations.

  The linear dependence on $\mathcal{D}_{\mathrm{loc}}\chi$ ensures that the effective
  relaxation flow remains monotonic and constrained, without introducing additional
  propagating degrees of freedom.
  The square-root structure acts as a non-linear regulator enforcing the universal
  upper bound on effective spatial variations, in direct analogy with the original
  role of Born--Infeld electrodynamics.

  Within this effective variational framework, the Euler--Lagrange equation
  associated with $\chi$ reproduces the non-linear elliptic relation governing the
  spatial distribution of relaxation slowdown:
  \begin{equation}
    \nabla \cdot
    \left(
      \frac{\nabla \chi}{\sqrt{1 - |\nabla \chi|^2 / c^2}}
    \right)
    = \frac{4\pi G_{\mathrm{eff}}}{c^2} \rho ,
    \label{eq:nonlinear_poisson}
  \end{equation}
  which coincides with the effective Poisson-type equation obtained in
  Sec.~\ref{subsec:microscopic-origin-of-the-coupling-tensor-and-the-poisson-equation}.

  This variational formulation should be understood as a compact, regularized
  representation of the effective dynamics of $\chi$, not as a fundamental action
  principle.
  Its role is to ensure internal consistency and to provide a convenient link to
  standard gravitational phenomenology in the appropriate weak-field regime.
