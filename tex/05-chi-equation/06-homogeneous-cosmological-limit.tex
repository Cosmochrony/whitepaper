\subsection{Homogeneous Cosmological Limit}
  \label{subsec:homogeneous-cosmological-limit}

  In a homogeneous and isotropic configuration, effective spatial variations of
  $\chi$ vanish, and the relaxation proceeds uniformly.
  In this regime, the local relaxation rate reaches its maximal value,
  \begin{equation}
    \mathcal{D}_{\mathrm{loc}}\chi = c ,
  \end{equation}
  where $c$ denotes the universal bound on the relaxation rate.

  When described in terms of an effective cosmological time parameter $t$, this
  uniform relaxation is conveniently represented as a linear relation,
  \begin{equation}
    \chi(t) = \chi_{\min} + c\, t ,
  \end{equation}
  where $\chi_{\min}$ denotes the minimal physically meaningful value of $\chi$.
  This parametrization does not introduce a fundamental time variable, but provides
  a useful representation of the cumulative relaxation in a homogeneous cosmological
  regime.

  Interpreting spatial distances as accumulated relational differences in $\chi$,
  this linear relaxation directly leads to a Hubble-like expansion law, as discussed
  in Sec.~\ref{sec:cosmology}.
  Cosmic expansion thus reflects the global relaxation of $\chi$, rather than the
  presence of an external energy component.

  As shown in Appendix~\ref{subsec:minimal-kinematic-constraint}, the requirement that the relaxation
  flow remains monotonic in an expanding background implies the existence of a
  minimal residual structural inhomogeneity of $\chi$.
  In effective geometric terms, this manifests as a non-vanishing lower bound on
  gravitational acceleration, providing a natural explanation for MOND-like
  phenomenology without invoking dark matter particles.
