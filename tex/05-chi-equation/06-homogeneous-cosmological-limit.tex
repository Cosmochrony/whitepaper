\subsection{Homogeneous Cosmological Limit}
  \label{subsec:homogeneous-cosmological-limit}

  In a homogeneous and isotropic regime, projected $\chi$ configurations exhibit no
  effective spatial variations.
  The admissible relaxation ordering is then uniform across the emergent description,
  and the effective local relaxation rate attains its maximal allowed value,
  \begin{equation}
    \mathcal{D}_{\mathrm{loc}}\chi_{\mathrm{eff}} = c ,
  \end{equation}
  where $c$ denotes the universal bound constraining effective relaxation ordering.

  When expressed in terms of an effective cosmological time parameter $t$, introduced
  solely as a convenient label of the relaxation ordering, this uniform regime may be
  represented by the linear relation
  \begin{equation}
    \chi_{\mathrm{eff}}(t) = \chi_{\mathrm{eff},0} + c\, t ,
  \end{equation}
  where $\chi_{\mathrm{eff},0}$ denotes a reference value fixing the origin of the
  effective description.
  This parametrization does not introduce a fundamental time variable, nor does it
  assign intrinsic values to the $\chi$ substrate.
  This linear regime underlies the emergent Hubble law derived in Section~\ref{subsec:emergent-hubble-law}.
  It serves only as a compact representation of cumulative relaxation ordering in a
  homogeneous cosmological regime.

  Interpreting effective spatial distances as accumulated relational differentiation
  between projected configurations, this uniform ordering leads directly to a
  Hubble-like expansion law, as discussed in
  Sec.~\ref{sec:cosmology}.
  Cosmic expansion therefore reflects the global ordering of projected $\chi$
  configurations, rather than the presence of an external energy component or a
  fundamental expansion of spacetime itself.

  As shown in Appendix~\ref{subsec:minimal-kinematic-constraint}, the requirement that
  effective relaxation ordering remain monotonic in an expanding regime implies the
  existence of a minimal residual structural inhomogeneity in projected $\chi$
  configurations.
  In effective geometric terms, this manifests as a non-vanishing lower bound on
  gravitational acceleration.
  This bound provides a natural route to MOND-like phenomenology in appropriate
  regimes, without postulating additional dark matter particles~\cite{Milgrom2002,FamaeyMcGaugh2012}.
