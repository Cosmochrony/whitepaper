\section{Cosmological and Observational Implications of Cosmochrony}
  \label{sec:appendix-cosmo}

  This appendix examines the cosmological and observational implications of the
  Cosmochrony framework.
  Its purpose is not to construct a fully parameterized cosmological model, nor to
  perform precision fits to existing datasets, but to establish conceptual and
  phenomenological consistency with key observations and to identify distinctive
  signatures that may allow the framework to be tested or falsified.

  All results presented here should be understood as consequences of the same
  underlying scalar relaxation dynamics governing the \(\chi\) field.
  No additional cosmological degrees of freedom are introduced beyond those already
  discussed in the main body of the work.
  Standard cosmological observables arise as effective descriptions once a smooth
  geometric projection of \(\chi\) becomes applicable.

  The appendix is organized as follows:
  \begin{itemize}
    \item Section~\ref{app:lowell_attenuation} analyzes the spectrum of large-scale
    fluctuations of the projected field \(\chi_{\mathrm{eff}}\) and their imprint on
    the cosmic microwave background, with particular emphasis on the suppression of
    low-\(\ell\) multipoles.
    \item Section~\ref{subsec:cosmochrony_horizon_flatness} shows how the horizon and
    flatness problems are resolved dynamically in Cosmochrony, without invoking an
    inflationary phase or fine-tuned initial conditions.
    \item Section~\ref{app:hubble_tension} discusses the evolution of the effective
    Hubble parameter \(H(z)\) and its implications for the observed Hubble tension,
    highlighting how deviations from standard expansion histories may arise from
    relaxation dynamics.
    \item Section~\ref{subsec:observational-estimates} provides order-of-magnitude
    estimates of the characteristic \(\chi\)-field parameters and relates them to
    observable cosmological quantities, without assuming a specific cosmological
    parameter set.
    \item Section~\ref{subsec:phenomenology} explores broader phenomenological
    consequences, including modified gravitational-wave propagation, MOND-like
    effective dynamics at galactic scales, and lensing anomalies.
  \end{itemize}

  Throughout this appendix, the emphasis is placed on identifying robust qualitative
  and semi-quantitative features that follow generically from the Cosmochrony
  framework.
  Where numerical estimates are provided, they are intended as consistency checks
  and scaling guides rather than as precision predictions.

  Taken together, these results demonstrate that Cosmochrony is compatible with the
  broad structure of contemporary cosmological observations while predicting
  systematic deviations from standard \(\Lambda\)CDM expectations.
  These deviations define concrete targets for future observational tests and
  numerical investigations.

  \subsection{Spectrum of \(\chi\)-Field Fluctuations and CMB Anisotropies}
  \label{subsec:chi_cmb_spectrum}

  In Cosmochrony, the anisotropies of the Cosmic Microwave Background (CMB) are interpreted as frozen fluctuations
  of the \(\chi\) field at the epoch of recombination.
  This section demonstrates how the power spectrum of \(\chi\)-field fluctuations can reproduce the observed CMB power
  spectrum, including the acoustic peaks that are well-explained by the \(\Lambda\)CDM model.

  \subsubsection{Fluctuations of \(\chi\) and Temperature Anisotropies}

    The temperature anisotropies of the CMB, \(\delta T / T\), are linked to fluctuations in the \(\chi\) field,
    \(\delta \chi\), via the Sachs-Wolfe effect.
    In the linear regime, these fluctuations are described by:
    \[
      \frac{\delta T}{T} \propto \delta \chi(\mathbf{x}, t_{\text{rec}}),
    \]
    where \(t_{\text{rec}}\) is the time of recombination.
    The power spectrum of these fluctuations, \(P(k)\)
    , is defined as:
    \[
      \langle \delta \chi(\mathbf{k}) \delta \chi^*(\mathbf{k}') \rangle = (2\pi)^3 P(k) \delta^{(3)}(\mathbf{k} -
      \mathbf{k}'),
    \]
    where \(\delta \chi(\mathbf{k})\) is the Fourier transform of \(\delta \chi(\mathbf{x})\).

  \subsubsection{Power Spectrum of \(\chi\)-Field Fluctuations}

    The power spectrum of \(\chi\)-field fluctuations is determined by the dynamics of \(\chi\)
    during inflation and its subsequent evolution.
    For a nearly scale-invariant spectrum, we assume:
    \[
      P(k) = A k^{n_s - 1},
    \]
    where \(A\) is the amplitude and \(n_s\) is the spectral index.
    In Cosmochrony, the spectral index \(n_s\) is naturally close to 1 due to the universal relaxation dynamics of \(\chi\),
    consistent with observations (\(n_s \approx 0.96\)).

    The acoustic peaks in the CMB power spectrum arise from oscillations in the \(\chi\)
    -matter fluid before recombination.
    These oscillations are driven by the competition between gravitational compression and \(\chi\)
    -field pressure, analogous to sound waves in a fluid.
    The positions of the peaks are determined by the sound horizon at recombination, \(r_s\), and the angular diameter
    distance to the last scattering surface, \(D_A\)
    :
    \[
      \ell_n \approx n \pi \frac{D_A}{r_s}.
    \]

  \subsubsection{Comparison with \(\Lambda\)CDM Acoustic Peaks}
    \label{subsubsec:comparison-with-lambda-cdm-acoustic-peaks}

    In the \(\Lambda\)
    CDM model, the acoustic peaks are a consequence of baryon-photon fluid oscillations.
    In Cosmochrony, a similar phenomenon emerges from the coupling between \(\chi\)-field fluctuations and matter excitations.
    The key differences and similarities are:

    \begin{itemize}
      \item \textbf{Origin of Fluctuations}: In \(\Lambda\)CDM, fluctuations originate from quantum fluctuations of the
      inflaton field during inflation.
      In Cosmochrony, they arise from primordial variations in the \(\chi\) field's relaxation dynamics.
      Crucially, the phase of these acoustic oscillations is locked to the initial relaxation onset of $\chi$.
      Unlike inflation, which requires a separate ``reheating'' phase to populate the universe with particles,
      the intrinsic coupling between $\chi$-fluctuations and matter ensures that baryonic matter is born directly within
      these geometric ripples.
      This provides a natural mechanism for the phase coherence of the acoustic peaks without invoking super-horizon
      inflationary correlation.

      \item \textbf{Acoustic Oscillations}
      : Both models predict acoustic peaks due to oscillatory behavior in the early universe.
      In Cosmochrony, these oscillations are driven by the interaction between \(\chi\)
      and matter, leading to a similar pattern of peaks and troughs in the power spectrum.

      \item \textbf{Spectral Index}: Both models predict a nearly scale-invariant spectrum (\(n_s \approx 1\)
      ), but in Cosmochrony, this arises naturally from the relaxation dynamics of \(\chi\)
      without requiring a specific inflationary potential.

      \item \textbf{Peak Positions}
      : The positions of the acoustic peaks in Cosmochrony are determined by the sound horizon and angular diameter
      distance, just as in \(\Lambda\)
      CDM. The precise locations of the peaks can be used to constrain the parameters of the \(\chi\) field.
    \end{itemize}

  \subsubsection{Quantitative Estimation of the Power Spectrum}

    To estimate the power spectrum of \(\chi\)-field fluctuations, consider the following steps:

    \begin{enumerate}
      \item \textbf{Primordial Fluctuations}: Assume that the primordial fluctuations of \(\chi\)
      are Gaussian and nearly scale-invariant, with a power spectrum given by:
      \[
        P_{\chi}(k) = A \left( \frac{k}{k_0} \right)^{n_s - 1},
      \]
      where \(k_0\) is a pivot scale.

      \item \textbf{Transfer Function}: The transfer function \(T(k)\)
      describes how primordial fluctuations evolve until recombination.
      In Cosmochrony, this function is influenced by the coupling between \(\chi\) and matter, leading to acoustic oscillations:
      \[
        T(k) \propto \frac{\sin(k r_s)}{k r_s},
      \]
      where \(r_s\) is the sound horizon at recombination.

      \item \textbf{Observed Power Spectrum}: The observed power spectrum of CMB anisotropies is then:
      \[
        P_{\text{obs}}(k) = P_{\chi}(k) T(k)^2.
      \]
      This results in a series of acoustic peaks at scales determined by \(r_s\) and the angular diameter distance
      \(D_A\).
    \end{enumerate}

  \subsubsection{Implications for Cosmochrony}

    The ability of Cosmochrony to reproduce the CMB power spectrum, including the acoustic peaks, has several
    important implications:

    \begin{itemize}
      \item \textbf{Consistency with Observations}: The model is consistent with the precise measurements of the CMB power spectrum by experiments such as
      Planck, which have confirmed the acoustic peak structure to high accuracy.

      \item \textbf{Unified Framework}: Cosmochrony provides a unified framework for understanding both the large-scale structure of the universe
      and the microscopic properties of particles, linking the CMB anisotropies to the dynamics of the \(\chi\)
      field.

      \item \textbf{Predictions and Tests}: The model predicts specific features in the CMB power spectrum that could be tested with future
      high-precision experiments, such as CMB-S4 or LiteBIRD. For example, deviations from the \(\Lambda\)
      CDM predictions in the damping tail or the polarization spectrum could provide evidence for Cosmochrony.
    \end{itemize}

  \subsection{Resolution of the Horizon and Flatness Problems Without Inflation}
  \label{subsec:cosmochrony_horizon_flatness}

  In standard cosmology, the horizon and flatness problems arise from extrapolating
  a spacetime-based notion of causality and geometry back to the earliest stages of
  cosmic evolution.
  Within this framework, regions of the universe that appear widely separated today
  should not have been in causal contact, and the near-flatness of spatial geometry
  requires fine-tuned initial conditions.
  Inflation addresses these issues by postulating a brief phase of accelerated
  expansion in a pre-existing metric background.

  Cosmochrony adopts a fundamentally different standpoint.
  Spacetime geometry, causal structure, and metric notions of distance are not
  assumed to be fundamental.
  They emerge only at a later stage, as effective descriptions of the relaxation
  dynamics of the scalar field \(\chi\).
  As a result, the assumptions underlying the horizon and flatness problems do not
  apply at the fundamental level.

  \paragraph{Horizon problem: pre-geometric connectivity.}
    In Cosmochrony, large-scale correlations do not need to be established through
    signal propagation within spacetime.
    Instead, they originate from the fact that \(\chi\) constitutes a single,
    globally connected dynamical substrate whose relaxation precedes the emergence of
    any effective spacetime description.

    At early stages, before a metric notion of causality becomes meaningful, the
    configuration of \(\chi\) is defined globally.
    Regions that later appear causally disconnected in the emergent spacetime may
    therefore share correlated configurations inherited from earlier phases of the
    relaxation process.

    In this sense, Cosmochrony replaces inflationary causal contact with
    \emph{pre-geometric connectivity}:
    correlations are established at the level of the fundamental field itself, rather
    than through superluminal expansion or specially prepared initial conditions on a
    metric background.

  \paragraph{Flatness problem: relaxation toward geometric uniformity.}
    The flatness problem is addressed through the same underlying mechanism.
    In Cosmochrony, effective spatial curvature reflects large-scale gradients and
    inhomogeneities in the relaxation rate of the projected field
    \(\chi_{\mathrm{eff}}\).
    As relaxation proceeds, configurations with large curvature gradients are
    dynamically disfavored, since they correspond to sustained resistance to global
    relaxation.

    As a consequence, near-flat spatial geometry emerges as a natural attractor of the
    relaxation dynamics.
    Curvature dilution does not require exponential expansion or fine-tuning of
    initial curvature parameters.
    It reflects the tendency of the \(\chi\) field to minimize large-scale geometric
    tension as it approaches a homogeneous relaxation state.

    This mechanism operates independently of any inflationary phase and does not rely
    on a specific initial curvature value.

  \paragraph{Implications for primordial correlations.}
    Because large-scale coherence arises from the global organization of \(\chi\)
    rather than from the amplification of quantum vacuum fluctuations, Cosmochrony
    does not predict exact scale invariance at the largest wavelengths.
    Instead, the longest-wavelength modes are subject to global relaxation
    constraints, which may lead to deviations from scale invariance at the lowest
    multipoles of the cosmic microwave background.

    Such deviations are interpreted as structural tendencies rather than sharp
    predictions.
    They provide a qualitative distinction from inflation-based scenarios and
    motivate the phenomenological analysis of low-\(\ell\) CMB anomalies discussed in
    Section~\ref{app:lowell_attenuation}.

  \paragraph{Status and limitations.}
    The arguments presented here establish that the horizon and flatness problems do
    not arise as fundamental inconsistencies within the Cosmochrony framework.
    They are artifacts of applying metric-based reasoning beyond its domain of
    validity.

    A quantitative derivation of primordial correlation functions and power spectra,
    including detailed predictions for CMB anisotropies, requires dedicated numerical
    simulations of the \(\chi\)-field relaxation dynamics and lies beyond the scope of
    the present work.

    Nevertheless, Cosmochrony provides a conceptually coherent, inflation-free
    resolution of large-scale causal coherence and near-flat spatial geometry, rooted
    in the pre-geometric dynamics of a single scalar field.

  \subsection{Evolution of the Hubble Parameter and the Hubble Tension}
  \label{app:hubble_tension}

  In the Cosmochrony framework, cosmological expansion is not governed by the
  competition between matter, radiation, and a dark energy component.
  Instead, it reflects the relaxation dynamics of the scalar field \(\chi\), from
  which spacetime geometry and its associated expansion rate emerge as effective
  descriptions.

  The Hubble parameter therefore encodes the instantaneous relaxation rate of
  \(\chi\) relative to its global configuration, rather than the response of a
  metric to an energy--momentum content.

  \paragraph{Global expansion rate.}
    At the homogeneous background level, the effective scale factor is proportional
    to the global value of the projected field,
    \begin{equation}
      a(t) \propto \chi(t),
    \end{equation}
    so that the Hubble parameter may be written as
    \begin{equation}
      H(t) = \frac{\dot{\chi}}{\chi}.
    \end{equation}

    In the idealized homogeneous limit, spatial gradients vanish
    (\(\nabla \chi = 0\)) and the relaxation dynamics reduce to a uniform evolution
    with maximal relaxation speed,
    \begin{equation}
      \dot{\chi} = c .
    \end{equation}
    In this limit, the global expansion rate becomes
    \begin{equation}
      H(t) = \frac{c}{\chi(t)} .
    \end{equation}

    This relation defines the \emph{global} expansion rate in Cosmochrony.
    Its detailed redshift dependence away from perfect homogeneity is not assumed to
    follow a fixed power law and depends on how relaxation gradients contribute to
    the averaged dynamics.

  \paragraph{Relaxation budget and effective expansion.}
    In a realistic universe, part of the relaxation capacity of \(\chi\) is stored
    in spatial gradients associated with inhomogeneities.
    To quantify this effect, we introduce a dimensionless \emph{relaxation budget}
    parameter,
    \begin{equation}
      \Omega_\chi \equiv \langle \beta^2 \rangle,
      \qquad
      \beta \equiv \frac{|\nabla \chi|}{c},
      \label{eq:chi-relaxation}
    \end{equation}
    which measures the fraction of the total relaxation capacity diverted into
    spatial structure rather than global evolution.

    At late times, these gradients are dominated by localized solitonic
    configurations and therefore track the large-scale matter distribution.
    The effective global expansion rate is then reduced to
    \begin{equation}
      \bar{H}
      =
      \frac{c}{\chi}
      \sqrt{1 - \Omega_\chi}.
    \end{equation}

    Empirically, consistency with large-scale observations suggests
    \(\Omega_\chi\) is of order the observed matter fraction,
    \(\Omega_\chi \sim 0.3\).
    In Cosmochrony, this suppression of the global expansion rate arises naturally
    from relaxation dynamics and does not require a dark energy component.

  \paragraph{Local expansion and environmental dependence.}
    In an inhomogeneous universe, the relaxation budget is not spatially uniform.
    Regions with different matter densities redistribute relaxation capacity
    differently between global evolution and local gradients.

    For a region characterized by a density contrast
    \begin{equation}
      \delta = \frac{\rho - \bar{\rho}}{\bar{\rho}},
    \end{equation}
    we adopt a minimal mean-field closure relation,
    \begin{equation}
      \beta_{\mathrm{loc}}^{2}
      =
      \Omega_\chi (1 + \delta),
    \end{equation}
    which encodes the intuitive scaling between matter density and
    \(\chi\)-gradient energy.

    The locally inferred Hubble parameter then takes the form
    \begin{equation}
      H_{\mathrm{loc}}
      =
      \bar{H}
      \sqrt{
        \frac{1 - \Omega_\chi (1+\delta)}
        {1 - \Omega_\chi}
      } .
    \end{equation}

    In underdense regions (\(\delta < 0\)), a larger fraction of the relaxation
    capacity is available for global evolution, leading to
    \(H_{\mathrm{loc}} > \bar{H}\).

  \paragraph{Numerical consistency and the Hubble tension.}
    For representative values
    \(\Omega_\chi \approx 0.3\) and a local underdensity consistent with the
    KBC void (\(\delta \approx -0.4\) on scales of a few hundred megaparsecs), one
    finds
    \begin{equation}
      \frac{H_{\mathrm{loc}}}{\bar{H}} \approx 1.08 ,
    \end{equation}
    corresponding to an enhancement of order \(8\%\) in the locally inferred Hubble
    constant.

    This magnitude is comparable to the observed discrepancy between local
    distance-ladder measurements and global CMB-based inferences.
    In Cosmochrony, this discrepancy arises naturally as an environmental effect,
    without invoking new energy components or modifications of early-universe
    physics.

  \paragraph{Interpretation and status.}
    Within Cosmochrony, the Hubble tension does not signal a breakdown of cosmological
    consistency.
    It reflects the fact that cosmological expansion is an emergent relaxation
    phenomenon whose effective rate depends on the local redistribution of
    \(\chi\)-field gradients.

    While the framework robustly predicts a separation between local and global
    expansion rates, a fully quantitative determination of \(H(z)\) across all
    redshifts requires dedicated numerical simulations of the \(\chi\) relaxation
    dynamics.
    Such simulations lie beyond the scope of the present work.

    Nevertheless, the qualitative resolution of the Hubble tension follows directly
    from the relaxation-based interpretation of cosmological expansion and
    constitutes a distinctive and testable signature of the Cosmochrony framework.

  \subsection{Relation to Observational Units and Numerical Estimates}
  \label{subsec:observational-estimates}

  This subsection establishes order-of-magnitude relations linking the Cosmochrony
  framework to observed cosmological quantities.
  Its purpose is not to perform parameter fitting or to derive precision
  predictions, but to assess the internal consistency of the theory and to verify
  that its fundamental relaxation-based interpretation naturally reproduces the
  correct empirical scales.

  All numerical relations presented here should be understood as effective
  normalizations arising in the projectable regime of the \(\chi\) dynamics.
  They do not define fundamental constants and do not fix the microscopic structure
  of the theory.

  \subsubsection*{Normalization of the \texorpdfstring{$\chi$}{χ} Field}
    \label{subsec:normalization-of-the-chi-field}

    To relate the projected field \(\chi_{\mathrm{eff}}\) to observable quantities, a
    reference normalization must be specified.
    At the effective cosmological level, the scale factor is defined up to a global
    multiplicative constant.
    In Cosmochrony, this freedom is fixed by identifying the present-day value
    \(\chi(t_0)\) with the characteristic geometric scale governing large-scale
    expansion.

    Operationally, \(\chi(t_0)\) represents the cumulative geometric scale associated
    with the global relaxation of the \(\chi\) field up to the present epoch.
    This identification does not assume a unique microscopic origin for
    \(\chi(t_0)\); it provides a minimal and observationally anchored normalization
    consistent with the effective relation \(a(t) \propto \chi(t)\).

  \subsubsection*{Emergent Gravitational Coupling}
    \label{subsec:K0-chic-constraints}

    In the effective geometric description, the Newtonian gravitational constant
    \(G\) emerges from the constitutive relation governing the coupling between
    neighboring configurations of the projected \(\chi\) field.
    This coupling is controlled by two parameters of the relaxation dynamics: the
    maximal stiffness scale \(K_0\) and the characteristic correlation length
    \(\chi_c\).

    Although \(K_0\) and \(\chi_c\) are not individually fixed at the present stage,
    their combination is constrained by matching the observed gravitational coupling:
    \begin{equation}
      K_0 \chi_c^2 \sim \frac{c^4}{16 \pi G}.
    \end{equation}

    This relation fixes the overall stiffness scale of the effective \(\chi\)
    network.
    It does not require committing to a specific microscopic interpretation of
    \(\chi_c\), which may correspond to a fundamental correlation scale or to an
    emergent coarse-graining length.
    At this level, only the product \(K_0 \chi_c^2\) is observationally relevant.

  \subsubsection*{Hubble Constant}
    \label{subsec:hubble-constant}

    In the homogeneous limit, the effective Hubble parameter is defined by the
    relative relaxation rate of the projected field,
    \begin{equation}
      H(t) = \frac{\dot{\chi}}{\chi}.
    \end{equation}

    Assuming that the present universe lies close to the maximal relaxation regime,
    \(\dot{\chi}(t_0) \simeq c\), the present-day Hubble constant follows as
    \begin{equation}
      H_0 \simeq \frac{c}{\chi(t_0)}.
    \end{equation}

    Using the observed value
    \(H_0 \approx 70~\mathrm{km\,s^{-1}\,Mpc^{-1}}\) yields
    \begin{equation}
      \chi(t_0) \sim 4 \times 10^{26}~\mathrm{m},
    \end{equation}
    which is of the order of the observed Hubble radius.
    This correspondence arises directly from the relaxation-based interpretation of
    cosmic expansion and does not require the introduction of additional cosmological
    parameters.

  \subsubsection*{Age of the Universe}
    \label{subsec:age-of-the-universe}

    In the homogeneous relaxation regime, the evolution of \(\chi\) may be
    approximated as
    \begin{equation}
      \dot{\chi} \simeq c ,
    \end{equation}
    leading to
    \begin{equation}
      \chi(t) \simeq c t + \chi_{\mathrm{init}},
    \end{equation}
    where \(\chi_{\mathrm{init}}\) denotes the effective value of \(\chi\) at the onset
    of the relaxation regime relevant for cosmological observations.

    Neglecting \(\chi_{\mathrm{init}}\) compared to present values yields
    \begin{equation}
      t_0 \simeq \frac{\chi(t_0)}{c} \sim 4 \times 10^{17}~\mathrm{s},
    \end{equation}
    corresponding to approximately \(13.8\) billion years.
    This estimate is consistent with standard cosmological age determinations and
    follows directly from the bounded relaxation dynamics.

  \subsubsection*{Redshift Interpretation}
    \label{subsec:redshift-interpretation}

    In Cosmochrony, cosmological redshift is interpreted as a consequence of the
    relative change in the projected \(\chi\) field between emission and observation,
    \begin{equation}
      1 + z = \frac{\chi(t_{\mathrm{obs}})}{\chi(t_{\mathrm{emit}})}.
    \end{equation}

    This relation reproduces standard redshift phenomenology while attributing it to
    geometric scaling induced by \(\chi\) relaxation, rather than to recessional
    motion within a pre-existing spacetime background.

  \subsubsection*{Cosmic Microwave Background Scale}
    \label{subsec:cosmic-microwave-background-scale}

    At recombination, characterized observationally by
    \(z_{\mathrm{rec}} \simeq 1100\), the effective value of the projected field was
    smaller by the corresponding scaling factor,
    \begin{equation}
      \chi(t_{\mathrm{rec}}) \simeq \frac{\chi(t_0)}{1 + z_{\mathrm{rec}}}.
    \end{equation}

    Fluctuations imprinted at that epoch are subsequently stretched by the monotonic
    growth of \(\chi\), providing a natural geometric interpretation of the angular
    scales observed in the cosmic microwave background without invoking an inflationary
    stretching phase.

  \subsubsection*{Orders of Magnitude and Robustness}
    \label{subsec:orders-of-magnitude-and-robustness}

    All numerical estimates presented in this subsection rely solely on observed
    cosmological quantities and on the bounded relaxation dynamics of the \(\chi\)
    field.
    No fine-tuning of parameters, no detailed cosmological fitting, and no additional
    degrees of freedom are assumed.

    While a fully predictive cosmological model requires explicit numerical
    simulations of the \(\chi\) dynamics, these order-of-magnitude relations
    demonstrate that Cosmochrony naturally reproduces the correct scales for the
    Hubble constant, the age of the universe, redshift evolution, and characteristic
    CMB features.

  \subsubsection*{Summary}
    \label{subsec:summary}

    The Cosmochrony framework admits a consistent normalization in observational
    units and reproduces key cosmological scales without introducing new fundamental
    parameters.
    These order-of-magnitude relations support the internal coherence of the theory
    and motivate further quantitative investigation of its cosmological dynamics.

  \subsection{Phenomenological Implications}
  \label{subsec:phenomenology}

  This subsection summarizes the principal phenomenological consequences of
  Cosmochrony that are accessible to observation.
  The emphasis is placed on effects that follow robustly from the kinematic and
  relaxation structure of the \(\chi\) field itself, without introducing auxiliary
  degrees of freedom, adjustable interpolation functions, or phenomenological
  potentials.

  All results presented here arise in the projectable regime of the theory and
  should be understood as effective manifestations of the underlying relaxation
  dynamics, not as fundamental postulates.

  \paragraph{Propagation speed of gravitational perturbations.}
    To determine the propagation speed of gravitational information in Cosmochrony,
    consider small perturbations \(\delta\chi\) around a homogeneous relaxation
    background,
    \begin{equation}
      \chi_0(t) = c t ,
    \end{equation}
    such that
    \begin{equation}
      \chi(\mathbf{x},t) = c t + \delta\chi(\mathbf{x},t),
      \qquad |\nabla \delta\chi| \ll c .
    \end{equation}

    Substituting this form into the fundamental kinematic constraint governing
    \(\chi\)-field relaxation (Eq.~\ref{eq:chi_dynamics}) yields, to leading order,
    \begin{equation}
      c + \partial_t \delta\chi
      =
      c \sqrt{1 - \frac{|\nabla \delta\chi|^2}{c^2}} .
    \end{equation}

    Expanding for small spatial gradients gives
    \begin{equation}
      \partial_t \delta\chi
      \simeq
      -\frac{|\nabla \delta\chi|^2}{2c},
    \end{equation}
    reflecting the irreversible character of the relaxation process.
    While this first-order relation governs dissipation, the propagation of
    perturbations is more transparently captured by considering the second-order
    operator associated with the squared constraint.

    Linearizing this operator leads to the effective wave equation
    \begin{equation}
      \left(
        \frac{1}{c^2}\partial_t^2 - \nabla^2
      \right)
      \delta\chi = 0 ,
      \label{eq:gw_wave}
    \end{equation}
    which admits propagating solutions with characteristic speed
    \begin{equation}
      v_{\mathrm{prop}} = c .
    \end{equation}

    Gravitational perturbations therefore propagate exactly at the invariant speed
    \(c\).
    This equality is not imposed by hand but follows directly from the fundamental
    kinematic bound on \(\chi\) relaxation.
    As a result, Cosmochrony is automatically consistent with multi-messenger
    observations, including the near-simultaneous arrival of gravitational and
    electromagnetic signals in events such as GW170817.

  \paragraph{Emergent acceleration scale and MOND-like phenomenology.}
    In Cosmochrony, the arrow of time is encoded in the monotonic evolution of the
    fundamental field, \(\partial_t \chi \ge 0\).
    At late cosmic times and on sufficiently large scales, where
    \(\chi_{\mathrm{eff}}\) admits an approximately homogeneous description, the
    relaxation dynamics may be coarse-grained into an effective cosmological clock.

    In this effective regime, the temporal evolution of \(\chi\) can be written as
    \begin{equation}
      \partial_t \chi \simeq H(t)\,\chi ,
    \end{equation}
    where \(H(t)\) denotes the emergent Hubble parameter associated with global
    relaxation.

    The local kinematic constraint
    \begin{equation}
    (\partial_t \chi)^2 + |\nabla \chi|^2 = c^2
    \end{equation}
    then implies that even in the absence of localized matter excitations, the
    cosmological evolution of \(\chi\) enforces a non-vanishing residual spatial
    gradient.
    In the homogeneous limit, this minimal gradient is
    \begin{equation}
      |\nabla \chi|_{\min}
      =
      \sqrt{c^2 - (H\chi)^2} .
    \end{equation}

    This residual gradient defines a background kinematic scale that constrains the
    superposition of additional, locally induced gradients.
    Operationally, it corresponds to an effective acceleration scale
    \begin{equation}
      a_0(t) \sim c\,H(t) .
    \end{equation}

    When localized matter excitations are present, they induce additional gradients
    \(\nabla\chi_N\) that reproduce the Newtonian scaling
    \(|\nabla\chi_N| \propto M/r^2\) at short distances.
    Because the kinematic constraint is nonlinear, the total gradient does not
    superpose linearly.
    At sufficiently large radii, the effective acceleration asymptotically approaches
    \begin{equation}
      g_{\mathrm{eff}}
      \simeq
      \sqrt{g_N\,a_0(t)} ,
    \end{equation}
    recovering the characteristic deep-MOND scaling without introducing interpolation
    functions, dark matter particles, or additional fields.

    In this framework, the acceleration scale \(a_0\) is not fundamental.
    It evolves slowly with cosmic time through its dependence on \(H(t)\), providing a
    potential observational discriminator at high redshift.

  \paragraph{Gravitational lensing.}
    In Cosmochrony, light propagation follows wavefronts of constant \(\chi\).
    An effective refractive index for the vacuum may be defined operationally as
    \begin{equation}
      n(r)
      =
      \frac{c}{\partial_t \chi}
      =
      \frac{1}{\sqrt{1 - |\nabla \chi|^2/c^2}} .
    \end{equation}

    Near a localized mass \(M\), where
    \(|\nabla \chi| \simeq GM/(c^2 r)\),
    a weak-field expansion yields
    \begin{equation}
      n(r) \simeq 1 + \frac{GM}{c^2 r} .
    \end{equation}

    Integrating the transverse gradient of \(n(r)\) along a photon trajectory leads to
    a deflection angle
    \begin{equation}
      \alpha = \frac{4GM}{b c^2},
    \end{equation}
    where \(b\) is the impact parameter.
    This reproduces the general-relativistic prediction for gravitational lensing.

    In Cosmochrony, the enhancement relative to the Newtonian deflection does not
    originate from a fundamental spacetime curvature.
    It arises from the nonlinear structure of the \(\chi\) relaxation dynamics, which
    modifies the effective propagation geometry experienced by light.

  \paragraph{Summary.}
    The phenomenology of Cosmochrony reproduces key observational signatures of gravity
    and cosmology while relying on a single scalar degree of freedom.
    Gravitational perturbations propagate at exactly the invariant speed \(c\), a
    MOND-like acceleration scale emerges naturally from cosmological relaxation, and
    gravitational lensing is recovered without postulating a fundamental metric.

    These results illustrate how classical gravitational phenomena arise as
    coarse-grained manifestations of the underlying \(\chi\) dynamics and define a
    set of observationally testable signatures distinguishing Cosmochrony from
    standard metric-based theories.

  \section{Appendix E: Toy-Model of Spectral Gravitational Susceptibility}
  \label{app:spectral-gravitational-toy-model}

  This appendix provides the mathematical foundations for a non-particulate interpretation of dark matter phenomena,
  treating galactic dynamics as the non-linear elastic response of the $\chi$ substrate.

  \subsection{Modified Poisson Equation and Field Strength}
    In the Cosmochrony framework, gravitational acceleration is not a force acting in a passive vacuum but the emergent
    manifestation of a relaxation gradient. We define the \textbf{local relaxation field strength}
    as the gradient of the scalar relaxation potential:
    \begin{equation}
      \mathbf{E}_\chi = - \nabla \Phi_\chi
    \end{equation}
    The dynamics are governed by a modified Poisson equation analogous to electrodynamics in continuous media:
    \begin{equation}
      \nabla \cdot \left[ \epsilon_{\text{spec}}(\mathbf{E}_\chi) \mathbf{E}_\chi \right] = 4\pi G_0 \rho_b
    \end{equation}
    where $\rho_b$ is the baryonic mass density and $\epsilon_{\text{spec}}$ is the \textbf{spectral permittivity}
    of the substrate, defined by the relation $\epsilon_{\text{spec}} = 1 + \phi(\mathbf{E}_\chi)$.

  \subsection{Spectral Susceptibility and the Stiffness Threshold $\mathcal{K}_c$}
    To recover the observed galactic phenomenology, we define the \textbf{spectral gravitational susceptibility} $\phi$
    as a function of the field strength relative to a saturation threshold $\mathcal{K}_c$:
    \begin{equation}
      \phi(\mathbf{E}_\chi) =
      \begin{cases}
        0 & \text{for } |\mathbf{E}_\chi| \gg \mathcal{K}_c \text{ (Linear/Newtonian Regime)} \\
        \frac{\mathcal{K}_c}{|\mathbf{E}_\chi|} & \text{for } |\mathbf{E}_\chi| \ll \mathcal{K}_c
        \text{ (Saturation Regime)}
      \end{cases}
    \end{equation}
    Crucially, $\mathcal{K}_c$ is not a universal constant of nature but a \textbf{local state property}
    of the substrate. It represents the threshold where the relaxation flux reaches the elastic limit of the $\chi$
    field.

  \subsection{Emergence of Flat Rotation Curves}
    In the low-field limit ($|\mathbf{E}_\chi| \ll \mathcal{K}_c$
    ) typical of galactic peripheries, the effective acceleration $g_{\text{eff}}$ follows:
    \begin{equation}
      \nabla \cdot \left( \mathcal{K}_c \frac{\mathbf{E}_\chi}{|\mathbf{E}_\chi|} \right) \sim 4\pi G_0 \rho_b \implies
      g_{\text{eff}} \approx \frac{\sqrt{G_0 M \mathcal{K}_c}}{r}
    \end{equation}
    This leads directly to a constant orbital velocity $v^4 = G_0 M \mathcal{K}_c$, recovering the
    \textbf{Baryonic Tully-Fisher Relation}.
    Here, ``dark matter'' is reinterpreted as the increased elastic response of the substrate in regions of diluted
    relaxation flux.

  \subsection{Comparative Framework: MOND vs. Cosmochrony}
    The following table summarizes the conceptual shift from modified gravity to substrate dynamics.

    \begin{table}[h]
      \centering
      \begin{tabular}{|l|l|l|}
        \hline
        \textbf{Feature}        & \textbf{MOND (Milgrom)}           & \textbf{Cosmochrony ($\chi$ Substrate)}    \\
        \hline
        \textbf{Origin}         & Modified law of inertia/force.    & Non-linear susceptibility of the medium.   \\
        \hline
        \textbf{Threshold}      & Universal constant $a_0$.         & Local stiffness threshold $\mathcal{K}_c$. \\
        \hline
        \textbf{Bullet Cluster} & Requires additional DM particles. & Natural: Relaxation hysteresis (wake).     \\
        \hline
        \textbf{GR Relation}    & Requires $TeVeS$ or similar.      & GR is the linear-response limit.           \\
        \hline
        \textbf{DM Nature}      & Force discrepancy.                & Residual non-projected energy.             \\
        \hline
      \end{tabular}
      \caption{Comparison between MOND phenomenology and Cosmochrony substrate response.}
    \end{table}

  \subsection{Limitations and Outlook}

    \paragraph{Theoretical Refinement.}
      The current form of $\phi(\mathbf{E}_\chi)$
      is phenomenological. A rigorous derivation from the microscopic relaxation equations in Appendix D is required
      to link $\mathcal{K}_c$ to the global Hubble relaxation rate.

    \paragraph{The Relaxation Wake.}
      Cosmochrony predicts that in high-energy collisions (e.g., Bullet Cluster), the geometric deformation of the
      substrate exhibits a \textbf{phase lag} (hysteresis).
      Gravitational lensing tracks this ``residual wake'' of the mass-solitons, explaining the offset from dissipative
      gas.
      A specific prediction of this model is the existence of \textbf{spectral echoes}: residual curvature in regions
      where matter has recently passed, a signature that could distinguish Cosmochrony from WIMP-based models.

    \paragraph{General Relativity Limit.}
      Finally, it is emphasized that Cosmochrony reduces to General Relativity in the linear-response limit of the
      $\chi$substrate.
      Spacetime curvature is the refractive manifestation of the substrate's spectral density, and gravity is its
      macroscopic relaxation.

