\subsection{Phenomenological Implications}
  \label{subsec:phenomenology}

  This subsection summarizes the principal phenomenological consequences of
  Cosmochrony that are accessible to observation.
  The emphasis is placed on effects that follow robustly from the kinematic and
  relaxation structure of the \(\chi\) field itself, without introducing auxiliary
  degrees of freedom, adjustable interpolation functions, or phenomenological
  potentials.

  All results presented here arise in the projectable regime of the theory and
  should be understood as effective manifestations of the underlying relaxation
  dynamics, not as fundamental postulates.

  \paragraph{Propagation speed of gravitational perturbations.}
    To determine the propagation speed of gravitational information in Cosmochrony,
    consider small perturbations \(\delta\chi\) around a homogeneous relaxation
    background,
    \begin{equation}
      \chi_0(t) = c t ,
    \end{equation}
    such that
    \begin{equation}
      \chi(\mathbf{x},t) = c t + \delta\chi(\mathbf{x},t),
      \qquad |\nabla \delta\chi| \ll c .
    \end{equation}

    Substituting this form into the fundamental kinematic constraint governing
    \(\chi\)-field relaxation (Eq.~\ref{eq:chi_dynamics}) yields, to leading order,
    \begin{equation}
      c + \partial_t \delta\chi
      =
      c \sqrt{1 - \frac{|\nabla \delta\chi|^2}{c^2}} .
    \end{equation}

    Expanding for small spatial gradients gives
    \begin{equation}
      \partial_t \delta\chi
      \simeq
      -\frac{|\nabla \delta\chi|^2}{2c},
    \end{equation}
    reflecting the irreversible character of the relaxation process.
    While this first-order relation governs dissipation, the propagation of
    perturbations is more transparently captured by considering the second-order
    operator associated with the squared constraint.

    Linearizing this operator leads to the effective wave equation
    \begin{equation}
      \left(
        \frac{1}{c^2}\partial_t^2 - \nabla^2
      \right)
      \delta\chi = 0 ,
      \label{eq:gw_wave}
    \end{equation}
    which admits propagating solutions with characteristic speed
    \begin{equation}
      v_{\mathrm{prop}} = c .
    \end{equation}

    Gravitational perturbations therefore propagate exactly at the invariant speed
    \(c\).
    This equality is not imposed by hand but follows directly from the fundamental
    kinematic bound on \(\chi\) relaxation.
    As a result, Cosmochrony is automatically consistent with multi-messenger
    observations, including the near-simultaneous arrival of gravitational and
    electromagnetic signals in events such as GW170817.

  \paragraph{Emergent acceleration scale and MOND-like phenomenology.}
    In Cosmochrony, the arrow of time is encoded in the monotonic evolution of the
    fundamental field, \(\partial_t \chi \ge 0\).
    At late cosmic times and on sufficiently large scales, where
    \(\chi_{\mathrm{eff}}\) admits an approximately homogeneous description, the
    relaxation dynamics may be coarse-grained into an effective cosmological clock.

    In this effective regime, the temporal evolution of \(\chi\) can be written as
    \begin{equation}
      \partial_t \chi \simeq H(t)\,\chi ,
    \end{equation}
    where \(H(t)\) denotes the emergent Hubble parameter associated with global
    relaxation.

    The local kinematic constraint
    \begin{equation}
    (\partial_t \chi)^2 + |\nabla \chi|^2 = c^2
    \end{equation}
    then implies that even in the absence of localized matter excitations, the
    cosmological evolution of \(\chi\) enforces a non-vanishing residual spatial
    gradient.
    In the homogeneous limit, this minimal gradient is
    \begin{equation}
      |\nabla \chi|_{\min}
      =
      \sqrt{c^2 - (H\chi)^2} .
    \end{equation}

    This residual gradient defines a background kinematic scale that constrains the
    superposition of additional, locally induced gradients.
    Operationally, it corresponds to an effective acceleration scale
    \begin{equation}
      a_0(t) \sim c\,H(t) .
    \end{equation}

    When localized matter excitations are present, they induce additional gradients
    \(\nabla\chi_N\) that reproduce the Newtonian scaling
    \(|\nabla\chi_N| \propto M/r^2\) at short distances.
    Because the kinematic constraint is nonlinear, the total gradient does not
    superpose linearly.
    At sufficiently large radii, the effective acceleration asymptotically approaches
    \begin{equation}
      g_{\mathrm{eff}}
      \simeq
      \sqrt{g_N\,a_0(t)} ,
    \end{equation}
    recovering the characteristic deep-MOND scaling without introducing interpolation
    functions, dark matter particles, or additional fields.

    In this framework, the acceleration scale \(a_0\) is not fundamental.
    It evolves slowly with cosmic time through its dependence on \(H(t)\), providing a
    potential observational discriminator at high redshift.

  \paragraph{Gravitational lensing.}
    In Cosmochrony, light propagation follows wavefronts of constant \(\chi\).
    An effective refractive index for the vacuum may be defined operationally as
    \begin{equation}
      n(r)
      =
      \frac{c}{\partial_t \chi}
      =
      \frac{1}{\sqrt{1 - |\nabla \chi|^2/c^2}} .
    \end{equation}

    Near a localized mass \(M\), where
    \(|\nabla \chi| \simeq GM/(c^2 r)\),
    a weak-field expansion yields
    \begin{equation}
      n(r) \simeq 1 + \frac{GM}{c^2 r} .
    \end{equation}

    Integrating the transverse gradient of \(n(r)\) along a photon trajectory leads to
    a deflection angle
    \begin{equation}
      \alpha = \frac{4GM}{b c^2},
    \end{equation}
    where \(b\) is the impact parameter.
    This reproduces the general-relativistic prediction for gravitational lensing.

    In Cosmochrony, the enhancement relative to the Newtonian deflection does not
    originate from a fundamental spacetime curvature.
    It arises from the nonlinear structure of the \(\chi\) relaxation dynamics, which
    modifies the effective propagation geometry experienced by light.

  \paragraph{Summary.}
    The phenomenology of Cosmochrony reproduces key observational signatures of gravity
    and cosmology while relying on a single scalar degree of freedom.
    Gravitational perturbations propagate at exactly the invariant speed \(c\), a
    MOND-like acceleration scale emerges naturally from cosmological relaxation, and
    gravitational lensing is recovered without postulating a fundamental metric.

    These results illustrate how classical gravitational phenomena arise as
    coarse-grained manifestations of the underlying \(\chi\) dynamics and define a
    set of observationally testable signatures distinguishing Cosmochrony from
    standard metric-based theories.
