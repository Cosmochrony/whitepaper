\subsection{Phenomenological Implications}
  \label{subsec:phenomenology}

  \subsubsection{Speed of Gravitational Perturbations}
  \label{sec:gw_speed}

  To determine the propagation speed of gravitational information, we consider a small perturbation $\delta\chi$ around
  a homogeneous background $\chi_0(t) = ct$. Let $\chi(\mathbf{x}, t) = ct + \delta\chi(\mathbf{x}, t)$, where
  $|\nabla \delta\chi| \ll c$. Substituting this into the evolution equation~\eqref{eq:chi_dynamics}:
  \begin{equation}
    c + \partial_t \delta\chi = c \sqrt{1 - \frac{|\nabla \delta\chi|^2}{c^2}}
  \end{equation}
  Using the Taylor expansion $\sqrt{1-u} \approx 1 - u/2$ for small $u$:
  \begin{equation}
    c + \partial_t \delta\chi \approx c \left( 1 - \frac{|\nabla \delta\chi|^2}{2c^2} \right) = c - \frac{|\nabla \delta\chi|^2}{2c}
  \end{equation}
  This gives $\partial_t \delta\chi \approx -\frac{1}{2c} |\nabla \delta\chi|^2$.
  To find the wave equation, we take the time derivative of this expression and assume the perturbations follow a
  harmonic or eikonal form.
  More fundamentally, by squaring the Hamiltonian constraint~\eqref{eq:hamiltonian_constraint} and linearizing the
  resulting second-order operator, we obtain the d'Alembertian:
  \begin{equation}
    \left( \frac{1}{c^2} \partial_t^2 - \nabla^2 \right) \delta\chi = 0
  \end{equation}
  The characteristic speed is identically $c$.
  This result is robust and independent of any coupling constant, ensuring that Cosmochrony is strictly consistent with
  the GW170817 multi-messenger observation.

\subsubsection{Emergence and Evolution of an Effective MOND-like Acceleration Scale}
  \label{sec:mond_derivation}

  In Cosmochrony, the intrinsic arrow of time is encoded in the monotonic evolution of the fundamental
  field $\chi$, with $\partial_t \chi \geq 0$.
  At large scales and late cosmic times, when $\chi$ admits a quasi-homogeneous and isotropic description, this
  monotonic relaxation can be coarse-grained into an effective cosmological clock, analogous to the role played by
  cosmic time in standard FLRW cosmology~\cite{Ellis1971,Weinberg2008}.

  In such regimes, and only as an effective description, the temporal evolution of $\chi$ may be
  approximated by
  \begin{equation}
    \partial_t \chi \simeq H(t)\,\chi,
  \end{equation}
  where $H(t)$ denotes the emergent Hubble parameter associated with the global relaxation of the
  field.
  This relation should not be interpreted as a fundamental equation of motion, but as a phenomenological correspondence
  valid in the FLRW-like limit of the theory, in the same spirit as coarse-grained descriptions commonly employed in
  emergent gravity and cosmological averaging approaches~\cite{Buchert2000,Padmanabhan2010}.

  The local kinematic constraint governing the $\chi$ field,
  \begin{equation}
  (\partial_t \chi)^2 + |\nabla \chi|^2 = c^2,
  \end{equation}
  then implies that even in the absence of localized matter excitations, the cosmological evolution
  of $\chi$ generically induces a non-vanishing residual spatial gradient.
  In the homogeneous limit, this minimal gradient magnitude is given by
  \begin{equation}
    |\nabla \chi|_{\mathrm{\min}} = \sqrt{c^2 - (H(t)\chi)^2}.
  \end{equation}

  This residual gradient does not correspond to a directional force acting on test particles.
  Rather, it defines a background kinematic scale that constrains how additional, locally induced
  gradients can contribute to the effective dynamics.
  For a local observer, the associated scale may be expressed as an effective acceleration floor,
  \begin{equation}
    a_0(t) \sim c\,H(t),
  \end{equation}
  a relation that has long been noted empirically in the context of galactic dynamics and MOND-like
  phenomenology~\cite{Milgrom1983a,Milgrom2002,FamaeyMcGaugh2012}, but which here arises dynamically
  from the global relaxation of the $\chi$ field.
  Unlike Milgromian dynamics, where $a_0$ is postulated as a universal constant, Cosmochrony predicts that this scale
  evolves slowly with cosmic time, tracking the evolution of $H(t)$.

  When localized matter excitations are present, they induce additional spatial gradients
  $\nabla \chi_N$ that, in the weak-field and short-distance limit, reproduce the Newtonian scaling
  $|\nabla \chi_N| \propto M/r^2$.
  ue to the non-linear nature of the kinematic constraint, the total
  effective gradient is not a linear superposition of the cosmological and local contributions.
  Instead, the field dynamics enforces a saturation behavior: at sufficiently large radii, where the
  Newtonian contribution would otherwise vanish, the total gradient asymptotically approaches the
  cosmological floor set by $|\nabla \chi|_{\mathrm{\min}}$.

  In this regime, the resulting effective gravitational acceleration naturally approaches
  \begin{equation}
    g_{\mathrm{eff}} \simeq \sqrt{g_N\,a_0(t)},
  \end{equation}
  recovering the characteristic deep-MOND scaling originally identified by Milgrom~\cite{Milgrom1983a}
  without introducing an ad hoc interpolation function or modifying the underlying gravitational law.

  This interpretation offers two conceptual advantages over phenomenological MOND formulations:
  \begin{enumerate}
    \item \textbf{Cosmological scaling.} The acceleration scale $a_0$ is not fundamental but emerges
    from the cosmological state of the $\chi$ field. As a result, $a_0(t)$ is expected to be larger at
    earlier cosmic epochs, providing a potential observational handle through high-redshift galaxy
    kinematics, now becoming accessible with modern surveys~\cite{McGaugh2015}.
    \item \textbf{Environmental dependence.} Because the relaxation of $\chi$ is sensitive to the
    global and local distribution of gradients, the effective acceleration scale can be weakly modulated
    by environmental factors. This naturally incorporates effects analogous to the external field effect
    (EFE) discussed in MOND literature~\cite{BekensteinMilgrom1984,FamaeyMcGaugh2012} and allows for
    deviations from perfectly flat rotation curves in the far field or in strongly inhomogeneous
    environments.
  \end{enumerate}

  In this framework, flat galactic rotation curves do not signal a breakdown of Newtonian gravity
  nor the presence of unseen matter components.
  Instead, they arise as a kinematic projection of the global cosmological relaxation of $\chi$ onto local gravitational
  dynamics, with MOND-like behavior emerging as an effective, scale-dependent regime rather than as a fundamental
  modification of gravity.

\subsubsection{Gravitational Lensing in the Scalar Framework}
  \label{sec:lensing_derivation}

  Light deflection is modeled as the propagation of a wave front where $\chi = \text{const}$.
  The effective refractive index of the vacuum $n(r)$ is derived from the ratio of the global evolution rate to the local rate:
  \begin{equation}
    n(r) = \frac{c}{\partial_t \chi} = \frac{1}{\sqrt{1 - |\nabla \chi|^2/c^2}}
  \end{equation}
  Near a mass $M$, $|\nabla \chi| \approx \frac{GM}{c^2r}$. For small deflections, $n(r) \approx 1 + \frac{GM}{c^2r}$.
  Integrating the gradient of $n$ along the photon path $z$ gives the deflection angle $\alpha$:
  \begin{equation}
    \alpha = \int_{-\infty}^{\infty} \nabla_\perp n \, dz = \frac{4GM}{bc^2}
  \end{equation}
  This matches the General Relativity prediction.
  The factor of 2, which Newton's theory lacks, arises here from the non-linear square-root structure of the evolution equation~\eqref{eq:chi_dynamics}.
