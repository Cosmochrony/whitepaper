\subsection{Low-\texorpdfstring{$\ell$}{ℓ} CMB Power Suppression from Global
\texorpdfstring{$\chi$}{χ} Relaxation}
  \label{app:lowell_attenuation}

  One of the most persistent large-scale anomalies of the cosmic microwave
  background (CMB) concerns the suppression of temperature anisotropy power at the
  largest angular scales (\(\ell \lesssim 30\)), most notably in the quadrupole and
  octupole moments.
  Within the standard \(\Lambda\)CDM framework, such deviations are commonly
  attributed to cosmic variance, and no specific physical mechanism is associated
  with their occurrence.

  Within the Cosmochrony framework, by contrast, the lowest multipoles probe global
  properties of the projected field \(\chi_{\mathrm{eff}}\) rather than independent
  local perturbations.
  Because the fundamental field \(\chi\) evolves through a monotonic relaxation
  process constrained by finite connectivity and a maximal relaxation speed, the
  longest-wavelength modes correspond to collective configurations whose amplitudes
  are not freely adjustable.

  \paragraph{Structural attenuation of global modes.}
    At very low multipoles, the associated angular modes span regions comparable to the
    full causal domain of the projected \(\chi_{\mathrm{eff}}\) configuration.
    As a result, these modes are subject to global relaxation constraints: their
    amplitude is systematically attenuated relative to the scale-invariant
    expectation.

    This suppression is not the result of stochastic damping or fine-tuned initial
    conditions.
    It arises because the finite relaxation capacity of the \(\chi\) field limits the
    degree to which globally coherent configurations can deviate from the relaxed
    background.
    The effect is deterministic in origin, while its detailed realization in any
    given universe remains statistical.

    Cosmochrony therefore does not predict exact multipole amplitudes.
    Instead, it predicts a robust suppression tendency affecting the lowest
    \(\ell\)-modes, whose precise pattern depends on the detailed global configuration
    of \(\chi\) at last scattering.

  \paragraph{Illustrative comparison with observations.}
    Figure~\ref{fig:cmb_lowell_unsmoothed} shows the observed CMB temperature power
    spectrum at low multipoles, displayed without aggressive smoothing, together with
    a schematic attenuation envelope representative of the Cosmochrony mechanism.

    This comparison is intended to illustrate the qualitative structural deviation
    from scale invariance implied by global relaxation constraints.
    It does \emph{not} constitute a multipole-by-multipole prediction, nor does it
    replace a full Boltzmann analysis.

    \begin{figure}[htbp]
      \centering
      \includegraphics[width=0.85\textwidth]{C-appendix-cosmo/cmb_lowell_unsmoothed}
      \caption{
        Observed CMB temperature power spectrum at low multipoles
        (\(\ell \lesssim 30\)), shown without heavy smoothing.
        The shaded region illustrates a qualitative attenuation envelope expected
        from global relaxation constraints on the projected field
        \(\chi_{\mathrm{eff}}\) in Cosmochrony.
        Unlike \(\Lambda\)CDM, where low-\(\ell\) suppression is treated as a
        statistical accident, Cosmochrony interprets it as a structural consequence
        of the finite relaxation capacity of globally coherent configurations.
        The envelope may be summarized phenomenologically by
        Eq.~\eqref{eq:cc_lowell_envelope}.
      }
      \label{fig:cmb_lowell_unsmoothed}
    \end{figure}

  \paragraph{Phenomenological parametrization.}
    To render this schematic attenuation minimally quantitative without introducing a
    full cosmological perturbation theory, we introduce a two-parameter
    phenomenological envelope in which the low-\(\ell\) power is multiplicatively
    suppressed relative to the \(\Lambda\)CDM best-fit spectrum:
    \begin{equation}
      C_\ell^{\mathrm{CC}}
      =
      C_\ell^{\Lambda\mathrm{CDM}}
      \left[
        1 - \alpha \exp\!\left(-\frac{\ell}{\ell_0}\right)
      \right],
      \qquad
      \alpha \in [0,1], \;\; \ell_0 > 0 .
      \label{eq:cc_lowell_envelope}
    \end{equation}

    Equivalently, in terms of
    \(D_\ell \equiv \ell(\ell+1)C_\ell/(2\pi)\),
    \[
      D_\ell^{\mathrm{CC}}
      =
      D_\ell^{\Lambda\mathrm{CDM}}
      \left[
        1 - \alpha e^{-\ell/\ell_0}
      \right].
    \]
    In this parametrization, \(\alpha\) controls the overall amplitude of large-angle
    suppression, while \(\ell_0\) sets the angular scale beyond which the spectrum
    rapidly converges back to the standard \(\Lambda\)CDM behavior.

  \paragraph{Indicative low-\(\ell\) characterization.}
    An indicative estimate of \((\alpha,\ell_0)\) may be obtained by fitting the ratio
    \(R_\ell \equiv D_\ell^{\mathrm{obs}}/D_\ell^{\Lambda\mathrm{CDM}}\) over a restricted
    low-\(\ell\) range (e.g.\ \(\ell = 2\ldots 30\)), using cosmic-variance-dominated
    uncertainties
    \(\sigma(R_\ell) \simeq \sqrt{2/(2\ell+1)}\).

    This procedure is not intended as a detection claim.
    It provides a compact and reproducible summary of the suppression tendency,
    replacing heuristic hand-drawn envelopes by a controlled two-parameter
    characterization.

  \paragraph{Conceptual distinction from \(\Lambda\)CDM.}
    In \(\Lambda\)CDM, low-\(\ell\) deviations are interpreted \emph{a posteriori} as
    statistical fluctuations around an ensemble mean defined by inflationary initial
    conditions.
    In Cosmochrony, the ensemble itself is constrained:
    the global relaxation dynamics of \(\chi\) restrict the admissible configuration
    space for the longest-wavelength modes.

    This leads to a qualitative physical distinction between large-scale and
    small-scale fluctuations.
    Small-scale modes probe local relaxation and behave approximately as independent
    perturbations, while large-scale modes encode global structural properties of the
    field.

  \paragraph{Scope and limitations.}
    The present analysis does not replace full Boltzmann calculations and does not aim
    to reproduce the entire angular power spectrum.
    Its purpose is to identify a robust qualitative signature of Cosmochrony: a
    systematic suppression tendency affecting the lowest CMB multipoles, arising from
    global relaxation constraints on the fundamental field.

    Quantitative refinement of this effect, including detailed parameter inference
    and polarization observables, is deferred to future numerical studies of the
    \(\chi\) dynamics.
