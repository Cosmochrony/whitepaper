\subsection{Substrate Origin of the Effective Galactic Potential}
  \label{app:substrate-origin-phi-eff}

  This appendix clarifies how the effective gravitational potential introduced in
  Section~\ref{subsec:phi-eff-galaxies} arises from infra-physical properties of the
  $\chi$ substrate, rather than from a fundamental interaction law.

  \paragraph{Relaxation flux and relational stiffness.}
    At the fundamental level, localized excitations correspond to constrained
    configurations of the relational substrate $\chi$.
    Their large-scale influence is mediated by a relaxation flux $\Phi_\chi$,
    whose magnitude depends both on local gradients and on the relational stiffness
    of the substrate,
    \begin{equation}
      \Phi_\chi(r) \sim \frac{|\nabla \chi|}{K(r)}.
    \end{equation}
    The effective stiffness $K(r)$ increases when the projective connectivity of
    $\chi$ becomes sparse, as occurs at large distances from localized excitations.

  \paragraph{Saturation threshold and loss of injectivity.}
    Because the relaxation dynamics of $\chi$ are bounded, there exists a critical
    stiffness scale $K_c$ beyond which additional gradients cannot be transmitted
    linearly.
    When $K(r) \ll K_c$, relaxation remains unsaturated and the projected description
    is effectively injective, leading to Newtonian behavior.
    When $K(r) \gtrsim K_c$, relaxation saturates and the projection from $\chi$ to
    effective observables becomes non-injective.

  \paragraph{Emergent acceleration scale.}
    In the projected geometric description, the saturation condition $K(r)\simeq K_c$
    manifests as a threshold on the Newtonian baryonic acceleration,
    \begin{equation}
      g_N(r) \simeq a_0(t).
    \end{equation}
    The scale $a_0$ is therefore not a fundamental constant but an operational
    re-expression of the substrate saturation threshold.

  \paragraph{Cosmological origin of $a_0(t)$.}
    At the largest scales, the relaxation of $\chi$ is constrained by the global
    relational expansion of the Universe.
    The characteristic relaxation rate is set by the inverse cosmological timescale,
    leading naturally to
    \begin{equation}
      a_0(t) \sim c\,H(t),
    \end{equation}
    where $c$ encodes the maximal relaxation speed of $\chi$ and $H(t)$ the global
    relational expansion rate.
    This relation predicts a slow cosmological evolution of the effective acceleration
    scale.

  \paragraph{From substrate saturation to logarithmic potential.}
    In the saturated regime, the relaxation flux transmitted by $\chi$ becomes
    effectively scale-invariant, implying an effective acceleration
    $g_{\mathrm{eff}}(r)\propto 1/r$.
    The corresponding projected potential therefore takes the logarithmic form
    \begin{equation}
      \Phi_{\mathrm{eff}}(r)\propto \ln r,
    \end{equation}
    which directly accounts for asymptotically flat galactic rotation curves.

  \paragraph{Interpretational status.}
    The effective potential $\Phi_{\mathrm{eff}}$ thus summarizes a specific regime of
    substrate relaxation.
    It does not correspond to an additional field or interaction, but to the
    geometrically admissible description of a saturated projection of the underlying
    $\chi$ dynamics.
