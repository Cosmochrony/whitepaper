\section{Appendix E: Toy-Model of Spectral Gravitational Susceptibility}
  \label{app:spectral-gravitational-toy-model}

  This appendix provides the mathematical foundations for a non-particulate interpretation of dark matter phenomena,
  treating galactic dynamics as the non-linear elastic response of the $\chi$ substrate.

  \subsection{Modified Poisson Equation and Field Strength}
    In the Cosmochrony framework, gravitational acceleration is not a force acting in a passive vacuum but the emergent
    manifestation of a relaxation gradient. We define the \textbf{local relaxation field strength}
    as the gradient of the scalar relaxation potential:
    \begin{equation}
      \mathbf{E}_\chi = - \nabla \Phi_\chi
    \end{equation}
    The dynamics are governed by a modified Poisson equation analogous to electrodynamics in continuous media:
    \begin{equation}
      \nabla \cdot \left[ \epsilon_{\text{spec}}(\mathbf{E}_\chi) \mathbf{E}_\chi \right] = 4\pi G_0 \rho_b
    \end{equation}
    where $\rho_b$ is the baryonic mass density and $\epsilon_{\text{spec}}$ is the \textbf{spectral permittivity}
    of the substrate, defined by the relation $\epsilon_{\text{spec}} = 1 + \phi(\mathbf{E}_\chi)$.

  \subsection{Spectral Susceptibility and the Stiffness Threshold $\mathcal{K}_c$}
    To recover the observed galactic phenomenology, we define the \textbf{spectral gravitational susceptibility} $\phi$
    as a function of the field strength relative to a saturation threshold $\mathcal{K}_c$:
    \begin{equation}
      \phi(\mathbf{E}_\chi) =
      \begin{cases}
        0 & \text{for } |\mathbf{E}_\chi| \gg \mathcal{K}_c \text{ (Linear/Newtonian Regime)} \\
        \frac{\mathcal{K}_c}{|\mathbf{E}_\chi|} & \text{for } |\mathbf{E}_\chi| \ll \mathcal{K}_c
        \text{ (Saturation Regime)}
      \end{cases}
    \end{equation}
    Crucially, $\mathcal{K}_c$ is not a universal constant of nature but a \textbf{local state property}
    of the substrate. It represents the threshold where the relaxation flux reaches the elastic limit of the $\chi$
    field.

  \subsection{Emergence of Flat Rotation Curves}
    In the low-field limit ($|\mathbf{E}_\chi| \ll \mathcal{K}_c$
    ) typical of galactic peripheries, the effective acceleration $g_{\text{eff}}$ follows:
    \begin{equation}
      \nabla \cdot \left( \mathcal{K}_c \frac{\mathbf{E}_\chi}{|\mathbf{E}_\chi|} \right) \sim 4\pi G_0 \rho_b \implies
      g_{\text{eff}} \approx \frac{\sqrt{G_0 M \mathcal{K}_c}}{r}
    \end{equation}
    This leads directly to a constant orbital velocity $v^4 = G_0 M \mathcal{K}_c$, recovering the
    \textbf{Baryonic Tully-Fisher Relation}.
    Here, ``dark matter'' is reinterpreted as the increased elastic response of the substrate in regions of diluted
    relaxation flux.

  \subsection{Comparative Framework: MOND vs. Cosmochrony}
    The following table summarizes the conceptual shift from modified gravity to substrate dynamics.

    \begin{table}[h]
      \centering
      \begin{tabular}{|l|l|l|}
        \hline
        \textbf{Feature}        & \textbf{MOND (Milgrom)}           & \textbf{Cosmochrony ($\chi$ Substrate)}    \\
        \hline
        \textbf{Origin}         & Modified law of inertia/force.    & Non-linear susceptibility of the medium.   \\
        \hline
        \textbf{Threshold}      & Universal constant $a_0$.         & Local stiffness threshold $\mathcal{K}_c$. \\
        \hline
        \textbf{Bullet Cluster} & Requires additional DM particles. & Natural: Relaxation hysteresis (wake).     \\
        \hline
        \textbf{GR Relation}    & Requires $TeVeS$ or similar.      & GR is the linear-response limit.           \\
        \hline
        \textbf{DM Nature}      & Force discrepancy.                & Residual non-projected energy.             \\
        \hline
      \end{tabular}
      \caption{Comparison between MOND phenomenology and Cosmochrony substrate response.}
    \end{table}

  \subsection{Limitations and Outlook}

    \paragraph{Theoretical Refinement.}
      The current form of $\phi(\mathbf{E}_\chi)$
      is phenomenological. A rigorous derivation from the microscopic relaxation equations in Appendix D is required
      to link $\mathcal{K}_c$ to the global Hubble relaxation rate.

    \paragraph{The Relaxation Wake.}
      Cosmochrony predicts that in high-energy collisions (e.g., Bullet Cluster), the geometric deformation of the
      substrate exhibits a \textbf{phase lag} (hysteresis).
      Gravitational lensing tracks this ``residual wake'' of the mass-solitons, explaining the offset from dissipative
      gas.
      A specific prediction of this model is the existence of \textbf{spectral echoes}: residual curvature in regions
      where matter has recently passed, a signature that could distinguish Cosmochrony from WIMP-based models.

    \paragraph{General Relativity Limit.}
      Finally, it is emphasized that Cosmochrony reduces to General Relativity in the linear-response limit of the
      $\chi$substrate.
      Spacetime curvature is the refractive manifestation of the substrate's spectral density, and gravity is its
      macroscopic relaxation.
