\subsection{Relation to Observational Units and Numerical Estimates}
  \label{subsec:observational-estimates}

\subsubsection{Normalization of the $\chi$ Field}
  \label{subsec:normalization-of-the-chi-field}
  To connect the $\chi$ field to observable quantities, a normalization must be specified. We identify the present-day value $\chi(t_0)$ with the characteristic cosmological length scale governing expansion. Operationally, $\chi(t_0)$ may be interpreted as the proper wavelength accumulated since the epoch at which coherent propagation of radiation became possible, approximately recombination.

  \subsubsubsection*{Constraints on $K_0$ and $\chi_c$}
    \label{subsec:K0-chic-constraints}
    The fundamental parameters $K_0$ and $\chi_c$ in the constitutive relation for $K_{ij}$ (Equation~\eqref{eq:Kij-def}) can be constrained by matching the emergent gravitational constant $G$ (Equation~\eqref{eq:G-emergent}) to its observed value. Using $G \approx 6.674 \times 10^{-11} \, \text{m}^3 \, \text{kg}^{-1} \, \text{s}^{-2}$ and $c \approx 3 \times 10^8 \, \text{m/s}$, we find:
    \begin{equation}
      K_0 \chi_c^2 = \frac{c^4}{16 \pi G} \approx 1.1 \times 10^{44} \, \text{N}
    \end{equation}

    If we associate $\chi_c$ with the \textbf{Planck length} $\ell_P \approx 1.6 \times 10^{-35} \, \text{m}$, then:
    \begin{equation}
      K_0 \approx \frac{c^4}{16 \pi G \ell_P^2} \approx 1.3 \times 10^{93} \, \text{m}^{-2}
    \end{equation}
    This value of $K_0$ sets the \textbf{stiffness scale} of the $\chi$-field network, while $\chi_c \approx \ell_P$ ensures that quantum gravitational effects become significant at the Planck scale.

    Alternatively, if $\chi_c$ is associated with the \textbf{Hubble scale} $c / H_0 \approx 1.4 \times 10^{26} \, \text{m}$, then:
    \begin{equation}
      K_0 \approx \frac{H_0^2}{c^2} \approx 1.1 \times 10^{-52} \, \text{m}^{-2}
    \end{equation}
    This smaller value of $K_0$ would imply a \textbf{softer network}, with cosmological-scale effects dominating the dynamics. The precise values of $K_0$ and $\chi_c$ remain open to empirical constraints, but their product $K_0 \chi_c^2$ is fixed by the observed $G$.

  \subsubsubsection*{Implications for Particle Masses}
    The soliton mass scale $m \propto \sqrt{\lambda}$ (from Appendix~\ref{subsec:soliton_energy_mass}) requires
    $\lambda \sim 10^{-116} \, \text{m}^{-2}$ to reproduce the electron mass $m_e \approx 9.11 \times 10^{-31} \, \text{kg}$.
    This tiny value suggests that $\lambda$ may be \textbf{dynamically generated} rather than fundamental, reflecting the hierarchical structure of the $\chi$ field potential.

\subsubsection{Hubble Constant}
  \label{subsec:hubble-constant}

  From the fundamental relation
  \begin{equation}
    H(t) = \frac{\dot{\chi}}{\chi},
  \end{equation}
  and assuming maximal relaxation speed $\dot{\chi} \simeq c$, the present Hubble constant follows as
  \begin{equation}
    H_0 \simeq \frac{c}{\chi(t_0)}.
  \end{equation}

  Using the observed value $H_0 \sim 70~\mathrm{km\,s^{-1}\,Mpc^{-1}}$, one infers
  \begin{equation}
    \chi(t_0) \sim 4 \times 10^{26}~\mathrm{m},
  \end{equation}
  consistent with the current Hubble radius.

  This correspondence arises without introducing free cosmological parameters.

  The soliton mass scale $m \propto \sqrt{\lambda}$ then requires $\lambda \sim 10^{-116} \, \text{m}^{-2}$
  to reproduce the electron mass $m_e \approx 9.11 \times 10^{-31} \, \text{kg}$. This tiny value suggests that
  $\lambda$ may be dynamically generated rather than fundamental.

\subsubsection{Age of the Universe}
  \label{subsec:age-of-the-universe}

  Integrating the relation $\dot{\chi} \simeq c$ yields
  \begin{equation}
    \chi(t) \simeq c t + \chi_{\mathrm{init}},
  \end{equation}
  where $\chi_{\mathrm{init}}$ denotes the effective value at the onset of coherent $\chi$ relaxation.

  Neglecting $\chi_{\mathrm{init}}$ compared to present values gives
  \begin{equation}
    t_0 \simeq \frac{\chi(t_0)}{c} \sim 4 \times 10^{17}~\mathrm{s},
  \end{equation}
  corresponding to approximately 13.8 billion years, in agreement with standard cosmological estimates.

\subsubsection{Redshift Interpretation}
  \label{subsec:redshift-interpretation}

  Cosmological redshift arises from the increase of $\chi$ between emission and observation:
  \begin{equation}
    1 + z = \frac{\chi(t_{\mathrm{obs}})}{\chi(t_{\mathrm{emit}})}.
  \end{equation}

  This interpretation reproduces standard redshift relations while attributing them to geometric scaling rather than
  recessional motion through spacetime.

\subsubsection{Cosmic Microwave Background Scale}
  \label{subsec:cosmic-microwave-background-scale}

  At recombination ($z_{\mathrm{rec}} \simeq 1100$), the characteristic scale of $\chi$
  was smaller by the same factor:
  \begin{equation}
    \chi(t_{\mathrm{rec}}) \simeq \frac{\chi(t_0)}{1 + z_{\mathrm{rec}}}.
  \end{equation}

  Fluctuations imprinted at that epoch are stretched by subsequent $\chi$
  growth, explaining the observed angular power spectrum of the CMB\@.

\subsubsection{Orders of Magnitude and Robustness}
  \label{subsec:orders-of-magnitude-and-robustness}

  All numerical estimates presented here rely solely on observed cosmological quantities and the assumption of
  bounded $\chi$ relaxation.
  No fine-tuning of parameters is required.

  While precise numerical modeling remains to be developed, these estimates demonstrate that cosmochrony naturally
  reproduces the correct orders of magnitude for key cosmological observables.

\subsubsection{Summary}
  \label{subsec:summary}

  The $\chi$ framework connects directly to measured cosmological quantities through simple scaling relations.
  The Hubble constant, cosmic age, redshift, and CMB scales emerge consistently from the same underlying dynamics.
