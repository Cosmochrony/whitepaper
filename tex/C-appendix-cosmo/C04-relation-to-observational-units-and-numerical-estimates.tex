\subsection{Relation to Observational Units and Numerical Estimates}
  \label{subsec:observational-estimates}

  This subsection establishes order-of-magnitude relations linking the Cosmochrony
  framework to observed cosmological quantities.
  Its purpose is not to perform parameter fitting or to derive precision
  predictions, but to assess the internal consistency of the theory and to verify
  that its fundamental relaxation-based interpretation naturally reproduces the
  correct empirical scales.

  All numerical relations presented here should be understood as effective
  normalizations arising in the projectable regime of the \(\chi\) dynamics.
  They do not define fundamental constants and do not fix the microscopic structure
  of the theory.

  \subsubsection*{Normalization of the \texorpdfstring{$\chi$}{χ} Field}
    \label{subsec:normalization-of-the-chi-field}

    To relate the projected field \(\chi_{\mathrm{eff}}\) to observable quantities, a
    reference normalization must be specified.
    At the effective cosmological level, the scale factor is defined up to a global
    multiplicative constant.
    In Cosmochrony, this freedom is fixed by identifying the present-day value
    \(\chi(t_0)\) with the characteristic geometric scale governing large-scale
    expansion.

    Operationally, \(\chi(t_0)\) represents the cumulative geometric scale associated
    with the global relaxation of the \(\chi\) field up to the present epoch.
    This identification does not assume a unique microscopic origin for
    \(\chi(t_0)\); it provides a minimal and observationally anchored normalization
    consistent with the effective relation \(a(t) \propto \chi(t)\).

  \subsubsection*{Emergent Gravitational Coupling}
    \label{subsec:K0-chic-constraints}

    In the effective geometric description, the Newtonian gravitational constant
    \(G\) emerges from the constitutive relation governing the coupling between
    neighboring configurations of the projected \(\chi\) field.
    This coupling is controlled by two parameters of the relaxation dynamics: the
    maximal stiffness scale \(K_0\) and the characteristic correlation length
    \(\chi_c\).

    Although \(K_0\) and \(\chi_c\) are not individually fixed at the present stage,
    their combination is constrained by matching the observed gravitational coupling:
    \begin{equation}
      K_0 \chi_c^2 \sim \frac{c^4}{16 \pi G}.
    \end{equation}

    This relation fixes the overall stiffness scale of the effective \(\chi\)
    network.
    It does not require committing to a specific microscopic interpretation of
    \(\chi_c\), which may correspond to a fundamental correlation scale or to an
    emergent coarse-graining length.
    At this level, only the product \(K_0 \chi_c^2\) is observationally relevant.

  \subsubsection*{Hubble Constant}
    \label{subsec:hubble-constant}

    In the homogeneous limit, the effective Hubble parameter is defined by the
    relative relaxation rate of the projected field,
    \begin{equation}
      H(t) = \frac{\dot{\chi}}{\chi}.
    \end{equation}

    Assuming that the present universe lies close to the maximal relaxation regime,
    \(\dot{\chi}(t_0) \simeq c\), the present-day Hubble constant follows as
    \begin{equation}
      H_0 \simeq \frac{c}{\chi(t_0)}.
    \end{equation}

    Using the observed value
    \(H_0 \approx 70~\mathrm{km\,s^{-1}\,Mpc^{-1}}\) yields
    \begin{equation}
      \chi(t_0) \sim 4 \times 10^{26}~\mathrm{m},
    \end{equation}
    which is of the order of the observed Hubble radius.
    This correspondence arises directly from the relaxation-based interpretation of
    cosmic expansion and does not require the introduction of additional cosmological
    parameters.

  \subsubsection*{Age of the Universe}
    \label{subsec:age-of-the-universe}

    In the homogeneous relaxation regime, the evolution of \(\chi\) may be
    approximated as
    \begin{equation}
      \dot{\chi} \simeq c ,
    \end{equation}
    leading to
    \begin{equation}
      \chi(t) \simeq c t + \chi_{\mathrm{init}},
    \end{equation}
    where \(\chi_{\mathrm{init}}\) denotes the effective value of \(\chi\) at the onset
    of the relaxation regime relevant for cosmological observations.

    Neglecting \(\chi_{\mathrm{init}}\) compared to present values yields
    \begin{equation}
      t_0 \simeq \frac{\chi(t_0)}{c} \sim 4 \times 10^{17}~\mathrm{s},
    \end{equation}
    corresponding to approximately \(13.8\) billion years.
    This estimate is consistent with standard cosmological age determinations and
    follows directly from the bounded relaxation dynamics.

  \subsubsection*{Redshift Interpretation}
    \label{subsec:redshift-interpretation}

    In Cosmochrony, cosmological redshift is interpreted as a consequence of the
    relative change in the projected \(\chi\) field between emission and observation,
    \begin{equation}
      1 + z = \frac{\chi(t_{\mathrm{obs}})}{\chi(t_{\mathrm{emit}})}.
    \end{equation}

    This relation reproduces standard redshift phenomenology while attributing it to
    geometric scaling induced by \(\chi\) relaxation, rather than to recessional
    motion within a pre-existing spacetime background.

  \subsubsection*{Cosmic Microwave Background Scale}
    \label{subsec:cosmic-microwave-background-scale}

    At recombination, characterized observationally by
    \(z_{\mathrm{rec}} \simeq 1100\), the effective value of the projected field was
    smaller by the corresponding scaling factor,
    \begin{equation}
      \chi(t_{\mathrm{rec}}) \simeq \frac{\chi(t_0)}{1 + z_{\mathrm{rec}}}.
    \end{equation}

    Fluctuations imprinted at that epoch are subsequently stretched by the monotonic
    growth of \(\chi\), providing a natural geometric interpretation of the angular
    scales observed in the cosmic microwave background without invoking an inflationary
    stretching phase.

  \subsubsection*{Orders of Magnitude and Robustness}
    \label{subsec:orders-of-magnitude-and-robustness}

    All numerical estimates presented in this subsection rely solely on observed
    cosmological quantities and on the bounded relaxation dynamics of the \(\chi\)
    field.
    No fine-tuning of parameters, no detailed cosmological fitting, and no additional
    degrees of freedom are assumed.

    While a fully predictive cosmological model requires explicit numerical
    simulations of the \(\chi\) dynamics, these order-of-magnitude relations
    demonstrate that Cosmochrony naturally reproduces the correct scales for the
    Hubble constant, the age of the universe, redshift evolution, and characteristic
    CMB features.

  \subsubsection*{Summary}
    \label{subsec:summary}

    The Cosmochrony framework admits a consistent normalization in observational
    units and reproduces key cosmological scales without introducing new fundamental
    parameters.
    These order-of-magnitude relations support the internal coherence of the theory
    and motivate further quantitative investigation of its cosmological dynamics.
