\subsection{Spectral Interpretation of the Galactic Saturation Regime}
  \label{app:spectral-saturation-galaxies}

  This appendix provides a complementary operator-based interpretation of the
  effective galactic potential introduced in Section~\ref{subsec:phi-eff-galaxies},
  by relating the saturation of $\chi$-relaxation to spectral properties of the
  projected relational operator.

  \paragraph{Effective operator and spectral stiffness.}
    In projectable regimes, the collective response of the $\chi$ substrate to
    localized excitations may be characterized by an effective elliptic operator
    $\mathcal{L}_{\mathrm{eff}}$, whose spectrum encodes the relational stiffness
    of the medium,
    \begin{equation}
      \mathcal{L}_{\mathrm{eff}} \,\psi_n = \lambda_n \psi_n,
    \end{equation}
    with increasing eigenvalues $\lambda_n$ corresponding to increasing resistance
    to long-wavelength deformations.

  \paragraph{Spectral saturation and loss of long modes.}
    As the scale increases, the density of low-lying modes decreases.
    Beyond a critical scale, the smallest admissible eigenvalue $\lambda_{\min}$
    approaches a saturation threshold $\lambda_c$, such that additional long-range
    modes cannot be supported.
    This spectral depletion corresponds to the stiffness threshold $K_c$ discussed
    in Appendix~\ref{app:substrate-origin-phi-eff}.

  \paragraph{Effective acceleration threshold.}
    In the projected geometric description, the spectral threshold $\lambda_c$
    translates into an effective acceleration scale $a_0$ through
    \begin{equation}
      a_0 \sim \frac{\lambda_c}{\tau_\chi^2},
    \end{equation}
    where $\tau_\chi$ denotes the characteristic relaxation time of the substrate.
    This relation expresses the operational equivalence between spectral saturation
    and the acceleration crossover observed in galactic dynamics.

  \paragraph{Emergence of the logarithmic potential.}
    When the spectrum becomes marginally scale-invariant near $\lambda_c$,
    the Green function of $\mathcal{L}_{\mathrm{eff}}$ in the effective
    two-dimensional asymptotic regime behaves as
    \begin{equation}
      G(r) \propto \ln r.
    \end{equation}
    The projected potential $\Phi_{\mathrm{eff}}$ inherits this logarithmic behavior,
    providing a spectral explanation for asymptotically flat rotation curves.

  \paragraph{Consistency with Newtonian behavior.}
    At scales where $\lambda_{\min} \ll \lambda_c$, the spectrum is dense and
    the Green function reduces to the familiar $1/r$ behavior, recovering
    Newtonian gravity as an unsaturated spectral regime.

  \paragraph{Interpretational status.}
    The spectral description introduced here does not add new degrees of freedom.
    It provides an operator-level characterization of the same saturation phenomenon
    described infra-physically in Appendix~\ref{app:substrate-origin-phi-eff}.
