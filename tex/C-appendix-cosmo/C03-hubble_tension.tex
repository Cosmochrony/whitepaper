\subsection{Evolution of the Hubble Parameter and the Hubble Tension}
  \label{app:hubble_tension}

  In the Cosmochrony framework, cosmological expansion is not governed by the
  competition between matter, radiation, and a dark energy component.
  Instead, it reflects the relaxation dynamics of the scalar field \(\chi\), from
  which spacetime geometry and its associated expansion rate emerge as effective
  descriptions.

  The Hubble parameter therefore encodes the instantaneous relaxation rate of
  \(\chi\) relative to its global configuration, rather than the response of a
  metric to an energy--momentum content.

  \paragraph{Global expansion rate.}
    At the homogeneous background level, the effective scale factor is proportional
    to the global value of the projected field,
    \begin{equation}
      a(t) \propto \chi(t),
    \end{equation}
    so that the Hubble parameter may be written as
    \begin{equation}
      H(t) = \frac{\dot{\chi}}{\chi}.
    \end{equation}

    In the idealized homogeneous limit, spatial gradients vanish
    (\(\nabla \chi = 0\)) and the relaxation dynamics reduce to a uniform evolution
    with maximal relaxation speed,
    \begin{equation}
      \dot{\chi} = c .
    \end{equation}
    In this limit, the global expansion rate becomes
    \begin{equation}
      H(t) = \frac{c}{\chi(t)} .
    \end{equation}

    This relation defines the \emph{global} expansion rate in Cosmochrony.
    Its detailed redshift dependence away from perfect homogeneity is not assumed to
    follow a fixed power law and depends on how relaxation gradients contribute to
    the averaged dynamics.

  \paragraph{Relaxation budget and effective expansion.}
    In a realistic universe, part of the relaxation capacity of \(\chi\) is stored
    in spatial gradients associated with inhomogeneities.
    To quantify this effect, we introduce a dimensionless \emph{relaxation budget}
    parameter,
    \begin{equation}
      \Omega_\chi \equiv \langle \beta^2 \rangle,
      \qquad
      \beta \equiv \frac{|\nabla \chi|}{c},
      \label{eq:chi-relaxation}
    \end{equation}
    which measures the fraction of the total relaxation capacity diverted into
    spatial structure rather than global evolution.

    At late times, these gradients are dominated by localized solitonic
    configurations and therefore track the large-scale matter distribution.
    The effective global expansion rate is then reduced to
    \begin{equation}
      \bar{H}
      =
      \frac{c}{\chi}
      \sqrt{1 - \Omega_\chi}.
    \end{equation}

    Empirically, consistency with large-scale observations suggests
    \(\Omega_\chi\) is of order the observed matter fraction,
    \(\Omega_\chi \sim 0.3\).
    In Cosmochrony, this suppression of the global expansion rate arises naturally
    from relaxation dynamics and does not require a dark energy component.

  \paragraph{Local expansion and environmental dependence.}
    In an inhomogeneous universe, the relaxation budget is not spatially uniform.
    Regions with different matter densities redistribute relaxation capacity
    differently between global evolution and local gradients.

    For a region characterized by a density contrast
    \begin{equation}
      \delta = \frac{\rho - \bar{\rho}}{\bar{\rho}},
    \end{equation}
    we adopt a minimal mean-field closure relation,
    \begin{equation}
      \beta_{\mathrm{loc}}^{2}
      =
      \Omega_\chi (1 + \delta),
    \end{equation}
    which encodes the intuitive scaling between matter density and
    \(\chi\)-gradient energy.

    The locally inferred Hubble parameter then takes the form
    \begin{equation}
      H_{\mathrm{loc}}
      =
      \bar{H}
      \sqrt{
        \frac{1 - \Omega_\chi (1+\delta)}
        {1 - \Omega_\chi}
      } .
    \end{equation}

    In underdense regions (\(\delta < 0\)), a larger fraction of the relaxation
    capacity is available for global evolution, leading to
    \(H_{\mathrm{loc}} > \bar{H}\).

  \paragraph{Numerical consistency and the Hubble tension.}
    For representative values
    \(\Omega_\chi \approx 0.3\) and a local underdensity consistent with the
    KBC void (\(\delta \approx -0.4\) on scales of a few hundred megaparsecs), one
    finds
    \begin{equation}
      \frac{H_{\mathrm{loc}}}{\bar{H}} \approx 1.08 ,
    \end{equation}
    corresponding to an enhancement of order \(8\%\) in the locally inferred Hubble
    constant.

    This magnitude is comparable to the observed discrepancy between local
    distance-ladder measurements and global CMB-based inferences.
    In Cosmochrony, this discrepancy arises naturally as an environmental effect,
    without invoking new energy components or modifications of early-universe
    physics.

  \paragraph{Interpretation and status.}
    Within Cosmochrony, the Hubble tension does not signal a breakdown of cosmological
    consistency.
    It reflects the fact that cosmological expansion is an emergent relaxation
    phenomenon whose effective rate depends on the local redistribution of
    \(\chi\)-field gradients.

    While the framework robustly predicts a separation between local and global
    expansion rates, a fully quantitative determination of \(H(z)\) across all
    redshifts requires dedicated numerical simulations of the \(\chi\) relaxation
    dynamics.
    Such simulations lie beyond the scope of the present work.

    Nevertheless, the qualitative resolution of the Hubble tension follows directly
    from the relaxation-based interpretation of cosmological expansion and
    constitutes a distinctive and testable signature of the Cosmochrony framework.
