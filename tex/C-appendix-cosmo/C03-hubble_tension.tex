\subsection{Evolution of the Hubble Parameter and the Hubble Tension}
  \label{app:hubble_tension}

  This appendix provides the mathematical details underlying the Hubble tension
  interpretation presented in Section~\ref{subsec:hubble-tension}.

  In the Cosmochrony framework, cosmological expansion is not governed by the
  competition between matter, radiation, and a dark energy component.
  Instead, it reflects the relaxation dynamics of the scalar field \(\chi\), from
  which spacetime geometry and its associated expansion rate emerge as effective
  descriptions.

  The Hubble parameter therefore encodes the instantaneous relaxation rate of
  \(\chi\) relative to its global configuration, rather than the response of a
  metric to an energy--momentum content.

  \paragraph{Global expansion rate.}
    At the homogeneous background level, the effective scale factor is proportional
    to the global value of the projected field,
    \begin{equation}
      a(t) \propto \chi(t),
    \end{equation}
    so that the Hubble parameter may be written as
    \begin{equation}
      H(t) = \frac{\dot{\chi}}{\chi}.
    \end{equation}

    In the idealized homogeneous limit, spatial gradients vanish
    (\(\nabla \chi = 0\)) and the relaxation dynamics reduce to a uniform evolution
    with maximal relaxation speed,
    \begin{equation}
      \dot{\chi} = c .
    \end{equation}
    In this limit, the global expansion rate becomes
    \begin{equation}
      H(t) = \frac{c}{\chi(t)} .
    \end{equation}

    This relation defines the \emph{global} expansion rate in Cosmochrony.
    Its detailed redshift dependence away from perfect homogeneity is not assumed to
    follow a fixed power law and depends on how relaxation gradients contribute to
    the averaged dynamics.

  \paragraph{Illustrative $H(z)$ profile and comparison with $\Lambda$CDM.}
    To provide a visual benchmark against standard cosmology, we compare an
    \emph{illustrative} Cosmochrony profile to the $\Lambda$CDM expansion history.
    This is not a fit and does not claim a derived redshift law: it only translates the
    baseline relations $a(t)\propto \chi(t)$ and $H=\dot{\chi}/\chi$ into a convenient
    parametrization for plotting purposes.

    Using the effective redshift interpretation
    \begin{equation}
      1+z = \frac{\chi(t_0)}{\chi(t)} ,
    \end{equation}
    one may write, in the simplest background closure where the relaxation budget
    is treated as a slowly varying effective fraction,
    \begin{equation}
      H_{\chi}(z)
      \;\equiv\;
      H_0 \,(1+z)\,\sqrt{1-\Omega_{\chi}}
      \qquad
      ( \text{illustrative closure} ).
      \label{eq:Hz_cosmochrony_illustrative}
    \end{equation}
    For comparison, the $\Lambda$CDM benchmark is
    \begin{equation}
      H_{\Lambda\mathrm{CDM}}(z)
      \;=\;
      H_0 \sqrt{\Omega_m(1+z)^3+\Omega_r(1+z)^4+\Omega_\Lambda},
      \label{eq:Hz_LCDM}
    \end{equation}
    with $(\Omega_m,\Omega_\Lambda)\simeq(0.3,0.7)$ and $\Omega_r$ negligible at late times.

    Figure~\ref{fig:Hz_comparison} shows the two curves, together with a shaded
    Cosmochrony band obtained by varying $(H_0,\Omega_\chi)$ within representative
    late-universe ranges. The purpose is to highlight (i) the qualitatively distinct scaling
    of $H(z)$ induced by monotonic $\chi$ relaxation and (ii) how modest variations in the
    effective relaxation budget translate into percent-level differences in inferred expansion.

    \begin{figure}[t]
      \centering
      \begin{tikzpicture}
        \begin{axis}[
          width=0.92\linewidth,
        height=0.48\linewidth,
        xlabel={$z$},
        ylabel={$H(z)$ [km\,s$^{-1}$\,Mpc$^{-1}$]},
        xmin=0, xmax=3.0,
        ymin=60, ymax=260,
        legend pos=north west,
        grid=both,
        samples=200,
        domain=0:3
        ]

          % --- LCDM benchmark (late-time; radiation neglected in the plot range) ---
          \addplot+[thick]
            {70*sqrt(0.3*(1+x)^3 + 0.7)};
          \addlegendentry{$\Lambda$CDM ($H_0=70,\ \Omega_m=0.3,\ \Omega_\Lambda=0.7$)}

          % --- Cosmochrony illustrative band: vary H0 in [67,73] and Omega_chi in [0.25,0.35] ---
          % Lower envelope: H0=67, Omega_chi=0.35
          \addplot[name path=chiLow, draw=none]
          {67*(1+x)*sqrt(1-0.35)};

          % Upper envelope: H0=73, Omega_chi=0.25
          \addplot[name path=chiHigh, draw=none]
          {73*(1+x)*sqrt(1-0.25)};

          \addplot[fill opacity=0.25]
          fill between[of=chiLow and chiHigh];
          \addlegendentry{Cosmochrony (illustrative band: $H_0\in[67,73]$, $\Omega_\chi\in[0.25,0.35]$)}

          % Central illustrative Cosmochrony curve: H0=70, Omega_chi=0.30
          \addplot+[thick, dashed]
            {70*(1+x)*sqrt(1-0.30)};
          \addlegendentry{Cosmochrony central ($H_0=70,\ \Omega_\chi=0.30$)}

        \end{axis}
      \end{tikzpicture}
      \caption{
        Illustrative comparison of $H(z)$ in Cosmochrony versus a $\Lambda$CDM benchmark.
        The Cosmochrony curve uses the minimal plotting closure
        $H_\chi(z)=H_0(1+z)\sqrt{1-\Omega_\chi}$, shown with a representative uncertainty band
        from $(H_0,\Omega_\chi)$ variations. This figure is a visualization aid, not a fit.
      }
      \label{fig:Hz_comparison}
    \end{figure}

  \paragraph{Relaxation budget and effective expansion.}
    In a realistic universe, part of the relaxation capacity of \(\chi\) is stored
    in spatial gradients associated with inhomogeneities.
    To quantify this effect, we introduce a dimensionless \emph{relaxation budget}
    parameter,
    \begin{equation}
      \Omega_\chi \equiv \langle \beta^2 \rangle,
      \qquad
      \beta \equiv \frac{|\nabla \chi|}{c},
      \label{eq:chi-relaxation}
    \end{equation}
    which measures the fraction of the total relaxation capacity diverted into
    spatial structure rather than global evolution.

    At late times, these gradients are dominated by localized solitonic
    configurations and therefore track the large-scale matter distribution.
    The effective global expansion rate is then reduced to
    \begin{equation}
      \bar{H}
      =
      \frac{c}{\chi}
      \sqrt{1 - \Omega_\chi}.
    \end{equation}

    Empirically, consistency with large-scale observations suggests
    \(\Omega_\chi\) is of order the observed matter fraction,
    \(\Omega_\chi \sim 0.3\).
    In Cosmochrony, this suppression of the global expansion rate arises naturally
    from relaxation dynamics and does not require a dark energy component.

  \paragraph{Local expansion and environmental dependence.}
    In an inhomogeneous universe, the relaxation budget is not spatially uniform.
    Regions with different matter densities redistribute relaxation capacity
    differently between global evolution and local gradients.

    For a region characterized by a density contrast
    \begin{equation}
      \delta = \frac{\rho - \bar{\rho}}{\bar{\rho}},
    \end{equation}
    we adopt a minimal mean-field closure relation,
    \begin{equation}
      \beta_{\mathrm{loc}}^{2}
      =
      \Omega_\chi (1 + \delta),
    \end{equation}
    which encodes the intuitive scaling between matter density and
    \(\chi\)-gradient energy.

    The locally inferred Hubble parameter then takes the form
    \begin{equation}
      H_{\mathrm{loc}}
      =
      \bar{H}
      \sqrt{
        \frac{1 - \Omega_\chi (1+\delta)}
        {1 - \Omega_\chi}
      } .\label{eq:hubble_local}
    \end{equation}

    In underdense regions (\(\delta < 0\)), a larger fraction of the relaxation
    capacity is available for global evolution, leading to
    \(H_{\mathrm{loc}} > \bar{H}\).

  \paragraph{Numerical consistency and the Hubble tension.}
    For representative values
    \(\Omega_\chi \approx 0.3\) and a local underdensity consistent with the
    KBC void (\(\delta \approx -0.4\) on scales of a few hundred megaparsecs), one
    finds
    \begin{equation}
      \frac{H_{\mathrm{loc}}}{\bar{H}} \approx 1.08 ,
    \end{equation}
    corresponding to an enhancement of order \(8\%\) in the locally inferred Hubble
    constant.

    This magnitude is comparable to the observed discrepancy between local
    distance-ladder measurements and global CMB-based inferences.
    In Cosmochrony, this discrepancy arises naturally as an environmental effect,
    without invoking new energy components or modifications of early-universe
    physics.

  \paragraph{Predicted amplitude of the deviation at $z\sim 1$.}
    A distinctive feature of the Cosmochrony framework is that the magnitude of the
    departure from the $\Lambda$CDM expansion history is not freely adjustable.
    At late times, the relative deviation between the locally inferred and globally
    averaged expansion rates is fully controlled by the relaxation budget parameter
    $\Omega_\chi$ and by the effective density contrast $\delta$ in the redshift range
    under consideration.

    Within Cosmochrony, structural consistency imposes
    \begin{equation}
      \Omega_\chi \in [0.25,\,0.35],
    \end{equation}
    reflecting the fraction of relaxation capacity stored in large-scale gradients,
    while the partially developed cosmic web at $z\sim 1$ naturally corresponds to
    moderate underdensities in the range
    \begin{equation}
      \delta(z\sim 1) \in [-0.2,\,-0.5].
    \end{equation}
    Inserting these bounds into Eq.~(C.\ref{eq:hubble_local}) yields a predicted fractional
    enhancement
    \begin{equation}
      \frac{\Delta H}{H}
      \;\equiv\;
      \frac{H_{\mathrm{loc}}-\bar H}{\bar H}
      \;\in\;
      [3.5\%,\,9.5\%]
      \qquad (z\sim 1).
    \end{equation}

    This narrow interval is not a phenomenological choice but a direct consequence of the
    relaxation-based origin of expansion in Cosmochrony.
    Deviations significantly outside this range would require either unphysical values
    of $\Omega_\chi$ or density contrasts incompatible with the observed state of
    large-scale structure at intermediate redshifts.

  \paragraph{Uncertainty estimates and parameter degeneracies.}
    At the present stage, Cosmochrony does not claim a precision reconstruction of $H(z)$.
    Nevertheless, it is useful to clarify how uncertainties in the effective parameters map
    to uncertainties in the plotted $H(z)$ profiles and in the underlying stiffness scale.

    \emph{Late-time effective uncertainty in $(H_0,\Omega_\chi)$.}
    Within the illustrative closure of Eq.~\eqref{eq:Hz_cosmochrony_illustrative},
    small variations propagate as
    \begin{equation}
      \frac{\delta H}{H}
      \;\simeq\;
      \frac{\delta H_0}{H_0}
      \;-\;
      \frac{1}{2}\frac{\delta\Omega_\chi}{1-\Omega_\chi}.
    \end{equation}
    For $\Omega_\chi\simeq 0.30$, an uncertainty $\delta\Omega_\chi\simeq 0.05$ corresponds
    to a fractional variation $\delta H/H \simeq 3.6\%$, comparable to the width used in
    Fig.~\ref{fig:Hz_comparison}. This motivates representing Cosmochrony as a band rather
    than a single curve until dedicated simulations determine $\Omega_\chi(z)$.

    \emph{Degeneracy between $K_0$ and $\chi_c$.}
    The effective gravitational coupling constrains the product
    \begin{equation}
      K_0\,\chi_c^2 \;\sim\; \frac{c^4}{16\pi G},
      \label{eq:K0_chic_deg}
    \end{equation}
    so that $K_0$ and $\chi_c$ cannot be fixed independently without additional input.
    Propagating uncertainties in logarithmic form gives
    \begin{equation}
      \delta\ln K_0 \;\simeq\; -2\,\delta\ln \chi_c \quad (\text{at fixed } G).
    \end{equation}
    Hence, a factor-of-10 uncertainty on $\chi_c$ implies a factor-of-100 uncertainty on $K_0$.
    This is consistent with the fact that both Planck-scale and cosmological-scale
    normalizations can remain internally viable at the level of dimensional analysis,
    until the microscopic origin of $\chi_c$ is constrained by simulations and/or additional
    observables.

  \paragraph{Interpretation and status.}
    Within Cosmochrony, the Hubble tension does not signal a breakdown of cosmological
    consistency.
    It reflects the fact that cosmological expansion is an emergent relaxation
    phenomenon whose effective rate depends on the local redistribution of
    \(\chi\)-field gradients.

    While the framework robustly predicts a separation between local and global
    expansion rates, a fully quantitative determination of \(H(z)\) across all
    redshifts requires dedicated numerical simulations of the \(\chi\) relaxation
    dynamics.
    Such simulations lie beyond the scope of the present work.

    Nevertheless, the qualitative resolution of the Hubble tension follows directly
    from the relaxation-based interpretation of cosmological expansion and
    constitutes a distinctive and testable signature of the Cosmochrony framework.
