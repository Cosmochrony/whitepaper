\subsubsection{Evolution of the Hubble Parameter \(H(z)\) in Cosmochrony}
\label{subsec:hubble_tension}

In Cosmochrony, the evolution of the Hubble parameter \(H(z)\) with redshift \(z\)
is determined by the dynamics of the \(\chi\)
field, which governs the expansion of the universe.
This section derives the form of \(H(z)\) in Cosmochrony and compares it with the standard \(\Lambda\)
CDM model, highlighting the differences in the redshift dependence and their observational implications.

\subsubsubsection{Hubble Parameter in Cosmochrony}

  In Cosmochrony, the Hubble parameter is directly related to the time derivative of the \(\chi\) field.
  The scale factor \(a(t)\) is proportional to \(\chi(t)\), such that:
  \[
    a(t) \propto \chi(t).
  \]

  The Hubble parameter \(H(t)\) is then given by:
  \[
    H(t) = \frac{\dot{a}}{a} = \frac{\dot{\chi}}{\chi}.
  \]

  Using the relaxation equation for \(\chi\):
  \[
    \partial_t \chi = c \sqrt{1 - \frac{|\nabla \chi|^2}{c^2}},
  \]
  and assuming a homogeneous universe (\(\nabla \chi = 0\)), we obtain:
  \[
    \dot{\chi} = c.
  \]

  Thus, the Hubble parameter in Cosmochrony is:
  \[
    H(t) = \frac{c}{\chi(t)}.
  \]

  Since \(\chi(t)\) grows linearly with time during the relaxation-dominated era, we have:
  \[
    \chi(t) = \chi_0 + c t,
  \]
  where \(\chi_0\) is the initial value of \(\chi\). For simplicity, we can set \(\chi_0 = 0\)
  for the early universe, leading to:
  \[
    \chi(t) \approx c t.
  \]

  The Hubble parameter then becomes:
  \[
    H(t) = \frac{c}{\chi(t)} = \frac{1}{t}.
  \]

  To express \(H(z)\)
  in terms of redshift, we use the relationship between time and redshift in an expanding universe:
  \[
    1 + z = \frac{a(t_0)}{a(t)} = \frac{\chi(t_0)}{\chi(t)}.
  \]

  Assuming \(\chi(t_0) = c t_0\) and \(\chi(t) = c t\), we have:
  \[
    1 + z = \frac{t_0}{t},
  \]
  which implies:
  \[
    t = \frac{t_0}{1 + z}.
  \]

  Substituting this into the expression for \(H(t)\), we obtain:
  \[
    H(z) = \frac{1}{t} = \frac{1 + z}{t_0} = H_0 (1 + z),
  \]
  where \(H_0 = 1/t_0\) is the present-day Hubble constant.

\subsubsubsection{Hubble Parameter in \(\Lambda\)CDM}

  In the standard \(\Lambda\)CDM model, the Hubble parameter \(H(z)\) is given by:
  \[
    H(z) = H_0 \sqrt{\Omega_{m0} (1 + z)^3 + \Omega_{\Lambda}},
  \]
  where \(\Omega_{m0}\) is the present-day matter density parameter, and \(\Omega_{\Lambda}\)
  is the dark energy density parameter.

  For comparison, we use the Planck 2018 best-fit values:
  \[
    \Omega_{m0} \approx 0.315, \quad \Omega_{\Lambda} \approx 0.685, \quad H_0 \approx 67.4 \, \text{km/s/Mpc}.
  \]

\subsubsubsection{Comparison of \(H(z)\) in Cosmochrony and \(\Lambda\)CDM}

  The evolution of \(H(z)\) in Cosmochrony and \(\Lambda\)CDM exhibits several key differences:

  \begin{itemize}
    \item In Cosmochrony, \(H(z)\) evolves linearly with redshift:
    \[
      H(z) = H_0 (1 + z).
    \]
    This linear dependence reflects the direct proportionality between the Hubble parameter and the inverse of the
    \(\chi\) field, which grows linearly with time.

    \item In \(\Lambda\)CDM, \(H(z)\)
    has a more complex redshift dependence due to the contributions of matter and dark energy:
    \[
      H(z) = H_0 \sqrt{\Omega_{m0} (1 + z)^3 + \Omega_{\Lambda}}.
    \]
    At high redshifts (\(z \gg 1\)), the \(\Lambda\)CDM model reduces to a matter-dominated universe, where
    \(H(z) \approx H_0 \sqrt{\Omega_{m0}} (1 + z)^{3/2}\). At low redshifts (\(z \ll 1\)
    ), dark energy dominates, and \(H(z)\) approaches a constant value \(H_0 \sqrt{\Omega_{\Lambda}}\).

    \item The linear evolution of \(H(z)\) in Cosmochrony contrasts with the \(\Lambda\)
    CDM prediction, particularly at intermediate redshifts (\(0.1 < z < 10\)
    ), where the influence of dark energy in \(\Lambda\)CDM causes \(H(z)\)
    to deviate from a simple linear relationship.
  \end{itemize}

\subsubsubsection{Quantitative Comparison}

  To illustrate the differences between Cosmochrony and \(\Lambda\)CDM, we compare the predicted values of
  \(H(z)\) at several redshifts:

  \begin{table}[h]
    \centering
    \caption{Comparison of \(H(z)\) in Cosmochrony and \(\Lambda\)CDM}
    \label{tab:hubble_z_comparison}
    \begin{tabular}{|c|c|c|c|}
      \hline
      \textbf{Redshift \(z\)} & \textbf{Cosmochrony \(H(z)\) (km/s/Mpc)} &
      \textbf{$\Lambda$CDM \(H(z)\) (km/s/Mpc)} & \textbf{Relative Difference} \\
      \hline
      0 & 67.4 & 67.4
      & 0\% \\
      0.5 & 101.1 & 95.6
      & +5.8\% \\
      1 & 134.8 & 129.5
      & +4.1\% \\
      3 & 269.6 & 238.5
      & +13.0\% \\
      10 & 741.4 & 560.3
      & +32.3\% \\
      \hline
    \end{tabular}
  \end{table}

  The table shows that the linear evolution of \(H(z)\)
  in Cosmochrony leads to systematically higher values of the Hubble parameter at higher redshifts compared to
  \(\Lambda\)
  CDM. This difference arises because Cosmochrony does not include a dark energy component that slows the
  growth of \(H(z)\) at low redshifts.

\subsubsubsection{Observational Implications}

  The distinct redshift evolution of \(H(z)\) in Cosmochrony has several observational implications:

  \begin{itemize}
    \item \textbf{Baryon Acoustic Oscillations (BAO)}:
    \textbf{Baryon Acoustic Oscillations (BAO)}:
    Measurements of BAO at various redshifts constrain the evolution of $H(z)$.
    In this framework, the acoustic scale is a ``frozen'' imprint of the primordial $\chi$-fluctuations described
    in~\ref{subsubsec:comparison-with-lambda-cdm-acoustic-peaks}.
    However, Cosmochrony predicts a faster increase in $H(z)$ with redshift compared to $\Lambda$CDM.
    Consequently, the apparent angular diameter distance of the BAO scale at $z > 1$ should exhibit a systematic shift,
    offering a clear discriminant between the $\chi$-field relaxation and a constant dark energy density ($\Lambda$).
    This prediction is directly testable with upcoming high-precision surveys such as DESI or Euclid.

    \item \textbf{Type Ia Supernovae}
    : The distance--redshift relation for Type Ia supernovae depends on the integrated expansion
    history encoded in the Hubble parameter \(H(z)\), through the luminosity distance
    \(d_L(z) = (1+z)\int_0^z c\,dz'/H(z')\)~\cite{Riess1998,Perlmutter1999,Hogg1999}.
    The linear evolution of \(H(z)\) in Cosmochrony would result in slightly different distance
    moduli compared to \(\Lambda\)CDM, particularly at intermediate redshifts, where the inferred
    expansion history is most sensitive to the detailed form of \(H(z)\)~\cite{Betoule2014}.

    \item \textbf{CMB Anisotropies}
    : The angular diameter distance to the surface of last scattering and the growth of cosmological
    structures imprinted in the CMB anisotropies are influenced by the integrated history of
    \(H(z)\)~\cite{HuSugiyama1995,Dodelson2003}.
    Cosmochrony's linear \(H(z)\) could therefore lead to subtle differences in the CMB power
    spectrum relative to \(\Lambda\)CDM, particularly in the damping tail at high multipoles and in
    the late-time integrated Sachs--Wolfe effect~\cite{SachsWolfe1967,HuWhite1997}.
  \end{itemize}

\subsubsubsection{Non-linear Resolution of the Hubble Tension}
  \label{appendix:hubble_tension}

  The discrepancy between local and global measurements of $H_0$ can be naturally accounted for through the internal
  kinematics of the $\chi$ field.
  We depart from the fundamental equation of motion $\dot{\chi} = c \sqrt{1 - \beta^2}$, where $\beta = |\nabla \chi|/c$ represents the local field gradient density.

  \paragraph{The Relaxation Budget Parameter $\Omega_\chi$}
    We introduce a dimensionless parameter $\Omega_\chi$, defined as the global fraction of the $\chi$-field relaxation budget stored in spatial gradients:
    \begin{equation}
      \Omega_\chi \equiv \langle \beta^2 \rangle
      \label{eq:chi-relaxation}
    \end{equation}
    In the late universe, $\Omega_\chi$ is empirically constrained to be close to the observed matter
    fraction $\Omega_m$ ($\Omega_\chi \approx \Omega_m \approx 0.31$), reflecting the fact that matter (solitons) is the
    primary source of field gradients.
    Consequently, the global expansion rate is governed by the average available relaxation
    speed: $\bar{H} = \frac{c}{\chi}\sqrt{1 - \Omega_\chi}$.

    The linear identification $\beta_{loc}^2 = \Omega_\chi(1+\delta)$ assumes a ``mean-field'' approximation where the
    energy density of the gradients (the solitons) is locally proportional to the number density of those solitons.
    In this regime, the collective resistance to relaxation scales linearly with the local concentration of field
    deformations, much like the elastic energy density in a deformed medium scales with the density of defects.
    This ensures that in the limit $\delta \to -1$ (absolute vacuum), $\beta^2 \to 0$ and the relaxation
    speed $\dot{\chi}$ reaches its upper bound $c$.

  \paragraph{Local Variation and the Hubble Tension}
    In a local region characterized by a density contrast $\delta = (\rho - \bar{\rho})/\bar{\rho}$, the local gradient density scales as $\beta_{loc}^2 = \Omega_\chi(1+\delta)$.
    This linear scaling constitutes the minimal closure relation between matter inhomogeneities and $\chi$-field gradients, sufficient to capture the leading non-linear effect.
    The local Hubble parameter $H_{loc}$ then deviates from the global average according to:
    \begin{equation}
      H_{loc} = \bar{H} \sqrt{\frac{1 - \Omega_\chi(1 + \delta)}{1 - \Omega_\chi}}
    \end{equation}
    For an underdense region (void) where $\delta < 0$, the available relaxation budget is locally higher, leading to $H_{loc} > \bar{H}$.

  \paragraph{Numerical Consistency}
    Assuming $\Omega_\chi \approx 0.31$ and a local underdensity corresponding to the KBC void ($\delta \approx -0.4$ on a scale of $300$ Mpc), the ratio becomes:
    \begin{equation}
      \frac{H_{loc}}{\bar{H}} = \sqrt{\frac{1 - 0.31(0.6)}{0.69}} \approx 1.084
    \end{equation}
    This $8.4\%$ increase naturally accounts for the Hubble tension.
    This analysis shows that the Hubble tension does not necessarily signal missing components or modifications of the
    cosmological model, but may instead reflect a non-linear environmental effect arising from the relaxation dynamics
    of the $\chi$ field.

\subsubsubsection{Conclusion}

  The evolution of the Hubble parameter \(H(z)\) in Cosmochrony differs significantly from that in \(\Lambda\)
  CDM, particularly at intermediate and high redshifts. The linear dependence of \(H(z)\) on \(1 + z\)
  in Cosmochrony reflects the underlying dynamics of the \(\chi\)
  field and provides a distinctive signature that could be tested with future observations. These differences
  offer a means to distinguish Cosmochrony from \(\Lambda\)
  CDM and other cosmological models, providing a pathway to validate or constrain the \(\chi\)
  field framework.
