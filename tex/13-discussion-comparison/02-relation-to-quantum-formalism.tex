\subsection{Relation to Quantum Formalism}
  \label{subsec:relation-to-quantum-formalism}

  Quantum mechanics and quantum field theory (QFT) introduce probabilistic
  wavefunctions, operators, and quantization rules as foundational postulates~\cite{PeskinSchroeder1995QFT}.
  In contrast, Cosmochrony treats continuous wave dynamics as primary and regards
  quantization as an emergent, interaction-dependent phenomenon.

  Within this framework, particles correspond to localized, topologically stable
  wave configurations (soliton-like excitations) of the $\chi$ field.
  Discrete observables arise from boundary conditions, topological constraints,
  and interaction-induced mode selection, rather than from intrinsic microscopic
  discreteness.
  The Planck relation $E = h\nu$ is interpreted as a geometric correspondence
  between oscillation frequency, field curvature, and the energetic cost of local
  $\chi$ deformation.

  Quantum correlations are described in relational terms.
  Entanglement corresponds to the persistence of a shared $\chi$ configuration
  across spatial separation, while decoherence reflects the irreversible
  fragmentation of this configuration through interactions with the surrounding
  field.
  This interpretation reproduces the standard quantum phenomenology, including
  nonlocal correlations, without invoking superluminal signaling, fundamental
  wavefunction collapse, or hidden variables.
