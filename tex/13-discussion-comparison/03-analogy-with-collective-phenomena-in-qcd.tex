\subsection{Analogy with Collective Phenomena in QCD}
  \label{subsec:analogy-with-collective-phenomena-in-qcd}

  A useful structural analogy may be drawn with quantum chromodynamics (QCD) in the
  low-energy regime, where the fundamental degrees of freedom (quarks and gluons)
  do not correspond directly to observable particles~\cite{Shifman2007QCDVacuum}.
  Instead, hadronic properties, effective masses, and confinement phenomena emerge
  from a strongly interacting collective vacuum structure, often described in terms
  of a quark--gluon medium.

  In a similar conceptual spirit, the Cosmochrony framework does not attribute
  gravitational phenomena to a fundamental interaction mediated by elementary
  gravitational degrees of freedom.
  Rather, gravity arises as a collective effect of localized excitations and
  modulations of the underlying $\chi$ field, whose large-scale behavior cannot be
  reduced to simple superpositions of microscopic dynamics.

  As in QCD, the appropriate physical description depends on the scale and regime
  considered.
  While the underlying dynamics may be simple in principle, the emergent macroscopic
  behavior is governed by nonlinear and collective effects that are more naturally
  captured by effective, phenomenological descriptions.
  This scale-dependent hierarchy of descriptions reinforces the view that geometry
  and gravitation in Cosmochrony are emergent constructs, rather than fundamental
  ontological primitives.
