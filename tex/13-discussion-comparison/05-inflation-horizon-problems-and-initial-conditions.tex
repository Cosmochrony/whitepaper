\subsection{Inflation, Horizon Problems, and Initial Conditions}
  \label{subsec:inflation-horizon-problems-and-initial-conditions}

  Standard inflationary theory addresses the horizon, flatness, and monopole
  problems by postulating a brief phase of accelerated expansion driven by an
  inflaton field.
  In the Cosmochrony framework, these issues are approached from a different
  conceptual standpoint.

  Because the fundamental field $\chi$ defines a global relaxation process rather
  than a metric expansion imposed externally, causal connectivity is preserved at
  the level of the underlying field dynamics.
  Large-scale coherence may therefore arise from the initial smoothness of $\chi$
  and its subsequent monotonic relaxation, potentially alleviating the need for a
  distinct inflationary epoch as a fundamental assumption.

  At this stage, this perspective should be regarded as an alternative
  interpretative framework rather than a complete replacement for inflationary
  cosmology.
  A detailed analysis of primordial perturbations, their spectrum, and their
  imprint on the cosmic microwave background (CMB) is required to determine the
  extent to which Cosmochrony reproduces, modifies, or departs from standard
  inflationary predictions.

  These questions define a clear direction for future work, in which the
  connection between early-time $\chi$ dynamics and observable cosmological
  signatures can be explored quantitatively.
