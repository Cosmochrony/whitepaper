\subsection{Conceptual Implications and Open Challenges}
  \label{subsec:conceptual-implications-and-open-challenges}

  Cosmochrony proposes a unifying conceptual framework in which time, distance,
  energy, gravitation, and quantization emerge from the dynamics of a single
  pre-geometric relational substrate.
  This ontological economy constitutes a central strength of the framework, while
  also requiring a careful reassessment of notions traditionally treated as
  independent physical primitives.

  In particular, the framework suggests that time, energy, and irreversibility do
  not correspond to distinct fundamental entities.
  Temporal ordering arises from the monotonic relaxation of the $\chi$ substrate,
  while energy quantifies the residual capacity of localized configurations to
  resist this relaxation.
  Irreversibility then reflects the progressive exhaustion of such relaxation
  capacity.
  From this perspective, temporal flow and energetic processes are not independent
  axioms of nature, but complementary effective descriptions of the same underlying
  relational dynamics.

  At the level of effective physical descriptions, these relations are encoded in
  coarse-grained quantities such as $\chi_{\mathrm{eff}}$, which summarize how the
  relaxation structure of $\chi$ manifests in spacetime-based observables.
  These effective constructs carry no independent ontological status and remain
  valid only within regimes where a geometric interpretation is applicable.

  A concrete realization of this unification, including an explicit formulation of
  the relaxation operator and its spectral role in mass generation, is outlined in
  Appendix~\ref{subsec:perspectives_mass_spectrum}.
  While this reinterpretation addresses several long-standing conceptual tensions
  ---including the origin of the arrow of time and the status of energy conservation
  ---it also raises important open challenges.

  Among these challenges are:
  \begin{itemize}
    \item the quantitative reconstruction of cosmic microwave background anisotropies
    from early-time $\chi$ dynamics,
    \item the detailed treatment of non-equilibrium quantum measurements, decoherence,
    and reprojection processes,
    \item the emergence of gauge symmetries and interaction hierarchies from
    topological and relational features of $\chi$,
    \item and the long-term stability of solitonic particle configurations under
    extreme gravitational or radiative conditions.
  \end{itemize}

  Addressing these issues will require a combination of analytical, numerical, and
  experimental approaches, including:
  \begin{enumerate}
    \item large-scale numerical simulations of $\chi$ dynamics to quantify structure
    formation and cosmological signatures,
    \item the exploration of discretized, network-based, or lattice realizations of
    $\chi$ at microscopic scales,
    \item and targeted experimental tests of predicted $\chi$-dependent effects in
    quantum coherence, gravitation, and radiation processes.
  \end{enumerate}

  Progress along these directions may elevate Cosmochrony from a unifying
  interpretative framework to a quantitatively predictive theory, while preserving
  its minimal ontological foundation.
