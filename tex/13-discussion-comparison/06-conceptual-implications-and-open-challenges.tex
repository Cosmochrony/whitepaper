\subsection{Conceptual Implications and Open Challenges}
  \label{subsec:conceptual-implications-and-open-challenges}

  Cosmochrony proposes a unifying geometric narrative in which time, distance,
  energy, gravitation, and quantization emerge from the dynamics of a single
  evolving field.
  This conceptual economy constitutes a central strength of the framework, while
  also requiring a careful reassessment of several notions traditionally treated
  as independent physical primitives.

  In particular, the framework suggests that time, energy, and irreversibility do
  not represent distinct fundamental entities.
  Rather, temporal ordering is provided by the monotonic relaxation of the
  $\chi$ field, while energy quantifies the residual capacity of $\chi$
  configurations to relax.
  Irreversibility then reflects the progressive exhaustion of this relaxation
  capacity.
  From this perspective, temporal flow and energetic processes constitute
  complementary descriptions of the same underlying geometric dynamics, rather
  than independent axioms of nature.

  A concrete realization of this unification, including an explicit formulation
  of the relaxation operator and its spectral role in mass generation, is outlined
  in Appendix~\ref{subsec:perspectives_mass_spectrum}.
  While this reinterpretation addresses several long-standing conceptual tensions
  ---such as the origin of the arrow of time and the status of energy conservation
  ---it also raises important open questions.

  Among these challenges are:
  \begin{itemize}
    \item the precise mapping between $\chi$ dynamics and observed cosmic microwave
    background anisotropies,
    \item the treatment of non-equilibrium quantum measurements and decoherence,
    \item the emergence of gauge symmetries and interaction hierarchies,
    \item and the stability of solitonic particle configurations under extreme
    conditions.
  \end{itemize}

  Addressing these issues will require a combination of analytical, numerical, and
  experimental approaches, including:
  \begin{enumerate}
    \item large-scale numerical simulations of $\chi$ dynamics to quantify
    structure formation and cosmological signatures,
    \item the exploration of discretized or network-based realizations of $\chi$
    at microscopic scales,
    \item and experimental tests of predicted $\chi$-dependent effects in quantum
    coherence, gravitation, and radiation processes.
  \end{enumerate}

  Progress along these directions may elevate Cosmochrony from a unifying
  conceptual framework to a quantitatively predictive theory, while preserving its
  minimal ontological foundation.
