\subsection{Relation to General Relativity}
  \label{subsec:relation-to-general-relativity}

  General Relativity (GR) describes gravitation as the curvature of spacetime induced
  by energy--momentum.
  In Cosmochrony, no \emph{a priori} metric dynamics is postulated at the fundamental
  level.
  Instead, an effective spacetime geometry emerges as a descriptive framework from
  variations in the local relaxation dynamics of the $\chi$ field.

  Matter configurations, modeled as stable or metastable topological excitations of
  $\chi$, locally constrain the relaxation of the field.
  This leads to differential rates of effective proper-time evolution between
  neighboring regions.
  When expressed in geometric terms, these differences can be reinterpreted as an
  effective deformation of the spacetime metric.

  In the weak-field regime, this mechanism reproduces Newtonian gravity, while in the
  strong-field limit it yields Schwarzschild-like solutions in effective geometric
  descriptions.
  The resulting phenomenology is therefore consistent with the empirical successes
  of GR across its tested domain.

  From this perspective, gravitation is not introduced as a fundamental interaction,
  but emerges as a macroscopic manifestation of inhomogeneous $\chi$ relaxation.
  General Relativity is recovered as the appropriate effective theory describing
  this regime, rather than being supplanted or modified at the level of observable
  predictions.
