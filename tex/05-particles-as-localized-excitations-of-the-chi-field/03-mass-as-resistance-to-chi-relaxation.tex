\subsection{Mass as Resistance to $\chi$ Relaxation}
  \label{subsec:mass-as-resistance-to-chi-relaxation}

  In this framework, mass is not an intrinsic attribute but an emergent measure of how strongly a localized
  excitation resists the local increase of $\chi$.
  Mass therefore characterizes the persistence of a non-relaxed configuration within an otherwise
  globally relaxing field.

  Regions containing particle excitations exhibit increased spatial gradients:
  \begin{equation}
    |\nabla \chi| > 0 ,
  \end{equation}
  which, according to Eq.~\eqref{eq:chi_dynamics}, reduces the local relaxation rate $\partial_t \chi$.
  This local slowdown of $\chi$-relaxation manifests macroscopically as time dilation and, at larger scales,
  as an effective gravitational mass.

  From an energetic perspective, such localized resistance corresponds to the storage of relaxation
  potential within the $\chi$ field.
  Maintaining a stable excitation requires continuously counterbalancing the global tendency of $\chi$
  to relax, and this cost is what is operationally identified as energy.
  In this sense, mass quantifies the amount of relaxation potential that remains trapped in a localized
  configuration.

  Mapping of the energy of solitons to the masses of observed particles is detailed in
  Section~\ref{subsec:soliton_energy_mass}.
