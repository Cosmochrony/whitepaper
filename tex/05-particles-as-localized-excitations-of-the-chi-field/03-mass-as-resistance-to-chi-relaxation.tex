\subsection{Mass as Resistance to $\chi$ Relaxation}
  \label{subsec:mass_as_resistance}

  In Cosmochrony, mass is not introduced as an intrinsic or fundamental property of matter.
  Instead, it emerges as a quantitative measure of how strongly a localized configuration of
  the $\chi$ field resists the global relaxation flow.

  A particle-like excitation is modeled as a stable, localized solitonic configuration
  $\chi_s$, characterized by sustained internal structure and non-vanishing gradients.
  Such configurations locally inhibit the relaxation of $\chi$, producing a slowdown
  relative to the homogeneous background evolution. In an effective geometric description,
  this slowdown manifests as inertial persistence and gravitational time dilation.

  We define the structural energy associated with a solitonic configuration $\chi_s$ as
  the excess relaxation capacity stored in its internal curvature:
  \begin{equation}
    E[\chi_s] \;\equiv\;
    \int_{\Sigma}
    \left(
      \frac{1}{\sqrt{1 - |\nabla \chi_s|^2 / c^2}} - 1
    \right)
    \, d\Sigma ,
    \label{eq:chi_soliton_energy}
  \end{equation}
  where $\Sigma$ denotes a hypersurface of constant effective ordering parameter,
  and $|\nabla \chi_s|$ quantifies the local structural deformation of the $\chi$ field.
  This expression measures the energetic cost of maintaining a non-relaxed configuration
  embedded within a globally relaxing substrate.

  The inertial mass associated with the soliton is then defined operationally as
  \begin{equation}
    m \;\equiv\; \frac{E[\chi_s]}{c^2}.
    \label{eq:mass_definition}
  \end{equation}

  This relation is not postulated but follows directly from the role of $E[\chi_s]$ as
  a measure of resistance to $\chi$ relaxation. The universal constant $c$ appears as
  the maximal relaxation rate of the field and therefore provides the unique conversion
  factor between relaxation energy and inertial response.

  Within this framework, the relation $E = mc^2$ is interpreted as a kinematic identity:
  mass quantifies the amount of relaxation potential locally trapped in a persistent
  $\chi$ configuration, while energy measures the same quantity expressed in relaxation units.

  In this sense, mass is not an independent attribute of matter, but a derived property
  encoding how strongly a localized $\chi$ configuration resists the irreversible
  relaxation that defines physical time.

  The question of how different particle masses arise from distinct solitonic
  configurations is addressed in Appendix~\ref{app:topological_solitons}, where a spectral characterization
  of $\chi$-field stability modes is proposed as the geometric origin of mass hierarchies.
