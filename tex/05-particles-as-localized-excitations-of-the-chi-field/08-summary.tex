\subsection{Summary}
  \label{subsec:summary7}

  In the Cosmochrony framework, particles emerge as stable, localized excitations of the
  $\chi$ field that locally resist its irreversible relaxation.
  Their physical properties are not postulated but arise from the internal structure
  and topology of these configurations.

  Mass is identified with the amount of relaxation capacity trapped in a solitonic
  $\chi$ configuration, quantified by its internal curvature and resistance to the
  global relaxation flow.
  In regimes where an effective relativistic description applies, this leads naturally
  to the relation $E = mc^2$, interpreted as a kinematic identity rather than a fundamental
  postulate.

  Spin and statistical behavior originate from topological obstructions in the space of
  localized $\chi$ configurations.
  Fermionic excitations exhibit a $4\pi$ periodicity, such that a $2\pi$ rotation
  corresponds to a non-contractible loop in configuration space, inducing a sign change
  of the effective wavefunction.
  This topological property provides a common origin for spin-$\tfrac{1}{2}$ behavior
  and fermionic antisymmetry, including the Pauli exclusion principle.

  In this perspective, different particle attributes correspond to distinct topological
  invariants of localized $\chi$ configurations.
  While spin is associated with non-trivial covering properties of configuration space,
  electric charge may be interpreted as an oriented topological defect or vortex-like
  structure of the $\chi$ field.
  These properties remain conceptually distinct but arise from a common topological
  substrate, suggesting a unified geometric origin of internal quantum numbers.

  Together, these results provide a unified account of particle properties compatible
  with both relativistic and quantum phenomena, without introducing particles or their
  attributes as fundamental ontological entities.
