\subsection{Fermions and Bosons}
    \label{subsec:fermions-and-bosons}

  Particle statistics arise from the topology of the underlying excitation.
  Configurations requiring a $4\pi$
  phase rotation to return to their original state correspond to fermions, while integer-winding configurations
  correspond to bosons.

  This topological distinction naturally reproduces the spin-statistics connection without invoking additional
  quantum postulates.

  Consider a localized soliton solution $\chi(\mathbf{x}) = \chi_0 \tanh(r/\xi)$, where $r$
  is the radial coordinate and $\xi$ sets the soliton size.
  For fermionic excitations, the phase of $\chi$ must wind by $4\pi$ to return to its original value, reflecting a
  M\"obius-like twist in the field configuration.
  This topological constraint enforces the spin-statistics theorem:
  only configurations with half-integer winding numbers (fermions) can exhibit such $4\pi$-periodicity, while integer
  windings (bosons) correspond to $2\pi$-periodic solutions.
