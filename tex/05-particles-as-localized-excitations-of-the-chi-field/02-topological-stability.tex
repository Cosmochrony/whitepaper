\subsection{Topological Stability}
  \label{subsec:topological-stability}

  The stability of particle-like excitations is attributed to topological constraints rather than to conserved
  charges postulated a priori\cite{rajaraman1982solitons}.
  Nontrivial phase winding, torsion, or knot-like structures in $\chi$
  prevent continuous deformation into the vacuum state.

  Such topological protection naturally explains the discreteness of particle species and their robustness under
  perturbations.

  For instance, an electron corresponds to a localized knot in $\chi$
  with a specific winding number, where the knot's energy (proportional to its curvature) determines the
  particle's mass, and its topological charge (e.g., $4\pi$-periodicity) determines its spin-1/2 nature.

  The stability of solitonic excitations arises from a balance between the nonlinear self-interaction term $V(\chi)$
  (which tends to localize the field) and the gradient energy $|\nabla \chi|^2$
  (which tends to disperse it).
  Topological invariants, such as the winding number $n = \frac{1}{2\pi} \oint \nabla \arg(\chi) \cdot d\mathbf{l}$,
  further protect these configurations from decay, ensuring their persistence as particle-like objects.

  More topological configurations are discussed in ~\ref{subsec:topological_solitons}
