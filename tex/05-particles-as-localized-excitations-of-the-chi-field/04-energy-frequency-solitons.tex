\subsection{Energy--Frequency Relation}
  \label{subsec:energy-frequency-solitons}

  The energy associated with a particle excitation is linked to the internal oscillation frequency of its wave
  configuration.
  Within the Cosmochrony framework, this frequency characterizes how strongly the excitation resists relaxation:
  higher-frequency structures correspond to tighter localization, stronger spatial gradients in $\chi$, and a
  greater capacity to relax.

  This provides a geometric interpretation of the relation
  \begin{equation}
    E \propto \nu ,
  \end{equation}
  in which energy measures the amount of relaxation potential stored in a given configuration, while frequency
  quantifies the rate at which this potential can be redistributed.
  Planck's constant then emerges as an effective proportionality factor determined by the properties of the $\chi$
  field and its coupling scales, rather than as a fundamental constant postulated a priori.

  A more explicit geometric derivation of this relation, in the context of radiation and photon-like
  excitations of the $\chi$ field, is presented in Section~\ref{subsec:energy-frequency-radiation}.
