\subsection{Collective Gravitational Coupling and Operational Geometry}
  \label{subsec:collective-gravitational-coupling-and-operational-geometry}

  The collective slowdown of $\chi$ relaxation described above affects not only the
  local flow of effective time but also the manner in which variations of $\chi$
  influence one another across extended regions.
  In the presence of localized excitations, the resistance they impose on the
  relaxation of $\chi$ modulates how efficiently structural variations of the field
  are transmitted.

  This collective behavior may be described, at an effective level, by a local and
  constitutive coupling function that characterizes the stiffness of the $\chi$
  field to relative variations.
  In regions where $\chi$ is nearly homogeneous, this coupling approaches a uniform
  vacuum value, while localized excitations weaken it by introducing additional
  structural constraints.
  Crucially, this coupling depends only on the local configuration of $\chi$ and
  does not presuppose any background spatial metric.

  Because no fundamental geometry is assumed, spatial distance is defined
  operationally.
  Two regions are considered close if variations of $\chi$ propagate efficiently
  between them, and distant otherwise.
  In the continuum and weak-variation regime, this operational notion naturally
  admits a description in terms of an effective spatial metric, which summarizes
  the collective response of the $\chi$ field.

  Within this framework, spacetime curvature does not arise as a primitive geometric
  property, but as an emergent manifestation of how localized excitations modulate
  the collective propagation and relaxation dynamics of $\chi$.

  A more explicit relational construction of the coupling mechanism and its
  connection to discrete formulations is presented in
  Appendix~\ref{subsec:collective-coupling}.
