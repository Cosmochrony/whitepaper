\subsection{Equivalence Principle}
  \label{subsec:equivalence-principle}

  Within the Cosmochrony framework, particle-like excitations do not couple to a
  fundamental gravitational field.
  Instead, they are described, at the level of effective descriptions, as localized
  projected configurations that impose constraints on admissible relaxation ordering.

  Because all such projected configurations resist admissible relaxation ordering in
  the same structural manner, the collective reduction of ordering is independent of
  their internal composition or detailed structure.
  As a result, all particle-like projected configurations respond identically to a
  given effective ordering environment.

  When expressed in effective geometric terms, this universal response appears as
  composition-independent gravitational acceleration.
  The equivalence between inertial and gravitational behavior therefore emerges as a
  direct consequence of the uniform manner in which admissible projected
  configurations constrain relaxation ordering, rather than as an independent
  postulate imposed on the theory.

  In this sense, the equivalence principle arises naturally within Cosmochrony as an
  emergent symmetry of admissible projected descriptions, reflecting the absence of
  any intrinsic distinction between inertial resistance and gravitational response at
  the effective level.
