\subsection{Strong Gravity and Black Holes}
  \label{subsec:strong-gravity-and-black-holes}

  In regions where the density of localized excitations becomes sufficiently high,
  the relaxation dynamics of the $\chi$ field may become extremely constrained.
  In effective spacetime descriptions, this corresponds to a regime in which the
  local relaxation rate is strongly suppressed relative to distant observers,
  defining an effective horizon.

  Within Cosmochrony, such regions are interpreted as black holes.
  Rather than being characterized by a fundamental spacetime singularity, black
  holes correspond to domains where the unfolding of physical processes becomes
  asymptotically inaccessible from the exterior due to the collective inhibition
  of $\chi$ relaxation.
  This naturally accounts for extreme time dilation effects without requiring
  divergent curvature invariants.

  These regions therefore mark not a terminal endpoint of physical description,
  but a transition toward a non-spatiotemporal regime of the underlying $\chi$
  structure.

  \subsubsection{Gravitational and Temporal Shadows}

    In the strong-gravity regime, the increasing concentration of excitations induces
    large structural constraints in the $\chi$ field.
    As a result, the effective rate at which $\chi$ relaxes relative to external
    parametrizations is progressively reduced, approaching an asymptotic freeze-out
    in effective geometric descriptions.

    This behavior reproduces the phenomenon commonly referred to as a
    \emph{gravitational shadow}.
    In general relativity, such shadows arise from the absence of escaping null
    geodesics within a characteristic angular region.
    In Cosmochrony, an equivalent observational signature emerges because propagating
    excitations of the $\chi$ field, including radiation-like modes, cannot be
    sustained in regions where the relaxation dynamics is effectively frozen.
    External observers therefore perceive a dark angular region corresponding to the
    projection of this dynamically inaccessible domain.

    Beyond this optical effect, the framework predicts a deeper phenomenon, which may
    be termed a \emph{temporal shadow}.
    As the local relaxation of $\chi$ becomes increasingly inhibited, the effective
    progression of time within the region slows asymptotically with respect to the
    external environment.
    From the external perspective, internal processes appear indefinitely delayed,
    providing a natural interpretation of horizon-induced time dilation.

    In this view, the observed gravitational shadow corresponds to the visible
    manifestation of an underlying temporal shadow.
    Both effects arise from the same collective relaxation dynamics of the $\chi$
    field and need not be attributed to a fundamental spacetime singularity or to
    divergent tensorial curvature.

  \subsubsection{Absence of Physical Singularities}

    In classical general relativity, black holes are associated with spacetime singularities characterized by divergent
    curvature and energy density.
    In Cosmochrony, such singularities are interpreted as artifacts of effective spacetime descriptions that neglect the
    structural bound imposed by $c_{\chi}$.

    Since the structural bound $c_{\chi}$ limits the maximal confinement of information
    within $\chi$, configurations corresponding to infinite mass density or curvature
    cannot arise physically.
    Apparent singularities therefore signal the breakdown of effective spacetime
    descriptions rather than genuine divergences of the underlying substrate.

    \paragraph{Structural bound and notation.}
      To avoid confusion between fundamental and emergent levels, we distinguish the
      dimensionless structural bound $c_{\chi}$, defined at the level of the pre-temporal
      $\chi$ substrate, from its emergent spacetime manifestation $c$, interpreted as the
      maximal signal propagation speed.
      While $c$ may exhibit effective regime-dependent variations, the bound $c_{\chi}$
      is invariant.

  \subsubsection{Black Holes, Deprojection, and Vacuum Reprojection}
    \label{subsec:black-hole-deprojection-cycle}

    Within Cosmochrony, the absence of physical singularities does not imply that
    black holes are dynamically inert.
    Rather, they correspond to regimes in which the structural bound $c_{\chi}$ is
    locally saturated, preventing any further confinement of relational information
    within emergent spacetime.

    In such regimes, information encoded in localized excitations cannot remain
    expressed in spatiotemporal relational form.
    Instead, it undergoes a deprojection: relational information ceases to be
    maintainable and reverts to a purely structural encoding within the $\chi$
    substrate.
    This process does not correspond to transport across a spatial boundary nor to a
    temporal reversal, but to a breakdown of the emergent spacetime description itself.

    Importantly, deprojected information is not lost.
    Due to intrinsic fluctuations of the $\chi$ substrate, structurally encoded
    information remains in principle reprojectable.
    Reprojection from $\chi$ occurs in discrete units governed by the fundamental
    granularity $\hbar_{\chi}$, and manifests in emergent spacetime as transient or
    stable excitations commonly interpreted as vacuum particles.

    In this view, black holes and vacuum fluctuations participate in a single closed
    informational cycle.
    Extreme confinement leads to deprojection into $\chi$, while intrinsic
    fluctuations enable reprojection elsewhere.
    The quantum vacuum is thus understood not as an empty background, but as the
    emergent expression of ongoing granular reprojection from the $\chi$ substrate.
