
Before turning to strong-gravity regimes, it is useful to summarize the different
projective regimes of the $\chi$ field discussed in this section.

\begin{figure}[h]
  \centering
  \begin{tabular}{c c c c c}
    \textbf{Projectable} &
    $\rightarrow$ &
    \textbf{Horizon} &
    $\rightarrow$ &
    \textbf{Non-projectable} \\[6pt]

    Gravitational &
    &
    Boundary of &
    &
    Deprojected \\
    Waves &
    &
    Projection &
    &
    $\chi$ Regime \\[6pt]

    $\delta \chi$ &
    &
    $\Phi$ non-injective &
    &
    No spacetime \\
    (projected) &
    &
    &
    &
    representation
  \end{tabular}
  \caption{Conceptual regimes of the $\chi$ field projection.
  Gravitational waves correspond to fully projectable collective modulations of $\chi$,
    while black holes mark the boundary beyond which spacetime descriptions cease to apply.
    Black hole evaporation reflects the gradual restoration of projectability, without
    any loss of information at the fundamental $\chi$ level.}
  \label{fig:chi_projection_regimes}
\end{figure}

This schematic overview highlights how gravitational waves, horizons, and black hole
evaporation correspond to distinct regimes of the same underlying $\chi$ dynamics.

\subsection{Strong Gravity and Black Holes}
  \label{subsec:strong-gravity-and-black-holes}

  In regions where the density of localized excitations becomes sufficiently high,
  the relaxation dynamics of the $\chi$ field may become extremely constrained.
  In effective spacetime descriptions, this corresponds to a regime in which the
  local relaxation rate is strongly suppressed relative to distant observers,
  defining an effective horizon.

  Within Cosmochrony, such regions are interpreted as black holes.
  Rather than being characterized by a fundamental spacetime singularity, black
  holes correspond to domains where the unfolding of physical processes becomes
  asymptotically inaccessible from the exterior due to the collective inhibition
  of $\chi$ relaxation.
  This naturally accounts for extreme time dilation effects without requiring
  divergent curvature invariants.

  These regions therefore mark not a terminal endpoint of physical description,
  but a transition toward a non-spatiotemporal regime of the underlying $\chi$
  structure.

  \subsubsection{Gravitational and Temporal Shadows}

    In the strong-gravity regime, the increasing concentration of excitations induces
    large structural constraints in the $\chi$ field.
    As a result, the effective rate at which $\chi$ relaxes relative to external
    parametrizations is progressively reduced, approaching an asymptotic freeze-out
    in effective geometric descriptions.

    This behavior reproduces the phenomenon commonly referred to as a
    \emph{gravitational shadow}.
    In general relativity, such shadows arise from the absence of escaping null
    geodesics within a characteristic angular region.
    In Cosmochrony, an equivalent observational signature emerges because propagating
    excitations of the $\chi$ field, including radiation-like modes, cannot be
    sustained in regions where the relaxation dynamics is effectively frozen.
    External observers therefore perceive a dark angular region corresponding to the
    projection of this dynamically inaccessible domain.

    Beyond this optical effect, the framework predicts a deeper phenomenon, which may
    be termed a \emph{temporal shadow}.
    As the local relaxation of $\chi$ becomes increasingly inhibited, the effective
    progression of time within the region slows asymptotically with respect to the
    external environment.
    From the external perspective, internal processes appear indefinitely delayed,
    providing a natural interpretation of horizon-induced time dilation.

    In this view, the observed gravitational shadow corresponds to the visible
    manifestation of an underlying temporal shadow.
    Both effects arise from the same collective relaxation dynamics of the $\chi$
    field and need not be attributed to a fundamental spacetime singularity or to
    divergent tensorial curvature.

  \subsubsection{Absence of Physical Singularities}

    In classical general relativity, black holes are associated with spacetime singularities characterized by divergent
    curvature and energy density.
    In Cosmochrony, such singularities are interpreted as artifacts of effective spacetime descriptions that neglect the
    structural bound imposed by $c_{\chi}$.

    Since the structural bound $c_{\chi}$ limits the maximal confinement of information
    within $\chi$, configurations corresponding to infinite mass density or curvature
    cannot arise physically.
    Apparent singularities therefore signal the breakdown of effective spacetime
    descriptions rather than genuine divergences of the underlying substrate.

    \paragraph{Structural bound and notation.}
      To avoid confusion between fundamental and emergent levels, we distinguish the
      dimensionless structural bound $c_{\chi}$, defined at the level of the pre-temporal
      $\chi$ substrate, from its emergent spacetime manifestation $c$, interpreted as the
      maximal signal propagation speed.
      While $c$ may exhibit effective regime-dependent variations, the bound $c_{\chi}$
      is invariant.

  \subsubsection{Black Holes, Deprojection, and Vacuum Reprojection}
    \label{subsec:black-hole-deprojection-cycle}

    Within Cosmochrony, the absence of physical singularities does not imply that
    black holes are dynamically inert.
    Rather, they correspond to regimes in which the structural bound $c_{\chi}$ is
    locally saturated, preventing any further confinement of relational information
    within emergent spacetime.

    The emergence of an effective spacetime description relies on a projection map
    \[
      \Phi : \mathcal{C}_{\chi} \longrightarrow \mathcal{M},
    \]
    from the space of $\chi$-field configurations to an effective spacetime manifold.
    In weak- and moderate-field regimes, this map is assumed to be locally injective,
    ensuring a faithful geometric encoding of relational information.

    In strong-gravity regimes associated with black holes, this injectivity breaks
    down.
    Multiple inequivalent $\chi$-configurations may correspond to the same effective
    spacetime event, signaling a loss of representability at the geometric level
    without any loss of information at the fundamental level.
    We refer to this loss of injectivity of $\Phi$ as \emph{deprojection}.

    Deprojection does not correspond to transport across a spatial boundary nor to a
    temporal reversal.
    Instead, relational information ceases to be expressible in spatiotemporal form
    and reverts to a purely structural encoding within the $\chi$ substrate.
    The apparent horizon thus marks a boundary of projection validity rather than a
    fundamental causal or physical discontinuity.

    Importantly, deprojected information is not destroyed.
    Due to intrinsic fluctuations of the $\chi$ substrate, structurally encoded
    information remains in principle reprojectable.
    Reprojection occurs in discrete units governed by the fundamental granularity
    $\hbar_{\chi}$ and manifests in emergent spacetime as propagating excitations,
    including radiation-like modes or particle--antiparticle pairs.

    In this view, black holes and vacuum fluctuations participate in a single closed
    informational cycle.
    Extreme confinement leads to deprojection into $\chi$, while intrinsic
    fluctuations enable reprojection elsewhere.
    The quantum vacuum is thus understood not as an empty background, but as the
    emergent expression of ongoing granular reprojection from the $\chi$ substrate.

\subsection{Black Hole Evaporation and the Information Problem}
  \label{subsec:black-hole-evaporation-information}

  Within the Cosmochrony framework, black holes are not associated with the formation
  of physical spacetime singularities, but with the emergence of regions where the
  projection of the $\chi$ field onto an effective spacetime description ceases to be
  injective.
  Such regions define domains of limited representability, bounded by an effective
  horizon, beyond which relational information encoded in $\chi$ can no longer be
  faithfully expressed in geometric or field-theoretic terms.

  Crucially, this loss of injectivity—referred to as \emph{deprojection}—does not
  correspond to a dynamical process occurring at a definite spacetime event.
  Rather, it marks the boundary of validity of emergent spacetime descriptions.
  The would-be singularity is therefore not a physical endpoint of evolution, but a
  non-spatiotemporal regime of the underlying $\chi$ substrate.

  \paragraph{Evaporation as a Projective Phenomenon.}
    Black hole evaporation, in contrast, is an entirely projective and effective process.
    It unfolds within the regime where an approximate spacetime description remains
    well-defined, namely in the vicinity of the horizon.
    In Cosmochrony, Hawking-like radiation does not originate from a transfer of matter
    or information from a singular interior, but from the progressive reconfiguration of
    the projection of $\chi$ near the boundary separating projectable and non-projectable
    domains.

    As the black hole evolves, the region of non-projectability gradually shrinks.
    This manifests to external observers as a slow loss of mass and energy through
    approximately thermal radiation.
    The evaporation process is thus completed before any effective description would
    encounter the deprojection regime associated with the would-be singularity.
    From the spacetime perspective, the singularity is never dynamically accessed.

  \paragraph{Resolution of the Information Paradox.}
    The apparent loss of information traditionally associated with black hole evaporation
    was first articulated in the seminal work of Hawking~\cite{Hawking1976}.
    In the standard semiclassical treatment, the emission of thermal radiation from an
    evaporating black hole leads to a final mixed state, apparently incompatible with
    unitary quantum evolution.

    Rather than modifying semiclassical quantum field theory or postulating non-unitary
    dynamics, Cosmochrony reinterprets the ontological status and domain of applicability
    of spacetime-based descriptions.

    In this framework, information is encoded in the global configuration of the $\chi$
    field, including its nonlocal relational structure, independently of its projection
    onto spacetime.
    Deprojection corresponds to a loss of geometric representability, not to the
    destruction of information.
    Throughout the formation and evaporation of a black hole, the $\chi$ configuration
    remains well-defined and retains all correlations.

    The Hawking evaporation process therefore does not violate information conservation.
    What is lost is not information itself, but access to a complete spacetime encoding
    of that information.
    The black hole information paradox is thus resolved by recognizing it as an artifact
    of extending effective spacetime concepts beyond their domain of applicability.

  \paragraph{Observational Implications.}
    From the perspective of external observers, the emitted radiation may appear nearly
    thermal and only weakly correlated with the initial infalling states.
    This does not signal a fundamental non-unitarity, but reflects the coarse-grained
    nature of the projection from $\chi$ to emergent spacetime observables.
    Full reconstruction of the underlying information from Hawking radiation is neither
    required nor generically possible within the effective description, and its absence
    does not indicate any inconsistency of the theory.

    In summary, black hole evaporation in Cosmochrony is a manifestation of the gradual
    restoration of projectability in the $\chi$ field, while information remains globally
    conserved at the fundamental level.
    The singularity is never reached, the horizon is not an information sink, and no
    modification of low-energy quantum mechanics is required.
