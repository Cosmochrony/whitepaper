\subsection{Gravitational Waves}
  \label{subsec:gravitational-waves}

  Time-dependent variations in the distribution of localized excitations, such as
  accelerating masses or mergers of compact configurations, induce collective
  fluctuations in the relaxation dynamics of the $\chi$ field.
  These fluctuations propagate as changes in the local relaxation regime and are
  transmitted at the maximal relaxation speed $c$.

  When described using an effective spacetime language, such propagating
  modulations correspond to gravitational waves.
  Unlike electromagnetic radiation, which consists of excitations propagating on
  top of a background field, gravitational waves in Cosmochrony represent collective
  variations of the $\chi$ field itself, reflecting the redistribution of
  relaxation constraints across extended regions.

  In this sense, gravitational waves do not introduce additional fundamental degrees
  of freedom, but arise as dynamical responses of the $\chi$ field to time-dependent
  reconfigurations of matter.

  It should be emphasized that such gravitational wave descriptions remain valid only
  within regimes where the projection of the $\chi$ field onto an effective spacetime
  is well-defined.
  In strong-gravity environments approaching the deprojection threshold discussed in
  Section~\ref{subsec:black-hole-deprojection-cycle}, these collective modulations are
  expected to become increasingly attenuated or ill-defined from the spacetime
  perspective.
