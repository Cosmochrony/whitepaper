\subsection{Gravitational Waves}
  \label{subsec:gravitational-waves}

  Time-dependent variations in the distribution of localized projected configurations,
  such as accelerating masses or mergers of compact systems, induce collective
  modulations in admissible relaxation ordering.
  These modulations propagate as changes in the effective ordering regime and are
  transmitted at the maximal admissible ordering speed $c$.

  When expressed in an effective spacetime language, such propagating modulations are
  described as gravitational waves.
  Unlike electromagnetic radiation, which corresponds to propagating particle-like
  projected excitations, gravitational waves represent collective variations in the
  admissible ordering and correlation structure of projected descriptions themselves.

  In this sense, gravitational waves do not introduce additional fundamental degrees of
  freedom.
  They arise as macroscopic, collective responses of admissible projected descriptions
  to time-dependent reconfigurations of localized constraints, rather than as
  excitations of an underlying physical field.

  It should be emphasized that gravitational-wave descriptions are valid only within
  regimes where the projection onto an effective spacetime remains well defined.
  In strong-gravity environments approaching the deprojection threshold discussed in
  Section~\ref{subsec:black-hole-deprojection-cycle}, these collective modulations are
  expected to become increasingly attenuated or lose a clear spacetime interpretation.
