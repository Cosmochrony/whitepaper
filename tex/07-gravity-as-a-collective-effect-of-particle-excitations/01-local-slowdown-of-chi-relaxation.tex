\subsection{Local Slowdown of $\chi$ Relaxation}
  \label{subsec:local-slowdown-of-chi-relaxation}

  In Cosmochrony, gravitation does not arise from a fundamental interaction but from
  the collective influence of particle-like excitations on the relaxation dynamics
  of the $\chi$ field.
  As established in Sec.~\ref{sec:particles-as-localized-excitations-of-the-chi-field},
  localized excitations locally resist the relaxation of $\chi$.

  When many such excitations are present, their combined influence leads, in the
  weak-coupling regime, to an effective macroscopic reduction of the relaxation
  rate.
  In an effective spacetime parametrization, this may be written as
  \begin{equation}
    \mathcal{D}_{\mathrm{eff}}\chi \simeq c \left( 1 - \alpha\, \rho \right),
  \end{equation}
  where $\rho$ denotes the effective density of localized excitations and $\alpha$
  encodes their average coupling to the $\chi$ relaxation flow.
  This expression should be understood as a first-order approximation valid when
  local constraints are sufficiently dilute.

  The coupling parameter $\alpha$ is not fundamental but emerges from the
  interaction between $\chi$ and stable excitations.
  In the weak-field limit, its scaling can be related to the observed gravitational
  constant by dimensional consistency, leading to $\alpha \propto G/c^2$ when
  expressed in terms of effective inertial mass densities.
  Within this approximation, the reduction of the relaxation rate admits an
  interpretation in terms of a Newtonian-like gravitational potential.

  Physically, this collective slowdown of $\chi$ relaxation manifests as
  gravitational time dilation in effective geometric descriptions.
