\subsection{Emergent Curvature}
  \label{subsec:emergent-curvature}

  Spatial variations in admissible relaxation ordering, combined with the collective
  modulation of effective coupling strength, lead to non-uniform correlation patterns
  within projected descriptions.
  When a smooth geometric parametrization becomes applicable, these non-uniformities
  are compactly summarized by gradients of an emergent metric structure.

  In Cosmochrony, spacetime curvature is therefore not a primitive geometric property
  nor the manifestation of an independent dynamical field.
  It is a descriptive construct encoding how localized projected configurations
  collectively modulate admissible ordering and correlation structure across extended
  regions.
  The metric does not act as a causal agent; it functions as a macroscopic summary of
  constrained relational organization within admissible projected descriptions.

  Within effective geometric regimes, this emergent curvature reproduces the
  phenomenology traditionally attributed to curved spacetime in general relativity,
  including gravitational time dilation, geodesic deviation, and lensing effects.
  These phenomena arise here not from a fundamental spacetime geometry, but from the
  spatial variation of admissible relaxation ordering and correlation efficiency.

  Crucially, this interpretation remains fully compatible with the pre-geometric and
  relational foundations of the framework.
  Geometry appears only as an effective and operational language, valid in regimes
  where projected $\chi$ configurations admit a smooth and slowly varying
  representation.

\paragraph{Einstein’s Equations as a Structural Equilibrium Principle}
  \label{subsec:einstein-equilibrium-principle}

  Within the Cosmochrony framework, Einstein’s field equations retain their full
  conceptual and physical legitimacy.
  They are not reinterpreted as approximations to a deeper gravitational dynamics,
  but as an exact and universal description of spacetime structure \emph{whenever a
geometric description is applicable}.

  This perspective is fully aligned with Einstein’s own methodological stance.
  General relativity does not describe the microscopic constitution of spacetime, but
  the necessary relations between geometry and physical content once spacetime itself
  is admitted as a meaningful concept.
  In this sense, Einstein’s equations are already formulated at the correct
  descriptive level: that of emergent geometry.

  In Cosmochrony, spacetime geometry is not assumed a priori, but arises from the
  admissible projection of the pre-geometric relational substrate $\chi$.
  When projected $\chi$ configurations admit a smooth, locally injective geometric
  description, their collective structural constraints can be summarized by an
  effective metric $g_{\mu\nu}$.
  In this regime, Einstein’s equations emerge \emph{necessarily} as the unique
  consistency condition relating curvature to the effective distribution of
  relaxation-resistant configurations.

  From this standpoint, the Einstein tensor does not encode a dynamical law acting on
  spacetime, but a geometric identity constraining admissible macroscopic descriptions.
  Likewise, the stress--energy tensor summarizes how localized projected
  configurations resist admissible relaxation ordering.
  The Einstein equations therefore express a balance condition between geometry and
  physical structure, not a force law.

  This interpretation does not weaken general relativity.
  On the contrary, it explains its extraordinary universality.
  As long as a smooth spacetime description exists and relaxation ordering is
  monotonic, bounded, and weakly inhomogeneous, the same geometric relations must hold,
  independently of the microscopic nature of the underlying substrate.
  General relativity thus plays a role analogous to thermodynamics: exact within its
  domain, silent outside it, and remarkably insensitive to deeper ontological details.

  Importantly, this view also clarifies the limits of applicability of Einstein’s
  equations without attributing any failure to the theory itself.
  When projection ceases to be locally injective—near deprojection boundaries or
  strong structural constraints—spacetime geometry itself loses operational meaning.
  In such regimes, Einstein’s equations do not break down; they simply no longer
  apply, because the concept of spacetime has not yet emerged.

  In this sense, Cosmochrony does not go beyond Einstein by correcting general
  relativity.
  It goes beneath it, by explaining why Einstein’s equations are inevitable wherever
  spacetime exists at all.
