\subsection{Unified Origin of Gravitational and Electromagnetic Effects}
  \label{subsec:unified-origin-of-gravitational-and-electromagnetic-effects}

  Within the Cosmochrony framework, gravitational and electromagnetic phenomena do
  not originate from distinct fundamental entities, but arise as complementary
  effective manifestations of the same underlying $\chi$ dynamics.
  At the fundamental level, only the scalar field $\chi$ and its relaxation
  properties are postulated.

  Gravitational effects correspond to sustained, quasi-static constraints on the
  relaxation of $\chi$, induced by persistent structural variations associated with
  localized excitations.
  When described in effective geometric terms, these constraints manifest as time
  dilation, attraction, and spacetime curvature.

  Electromagnetic phenomena, by contrast, arise from dynamic and phase-dependent
  modulations of $\chi$.
  These modulations admit an effective description in terms of propagating,
  oscillatory fields with both attractive and repulsive interactions, consistent
  with the observed behavior of electromagnetic radiation and forces.

  In this sense, gravity and electromagnetism differ not by their fundamental
  origin, but by the temporal character and organization of the $\chi$ modulations
  they involve: quasi-static and cumulative for gravitation, dynamic and
  oscillatory for electromagnetism.
  The familiar distinction between the two interactions thus emerges at the level
  of effective descriptions rather than from fundamentally separate fields.

  At the level of effective descriptions, these dynamic modulations of $\chi$
  admit a formulation equivalent to classical electrodynamics.
  An explicit derivation of the corresponding Maxwell-like equations from the
  underlying $\chi$ dynamics is provided in Appendix~\ref{app:emergent-electrodynamics}.
