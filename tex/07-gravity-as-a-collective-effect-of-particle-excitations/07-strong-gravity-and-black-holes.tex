Before turning to strong-gravity regimes, it is useful to summarize the different
projective regimes discussed in this section.

\begin{figure}[h]
  \centering
  \begin{tabular}{c c c c c}
    \textbf{Projectable} &
    $\rightarrow$ &
    \textbf{Horizon} &
    $\rightarrow$ &
    \textbf{Non-projectable} \\[6pt]

    Gravitational &
    &
    Boundary of &
    &
    Deprojected \\
    Waves &
    &
    Projection &
    &
    Regime \\[6pt]

    Effective &
    &
    $\Pi$ non-injective &
    &
    No spacetime \\
    Modulations &
    &
    &
    &
    Representation
  \end{tabular}
  \caption{Conceptual regimes of projection in Cosmochrony.
  Gravitational waves correspond to fully projectable collective modulations of
  admissible descriptions, while black holes mark the boundary beyond which spacetime
  representations cease to be injective.
  Black hole evaporation reflects the gradual restoration of projectability,
    without any loss of information at the fundamental relational level,
    but only a loss and recovery of spacetime representability.}
  \label{fig:chi_projection_regimes}
\end{figure}

This schematic overview highlights how gravitational waves, horizons, and black hole
evaporation correspond to distinct regimes of projectability of the same underlying
relational structure.

\subsection{Strong Gravity and Black Holes}
  \label{subsec:strong-gravity-and-black-holes}

  In regions where the density of localized projected configurations becomes
  sufficiently high, admissible relaxation ordering becomes strongly constrained.
  In effective spacetime descriptions, this corresponds to a regime in which the local
  accumulation of effective time is strongly suppressed relative to distant observers,
  defining an effective horizon.

  Within Cosmochrony, such regions are interpreted as black holes.
  Rather than being characterized by a fundamental spacetime singularity, black holes
  correspond to domains where physical processes become asymptotically inaccessible
  from the exterior due to the loss of injectivity of spacetime projection.
  This naturally accounts for extreme time dilation effects without requiring divergent
  curvature invariants.

  These regions therefore mark not a terminal endpoint of physical description, but a
  transition toward a non-projectable regime of the underlying relational structure.

  \subsubsection*{Gravitational and Temporal Shadows}

    In the strong-gravity regime, the increasing concentration of localized projected
    configurations induces severe constraints on admissible ordering.
    As a result, the effective progression of time within the region slows
    asymptotically with respect to external descriptions.

    This behavior reproduces the phenomenon commonly referred to as a
    \emph{gravitational shadow}.
    In general relativity, such shadows arise from the absence of escaping null
    geodesics within a characteristic angular region.
    In Cosmochrony, an equivalent observational signature emerges because effective
    propagating descriptions, including radiation-like modes, no longer admit a faithful
    spacetime representation once projectability is lost.
    External observers therefore perceive a dark angular region corresponding to the
    projection of a non-projectable domain.

    Beyond this optical effect, the framework predicts a deeper phenomenon, which may be
    termed a \emph{temporal shadow}.
    As projectability is progressively lost, internal processes become indefinitely
    delayed in effective spacetime descriptions.
    From the external perspective, physical evolution appears frozen, providing a
    natural interpretation of horizon-induced time dilation.

    In this view, the observed gravitational shadow corresponds to the visible
    manifestation of an underlying temporal shadow.
    Both effects arise from the same loss of projective representability and do not
    require a fundamental spacetime singularity.

  \subsubsection*{Absence of Physical Singularities}

    In classical general relativity, black holes are associated with spacetime
    singularities characterized by divergent curvature and energy density.
    In Cosmochrony, such singularities are interpreted as artifacts of extending
    effective spacetime descriptions beyond their domain of validity.

    Because admissible ordering is bounded, configurations corresponding to infinite
    curvature or density cannot be physically realized.
    Apparent singularities therefore signal the breakdown of spacetime representability
    rather than genuine divergences of the underlying relational structure.

    \paragraph{Structural bound and notation.}
      To avoid confusion between fundamental and emergent levels, we distinguish the
      dimensionless structural bound $c_{\chi}$, defined at the level of the pre-geometric
      relational substrate, from its emergent spacetime manifestation $c$, interpreted as
      the maximal signal propagation speed.
      While $c$ may exhibit effective regime-dependent variations, the bound $c_{\chi}$ is
      invariant.

  \subsubsection*{Black Holes, Deprojection, and Vacuum Reprojection}
    \label{subsec:black-hole-deprojection-cycle}

    Within Cosmochrony, the absence of physical singularities does not imply that black
    holes are dynamically inert.
    Rather, they correspond to regimes in which the projection of relational information
    onto spacetime ceases to be injective.

    The emergence of an effective spacetime description relies on a projection map
    \[
      \Pi : \mathcal{C}_{\mathrm{rel}} \longrightarrow \mathcal{M},
    \]
    from the space of relational configurations to an effective spacetime manifold.
    In weak- and moderate-field regimes, this map is locally injective, ensuring a faithful
    geometric encoding.

    In strong-gravity regimes, this injectivity breaks down.
    Multiple inequivalent relational configurations correspond to the same effective
    spacetime event, signaling a loss of representability without any loss of information.
    We refer to this loss of injectivity as \emph{deprojection}.

    Deprojection does not correspond to transport across a spatial boundary nor to a
    temporal reversal.
    Instead, relational information ceases to be expressible in spatiotemporal form and
    remains encoded structurally.

    Importantly, deprojected information is not destroyed.
    Because the underlying relational configuration remains globally defined, information
    is in principle reprojectable once projectability is restored.
    Reprojection occurs discretely and manifests in effective spacetime descriptions as
    radiation-like excitations or particle--antiparticle pairs.

    The deprojection regime associated with black hole horizons does not imply the
    absence of dynamical processes.
    While smooth metric evolution ceases, the $\chi$ substrate remains structurally active.
    In particular, reprojection may occur intermittently when local configurations
    reach the threshold required for effective visibility.
    In the following section, this process is formalized through an explicit
    reprojection flux equation governing black hole evaporation.

    \paragraph{Information conservation and unitarity.}
      Deprojection does not correspond to information loss.
      It marks the loss of spacetime encoding, not the destruction of correlations.
      At the fundamental relational level, global information is preserved.
      Apparent non-unitarity arises only within projected spacetime descriptions and
      reflects their limited domain of applicability.

      This reprojection mechanism is analogous to vacuum fluctuations
      (Section~\ref{subsec:vacuum-fluctuations-and-the-casimir-effect}), where structural information is temporarily
      non-projectable before re-emerging as radiation.
