\subsection{Intrinsic Structural Indeterminacy and Projective Variability}
  \label{subsec:intrinsic-structural-indeterminacy}

  A perfectly deterministic, fully closed, and maximally symmetric relational substrate
  would remain physically inert: no configuration would be distinguished, no effective
  ordering could arise, and no emergent structure would be selected.
  For this reason, Cosmochrony postulates the existence of an
  \emph{intrinsic structural indeterminacy} at the level of the pre-geometric $\chi$
  substrate.

  This indeterminacy does not correspond to randomness, stochastic dynamics, temporal
  fluctuations, or hidden variables evolving in time.
  Instead, it reflects a fundamental \emph{non-completeness of structural determination}:
  configurations of $\chi$ are not exhaustively specified by a finite, closed, or
  maximally constraining set of relational conditions.
  Intrinsic indeterminacy is therefore ontological rather than dynamical, and should be
  understood negatively, as the absence of perfect structural closure rather than as the
  presence of random motion or noise.

  Importantly, this indeterminacy does not introduce any form of temporal evolution,
  causal process, or exploration of alternatives at the level of $\chi$.
  The substrate itself does not fluctuate, evolve, or sample a space of possibilities
  in time.
  Rather, intrinsic indeterminacy prevents $\chi$ from collapsing into a perfectly
  symmetric and physically sterile configuration, thereby allowing relational
  distinctions to acquire physical relevance once projection into an effective
  description becomes applicable.

  For clarity, the term \emph{fluctuations} is used in this framework only in a strictly
  metaphorical sense when referring to the $\chi$ substrate.
  No underlying agitation, stochastic process, or microscopic dynamics is implied.
  Intrinsic indeterminacy should instead be understood as a structural openness of the
  relational substrate, analogous to an underdetermination of global relational structure
  rather than to random temporal behavior.

  Observable variability and probabilistic behavior arise only at the level of
  \emph{projected} descriptions.
  Because the projection from $\chi$ to effective spacetime observables is generically
  non-injective, a single underlying configuration of $\chi$ may correspond to multiple
  physically admissible effective realizations.
  This non-uniqueness of projection gives rise, at the effective level, to phenomena
  commonly described as fluctuations, probabilistic outcomes, or quantum uncertainty.

  In this sense, randomness is not fundamental but \emph{projective}.
  It reflects the multiplicity of effective descriptions compatible with a given
  pre-geometric relational structure, rather than any stochasticity intrinsic to
  $\chi$ itself.
  The apparent temporal character of fluctuations is therefore an emergent feature of
  projected spacetime descriptions, not a property of the underlying substrate.

  Intrinsic structural indeterminacy thus plays a dual conceptual role.
  At the ontological level, it prevents perfect structural symmetry and ensures that
  $\chi$ remains physically generative.
  At the effective level, when combined with non-injective projection, it provides the
  structural origin of observable variability, probabilistic outcomes, and quantum
  indeterminacy, without introducing fundamental randomness, hidden dynamics, or
  temporal evolution into the substrate itself.
