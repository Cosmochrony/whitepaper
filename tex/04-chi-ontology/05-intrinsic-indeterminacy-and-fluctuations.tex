\subsection{Intrinsic Structural Indeterminacy and Projective Variability}
  \label{subsec:intrinsic-structural-indeterminacy}

  A perfectly deterministic and fully symmetric relational substrate would remain
  physically inert: no configuration would be privileged, and no effective ordering
  could emerge.
  For this reason, Cosmochrony postulates the existence of \emph{intrinsic structural
indeterminacy} at the level of the pre-geometric $\chi$ substrate.

  This indeterminacy does not correspond to randomness, stochastic dynamics, or
  temporal fluctuations.
  It reflects instead a fundamental \emph{non-completeness of structural determination}:
  $\chi$ configurations are not exhaustively specifiable by a finite or closed set of
  relational constraints.
  In this sense, intrinsic indeterminacy is ontological rather than dynamical, and should
  be understood negatively, as the absence of perfect structural closure.

  Importantly, this indeterminacy does not introduce any form of temporal evolution
  or causal process at the level of $\chi$.
  The substrate itself does not fluctuate, evolve, or explore a space of possibilities
  in time.
  Rather, intrinsic indeterminacy prevents $\chi$ from collapsing into a perfectly
  symmetric and physically sterile configuration, thereby allowing relational distinctions
  to acquire physical relevance once projection into an effective description occurs.

  For clarity, the term \emph{fluctuations} is used in this framework only in a
  \emph{metaphorical} sense when referring to the $\chi$ substrate.
  No underlying agitation, noise, or stochastic process is implied.
  Intrinsic indeterminacy should instead be understood as a structural openness of the
  relational substrate, analogous to an underdetermination of global structure rather
  than to random motion.

  Observable variability and probabilistic behavior arise only at the level of
  \emph{projected} descriptions.
  Because the projection from $\chi$ to an effective spacetime representation is
  non-injective, a single underlying configuration of $\chi$ may admit multiple
  physically admissible projected realizations.
  This non-uniqueness of projection gives rise, at the effective level, to phenomena
  commonly described as fluctuations, probabilistic outcomes, or quantum uncertainty.

  In this sense, randomness is not fundamental but \emph{projective}:
  it reflects the multiplicity of effective descriptions compatible with a given
  pre-geometric structure, rather than any stochasticity intrinsic to $\chi$ itself.
  The apparent temporal character of fluctuations is therefore an emergent feature of
  projected descriptions, not a property of the underlying substrate.

  Intrinsic structural indeterminacy thus plays a dual conceptual role.
  At the ontological level, it prevents perfect structural symmetry and ensures that
  $\chi$ remains physically generative.
  At the effective level, when combined with non-injective projection, it provides the
  structural origin of observable variability, probabilistic outcomes, and quantum
  indeterminacy, without introducing fundamental randomness or temporal dynamics
  into the substrate itself.
