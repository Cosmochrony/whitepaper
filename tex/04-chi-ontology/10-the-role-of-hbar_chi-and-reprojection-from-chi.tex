\subsection{The Role of $\hbar_{\chi}$ and Reprojection from $\chi$}
  \label{subsec:the-role-of-hbar_chi-and-reprojection-from-chi}

  In the Cosmochrony framework, the parameter $\hbar_{\chi}$ is not identified with the
  quantum constant $\hbar$ as introduced in conventional quantum mechanics.
  Instead, it characterizes a fundamental structural scale of the pre-geometric
  $\chi$ substrate, determined by the intrinsic relational density and constraint
  structure of admissible configurations.

  Importantly, $\hbar_{\chi}$ does not represent a quantum of action evolving in time.
  At the level of $\chi$, no temporal evolution, Hilbert space, or dynamical operator
  algebra is defined.
  Rather, $\hbar_{\chi}$ specifies a minimal quantum of \emph{reprojection}:
  intrinsic structural information encoded in $\chi$ can enter an effective spacetime
  description only in discrete units set by this scale.

  Intrinsic structural indeterminacy of $\chi$ does not give rise to continuous emergence.
  Instead, reprojection into spacetime occurs in discrete events, each corresponding to
  a minimal admissible transfer of relational structure into projected observables.
  These reprojection quanta define the graininess of effective quantum phenomena.

  As spacetime structure stabilizes and projection becomes locally persistent,
  reprojection events become increasingly localized.
  Phenomenologically, this manifests as vacuum fluctuations or quantum noise within
  otherwise stable spacetime regions, without implying any fundamental stochastic
  dynamics at the level of $\chi$ itself.

\subsection{The Origin of Planck's Constant: $\hbar_\chi$ versus $\hbar_{\mathrm{eff}}$}
  \label{subsec:h-origin}

  A potential conceptual tension arises regarding the status of Planck's constant
  $\hbar$.
  Within Cosmochrony, $\hbar$ is not an independent postulate but a
  \textbf{spectral invariant} of the relaxation and projection process.

  \begin{itemize}
    \item \textbf{Fundamental substrate scale ($\hbar_\chi$):}
    At the level of the $\chi$ substrate, the quantum of reprojection is uniquely fixed
    by the intrinsic structural scales of the theory:
    \begin{equation}
      \hbar_\chi \;=\; \frac{c^3}{K_0\,\chi_c} .
    \end{equation}
    Here $K_0$ denotes the coupling density of relational constraints, $\chi_c$ the
    characteristic correlation scale of the substrate, and $c$ the effective
    projection of the invariant structural bound $c_\chi$.
    This relation expresses the \emph{spectral rigidity} of the substrate rather than
    a dynamical quantization rule.

    \item \textbf{Effective representation ($\hbar_{\mathrm{eff}}$):}
    In effective spacetime descriptions, the same structural scale appears as
    $\hbar_{\mathrm{eff}}$, functioning as the quantum of action in Hilbert-space
    formulations.
    The apparent universality of $\hbar$ across physical systems reflects the fact
    that $K_0$ and $\chi_c$ are global invariants of the current relaxation epoch.
  \end{itemize}

  The transition from $\hbar_\chi$ to $\hbar_{\mathrm{eff}}$ therefore does not involve
  a change in value, but a change in representation.
  A structural constraint defined on the pre-geometric substrate is re-expressed, after
  projection, as a universal dynamical constant governing effective quantum dynamics.

\subsection{Spectral Invariance of Planck's Constant and the Fine-Structure Constant}
  \label{subsec:quantum-invariants}

  The dependence of $\hbar_\chi$ on $K_0$ and $\chi_c$ does not signal arbitrariness, but
  is instead a requirement for spectral unification.
  By identifying
  \[
    \hbar_\chi \equiv \frac{c^3}{K_0\,\chi_c},
  \]
  the quantum of action is understood as a manifestation of the relational density and
  constraint structure of the $\chi$ substrate.

  Within this framework:
  \begin{itemize}
    \item \textbf{Resolution of the $\hbar$ tension:}
    The effective constant $\hbar_{\mathrm{eff}}$ is the projected expression of the
    fundamental reprojection scale $\hbar_\chi$.
    Its apparent constancy across spacetime follows from the near-homogeneity of the
    relaxation flux $\Phi_\chi$ in the current cosmological epoch.

    \item \textbf{Spectral origin of $\alpha$:}
    The fine-structure constant $\alpha$ emerges as a dimensionless spectral ratio,
    determined by the geometry of the projection fiber $\Pi$ and the spectral rigidity
    encoded by $K_0$:
    \begin{equation}
      \alpha \;=\;
      \mathcal{F}\!\left(
                     \frac{\text{geometry and topology of }\Pi}{K_0}
      \right),
    \end{equation}
    where $\mathcal{F}$ is a functional fixed by the structure of admissible projections.
  \end{itemize}

  This interpretation implies that if the structural parameters $K_0$ or $\chi_c$ were
  to vary—for example in primordial high-constraint regimes—both $\hbar$ and $\alpha$
  would scale accordingly.
  Such variation would not signal a breakdown of physical law, but a change in the
  projective representation of invariant relational constraints, preserving the
  internal structural coherence of the Cosmochrony framework.
