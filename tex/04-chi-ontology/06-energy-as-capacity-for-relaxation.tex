\subsection{Energy as Capacity for Relaxation}
  \label{subsec:energy-as-capacity-for-relaxation}

  Within the Cosmochrony framework, energy is reinterpreted as the effective capacity of
  projected $\chi$ configurations to relax unresolved structural constraints.
  Energy is not treated as a fundamental conserved substance, nor as an intrinsic
  attribute of the $\chi$ substrate itself.
  Rather, it is an emergent, descriptive quantity that characterizes the degree to which
  a given projected configuration retains the ability to undergo further relaxation.

  At the fundamental level, $\chi$ admits intrinsic structural indeterminacy but no
  notion of energy.
  Energy appears only after projection, as a measure of residual structural tension
  within effective descriptions compatible with spacetime interpretation.
  The progressive deployment of structural information corresponds to the conversion of
  intrinsic indeterminacy into relational differentiation.
  Energy thus quantifies the remaining potential for this conversion within projected
  configurations.

  From this perspective, processes commonly described as energy transfer, dissipation,
  or radiation correspond to the redistribution or release of relaxation capacity across
  projected degrees of freedom.
  Conservation laws associated with energy arise only at the effective level, as
  structural regularities of projected dynamics, and do not reflect a fundamental
  conservation principle at the level of $\chi$ itself.

  Importantly, without intrinsic structural indeterminacy, the notion of energy would be
  ill-defined.
  A perfectly closed and fully determined relational substrate would admit no unresolved
  tension and therefore no capacity for relaxation.
  Energy is thus inseparable from the structural openness of the $\chi$ substrate and
  from the non-injective character of its projection into effective physical
  descriptions.
