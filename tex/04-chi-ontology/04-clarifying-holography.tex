\subsection{Clarifying the Relation to Holographic Descriptions}
  \label{subsec:clarifying-holography}

  The preceding considerations naturally invite comparison with the holographic
  principle, as originally proposed by ’t~Hooft and Susskind, according to which the
  effective information content of a spacetime region scales with its boundary rather
  than with its volume.

  Cosmochrony is \emph{not} a holographic theory in the technical sense.
  It does not posit a lower-dimensional boundary description, nor does it assume a
  dual equivalence between bulk and boundary physics.
  No independent boundary degrees of freedom or dimensional reduction are introduced
  at the fundamental level.
  Any holographic-like behavior that may arise is therefore not postulated as a
  principle, but emerges indirectly from the projection of a non-factorizable,
  pre-geometric relational substrate.

  In particular, the limitation of physically accessible information within a given
  spacetime region reflects the degeneracy of underlying $\chi$ configurations that
  correspond to the same effective projection.
  This degeneracy is a direct consequence of the non-injective character of the
  projection from $\chi$ to emergent observables, and is intrinsic to the very process
  by which spacetime descriptions arise.
  It does not require the introduction of boundary-localized degrees of freedom, nor
  the assumption that information is fundamentally stored on lower-dimensional
  surfaces.

  From this perspective, scaling behaviors reminiscent of holography—such as the
  effective reduction of accessible degrees of freedom—are interpreted as emergent
  signatures of projection rather than as manifestations of an underlying holographic
  encoding principle.
  Holography thus appears in Cosmochrony as a derived phenomenological feature of
  spacetime emergence, not as a guiding postulate or a foundational equivalence.

  \paragraph{Relation to thermodynamic and entropic approaches (optional note).}
    The Cosmochrony framework also invites comparison with thermodynamic and entropic
    approaches to gravity, notably those developed by Jacobson and Verlinde, in which
    gravitational dynamics are interpreted as emerging from thermodynamic or informational
    principles.

    In Jacobson's approach, Einstein's equations are derived as an equation of state,
    assuming the proportionality of entropy to horizon area and the validity of local
    thermodynamic relations.
    In entropic gravity models, gravitational attraction is interpreted as an emergent
    entropic force arising from coarse-grained information constraints.

    Cosmochrony differs fundamentally in both ontological starting point and explanatory
    direction.
    It does not posit entropy, temperature, or information as primitive quantities.
    Instead, it introduces a pre-geometric relational substrate $\chi$ whose irreversible
    relaxation constitutes the primary physical process.
    Thermodynamic and entropic descriptions arise only at the effective level, as secondary
    languages applicable when projected $\chi$ configurations admit coarse-grained,
    statistical interpretations.

    From this perspective, the appearance of thermodynamic relations in gravitational
    contexts is not the origin of spacetime dynamics, but a consequence of projection and
    coarse-graining.
    Entropy-like quantities reflect the degeneracy of underlying $\chi$ configurations
    compatible with a given effective geometry, while temperature-like notions encode local
    rates of relaxation within projected descriptions.

    Accordingly, Cosmochrony does not reduce gravity to thermodynamics, nor does it interpret
    gravitation as an entropic force.
    Rather, it provides a deeper structural account of why gravitational dynamics admit a
    thermodynamic interpretation in appropriate regimes, without elevating entropy or
    information to a fundamental ontological status.
