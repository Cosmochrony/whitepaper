\subsection{Projection, Reality, and Ontological Asymmetry}
  \label{subsec:projection-reality-and-ontological-asymmetry}

  Within the Cosmochrony ontology, the emergence of spacetime is understood as a
  \emph{projection} from the $\chi$ substrate, rather than as a dual, equivalent, or bidirectional description.
  The projected universe is fully real at the level of physical experience and
  empirical description, but its reality is ontologically derivative: spacetime
  entities, effective fields, and dynamical laws do not possess ontological primacy~\cite{Rovelli2021}.

  This asymmetry is essential.
  While all physical descriptions depend on the projection of $\chi$, the converse is not true.
  The relational structure of $\chi$ exists independently of spacetime notions and does
  not admit a reformulation entirely in geometric, field-theoretic, or dynamical terms.
  Projection must therefore be understood as \emph{non-injective}: distinct structural
  features of $\chi$ may correspond to identical or physically indistinguishable effective configurations.

  \paragraph{Projection and non-circularity.}
    Because geometric notions arise only after projection, the construction of effective
    descriptions must not presuppose any metric, causal, or temporal structure defined at the effective level.
    In particular, coarse-graining procedures used to define $\chi_{\mathrm{eff}}$ must
    avoid relying implicitly on emergent geometric quantities.
    This requirement motivates the explicit separation between pre-geometric relational
    structures and geometry-dependent observables developed in Appendix~E, where
    combinatorial, spectral, and weighted distances are carefully distinguished.

  \paragraph{Projection and emergent time.}
    The parameter commonly interpreted as time in the effective description is not a
    fundamental attribute of the $\chi$ substrate.
    Temporal ordering and duration arise only after projection, as part of the emergent
    spacetime representation associated with $\chi_{\mathrm{eff}}$.
    No external, global, or fundamental time parameter is introduced at the level of
    $\chi$ itself.

    Accordingly, all averaging and coarse-graining operations involved in the definition
    of background descriptors (such as $\bar{\chi}$) and in the construction of
    $\chi_{\mathrm{eff}}$ are formulated relationally, without reference to an underlying
    temporal metric.
    Temporal concepts enter the framework only at the effective level, once a stable
    geometric regime has emerged through projection.

  \paragraph{Apparent fine-tuning.}
    In Cosmochrony, apparent fine-tuning does not reflect an improbable choice of initial
    conditions or a delicate adjustment of fundamental constants.
    Instead, it arises from the fact that only a restricted class of $\chi$
    configurations admits a coherent and stable physical projection.
    Most configurations of the $\chi$ substrate do not give rise to consistent spacetime
    descriptions, persistent physical structures, or well-defined effective laws.

    The apparent delicacy of physical parameters is therefore a selection effect imposed
    by projectability itself.
    Only those relational configurations compatible with a stable emergent geometry and
    sustained relaxation dynamics appear as physically realized universes.
    Fine-tuning is thus reinterpreted as a structural constraint on projection, rather
    than as a coincidence requiring external explanation.

  \paragraph{Absence of a multiverse.}
    Cosmochrony does not postulate a multiverse.
    While multiple configurations of the $\chi$ substrate may correspond to the same
    physical universe under non-injective projection, the framework provides no
    mechanism by which a single $\chi$ structure could sustain multiple independent and
    simultaneously realized physical projections.

    The universe is therefore unique at the level of physical reality, even though its
    underlying description in terms of $\chi$ may be non-unique.
    The absence of a multiverse is not an additional assumption, but a direct
    consequence of the ontological asymmetry and non-injective character of projection.

  \paragraph{Projection as constraint-based selection.}
    The projection $\Pi$ may be understood as a mechanism of constraint-based selection of
    admissible effective structures.
    Rather than realizing all relational possibilities encoded in $\chi$, projection filters
    them according to global consistency, stability, and projectability constraints.
    As a result, only a subset of structurally compatible configurations can appear as
    effective physical descriptions, while others are systematically excluded.
    Multiple effective realizations may therefore arise from a single underlying relational
    configuration, without any notion of execution, prescription, or teleology.

    Importantly, this selection does not constitute a dynamical process unfolding in time.
    It reflects a structural ordering of constraints that determines which projected
    descriptions can consistently exist, fully compatible with the pre-temporal and
    atemporal character of the $\chi$ substrate.

    To provide an intuitive image, one may compare this relation to the role played by DNA
    and morphogenesis in biological systems.
    In this analogy, $\chi$ plays a role analogous to DNA as a repository of structural
    possibilities, while projection plays a role analogous to morphogenesis, determining
    which forms are admissible and stable without encoding specific outcomes.
    This comparison is intended purely as an illustrative aid and does not imply any
    biological interpretation of the universe.

    \emph{Formal developments.}
    The geometric structure of projection, including its fiber-bundle formulation and the
    emergence of gauge interactions from projection symmetries, is developed in
    Section~\ref{sec:projection-gauge}.
