\subsection{Relational Ontology and Conceptual Lineage}
  \label{subsec:relational-ontology-and-conceptual-lineage}

  The relational character of the $\chi$ substrate bears a clear conceptual affinity
  with relational approaches in physics, most notably those advocated by
  Rovelli~\cite{Rovelli1996,Rovelli2004}.
  These approaches trace part of their philosophical lineage to Aristotelian
  relational ontology, in which properties are defined through relations rather than
  as intrinsic attributes~\cite{AristotleCategories,Shields2016}.

  Cosmochrony shares this rejection of intrinsic, observer-independent properties.
  However, it extends relationalism to a deeper ontological level.
  In relational quantum mechanics, relations are primary at the level of quantum
  states describing interactions between systems that are themselves taken as given
  within a spacetime framework.
  By contrast, Cosmochrony posits that the fundamental substrate $\chi$ is itself
  relational: there are no pre-existing entities between which relations are defined.

  In this sense, $\chi$ configurations are not relations \emph{between} fundamental
  objects, but relational structures that give rise to objects only upon projection
  into an effective spacetime description.
  Relationality is therefore not a feature of physical states \emph{within}
  spacetime, but an intrinsic property of the pre-geometric substrate from which
  spacetime, time ordering, and physical entities jointly emerge.

  This distinction is essential for understanding the ontological asymmetry between
  $\chi$ and its effective spacetime projections.
  While relational quantum mechanics reformulates quantum theory without altering the
  ontological status of spacetime itself, Cosmochrony relocates relationality at the
  level of the substrate that gives rise to spacetime and to the entities described
  within it.
  The non-injective character of the projection from $\chi$ to effective observables
  further implies that multiple effective spacetime descriptions may correspond to
  the same underlying relational structure.

  \paragraph{On the absence of fundamental values.}
    In Cosmochrony, relations are not defined between pre-existing fundamental values
    or field degrees of freedom.
    The relational structure of $\chi$ is ontologically primary and admits no intrinsic
    numerical, local, or field-like values.
    What appear, within effective spacetime descriptions, as scalar values, particles,
    or local degrees of freedom arise only as stable invariants of this relational
    structure under projection and coarse-graining.

  \paragraph{Relation to Rovelli's relational frameworks (optional note).}
    The Cosmochrony framework shares a broad conceptual affinity with relational approaches
    to physics, particularly those developed by Rovelli in the context of relational
    quantum mechanics and loop quantum gravity.
    In both cases, physical properties are not regarded as intrinsic attributes of isolated
    systems, but as relational features emerging from interactions or structural constraints.

    However, the point of departure differs in a crucial way.
    In relational quantum mechanics, relationality is introduced at the level of quantum
    states describing interactions between physical systems that are themselves assumed to
    exist within spacetime.
    Spacetime, while dynamically treated in general relativity, retains its ontological
    status as the arena in which relations are defined.

    By contrast, Cosmochrony relocates relationality to a deeper ontological level.
    The fundamental substrate $\chi$ is not a system embedded in spacetime, nor a field
    defined on a manifold, but a pre-geometric relational structure from which spacetime,
    time ordering, and physical entities jointly emerge through non-injective projection.
    Relations in Cosmochrony are therefore not relations \emph{between} objects, but
    relational structures that give rise to objects only at the effective level.

    In this sense, Cosmochrony should not be viewed as a reformulation or extension of
    Rovelli's relational frameworks, but as an ontologically prior approach that seeks to
    explain the origin of the relational degrees of freedom subsequently described within
    spacetime-based theories.
