\subsection{The $\chi$ Substrate as a Pre-Temporal Structural Plan}
  \label{subsec:the-chi-substrate-as-a-pre-temporal-structural-plan}

  In the Cosmochrony framework, the $\chi$ substrate is not interpreted as a physical
  field evolving within spacetime, but as a pre-temporal relational structure from
  which spacetime, matter, and effective physical laws emerge.
  It may be heuristically described as a \emph{structural plan} of the universe: not a
  dynamical history, but a complete relational organization encoding the set of
  physically admissible configurations and the constraints that relate them.

  This structural plan does not prescribe a unique evolution, nor does it encode a
  fixed sequence of events.
  Instead, it defines a constrained space of relational possibilities compatible with
  the intrinsic ordering structure of $\chi$.
  Temporal succession is therefore not fundamental, but emergent, corresponding to an
  oriented resolution of structural relations through irreversible relaxation once an
  effective spacetime description becomes applicable.

  Importantly, the notion of a structural plan should not be understood as introducing
  teleology, determinism, or a block-universe ontology.
  The ordering induced by $\chi$ constrains admissible effective descriptions, but does
  not select a unique realized history.
  Multiple effective histories may therefore correspond to the same underlying
  relational structure through the non-injective projection from $\chi$ to observable
  descriptions.

  \paragraph{Infra-physical status of $\chi$.}
    The $\chi$ substrate is not described by physics in the conventional sense of a
    theory of dynamical fields evolving in spacetime.
    Rather, it belongs to an infra-physical relational framework that specifies the
    structural conditions under which physical observables, spacetime geometry, and
    effective dynamical laws become meaningful.
    Physical theories formulated in spacetime thus arise as effective descriptions of
    particular projectable regimes of this deeper relational structure.
