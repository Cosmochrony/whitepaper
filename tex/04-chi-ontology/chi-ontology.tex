\section{Ontological Interpretation of the $\chi$ Field}
  \label{sec:chi_ontology}

  Throughout this section, we explicitly distinguish the invariant structural bound $c_{\chi}$,
  defined at the level of the pre-temporal $\chi$ substrate, from its emergent spacetime manifestation $c$.
  The latter appears only once spacetime notions such as distance, duration, and causal propagation become meaningful.

  \subsection{The $\chi$ Field as a Pre-Temporal Structural Plan}

    In the Cosmochrony framework, the $\chi$
    field is not interpreted as a physical field evolving within spacetime, but as a pre-temporal structural
    substrate from which spacetime, matter, and physical laws emerge.
    It may be heuristically described as a ``plan'' of the universe: a complete structure encoding the set of
    physically admissible configurations and their relations.

    Importantly, this plan is not a predetermined scenario.
    It does not specify a unique history nor a fixed sequence of events.
    Rather, it defines a constrained space of possibilities within which relational ordering may arise.
    Temporal succession is therefore not fundamental but emergent, corresponding to an oriented resolution of
    structural relations within $\chi$.

  \subsection{Intrinsic Indeterminacy and Fluctuations}
    \label{subsec:intrinsic-indeterminacy-and-fluctuations}

    A perfectly deterministic and fully symmetric structural plan would remain dynamically sterile: no configuration
    would be privileged, and no physically consequential ordering could emerge.
    For this reason, Cosmochrony postulates the existence of intrinsic indeterminacy within the $\chi$ structure.

    These fluctuations are not temporal events, nor stochastic processes unfolding in time.
    They represent a fundamental absence of complete determination at the structural level.
    Their role is not to generate dynamics, but to break perfect structural indifference, thereby allowing certain
    transitions to become physically consequential.

    In this sense, fluctuations constitute a condition of possibility for emergence, rather than a dynamical cause.

  \subsection{Energy as Capacity for Relaxation}

    Within this framework, energy is reinterpreted as the capacity of the $\chi$ structure to relax its constraints.
    Energy does not correspond to a substance or a conserved entity at the fundamental level, but to a measure of
    unresolved structural tension.

    The deployment of $\chi$
    corresponds to the progressive conversion of structural indeterminacy into relational information.
    Energy thus quantifies the potential for this conversion.
    Without intrinsic indeterminacy, energy would be undefined, as no relaxation could occur.

  \subsection{Mass as Frozen Information}

    Localized, stable configurations of the $\chi$
    field—interpreted as particle-like excitations—correspond to regions where relaxation is strongly inhibited.
    These configurations trap a fixed amount of unresolved structural information.

    In this interpretation, mass represents frozen energy: information that has lost its capacity to participate
    freely in further relaxation.
    At the level of emergent spacetime physics, this relation is expressed by the standard mass--energy equivalence,
    \[
      E_{\text{phys}} = m_{\text{phys}} c^2,
    \]
    where $c$ denotes the emergent spacetime limiting speed.
    This identity reflects, at the phenomenological level, a more fundamental structural relation
    governing the confinement and release of information within the $\chi$ substrate.

  \subsection{The Role of the Universal Bound $c_{\chi}$}

    The constant $c_{\chi}$ plays a central structural role in Cosmochrony.
    Rather than setting a signal propagation speed, it defines an absolute bound on the degree to which
    information can be structurally constrained within the $\chi$ substrate.

    In this sense, $c_{\chi}$
    sets the maximal ``knotting'' or confinement of structural information compatible with physical coherence.
    Beyond this bound, further constraint becomes impossible and relaxation is unavoidable.
    The emergent spacetime notions of inertial resistance, causal structure, and mass--energy equivalence
    are understood as projections of this invariant structural limit.

  \subsection{The Role of $\hbar_{\chi}$ and Reprojection from $\chi$}

    In Cosmochrony, the parameter \(\hbar_{\chi}\) is \textbf{not derived from \(\hbar\)}
    but emerges from the fundamental scales \(K_0\), \(\chi_c\), and \(c\).
    Its numerical coincidence with \(\hbar\) in quantum regimes reflects the
        \textbf{universality of action quantization} across physical theories.\footnote{
  Unlike in quantum gravity, where \(\hbar\) and \(G\) are fundamental, here \(\hbar_{\chi}\) is derived from \(\chi\)
  -field parameters alone. The standard \(\hbar\) is recovered only in regimes where
  \(\ell_{\text{spacetime}} \sim \chi_c\).
}
    It is derived purely from the fundamental scales $K_0$, $\chi_c$, and $c$,
    without reference to spacetime or quantum constants.
    It does not represent a quantum of action evolving in time, but a fundamental
    quantum of reprojection.

    Intrinsic fluctuations of $\chi$ do not give rise to continuous emergence.
    Rather, any reprojection of structural information into emergent spacetime
    occurs in discrete units set by $\hbar_{\chi}$.
    Transient excitations commonly interpreted as vacuum particles thus correspond
    to minimal reprojections allowed by this bound.

    In the early universe, where spacetime structure was weakly stabilized,
    such reprojections were diffuse and frequent.
    As the universe relaxed and large-scale structures formed, reprojection became
    progressively localized, manifesting as vacuum fluctuations in otherwise stable
    regions of spacetime.
