\section{Ontological Interpretation of the $\chi$ Substrate}
  \label{sec:chi_ontology}

  Throughout this section, we explicitly distinguish the invariant structural bound
  $c_{\chi}$, defined at the level of the pre-temporal $\chi$ substrate, from its
  emergent spacetime manifestation $c$~\cite{Rovelli2018}.
  The latter appears only once spacetime notions such as distance, duration, and
  causal propagation become meaningful within effective descriptions.

  \subsection{The $\chi$ Substrate as a Pre-Temporal Structural Plan}
  \label{subsec:the-chi-substrate-as-a-pre-temporal-structural-plan}

  In the Cosmochrony framework, the $\chi$ substrate is not interpreted as a physical
  field evolving within spacetime, but as a pre-temporal relational structure from
  which spacetime, matter, and effective physical laws emerge.
  It may be heuristically described as a \emph{structural plan} of the universe: not a
  dynamical history, but a complete relational organization encoding the set of
  physically admissible configurations and the constraints that relate them.

  This structural plan does not prescribe a unique evolution, nor does it encode a
  fixed sequence of events.
  Instead, it defines a constrained space of relational possibilities compatible with
  the intrinsic ordering structure of $\chi$.
  Temporal succession is therefore not fundamental, but emergent, corresponding to an
  oriented resolution of structural relations through irreversible relaxation once an
  effective spacetime description becomes applicable.

  Importantly, the notion of a structural plan should not be understood as introducing
  teleology, determinism, or a block-universe ontology.
  The ordering induced by $\chi$ constrains admissible effective descriptions, but does
  not select a unique realized history.
  Multiple effective histories may therefore correspond to the same underlying
  relational structure through the non-injective projection from $\chi$ to observable
  descriptions.

  \paragraph{Infra-physical status of $\chi$.}
    The $\chi$ substrate is not described by physics in the conventional sense of a
    theory of dynamical fields evolving in spacetime.
    Rather, it belongs to an infra-physical relational framework that specifies the
    structural conditions under which physical observables, spacetime geometry, and
    effective dynamical laws become meaningful.
    Physical theories formulated in spacetime thus arise as effective descriptions of
    particular projectable regimes of this deeper relational structure.

  \subsection{Relational Ontology and Conceptual Lineage}
  \label{subsec:relational-ontology-and-conceptual-lineage}

  The relational character of the $\chi$ substrate bears a clear conceptual affinity
  with relational approaches in physics, most notably those advocated by
  Rovelli~\cite{Rovelli1996,Rovelli2004}.
  These approaches trace part of their philosophical lineage to Aristotelian
  relational ontology, in which properties are defined through relations rather than
  as intrinsic attributes~\cite{AristotleCategories,Shields2016}.

  Cosmochrony shares this rejection of intrinsic, observer-independent properties.
  However, it extends relationalism to a deeper ontological level.
  In relational quantum mechanics, relations are primary at the level of quantum
  states describing interactions between systems that are themselves taken as given
  within a spacetime framework.
  By contrast, Cosmochrony posits that the fundamental substrate $\chi$ is itself
  relational: there are no pre-existing entities between which relations are defined.

  In this sense, $\chi$ configurations are not relations \emph{between} fundamental
  objects, but relational structures that give rise to objects only upon projection
  into an effective spacetime description.
  Relationality is therefore not a feature of physical states \emph{within}
  spacetime, but an intrinsic property of the pre-geometric substrate from which
  spacetime, time ordering, and physical entities jointly emerge.

  This distinction is essential for understanding the ontological asymmetry between
  $\chi$ and its effective spacetime projections.
  While relational quantum mechanics reformulates quantum theory without altering the
  ontological status of spacetime itself, Cosmochrony relocates relationality at the
  level of the substrate that gives rise to spacetime and to the entities described
  within it.
  The non-injective character of the projection from $\chi$ to effective observables
  further implies that multiple effective spacetime descriptions may correspond to
  the same underlying relational structure.

  \paragraph{On the absence of fundamental values.}
    In Cosmochrony, relations are not defined between pre-existing fundamental values
    or field degrees of freedom.
    The relational structure of $\chi$ is ontologically primary and admits no intrinsic
    numerical, local, or field-like values.
    What appear, within effective spacetime descriptions, as scalar values, particles,
    or local degrees of freedom arise only as stable invariants of this relational
    structure under projection and coarse-graining.

  \paragraph{Relation to Rovelli's relational frameworks (optional note).}
    The Cosmochrony framework shares a broad conceptual affinity with relational approaches
    to physics, particularly those developed by Rovelli in the context of relational
    quantum mechanics and loop quantum gravity.
    In both cases, physical properties are not regarded as intrinsic attributes of isolated
    systems, but as relational features emerging from interactions or structural constraints.

    However, the point of departure differs in a crucial way.
    In relational quantum mechanics, relationality is introduced at the level of quantum
    states describing interactions between physical systems that are themselves assumed to
    exist within spacetime.
    Spacetime, while dynamically treated in general relativity, retains its ontological
    status as the arena in which relations are defined.

    By contrast, Cosmochrony relocates relationality to a deeper ontological level.
    The fundamental substrate $\chi$ is not a system embedded in spacetime, nor a field
    defined on a manifold, but a pre-geometric relational structure from which spacetime,
    time ordering, and physical entities jointly emerge through non-injective projection.
    Relations in Cosmochrony are therefore not relations \emph{between} objects, but
    relational structures that give rise to objects only at the effective level.

    In this sense, Cosmochrony should not be viewed as a reformulation or extension of
    Rovelli's relational frameworks, but as an ontologically prior approach that seeks to
    explain the origin of the relational degrees of freedom subsequently described within
    spacetime-based theories.

  \subsection{Projection, Reality, and Ontological Asymmetry}
  \label{subsec:projection-reality-and-ontological-asymmetry}

  Within the Cosmochrony ontology, the emergence of spacetime is understood as a
  \emph{projection} from the $\chi$ substrate, rather than as a dual, equivalent, or bidirectional description.
  The projected universe is fully real at the level of physical experience and
  empirical description, but its reality is ontologically derivative: spacetime
  entities, effective fields, and dynamical laws do not possess ontological primacy~\cite{Rovelli2021}.

  This asymmetry is essential.
  While all physical descriptions depend on the projection of $\chi$, the converse is not true.
  The relational structure of $\chi$ exists independently of spacetime notions and does
  not admit a reformulation entirely in geometric, field-theoretic, or dynamical terms.
  Projection must therefore be understood as \emph{non-injective}: distinct structural
  features of $\chi$ may correspond to identical or physically indistinguishable effective configurations.

  \paragraph{Projection and non-circularity.}
    Because geometric notions arise only after projection, the construction of effective
    descriptions must not presuppose any metric, causal, or temporal structure defined at the effective level.
    In particular, coarse-graining procedures used to define $\chi_{\mathrm{eff}}$ must
    avoid relying implicitly on emergent geometric quantities.
    This requirement motivates the explicit separation between pre-geometric relational
    structures and geometry-dependent observables developed in Appendix~E, where
    combinatorial, spectral, and weighted distances are carefully distinguished.

  \paragraph{Projection and emergent time.}
    The parameter commonly interpreted as time in the effective description is not a
    fundamental attribute of the $\chi$ substrate.
    Temporal ordering and duration arise only after projection, as part of the emergent
    spacetime representation associated with $\chi_{\mathrm{eff}}$.
    No external, global, or fundamental time parameter is introduced at the level of
    $\chi$ itself.

    Accordingly, all averaging and coarse-graining operations involved in the definition
    of background descriptors (such as $\bar{\chi}$) and in the construction of
    $\chi_{\mathrm{eff}}$ are formulated relationally, without reference to an underlying
    temporal metric.
    Temporal concepts enter the framework only at the effective level, once a stable
    geometric regime has emerged through projection.

  \paragraph{Apparent fine-tuning.}
    In Cosmochrony, apparent fine-tuning does not reflect an improbable choice of initial
    conditions or a delicate adjustment of fundamental constants.
    Instead, it arises from the fact that only a restricted class of $\chi$
    configurations admits a coherent and stable physical projection.
    Most configurations of the $\chi$ substrate do not give rise to consistent spacetime
    descriptions, persistent physical structures, or well-defined effective laws.

    The apparent delicacy of physical parameters is therefore a selection effect imposed
    by projectability itself.
    Only those relational configurations compatible with a stable emergent geometry and
    sustained relaxation dynamics appear as physically realized universes.
    Fine-tuning is thus reinterpreted as a structural constraint on projection, rather
    than as a coincidence requiring external explanation.

  \paragraph{Absence of a multiverse.}
    Cosmochrony does not postulate a multiverse.
    While multiple configurations of the $\chi$ substrate may correspond to the same
    physical universe under non-injective projection, the framework provides no
    mechanism by which a single $\chi$ structure could sustain multiple independent and
    simultaneously realized physical projections.

    The universe is therefore unique at the level of physical reality, even though its
    underlying description in terms of $\chi$ may be non-unique.
    The absence of a multiverse is not an additional assumption, but a direct
    consequence of the ontological asymmetry and non-injective character of projection.

  \paragraph{Projection as constraint-based selection.}
    The projection $\Pi$ may be understood as a mechanism of constraint-based selection of
    admissible effective structures.
    Rather than realizing all relational possibilities encoded in $\chi$, projection filters
    them according to global consistency, stability, and projectability constraints.
    As a result, only a subset of structurally compatible configurations can appear as
    effective physical descriptions, while others are systematically excluded.
    Multiple effective realizations may therefore arise from a single underlying relational
    configuration, without any notion of execution, prescription, or teleology.

    Importantly, this selection does not constitute a dynamical process unfolding in time.
    It reflects a structural ordering of constraints that determines which projected
    descriptions can consistently exist, fully compatible with the pre-temporal and
    atemporal character of the $\chi$ substrate.

    To provide an intuitive image, one may compare this relation to the role played by DNA
    and morphogenesis in biological systems.
    In this analogy, $\chi$ plays a role analogous to DNA as a repository of structural
    possibilities, while projection plays a role analogous to morphogenesis, determining
    which forms are admissible and stable without encoding specific outcomes.
    This comparison is intended purely as an illustrative aid and does not imply any
    biological interpretation of the universe.

    \emph{Formal developments.}
    The geometric structure of projection, including its fiber-bundle formulation and the
    emergence of gauge interactions from projection symmetries, is developed in
    Section~\ref{sec:projection-gauge}.

  \subsection{Clarifying the Relation to Holographic Descriptions}
  \label{subsec:clarifying-holography}

  The preceding considerations naturally invite comparison with the holographic
  principle, as originally proposed by ’t~Hooft and Susskind, according to which the
  effective information content of a spacetime region scales with its boundary rather
  than with its volume.

  Cosmochrony is \emph{not} a holographic theory in the technical sense.
  It does not posit a lower-dimensional boundary description, nor does it assume a
  dual equivalence between bulk and boundary physics.
  No independent boundary degrees of freedom or dimensional reduction are introduced
  at the fundamental level.
  Any holographic-like behavior that may arise is therefore not postulated as a
  principle, but emerges indirectly from the projection of a non-factorizable,
  pre-geometric relational substrate.

  In particular, the limitation of physically accessible information within a given
  spacetime region reflects the degeneracy of underlying $\chi$ configurations that
  correspond to the same effective projection.
  This degeneracy is a direct consequence of the non-injective character of the
  projection from $\chi$ to emergent observables, and is intrinsic to the very process
  by which spacetime descriptions arise.
  It does not require the introduction of boundary-localized degrees of freedom, nor
  the assumption that information is fundamentally stored on lower-dimensional
  surfaces.

  From this perspective, scaling behaviors reminiscent of holography—such as the
  effective reduction of accessible degrees of freedom—are interpreted as emergent
  signatures of projection rather than as manifestations of an underlying holographic
  encoding principle.
  Holography thus appears in Cosmochrony as a derived phenomenological feature of
  spacetime emergence, not as a guiding postulate or a foundational equivalence.

  \paragraph{Relation to thermodynamic and entropic approaches (optional note).}
    The Cosmochrony framework also invites comparison with thermodynamic and entropic
    approaches to gravity, notably those developed by Jacobson and Verlinde, in which
    gravitational dynamics are interpreted as emerging from thermodynamic or informational
    principles.

    In Jacobson's approach, Einstein's equations are derived as an equation of state,
    assuming the proportionality of entropy to horizon area and the validity of local
    thermodynamic relations.
    In entropic gravity models, gravitational attraction is interpreted as an emergent
    entropic force arising from coarse-grained information constraints.

    Cosmochrony differs fundamentally in both ontological starting point and explanatory
    direction.
    It does not posit entropy, temperature, or information as primitive quantities.
    Instead, it introduces a pre-geometric relational substrate $\chi$ whose irreversible
    relaxation constitutes the primary physical process.
    Thermodynamic and entropic descriptions arise only at the effective level, as secondary
    languages applicable when projected $\chi$ configurations admit coarse-grained,
    statistical interpretations.

    From this perspective, the appearance of thermodynamic relations in gravitational
    contexts is not the origin of spacetime dynamics, but a consequence of projection and
    coarse-graining.
    Entropy-like quantities reflect the degeneracy of underlying $\chi$ configurations
    compatible with a given effective geometry, while temperature-like notions encode local
    rates of relaxation within projected descriptions.

    Accordingly, Cosmochrony does not reduce gravity to thermodynamics, nor does it interpret
    gravitation as an entropic force.
    Rather, it provides a deeper structural account of why gravitational dynamics admit a
    thermodynamic interpretation in appropriate regimes, without elevating entropy or
    information to a fundamental ontological status.

  \subsection{Intrinsic Structural Indeterminacy and Projective Variability}
  \label{subsec:intrinsic-structural-indeterminacy}

  A perfectly deterministic, fully closed, and maximally symmetric relational substrate
  would remain physically inert: no configuration would be distinguished, no effective
  ordering could arise, and no emergent structure would be selected.
  For this reason, Cosmochrony postulates the existence of an
  \emph{intrinsic structural indeterminacy} at the level of the pre-geometric $\chi$
  substrate.

  This indeterminacy does not correspond to randomness, stochastic dynamics, temporal
  fluctuations, or hidden variables evolving in time.
  Instead, it reflects a fundamental \emph{non-completeness of structural determination}:
  configurations of $\chi$ are not exhaustively specified by a finite, closed, or
  maximally constraining set of relational conditions.
  Intrinsic indeterminacy is therefore ontological rather than dynamical, and should be
  understood negatively, as the absence of perfect structural closure rather than as the
  presence of random motion or noise.

  Importantly, this indeterminacy does not introduce any form of temporal evolution,
  causal process, or exploration of alternatives at the level of $\chi$.
  The substrate itself does not fluctuate, evolve, or sample a space of possibilities
  in time.
  Rather, intrinsic indeterminacy prevents $\chi$ from collapsing into a perfectly
  symmetric and physically sterile configuration, thereby allowing relational
  distinctions to acquire physical relevance once projection into an effective
  description becomes applicable.

  For clarity, the term \emph{fluctuations} is used in this framework only in a strictly
  metaphorical sense when referring to the $\chi$ substrate.
  No underlying agitation, stochastic process, or microscopic dynamics is implied.
  Intrinsic indeterminacy should instead be understood as a structural openness of the
  relational substrate, analogous to an underdetermination of global relational structure
  rather than to random temporal behavior.

  Observable variability and probabilistic behavior arise only at the level of
  \emph{projected} descriptions.
  Because the projection from $\chi$ to effective spacetime observables is generically
  non-injective, a single underlying configuration of $\chi$ may correspond to multiple
  physically admissible effective realizations.
  This non-uniqueness of projection gives rise, at the effective level, to phenomena
  commonly described as fluctuations, probabilistic outcomes, or quantum uncertainty.

  In this sense, randomness is not fundamental but \emph{projective}.
  It reflects the multiplicity of effective descriptions compatible with a given
  pre-geometric relational structure, rather than any stochasticity intrinsic to
  $\chi$ itself.
  The apparent temporal character of fluctuations is therefore an emergent feature of
  projected spacetime descriptions, not a property of the underlying substrate.

  Intrinsic structural indeterminacy thus plays a dual conceptual role.
  At the ontological level, it prevents perfect structural symmetry and ensures that
  $\chi$ remains physically generative.
  At the effective level, when combined with non-injective projection, it provides the
  structural origin of observable variability, probabilistic outcomes, and quantum
  indeterminacy, without introducing fundamental randomness, hidden dynamics, or
  temporal evolution into the substrate itself.

  \subsection{Energy as Capacity for Relaxation}
  \label{subsec:energy-as-capacity-for-relaxation}

  Within the Cosmochrony framework, energy is reinterpreted as the effective capacity of
  projected $\chi$ configurations to relax unresolved structural constraints.
  Energy is not treated as a fundamental conserved substance, nor as an intrinsic
  attribute of the $\chi$ substrate itself.
  Rather, it is an emergent, descriptive quantity that characterizes the degree to which
  a given projected configuration retains the ability to undergo further relaxation.

  At the fundamental level, $\chi$ admits intrinsic structural indeterminacy but no
  notion of energy.
  Energy appears only after projection, as a measure of residual structural tension
  within effective descriptions compatible with spacetime interpretation.
  The progressive deployment of structural information corresponds to the conversion of
  intrinsic indeterminacy into relational differentiation.
  Energy thus quantifies the remaining potential for this conversion within projected
  configurations.

  From this perspective, processes commonly described as energy transfer, dissipation,
  or radiation correspond to the redistribution or release of relaxation capacity across
  projected degrees of freedom.
  Conservation laws associated with energy arise only at the effective level, as
  structural regularities of projected dynamics, and do not reflect a fundamental
  conservation principle at the level of $\chi$ itself.

  Importantly, without intrinsic structural indeterminacy, the notion of energy would be
  ill-defined.
  A perfectly closed and fully determined relational substrate would admit no unresolved
  tension and therefore no capacity for relaxation.
  Energy is thus inseparable from the structural openness of the $\chi$ substrate and
  from the non-injective character of its projection into effective physical
  descriptions.

  \subsection{Mass as Frozen Information}
  \label{subsec:mass-as-frozen-information}

  Localized and long-lived configurations of the $\chi$ substrate—describable in
  effective regimes as particle-like excitations—correspond to regions in which further
  relaxation is strongly inhibited.
  Such configurations trap a fixed amount of unresolved structural information,
  preventing it from participating freely in the global relaxation process.

  In this interpretation, mass represents \emph{frozen energy}: structural information
  whose capacity for further relaxation has been locally suppressed.
  The term ``information'' is used here in a structural sense, referring to persistent
  relational constraints within projected $\chi$ configurations, and should not be
  understood in terms of Shannon entropy or symbolic encoding.
  Mass therefore quantifies the degree to which relaxation capacity has been
  immobilized into stable relational patterns.

  At the level of emergent spacetime physics, this relation is expressed phenomenologically
  by the standard mass--energy equivalence,
  \[
    E_{\mathrm{phys}} = m_{\mathrm{phys}}\, c^2 ,
  \]
  where $c$ denotes the emergent spacetime limiting speed governing effective causal
  structure.
  Within Cosmochrony, this identity reflects a deeper structural constraint: the same
  quantity that appears as energy when relaxation capacity is mobile appears as mass
  when that capacity is structurally confined.

  From this perspective, processes such as particle annihilation, decay, or radiation
  emission correspond to the partial or complete release of frozen structural
  information, restoring its ability to participate in relaxation.
  Mass is thus not a fundamental attribute of the $\chi$ substrate, but an emergent
  measure of inhibited relaxation arising from stable, spectrally isolated
  configurations of the underlying relational structure.

  \subsection{Quarks as Non-Projectable Internal Modes}
  \label{subsec:quarks-non-projectable-modes}

  The ontological status of quarks provides a particularly clear illustration of the
  distinction between the pre-geometric $\chi$ substrate and its effective spacetime
  projections.

  In Cosmochrony, quarks are not interpreted as fundamental particle-like entities, nor
  as independent localized excitations of the $\chi$ substrate.
  Instead, they correspond to internal structural modes of composite solitonic
  configurations.
  These modes are required to characterize the internal organization, stability, and
  spectral structure of hadronic excitations, but do not admit an autonomous and
  coherent projection into spacetime.

  In effective quantum field descriptions, quarks appear as elementary degrees of
  freedom subject to confinement.
  Within Cosmochrony, this confinement is not imposed dynamically by an external
  interaction or force.
  Rather, it reflects a deeper structural constraint: isolated quark-like modes do not
  correspond to admissible standalone projections of $\chi_{\mathrm{eff}}$.
  Only collective configurations in which such internal modes are topologically and
  relationally closed admit a stable spacetime manifestation.

  In this sense, quarks are real at the structural level of the $\chi$ substrate, but
  incomplete at the level of physical projection.
  They are neither fictitious entities nor fundamental spacetime objects, but
  non-factorizable internal components of projected excitations.
  Their physical influence is therefore necessarily indirect, encoded in the spectral
  properties, symmetry patterns, and dynamical responses of hadrons rather than in
  localized detection events.

  This interpretation parallels the status of internal degrees of freedom in other
  collective systems: such modes are indispensable for an accurate effective
  description, yet do not correspond to independently realizable physical entities.
  Quark confinement thus appears not as a contingent feature of strong interactions,
  but as a direct and unavoidable consequence of the non-injective character of the
  projection from the $\chi$ substrate to emergent spacetime descriptions.

  \subsection{The Role of the Universal Bound $c_\chi$}
  \label{subsec:role-of-cchi}

  A central structural element of the Cosmochrony framework is the existence of a
  universal invariant bound, denoted $c_\chi$, defined at the level of the pre-temporal
  $\chi$ substrate.
  This bound does not correspond to a signal velocity, nor to the propagation of any
  field, excitation, or information in spacetime.
  Instead, $c_\chi$ characterizes an absolute structural limit on the degree to which
  relational information can be locally confined within admissible configurations of
  the $\chi$ substrate.

  At the fundamental level, $c_\chi$ is non-metric and non-temporal.
  It is not associated with distances, durations, lightcones, or causal ordering, since
  none of these notions are defined prior to projection.
  Rather, $c_\chi$ expresses a maximal admissible rate of structural ordering: beyond
  this limit, further confinement of relational degrees of freedom becomes impossible
  and relaxation necessarily occurs.

  Crucially, $c_\chi$ is not itself an observable quantity.
  It acquires operational meaning only through projection, when configurations of
  $\chi$ admit a locally injective representation in terms of effective spacetime
  variables.
  In such projectable regimes, the invariant structural bound $c_\chi$ manifests as an
  effective causal constraint $c$, defined through the maximal admissible local ordering
  rate of projected configurations:
  \[
    c \;\equiv\; \Pi(c_\chi),
  \]
  where $\Pi$ denotes the projection from the pre-geometric relational substrate to an
  effective spacetime description.

  The constant $c$, which coincides numerically with the observed speed of light, thus
  has a strictly derivative status.
  It is not an independent postulate, but the spacetime expression of the deeper
  structural bound $c_\chi$ once notions of locality, duration, and causal ordering
  become meaningful.
  All effective causal constraints appearing in projected descriptions are inherited
  from this underlying invariant bound.

  Accordingly, the local constraint on the effective relaxation functional,
  \[
    \left| \mathcal{D}_{\mathrm{loc}} \chi_{\mathrm{eff}} \right| \le c ,
  \]
  should be understood as the projected manifestation of the more fundamental structural
  limitation imposed by $c_\chi$.
  No causal or dynamical principle is imposed directly at the level of spacetime;
  effective causality arises solely as a consequence of the bounded projective
  realization of relational ordering.

  This distinction becomes essential in strong-gravity or near-deprojection regimes.
  While the effective constant $c$ may lose its geometric or causal interpretation when
  projection ceases to be locally injective, the structural bound $c_\chi$ remains
  invariant.
  The breakdown of spacetime description therefore signals not a violation of
  causality, but the loss of representability of relational configurations within a
  spacetime framework.

  In summary, $c_\chi$ defines a universal, pre-temporal structural bound governing the
  admissibility of $\chi$ configurations, while $c$ represents its effective spacetime
  manifestation.
  The latter inherits its numerical value and universality entirely from the former,
  ensuring conceptual continuity between the pre-geometric substrate and emergent
  relativistic causality.

  \subsection{The Role of $\hbar_{\chi}$ and Reprojection from $\chi$}
  \label{subsec:the-role-of-hbar_chi-and-reprojection-from-chi}

  In the Cosmochrony framework, the parameter $\hbar_{\chi}$ is not identified with the
  quantum constant $\hbar$ as introduced in conventional quantum mechanics.
  Instead, it characterizes a fundamental structural scale of the pre-geometric
  $\chi$ substrate, determined by the intrinsic relational density and constraint
  structure of admissible configurations.

  Importantly, $\hbar_{\chi}$ does not represent a quantum of action evolving in time.
  At the level of $\chi$, no temporal evolution, Hilbert space, or dynamical operator
  algebra is defined.
  Rather, $\hbar_{\chi}$ specifies a minimal quantum of \emph{reprojection}:
  intrinsic structural information encoded in $\chi$ can enter an effective spacetime
  description only in discrete units set by this scale.

  Intrinsic structural indeterminacy of $\chi$ does not give rise to continuous emergence.
  Instead, reprojection into spacetime occurs in discrete events, each corresponding to
  a minimal admissible transfer of relational structure into projected observables.
  These reprojection quanta define the graininess of effective quantum phenomena.

  As spacetime structure stabilizes and projection becomes locally persistent,
  reprojection events become increasingly localized.
  Phenomenologically, this manifests as vacuum fluctuations or quantum noise within
  otherwise stable spacetime regions, without implying any fundamental stochastic
  dynamics at the level of $\chi$ itself.

\subsection{The Origin of Planck's Constant: $\hbar_\chi$ versus $\hbar_{\mathrm{eff}}$}
  \label{subsec:h-origin}

  A potential conceptual tension arises regarding the status of Planck's constant
  $\hbar$.
  Within Cosmochrony, $\hbar$ is not an independent postulate but a
  \textbf{spectral invariant} of the relaxation and projection process.

  \begin{itemize}
    \item \textbf{Fundamental substrate scale ($\hbar_\chi$):}
    At the level of the $\chi$ substrate, the quantum of reprojection is uniquely fixed
    by the intrinsic structural scales of the theory:
    \begin{equation}
      \hbar_\chi \;=\; \frac{c^3}{K_0\,\chi_c} .
    \end{equation}
    Here $K_0$ denotes the coupling density of relational constraints, $\chi_c$ the
    characteristic correlation scale of the substrate, and $c$ the effective
    projection of the invariant structural bound $c_\chi$.
    This relation expresses the \emph{spectral rigidity} of the substrate rather than
    a dynamical quantization rule.

    \item \textbf{Effective representation ($\hbar_{\mathrm{eff}}$):}
    In effective spacetime descriptions, the same structural scale appears as
    $\hbar_{\mathrm{eff}}$, functioning as the quantum of action in Hilbert-space
    formulations.
    The apparent universality of $\hbar$ across physical systems reflects the fact
    that $K_0$ and $\chi_c$ are global invariants of the current relaxation epoch.
  \end{itemize}

  The transition from $\hbar_\chi$ to $\hbar_{\mathrm{eff}}$ therefore does not involve
  a change in value, but a change in representation.
  A structural constraint defined on the pre-geometric substrate is re-expressed, after
  projection, as a universal dynamical constant governing effective quantum dynamics.

\subsection{Spectral Invariance of Planck's Constant and the Fine-Structure Constant}
  \label{subsec:quantum-invariants}

  The dependence of $\hbar_\chi$ on $K_0$ and $\chi_c$ does not signal arbitrariness, but
  is instead a requirement for spectral unification.
  By identifying
  \[
    \hbar_\chi \equiv \frac{c^3}{K_0\,\chi_c},
  \]
  the quantum of action is understood as a manifestation of the relational density and
  constraint structure of the $\chi$ substrate.

  Within this framework:
  \begin{itemize}
    \item \textbf{Resolution of the $\hbar$ tension:}
    The effective constant $\hbar_{\mathrm{eff}}$ is the projected expression of the
    fundamental reprojection scale $\hbar_\chi$.
    Its apparent constancy across spacetime follows from the near-homogeneity of the
    relaxation flux $\Phi_\chi$ in the current cosmological epoch.

    \item \textbf{Spectral origin of $\alpha$:}
    The fine-structure constant $\alpha$ emerges as a dimensionless spectral ratio,
    determined by the geometry of the projection fiber $\Pi$ and the spectral rigidity
    encoded by $K_0$:
    \begin{equation}
      \alpha \;=\;
      \mathcal{F}\!\left(
                     \frac{\text{geometry and topology of }\Pi}{K_0}
      \right),
    \end{equation}
    where $\mathcal{F}$ is a functional fixed by the structure of admissible projections.
  \end{itemize}

  This interpretation implies that if the structural parameters $K_0$ or $\chi_c$ were
  to vary—for example in primordial high-constraint regimes—both $\hbar$ and $\alpha$
  would scale accordingly.
  Such variation would not signal a breakdown of physical law, but a change in the
  projective representation of invariant relational constraints, preserving the
  internal structural coherence of the Cosmochrony framework.

