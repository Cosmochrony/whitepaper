\section{Ontological Interpretation of the $\chi$ Substrate}
  \label{sec:chi_ontology}

  Throughout this section, we explicitly distinguish the invariant structural bound
  $c_{\chi}$, defined at the level of the pre-temporal $\chi$ substrate, from its
  emergent spacetime manifestation $c$~\cite{Rovelli2018}.
  The latter appears only once spacetime notions such as distance, duration, and
  causal propagation become meaningful within effective descriptions.

  \subsection{The $\chi$ Substrate as a Pre-Temporal Structural Plan}

    In the Cosmochrony framework, the $\chi$ substrate is not interpreted as a physical
    field evolving within spacetime, but as a pre-temporal relational structure from
    which spacetime, matter, and physical laws emerge.
    It may be heuristically described as a ``structural plan'' of the universe: a
    complete but non-dynamical organization encoding the set of physically admissible
    configurations and their internal relations.

    Importantly, this structural plan does not prescribe a unique history nor a fixed
    sequence of events.
    Rather, it defines a constrained space of relational possibilities.
    Temporal succession is therefore not fundamental, but emergent, corresponding to an
    oriented resolution of structural relations within $\chi$ once an effective spacetime
    description becomes applicable.

    \paragraph{Infra-physical status of $\chi$.}
      The $\chi$ substrate is not described by physics proper, understood as a theory of
      dynamical fields evolving in spacetime.
      Rather, it belongs to an infra-physical relational framework that specifies the
      structural conditions under which physical observables, spacetime geometry, and
      effective dynamical laws emerge.

\subsection{Relational Ontology and Conceptual Lineage}

  The relational character of the $\chi$ substrate bears a conceptual affinity with
  relational approaches in physics, notably those advocated by Rovelli~\cite{Rovelli1996,Rovelli2004}.
  These approaches trace their philosophical roots to Aristotelian relational ontology,
  in which properties are defined through relations rather than as intrinsic
  attributes~\cite{AristotleCategories,Shields2016}.

  Cosmochrony shares this rejection of intrinsic, observer-independent properties.
  However, it extends relationalism to a deeper ontological level.
  In relational quantum mechanics, relations are primary at the level of quantum states
  describing interactions between systems that are themselves taken as given.
  By contrast, Cosmochrony posits that the fundamental substrate $\chi$ is itself
  relational: there are no underlying entities between which relations are defined.

  In this sense, $\chi$ configurations are not relations \emph{between} pre-existing
  objects, but relational structures that constitute the objects themselves once a
  projection into an effective spacetime description occurs.
  Relationality is therefore not a feature of physical states within spacetime, but an
  intrinsic property of the pre-geometric substrate from which spacetime and physical
  entities emerge.

  This distinction is essential for understanding the ontological asymmetry between
  $\chi$ and its spacetime projections.
  While relational quantum mechanics reformulates quantum theory without modifying
  the ontological status of spacetime itself, Cosmochrony relocates relationality at
  the level of the substrate that gives rise to spacetime, time ordering, and physical
  objects.

  \paragraph{On the absence of fundamental values.}
    In Cosmochrony, relations are not defined between pre-existing fundamental values.
    The relational structure of $\chi$ is ontologically primary and admits no intrinsic
    numerical or field-like values.
    What appear, within effective spacetime descriptions, as scalar values, entities, or
    local degrees of freedom arise only as stable invariants of this relational structure
    under projection and coarse-graining.

\subsection{Projection, Reality, and Ontological Asymmetry}

Within this ontology, the emergence of spacetime should be understood as a projection
from the $\chi$ substrate, rather than as a dual or equivalent description.
The projected universe is fully real at the physical level, but its reality is
derivative: spacetime entities, effective fields, and dynamical laws do not possess
ontological primacy~\cite{Rovelli2021}.

This asymmetry is essential.
While physical descriptions depend on the projection of $\chi$, the converse is not
true.
The $\chi$ structure exists independently of spacetime notions and does not admit a
reformulation entirely in geometric or dynamical terms.
Projection must therefore be understood as non-injective: distinct structural
features of $\chi$ may correspond to identical or physically indistinguishable
spacetime configurations.

\paragraph{Apparent fine-tuning.}
  In Cosmochrony, apparent fine-tuning does not reflect an improbable choice of initial
  conditions or a delicate adjustment of fundamental constants.
  It arises from the fact that only a restricted class of $\chi$ configurations admits
  a coherent and stable physical projection.
  Most configurations of the $\chi$ substrate do not give rise to consistent spacetime
  descriptions, observable laws, or persistent physical structures.

\paragraph{Absence of a multiverse.}
  Cosmochrony does not postulate a multiverse.
  While multiple configurations of the $\chi$ substrate may correspond to the same
  physical universe under projection, the framework provides no mechanism by which a
  single $\chi$ structure could sustain multiple independent physical projections.
  The universe is therefore unique at the level of physical reality, even though its
  underlying description in terms of $\chi$ may be non-unique.

\subsection{Clarifying the Relation to Holographic Descriptions}
\label{subsec:clarifying-holography}

The preceding considerations naturally invite comparison with the holographic
principle, as originally proposed by ’t~Hooft and Susskind, which suggests that the
effective information content of a spacetime region scales with its boundary rather
than its volume.

Cosmochrony is \emph{not} a holographic theory in the technical sense.
It does not posit a lower-dimensional boundary description, nor a dual equivalence
between bulk and boundary physics.
Any holographic-like behavior arises here as a consequence of projection from a
non-factorizable, pre-geometric substrate, rather than from a fundamental encoding
principle.

In particular, the limitation of physically accessible information within a
spacetime region reflects the degeneracy of $\chi$ configurations compatible with a
given projection.
This constraint is intrinsic to the emergence of spacetime itself and does not
require the introduction of boundary degrees of freedom or dimensional reduction.

Thus, while Cosmochrony reproduces certain qualitative features commonly associated
with holographic descriptions, it differs fundamentally in its ontological grounding.
Holography appears here as an emergent signature of projection, not as a guiding
principle or foundational postulate.

\subsection{Intrinsic Indeterminacy and Fluctuations}
\label{subsec:intrinsic-indeterminacy-and-fluctuations}

A perfectly deterministic and fully symmetric relational substrate would remain
physically inert: no configuration would be privileged, and no effective ordering
could emerge.
For this reason, Cosmochrony postulates the existence of intrinsic indeterminacy
within the structure of $\chi$.

For clarity, the term ``fluctuations'' is used only metaphorically.
No underlying dynamical agitation or stochastic evolution is implied.
Intrinsic indeterminacy refers instead to a structural non-determination of $\chi$,
analogous to an incompleteness of specification rather than to random motion.

Its role is not to generate dynamics, but to prevent perfect structural symmetry and
indifference, thereby allowing relational distinctions to acquire physical relevance
once projection into an effective spacetime description occurs.

\subsection{Energy as Capacity for Relaxation}

Within this framework, energy is reinterpreted as the capacity of projected $\chi$
configurations to relax internal structural constraints.
Energy does not correspond to a conserved substance at the fundamental level, but to
a measure of unresolved structural tension within admissible projected descriptions.

The deployment of structural information corresponds to the progressive conversion of
intrinsic indeterminacy into relational differentiation.
Energy thus quantifies the remaining potential for this conversion.
Without intrinsic indeterminacy, energy itself would be undefined.

\subsection{Mass as Frozen Information}

Localized, stable configurations of the $\chi$ substrate—describable in effective
regimes as particle-like excitations—correspond to regions where further relaxation
is strongly inhibited.
These configurations trap a fixed amount of unresolved structural information.

In this interpretation, mass represents frozen energy: information that has lost its
capacity to participate freely in further relaxation.
At the level of emergent spacetime physics, this relation is expressed by the
standard mass--energy equivalence,
\[
  E_{\text{phys}} = m_{\text{phys}} c^2 ,
\]
where $c$ denotes the emergent spacetime limiting speed.
This phenomenological identity reflects a deeper structural constraint governing the
confinement and release of information within the $\chi$ substrate.

\subsection{Quarks as Non-Projectable Internal Modes}
\label{subsec:quarks-non-projectable-modes}

The ontological status of quarks provides a clarifying example of the distinction
between the pre-geometric $\chi$ substrate and its effective spacetime projections.

In Cosmochrony, quarks are not interpreted as fundamental particle-like entities,
nor as independent localized excitations of $\chi$.
Rather, they correspond to internal structural modes of composite solitonic
configurations—modes that are necessary to characterize the internal organization
and stability of hadronic excitations, but which do not admit an autonomous and
coherent projection into spacetime.

In effective quantum field descriptions, quarks appear as elementary degrees of
freedom subject to confinement.
Within Cosmochrony, this confinement is not imposed dynamically by an external
interaction, but reflects a deeper structural constraint: isolated quark-like modes
do not correspond to admissible standalone projections of $\chi_{\mathrm{eff}}$.
Only collective configurations in which these internal modes are topologically
and relationally closed admit a stable spacetime manifestation.

In this sense, quarks are real at the structural level of $\chi$, but incomplete at
the level of physical projection.
They are neither fictitious nor fundamental objects, but non-factorizable internal
components of projected excitations.
Their observability is therefore necessarily indirect, encoded in the spectral,
dynamical, and symmetry properties of hadrons rather than in localized detection
events.

This interpretation parallels the status of internal degrees of freedom in other
collective systems: they are indispensable for an accurate effective description,
yet do not correspond to independently realizable physical entities.
Quark confinement thus appears not as a contingent feature of strong interactions,
but as a direct consequence of the non-injective nature of the projection from $\chi$
to emergent spacetime.

\subsection{The Role of the Universal Bound $c_{\chi}$}

The constant $c_{\chi}$ plays a central structural role in Cosmochrony.
Rather than setting a signal propagation speed, it defines an absolute bound on the
degree to which information can be structurally constrained within the $\chi$
substrate.

In this sense, $c_{\chi}$ sets the maximal confinement of structural information
compatible with physical coherence.
Beyond this bound, further constraint becomes impossible and relaxation becomes
inevitable.
Emergent notions of inertial resistance, causal structure, and mass--energy relations
are understood as projections of this invariant structural limit.

\subsection{The Role of $\hbar_{\chi}$ and Reprojection from $\chi$}

In Cosmochrony, the parameter $\hbar_{\chi}$ is not identified with the quantum
constant $\hbar$, but emerges from the fundamental structural scales $K_0$, $\chi_c$,
and $c$.
Its numerical coincidence with $\hbar$ in quantum regimes reflects the universality
of action quantization across effective physical theories.

$\hbar_{\chi}$ does not represent a quantum of action evolving in time, but a
fundamental quantum of reprojection.
Intrinsic indeterminacy of $\chi$ does not give rise to continuous emergence;
rather, any reprojection of structural information into spacetime occurs in discrete
units set by $\hbar_{\chi}$.

As spacetime structure stabilizes, reprojection becomes increasingly localized,
manifesting phenomenologically as vacuum fluctuations within otherwise stable regions
of spacetime.
