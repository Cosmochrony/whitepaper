\section{Ontological Interpretation of the $\chi$ Field}
  \label{sec:chi_ontology}

  Throughout this section, we explicitly distinguish the invariant structural bound
  $c_{\chi}$, defined at the level of the pre-temporal $\chi$ substrate, from its
  emergent spacetime manifestation $c$~\cite{Rovelli2018}.
  The latter appears only once spacetime notions such as distance, duration, and
  causal propagation become meaningful.

  \subsection{The $\chi$ Field as a Pre-Temporal Structural Plan}

    In the Cosmochrony framework, the $\chi$ field is not interpreted as a physical
    field evolving within spacetime, but as a pre-temporal structural substrate from
    which spacetime, matter, and physical laws emerge.
    It may be heuristically described as a ``structural plan'' of the universe: a
    complete but non-dynamical organization encoding the set of physically admissible
    configurations and their internal relations.

    Importantly, this plan does not prescribe a unique history nor a fixed sequence
    of events.
    Rather, it defines a constrained space of relational possibilities.
    Temporal succession is therefore not fundamental, but emergent, corresponding to
    an oriented resolution of structural relations within $\chi$ once an effective
    spacetime description becomes applicable.

  \subsection{Projection, Reality, and Ontological Asymmetry}

    Within this ontology, the emergence of spacetime should be understood as a
    projection from the $\chi$ substrate, rather than as a dual or equivalent
    description.
    The projected universe is fully real at the physical level, but its reality is
    derivative: spacetime entities, fields, and dynamical laws do not possess
    ontological primacy~\cite{Rovelli2021}.

    This asymmetry is essential.
    While physical descriptions depend on the projection of $\chi$, the converse is
    not true.
    The $\chi$ structure exists independently of spacetime notions and does not admit
    a reformulation entirely in geometric or dynamical terms.
    Projection must therefore be understood as non-injective: distinct structural
    features of $\chi$ may correspond to identical or indistinguishable spacetime
    configurations.

  \subsection{Clarifying the Relation to Holographic Descriptions}
    \label{subsec:clarifying-holography}

    The preceding considerations naturally invite comparison with the holographic
    principle, as originally proposed by :contentReference[oaicite:0]{index=0}
    and :contentReference[oaicite:1]{index=1}, which suggests
    that the effective information content of a spacetime region scales with its
    boundary rather than its volume.

    Cosmochrony is \emph{not} a holographic theory in the technical sense.
    It does not posit a lower-dimensional boundary description, nor a dual equivalence
    between bulk and boundary physics.
    Any holographic-like behavior arises here as a consequence of projection from a
    non-factorizable, pre-geometric substrate, rather than from a fundamental encoding
    principle.

    In particular, the limitation of physically accessible information within a
    spacetime region reflects the degeneracy of $\chi$ configurations compatible with
    a given projection.
    This constraint is intrinsic to the emergence of spacetime itself, and does not
    require the introduction of boundary degrees of freedom or dimensional reduction.

    Thus, while Cosmochrony reproduces certain qualitative features commonly associated
    with holographic descriptions, it differs fundamentally in its ontological
    grounding.
    Holography appears here as an emergent signature of projection, not as a guiding
    principle or foundational postulate.

  \subsection{Intrinsic Indeterminacy and Fluctuations}
    \label{subsec:intrinsic-indeterminacy-and-fluctuations}

    A perfectly deterministic and fully symmetric structural substrate would remain
    physically inert: no configuration would be privileged, and no effective ordering
    could emerge.
    For this reason, Cosmochrony postulates the existence of intrinsic indeterminacy
    within the $\chi$ structure.

    For clarity, the term ``fluctuations'' is used only metaphorically.
    No underlying dynamical agitation or stochastic evolution is implied.
    Intrinsic indeterminacy refers instead to a structural non-determination of the
    $\chi$ field, analogous to an incompleteness of specification rather than to random
    motion.

    Its role is not to generate dynamics, but to prevent perfect structural symmetry
    and indifference, thereby allowing relational distinctions to acquire physical
    relevance once projection into an effective spacetime description occurs.

  \subsection{Energy as Capacity for Relaxation}

    Within this framework, energy is reinterpreted as the capacity of the $\chi$
    structure to relax its internal constraints.
    Energy does not correspond to a conserved substance at the fundamental level, but
    to a measure of unresolved structural tension.

    The deployment of $\chi$ corresponds to the progressive conversion of structural
    indeterminacy into relational information.
    Energy thus quantifies the potential for this conversion.
    Without intrinsic indeterminacy, energy itself would be undefined.

  \subsection{Mass as Frozen Information}

    Localized, stable configurations of the $\chi$ field—interpreted as particle-like
    excitations—correspond to regions where relaxation is strongly inhibited.
    These configurations trap a fixed amount of unresolved structural information.

    In this interpretation, mass represents frozen energy: information that has lost
    its capacity to participate freely in further relaxation.
    At the level of emergent spacetime physics, this relation is expressed by the
    standard mass--energy equivalence,
    \[
      E_{\text{phys}} = m_{\text{phys}} c^2 ,
    \]
    where $c$ denotes the emergent spacetime limiting speed.
    This phenomenological identity reflects a deeper structural constraint governing
    the confinement and release of information within the $\chi$ substrate.

  \subsection{The Role of the Universal Bound $c_{\chi}$}

    The constant $c_{\chi}$ plays a central structural role in Cosmochrony.
    Rather than setting a signal propagation speed, it defines an absolute bound on the
    degree to which information can be structurally constrained within the $\chi$
    substrate.

    In this sense, $c_{\chi}$ sets the maximal confinement of structural information
    compatible with physical coherence.
    Beyond this bound, further constraint becomes impossible and relaxation becomes
    inevitable.
    Emergent notions of inertial resistance, causal structure, and mass--energy
    relations are understood as projections of this invariant structural limit.

  \subsection{The Role of $\hbar_{\chi}$ and Reprojection from $\chi$}

    In Cosmochrony, the parameter $\hbar_{\chi}$ is not identified with the quantum
    constant $\hbar$, but emerges from the fundamental structural scales $K_0$,
    $\chi_c$, and $c$.
    Its numerical coincidence with $\hbar$ in quantum regimes reflects the
    universality of action quantization across effective physical theories.

    $\hbar_{\chi}$ does not represent a quantum of action evolving in time, but a
    fundamental quantum of reprojection.
    Intrinsic fluctuations of $\chi$ do not give rise to continuous emergence;
    rather, any reprojection of structural information into spacetime occurs in
    discrete units set by $\hbar_{\chi}$.

    As spacetime structure stabilizes, reprojection becomes increasingly localized,
    manifesting phenomenologically as vacuum fluctuations within otherwise stable
    regions of spacetime.
