\subsection{Derivation of the MOND Acceleration Floor}
  \label{sec:mond_derivation}

  In Cosmochrony, the ``arrow of time'' $\partial_t \chi \geq 0$ is coupled to the global expansion of the universe.
  In an FLRW-like limit, the field $\chi$ must follow the cosmological clock, such that
  $\partial_t \chi \approx H_0 \chi$.

  Substituting this into the constraint $(\partial_t \chi)^2 + |\nabla \chi|^2 = c^2$, we find that at any point in
  space, there exists a minimal residual gradient $\nabla \chi_{\min}$ even in the absence of local matter:
  \begin{equation}
    |\nabla \chi|_{\min} = \sqrt{c^2 - (H_0 \chi)^2}
  \end{equation}
  For a local observer, this residual gradient acts as a background acceleration $a_0 \approx c H_0$.
  When calculating the gravitational force via the non-linear Poisson equation~\eqref{eq:nonlinear_poisson}, the total
  gradient is the sum of the local Newtonian contribution and this cosmological floor.

  At large radii $r$, where the Newtonian gradient $\nabla \chi_N \propto M/r^2$ would normally vanish, the field
  ``saturates'' at the floor value. The effective gravitational acceleration then transitions from $1/r^2$ to a $1/r$
  dependence, naturally recovering the Deep-MOND regime:
  \begin{equation}
    g_{eff} = \sqrt{g_N a_0}
  \end{equation}
  This explains the flat rotation curves of galaxies as a kinematic projection of the global expansion onto local dynamics.
