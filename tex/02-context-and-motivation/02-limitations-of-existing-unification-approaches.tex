\subsection{Limitations of Existing Unification Approaches}
  \label{subsec:limitations-of-existing-unification-approaches}

  Several major research programs have sought to address these challenges.
  Quantum field theory in curved spacetime successfully accounts for particle creation
  and vacuum effects, but retains a classical spacetime background~\cite{weinberg1972gravitation}.
  Canonical and covariant approaches to quantum gravity attempt to quantize spacetime
  geometry itself, often at the cost of substantial mathematical complexity and
  interpretational ambiguity.

  String theory and related frameworks introduce extended fundamental objects and
  higher-dimensional structures, offering deep mathematical unification but leading
  to a large space of possible low-energy realizations~\cite{rovelli2004quantum}.
  While internally rich, these approaches face ongoing challenges concerning empirical
  testability and the physical interpretation of their fundamental degrees of freedom.

  Collectively, these limitations motivate the exploration of alternative perspectives
  in which spacetime geometry, matter, and quantum behavior are not independently
  postulated, but emerge from a common underlying mechanism operating at a more
  primitive, pre-geometric level.
