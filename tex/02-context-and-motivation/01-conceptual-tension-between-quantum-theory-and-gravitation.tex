\subsection{Conceptual Tension Between Quantum Theory and Gravitation}
  \label{subsec:conceptual-tension-between-quantum-theory-and-gravitation}

  Quantum mechanics and general relativity differ not only in their mathematical
  formalisms, but also in their foundational concepts.
  Quantum theory is intrinsically probabilistic, relies on a fixed causal structure,
  and treats time as an external parameter~\cite{Dirac1930,Born1926}.
  General relativity, by contrast, describes gravitation as the dynamics of spacetime
  geometry itself, with time acquiring a coordinate-dependent and observer-relative
  status~\cite{Einstein1915,MisnerThorneWheeler1973}.

  This conceptual mismatch becomes particularly acute in regimes where both quantum
  effects and strong gravitational fields are expected to be relevant, such as near
  spacetime singularities or in the early universe~\cite{penrose1989emperors,Prigogine1997}.
  Direct attempts to quantize gravity encounter persistent difficulties, including
  the problem of time, non-renormalizability, and the absence of a preferred background structure.
  These difficulties suggest that the tension may reflect not merely technical
  obstacles, but a deeper incompatibility in the assumed ontological status of time and geometry.
