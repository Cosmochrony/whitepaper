\subsection{Minimalism as a Guiding Principle}
  \label{subsec:minimalism-as-a-guiding-principle}

  The framework developed in this work adopts minimalism as a guiding principle.
  Rather than introducing multiple fundamental fields, additional dimensions, or
  independent quantization rules, we explore whether a single continuous fundamental
  entity can account for temporal ordering, spatial relations, and quantum features
  within a unified relational dynamics.

  The scalar quantity $\chi$ is not interpreted as a conventional matter field, nor
  as a component of spacetime geometry.
  Instead, it represents a pre-geometric substrate whose irreversible relaxation
  underlies the emergence of both duration and separation.
  In this view, time and space are not independent primitives, but complementary
  aspects of a single dynamical process.
  Effective geometric and quantum descriptions arise only through coarse-grained,
  generally non-injective projections of the underlying $\chi$-configurations.
