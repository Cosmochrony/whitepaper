\subsection{Hubble Constant from $\chi$ Dynamics}
  \label{subsec:hubble-constant-from-chi-dynamics}

  In Cosmochrony, the Hubble parameter is not introduced as a free cosmological
  constant, but arises as an effective quantity associated with the relaxation
  dynamics of the $\chi$ field.
  At the level of an effective spacetime description, it may be written as
  \begin{equation}
    H(t) = \frac{\dot{\chi}}{\chi},
  \end{equation}
  where the dot denotes differentiation with respect to an effective cosmological
  time parameter introduced to parametrize the relaxation ordering.

  In homogeneous regimes, the relaxation rate approaches its maximal admissible
  value.
  Assuming $\dot{\chi}_{\mathrm{eff}} \simeq c$, the present-day Hubble parameter can
  be estimated as
  \begin{equation}
    H_0 \simeq \frac{c}{\chi(t_0)}.
  \end{equation}

  This relation establishes a direct correspondence between the observed Hubble
  constant and the characteristic relaxation scale of $\chi$ at the current epoch.
  Early-universe probes (such as CMB-based inferences) and late-time distance-ladder
  measurements effectively sample $\chi$ at different stages of its relaxation,
  naturally leading to systematically different inferred values of $H_0$ without
  invoking additional cosmological components or fine-tuned initial conditions.
