\subsection{The Vacuum State as a Minimal Surface}
  \label{sec:vacuum-minimal-surface}

  In the absence of excitations, the fiber $\Pi$
      tends toward a state of minimal spectral tension. This ``vacuum'' is not empty but represents the smoothest
      possible configuration of the substrate $\chi$
      . Any deviation from this minimality—whether through torsion, curvature, or winding—manifests as the presence of
      fields or particles.

      This perspective replaces the concept of ``field quantization'' with the quantization of topological modes in a
      finite-volume fiber. This ensures that the vacuum energy remains finite and intrinsically linked to the spectral
      cutoff of the relational graph.
