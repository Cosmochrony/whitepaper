\subsection{The Geometry of the $\Pi$ Subspace}
  \label{sec:geometry-pi}

  The substrate $\chi$ is not observed in its full relational complexity but through a projection onto a local fiber
      $\Pi \cong S^3$
      . This projection acts as a ``spectral filter'', retaining only the modes of relaxation compatible with the
      $SU(2) \times U(1)$ symmetry of the Hopf fibration.

  In this framework, the metric of $\Pi$
      is not fixed but is dynamically induced by the local density of connections in the relational graph $G$
      . The mapping from the global Laplacian $\Delta_G$ to the projected Laplacian $\Delta_\Pi$ is defined by:
  \begin{equation}
    \Delta_\Pi = P^\dagger \Delta_G P
  \end{equation}
  where $P$
      is the projection operator. The emergence of a continuous 3-sphere geometry is a ``large-N limit'' effect of the
      underlying discrete connectivity.
