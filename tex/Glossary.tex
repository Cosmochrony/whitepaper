\section{Glossary of Core Quantities and Notation}
  \label{appendix:glossary}

  This appendix summarizes the meaning and status of the main quantities used
  throughout the Cosmochrony framework. It is intended as a reference guide and
  does not introduce new assumptions or definitions.

  \subsection{Fundamental and Effective Quantities}

    \paragraph{$\chi$ (Chi field).}
      The fundamental scalar quantity of the Cosmochrony framework.
      $\chi$ is not defined on a pre-existing spacetime manifold but constitutes a
      pre-geometric substrate whose monotonic relaxation provides an intrinsic ordering
      of physical processes.
      Localized, topologically stable configurations of $\chi$ correspond to particle-like
      excitations.

    \paragraph{$V(\chi)$ (Effective potential).}
      An effective, coarse-grained description used to model localization and stability
      properties of $\chi$ configurations.
      $V(\chi)$ is not assumed to be fundamental; its form may emerge from underlying
      discrete relaxation dynamics and is secondary to the spectral description of mass.

    \paragraph{$K_{ij}$ (Relaxation coupling).}
      Edge-dependent coupling coefficients defined on the relaxation network $G(V,E)$.
      $K_{ij}$ quantify the local resistance to relative variations of $\chi$ between
      neighboring nodes and encode geometric and topological information of the network.
      They may depend on the local configuration of $\chi$ in effective descriptions.

\subsection{Derived Operators and Dimensionless Parameters}

  \paragraph{$G(V,E)$ (Relaxation network).}
    A discrete graph representing the underlying relational structure on which the
    $\chi$ field is defined at the fundamental level.
    Vertices correspond to elementary degrees of freedom, and edges encode relaxation
    couplings.

  \paragraph{$\Delta_G$ (Graph Laplacian / relaxation operator).}
    The discrete Laplace--Beltrami operator associated with the network $G(V,E)$ and
    the couplings $K_{ij}$.
    It governs the stability and mode structure of localized $\chi$ configurations.
    Its spectral properties play a central role in the quantitative characterization
    of inertial mass.

  \paragraph{$S$ (Gradient saturation parameter).}
    A dimensionless quantity defined as
    \begin{equation}
      S \equiv \frac{1}{c^2}\sum_{j\sim i} K_{ij}(\chi_i-\chi_j)^2,
    \end{equation}
    measuring the local density of $\chi$ gradients.
    The condition $S \leq 1$ ensures causal consistency and bounds the local relaxation
    rate of $\chi$.

  \paragraph{$\lambda_n$ (Spectral eigenvalues).}
    The eigenvalues of the linearized relaxation or stability operator acting on
    small fluctuations around a localized configuration.
    In effective wave descriptions, $\sqrt{\lambda_n}$ determines the inertial mass
    scale of particle-like excitations.

  \paragraph{$\Omega_\chi$ (Relaxation budget parameter).}
    A dimensionless global quantity characterizing the fraction of the total $\chi$
    relaxation budget stored in spatial gradients.
    In cosmological regimes, $\Omega_\chi$ plays a role analogous to the matter
    density parameter in standard cosmology.

\subsection{Key Concepts}

\paragraph{Energy}
  Energy is a conserved quantity associated with time-translation symmetry and the capacity
  to induce change.
  In Cosmochrony, energy is interpreted as a measure of resistance of $\chi$-field
  configurations to dynamical evolution, while standard conservation laws and empirical
  relations remain unaffected.

\paragraph{Decoherence}
  In quantum mechanics, decoherence denotes the suppression of interference effects due to
  interaction with an environment.
  In Cosmochrony, decoherence is interpreted as the irreversible local deformation of the
  $\chi$ field induced by interaction, which destroys the phase correlations required for
  coherent superposition without altering the underlying structural configuration.

\paragraph{Fluctuations}
  Fluctuations refer to stochastic variations of the $\chi$ field around a given
  configuration.
  They modulate the localization and timing of individual events without altering the
  underlying structural constraints imposed by the $\chi$ topology.

\paragraph{Matter}
  Matter conventionally refers to localized physical entities carrying mass and energy.
  Within Cosmochrony, matter corresponds to stable topological configurations of the $\chi$
  field, whose persistence gives rise to particle-like behavior and inertial properties.

\paragraph{Measurement}
  In standard quantum mechanics, a measurement refers to an interaction resulting in a
  definite outcome drawn from a probability distribution described by the wavefunction.
  In Cosmochrony, measurement is interpreted as a localized interaction that selects a
  specific manifestation of an underlying $\chi$-field fluctuation, without altering the
  global probabilistic structure associated with the system.
  Observable predictions remain unchanged.
  This interpretation does not require a fundamental wavefunction collapse.

\paragraph{Probability}
  In quantum theory, probability quantifies the likelihood of different measurement outcomes.
  Within Cosmochrony, probabilities are not taken as primitive: they reflect a stable
  structural constraint imposed by the local configuration (topology) of the $\chi$ field,
  which defines an invariant pattern of allowed manifestations.
  Stochastic fluctuations of $\chi$ then modulate this pattern at the event level, governing
  the contingent localization and ordering of individual outcomes without altering the
  underlying structural (topological) configuration.

  \paragraph{Schrödinger Equation}
  An effective linear equation governing the evolution of quantum probability amplitudes.
  In Cosmochrony, the Schrödinger equation emerges as an approximate description of coherent,
  weak fluctuations of the $\chi$ field around a stable configuration.

\paragraph{Space--Time}
  In conventional physics, spacetime provides the geometric arena in which physical processes
  take place.
  In Cosmochrony, spacetime is an emergent relational structure arising from large-scale
  configurations of the $\chi$ field, while retaining its effective metric description at
  accessible scales.

\paragraph{Time}
  In standard physics, time parametrizes the ordering and duration of physical processes.
  Within Cosmochrony, time is interpreted as an emergent parameter associated with the local
  rate of evolution of the $\chi$ field.
  This reinterpretation does not modify operational time measurements or relativistic
  predictions.

\paragraph{Uncertainty Principle}
  In quantum mechanics, the uncertainty principle states that certain pairs of observables,
  such as position and momentum, cannot be simultaneously determined with arbitrary precision.
  In Cosmochrony, this limitation arises from the fact that any interaction locally modifies
  the configuration of the $\chi$ field: probing position necessarily alters the local
  dynamical state of $\chi$, thereby affecting momentum, and vice versa.
  The uncertainty principle thus reflects a dynamical constraint on wave configurations,
  rather than an intrinsic indeterminacy.

\paragraph{Wavefunction}
  The wavefunction $\psi$ provides a complete statistical description of quantum systems
  within standard quantum mechanics.
  In Cosmochrony, $\psi$ is interpreted as an effective statistical representation emerging
  from the dynamics and topology of the underlying $\chi$ field, without being itself a
  fundamental physical entity.

\paragraph{Wave--Particle Duality}
  Wave--particle duality refers to the ability of quantum systems to exhibit both wave-like
  and particle-like behavior.
  Within Cosmochrony, this duality is interpreted as a change in the local configuration of
  the $\chi$ field induced by interaction: the system remains fundamentally wave-like, while
  localized, particle-like manifestations arise from interaction-driven distortions of the
  field.
