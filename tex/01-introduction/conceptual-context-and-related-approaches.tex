\subsection{Conceptual Context and Related Approaches}
  \label{subsec:conceptual-context-and-related-approaches}

  The idea that spacetime geometry and gravitation may be emergent rather than fundamental
  has been explored in a variety of recent theoretical frameworks.
  Several approaches treat the spacetime metric as an effective description arising from
  deeper geometric, informational, or dynamical structures, and interpret gravitation as a
  collective or emergent phenomenon rather than a fundamental interaction\cite{Nye2024, Singh2025}.

  Like Loop Quantum Gravity (LQG), Cosmochrony holds that spacetime geometry is not fundamental~\cite{Rovelli2004}.
  However, the two frameworks operate at distinct conceptual levels.

  LQG provides a quantized description of geometry once a spacetime structure is already in place,
  encoding area and volume through spin networks and holonomies.
  Cosmochrony, by contrast, addresses an earlier stage: it proposes a pre-geometric substrate,
  described by a single scalar field $\chi$, from which geometric notions themselves emerge.

  In this sense, Cosmochrony does not compete with LQG but precedes it,
  offering a complementary framework that aims to explain the physical origin
  of the geometric degrees of freedom subsequently quantized in LQG.
