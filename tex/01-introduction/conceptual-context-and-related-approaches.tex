\subsection{Conceptual Context and Related Approaches}
  \label{subsec:conceptual-context-and-related-approaches}

  The idea that spacetime geometry and gravitation may be emergent rather than fundamental
  has been explored in a variety of contemporary theoretical frameworks.
  Several approaches interpret the spacetime metric as an effective description arising from
  deeper geometric, informational, or dynamical structures, and recast gravitation as a
  collective or emergent phenomenon rather than as a fundamental interaction~\cite{Nye2024,Singh2025}.

  Cosmochrony belongs to this broad conceptual lineage, while adopting a deliberately minimalist ontological stance.
  Rather than postulating multiple underlying structures or microscopic degrees of freedom,
  it assumes a single pre-geometric relational substrate, denoted $\chi$, whose irreversible
  relaxation governs the emergence of physical observables.

  Like Loop Quantum Gravity (LQG), Cosmochrony holds that spacetime geometry is not
  fundamental~\cite{Rovelli2004}.
  However, the two frameworks operate at distinct conceptual and ontological levels.

  LQG provides a quantized description of geometry once a spacetime structure is already
  assumed, encoding areas and volumes through spin networks and holonomies.
  In this sense, it addresses the quantization of geometric degrees of freedom defined on a
  kinematical spacetime arena.

  Cosmochrony, by contrast, addresses an earlier and more primitive level.
  It does not quantize geometry, but seeks to explain how geometric notions themselves arise
  as effective, coarse-grained descriptions of underlying $\chi$-configurations.
  The emergence of spacetime is mediated by a non-injective projection from the pre-geometric
  substrate to effective observables, allowing geometric, dynamical, and quantum features to
  appear only once specific relational and spectral conditions are met.

  From this perspective, Cosmochrony does not compete with LQG but conceptually precedes it.
  It aims to account for the physical origin of the geometric degrees of freedom that may
  subsequently be quantized within approaches such as LQG, while remaining agnostic about
  the detailed form of their quantization at the effective level.
