\subsection{Conceptual Context and Related Approaches}
  \label{subsec:conceptual-context-and-related-approaches}

  The idea that spacetime geometry and gravitation may be emergent rather than fundamental
  has been explored in a variety of recent theoretical frameworks.
  Several approaches treat the spacetime metric as an effective description arising from
  deeper geometric, informational, or dynamical structures, and interpret gravitation as a
  collective or emergent phenomenon rather than a fundamental interaction\cite{Nye2024, Singh2025}.

  Like Loop Quantum Gravity (LQG), Cosmochrony posits that spacetime geometry is not fundamental but emerges from a
  deeper, pre-geometric structure.
  However, while LQG relies on spin networks and holonomies, Cosmochrony introduces a single scalar field $\chi$ whose
  relational dynamics encode both spatial and temporal properties.
  This minimalist approach offers a complementary perspective on the emergence of gravity and quantum phenomena.
