\section{Introduction}
  \label{sec:introduction}

  Modern fundamental physics rests on two remarkably successful yet conceptually distinct
  theoretical frameworks: quantum mechanics and general relativity~\cite{dirac1930principles, einstein1915feldgleichungen}.
  Quantum theory provides an extraordinarily accurate description of microscopic phenomena,
  while general relativity offers a geometric account of gravitation and spacetime dynamics
  at macroscopic and cosmological scales\cite{Dirac1930,Einstein1915}.
  Despite their empirical success, these theories remain difficult to reconcile within a single
  coherent conceptual and mathematical framework~\cite{MisnerThorneWheeler1973,Weinberg1972,misner1973gravitation,rovelli2004quantum}.

  A central obstacle to this reconciliation lies in their incompatible foundational assumptions.
  Quantum mechanics presupposes a fixed spacetime background on which states evolve, whereas
  general relativity identifies spacetime geometry itself as a dynamical entity.
  Attempts to bridge this gap have led to a wide range of approaches, including quantum field
  theory in curved spacetime, canonical and covariant quantum gravity programs, as well as
  string-based and holographic frameworks.
  While these efforts have yielded deep insights, they typically rely on extended mathematical
  structures, additional dimensions, or large numbers of degrees of freedom, and often introduce
  elements whose empirical accessibility remains uncertain.

  In this article, we explore a complementary and deliberately minimalist approach, referred to
  as \emph{Cosmochrony}.
  The guiding hypothesis is that spacetime geometry, gravitation, and quantum phenomena may
  emerge from the dynamics of a single continuous geometric substrate, described by a scalar
  quantity denoted $\chi$.
  Crucially, $\chi$ is not introduced as a field propagating on a pre-existing spacetime
  background.
  Rather, spacetime notions themselves are treated as effective descriptions that arise from
  the relational and dynamical properties of $\chi$ configurations.

  The central dynamical postulate of Cosmochrony is that $\chi$ undergoes an irreversible
  relaxation process, locally bounded by the invariant propagation speed $c$.
  This monotonic evolution establishes an intrinsic ordering of physical processes, which is
  identified with the arrow of time.
  Spatial separation, in turn, is interpreted as a relational structure emerging from differences
  and gradients in $\chi$ once a stable geometric regime is reached.
  Within this framework, spacetime expansion, gravitation, particle-like excitations, radiation
  processes, and quantum correlations are not fundamental ingredients, but emergent phenomena
  associated with specific configurations or interactions of the underlying field.

  The aim of the present work is not to propose a complete or final unification of quantum
  theory and gravitation.
  Rather, it seeks to formulate a minimal and internally coherent dynamical framework, and to
  examine the extent to which its qualitative structure and phenomenological consequences are
  compatible with established physical observations.
  In particular, we investigate how cosmological expansion, particle properties, gravitational
  effects, radiation, and quantum entanglement may be consistently reinterpreted within a
  single geometric setting governed by the dynamics of $\chi$.

  Cosmochrony does not attempt to replace the Standard Model or General Relativity in their
  well-tested domains.
  Instead, it offers a unifying geometric interpretation in which quantization, spacetime
  curvature, and cosmic expansion emerge from a common relaxation dynamics rather than being
  introduced as independent postulates.
  In this sense, the framework is best viewed as an exploratory research program aimed at
  clarifying the conceptual relationships between time, geometry, and quantum phenomena.

  At the outset, it is important to clarify the ontological and epistemic status of the quantities
  introduced in this work.
  The scalar quantity $\chi$ is not defined on a pre-existing spacetime manifold, nor is it interpreted
  as a conventional physical field propagating within spacetime.
  Rather, $\chi$ is taken to represent a pre-geometric substrate whose irreversible relaxation
  underlies the emergence of both temporal ordering and spatial relations.

  In this perspective, familiar spacetime notions such as coordinates, metric structure, and
  differential geometry are not fundamental ingredients of the theory.
  They arise only as effective, coarse-grained descriptions of relational and dynamical properties
  of $\chi$ in regimes where a stable geometric interpretation becomes possible.
  Accordingly, variational principles, Lagrangian formulations, and metric-based descriptions
  employed later in the paper should be understood as emergent tools rather than as fundamental
  postulates.

  The structure of the paper is as follows.
  Sections~2--4 introduce the conceptual motivations and minimal dynamical assumptions governing
  the $\chi$ substrate.
  Subsequent sections examine how particle-like excitations, gravitation, quantum correlations,
  and cosmological behavior emerge in appropriate regimes.
  The standard formalisms of general relativity and quantum mechanics are recovered only at this
  effective level, as descriptive frameworks applicable once a spacetime interpretation has emerged.
  Technical reconstructions and mathematical details are collected in the appendices.

  \subsection{Conceptual Context and Related Approaches}
  \label{subsec:conceptual-context-and-related-approaches}

  The idea that spacetime geometry and gravitation may be emergent rather than fundamental
  has been explored in a variety of contemporary theoretical frameworks.
  Several approaches interpret the spacetime metric as an effective description arising from
  deeper geometric, informational, or dynamical structures, and recast gravitation as a
  collective or emergent phenomenon rather than as a fundamental interaction~\cite{Nye2024,Singh2025}.

  Cosmochrony belongs to this broad conceptual lineage, while adopting a deliberately minimalist ontological stance.
  Rather than postulating multiple underlying structures or microscopic degrees of freedom,
  it assumes a single pre-geometric relational substrate, denoted $\chi$, whose irreversible
  relaxation governs the emergence of physical observables.

  Like Loop Quantum Gravity (LQG), Cosmochrony holds that spacetime geometry is not
  fundamental~\cite{Rovelli2004}.
  However, the two frameworks operate at distinct conceptual and ontological levels.

  LQG provides a quantized description of geometry once a spacetime structure is already
  assumed, encoding areas and volumes through spin networks and holonomies.
  In this sense, it addresses the quantization of geometric degrees of freedom defined on a
  kinematical spacetime arena.

  Cosmochrony, by contrast, addresses an earlier and more primitive level.
  It does not quantize geometry, but seeks to explain how geometric notions themselves arise
  as effective, coarse-grained descriptions of underlying $\chi$-configurations.
  The emergence of spacetime is mediated by a non-injective projection from the pre-geometric
  substrate to effective observables, allowing geometric, dynamical, and quantum features to
  appear only once specific relational and spectral conditions are met.

  From this perspective, Cosmochrony does not compete with LQG but conceptually precedes it.
  It aims to account for the physical origin of the geometric degrees of freedom that may
  subsequently be quantized within approaches such as LQG, while remaining agnostic about
  the detailed form of their quantization at the effective level.

