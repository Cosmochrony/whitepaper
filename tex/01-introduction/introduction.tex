\clearpage

\section{Introduction}
  \label{sec:introduction}

  Modern fundamental physics is built upon two highly successful yet conceptually
  distinct frameworks: quantum mechanics and general relativity~\cite{Dirac1930,Einstein1915}.
  Quantum theory accurately describes microscopic phenomena, while general
  relativity provides a geometric account of gravitation and spacetime dynamics
  at macroscopic and cosmological scales.
  Despite their empirical success, these theories rely on incompatible foundational
  assumptions and resist unification within a single coherent conceptual
  framework~\cite{MisnerThorneWheeler1973,Weinberg1972,rovelli2004quantum}.

  Quantum mechanics presupposes a fixed spacetime arena in which physical states
  evolve, whereas general relativity identifies spacetime geometry itself as a
  dynamical entity.
  Numerous approaches have attempted to bridge this tension, including quantum
  field theory in curved spacetime, canonical and covariant quantum gravity
  programs, and string-based or holographic frameworks.
  While these approaches have led to important theoretical developments, they
  typically rely on extended mathematical structures or introduce additional
  degrees of freedom whose physical interpretation and empirical accessibility
  remain unclear.

  In this work, we explore a complementary and deliberately minimalist framework,
  referred to as \emph{Cosmochrony}\footnote{From \textit{$\kappa\acute{o}\sigma\mu o\varsigma$} and
\textit{$\chi\rho\acute{o}\nu o\varsigma$}, denoting a framework in which cosmic structure
and temporal ordering emerge from a common pre-geometric substrate.}.
  The guiding hypothesis is that spacetime geometry, gravitation, and quantum
  phenomena emerge from the dynamics of a single continuous underlying entity,
  denoted $\chi$, whose effective descriptions arise through a constrained
  projection process.
  This projection is generically non-injective, allowing distinct underlying
  $\chi$-configurations to correspond to identical effective observables and,
  conversely, allowing a single underlying configuration to admit multiple
  correlated effective realizations.
  A detailed and formal treatment of this projection asymmetry is given in
  Section~\ref{subsec:projection-reality-and-ontological-asymmetry}.

  The substrate $\chi$ is not defined on a pre-existing spacetime manifold, nor is
  it interpreted as a conventional physical field propagating within spacetime.
  Instead, spacetime notions themselves arise as effective and relational
  descriptions, applicable only once suitable stability and projection conditions
  are satisfied.
  The precise ontological status of $\chi$ and the minimal assumptions governing
  its dynamics are introduced systematically in
  Section~\ref{sec:definition-and-fundamental-properties-of-the-chi-field}.

  The fundamental dynamical postulate of Cosmochrony is that $\chi$ undergoes an
  irreversible relaxation process, locally bounded by an invariant structural
  propagation speed.
  The effective projection of this bound defines the observed causal limit $c$ and
  induces an intrinsic ordering of physical processes, identified with physical
  time.
  Spatial relations emerge relationally from differences, gradients, and
  correlations of $\chi$ once a stable geometric regime is reached.
  Within this perspective, spacetime expansion, gravitation, particle-like
  excitations, radiation processes, and quantum correlations are not fundamental
  ingredients, but emergent phenomena associated with specific configurations or
  interactions of the underlying substrate.
  In particular, discreteness, inertial mass, and quantum indeterminacy are shown
  later to arise from structural constraints on projection and relaxation, rather
  than from independent postulates.

  Cosmochrony does not aim to replace the Standard Model or general relativity in
  their empirically validated domains, nor does it claim to provide a final
  unification of quantum theory and gravitation.
  Instead, it offers an exploratory and internally coherent framework designed to
  clarify the physical origin of time, geometry, gravitation, and quantum
  correlations within a single relational dynamics.
  Standard geometric and quantum formalisms are recovered only at an effective,
  coarse-grained level, applicable when $\chi$ admits a stable spacetime
  interpretation.

  Accordingly, quantities such as coordinates, metric structure, variational
  principles, and differential geometry are not treated as fundamental.
  They are employed later in the paper as emergent descriptive tools, rather than
  as primary postulates of the theory.
  Technical reconstructions and mathematical details are therefore confined to the
  appropriate effective regimes and collected in the appendices.

  The unifying thread of the framework is the idea that apparent multiplicity,
  indeterminacy, and nonlocality reflect structural features of projection, rather
  than fundamental physical randomness or superluminal dynamics.

  The structure of the paper is as follows.
  Sections~2--4 introduce the conceptual motivations and minimal dynamical
  assumptions governing the $\chi$ substrate.
  Subsequent sections examine how particle-like excitations, gravitation, quantum
  correlations, and cosmological behavior emerge in appropriate regimes.

  \subsection{Conceptual Context and Related Approaches}
  \label{subsec:conceptual-context-and-related-approaches}

  The idea that spacetime geometry and gravitation may be emergent rather than fundamental
  has been explored in a variety of contemporary theoretical frameworks.
  Several approaches interpret the spacetime metric as an effective description arising from
  deeper geometric, informational, or dynamical structures, and recast gravitation as a
  collective or emergent phenomenon rather than as a fundamental interaction~\cite{Nye2024,Singh2025}.

  Cosmochrony belongs to this broad conceptual lineage, while adopting a deliberately minimalist ontological stance.
  Rather than postulating multiple underlying structures or microscopic degrees of freedom,
  it assumes a single pre-geometric relational substrate, denoted $\chi$, whose irreversible
  relaxation governs the emergence of physical observables.

  Like Loop Quantum Gravity (LQG), Cosmochrony holds that spacetime geometry is not
  fundamental~\cite{Rovelli2004}.
  However, the two frameworks operate at distinct conceptual and ontological levels.

  LQG provides a quantized description of geometry once a spacetime structure is already
  assumed, encoding areas and volumes through spin networks and holonomies.
  In this sense, it addresses the quantization of geometric degrees of freedom defined on a
  kinematical spacetime arena.

  Cosmochrony, by contrast, addresses an earlier and more primitive level.
  It does not quantize geometry, but seeks to explain how geometric notions themselves arise
  as effective, coarse-grained descriptions of underlying $\chi$-configurations.
  The emergence of spacetime is mediated by a non-injective projection from the pre-geometric
  substrate to effective observables, allowing geometric, dynamical, and quantum features to
  appear only once specific relational and spectral conditions are met.

  From this perspective, Cosmochrony does not compete with LQG but conceptually precedes it.
  It aims to account for the physical origin of the geometric degrees of freedom that may
  subsequently be quantized within approaches such as LQG, while remaining agnostic about
  the detailed form of their quantization at the effective level.


  For convenience, a glossary summarizing the main quantities and operators used
  throughout the article is provided in Appendix~\ref{appendix:glossary}.
