\subsection{Parameter-Independent Relaxation}
  \label{subsec:parameter-independent-relaxation}

  To avoid the conceptual pitfalls of a fundamental time coordinate, we define the dynamics not as a function of an
  absolute variable, but as a sequence of field configurations $(\chi_\lambda)$ where $\lambda$ is a strictly monotonic
  ordering parameter.
  At the fundamental level, the evolution of $\chi$ is defined by its relaxation flow:
  \begin{equation}
    \frac{d\chi}{d\lambda} = \mathcal{R}(\chi, \nabla\chi)
  \end{equation}
  where $\mathcal{R}$ represents the rate of geometric tension release toward equilibrium.
  The ``temporal'' derivative $\partial_t$ used in subsequent sections is then understood as a convenient
  reparametrization of this primary flow:
  \begin{equation}
    \partial_t \chi \equiv \frac{d\chi}{d\lambda} \frac{d\lambda}{dt}
  \end{equation}
  Since $t$ merely serves to label the relaxation ordering, all physical results are invariant under any monotonic
  reparameterization $t \rightarrow f(t)$ with $f'(t) > 0$.

  In this framework, the relaxation of $\chi$ is not occurring \textit{in} time; rather, the local rate of relaxation
  \textit{defines} the physical measure of time.
  What we perceive as duration is the cumulative displacement of the field toward its equilibrium state.
  Consequently, the ``speed of time'' is locally determined by the density of $\chi$-gradients, providing a direct
  link between field topology and temporal flow.
