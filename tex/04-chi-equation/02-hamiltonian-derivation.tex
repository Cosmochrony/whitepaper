\subsection{Hamiltonian Derivation of the Evolution Equation}
  \label{subsec:hamiltonian-derivation}

  \subsubsection*{Discrete Dynamics of $\chi$}
    Before introducing any metric structure, the dynamics of $\chi$ can be formulated purely in terms of its local relaxation on a discrete network.
    Let $\lambda$ be a monotonic ordering parameter, and define the local variation of $\chi$ at node $i$ as:
    \begin{equation}
      \frac{d \chi_i}{d \lambda} = c \sqrt{1 - \frac{1}{c^2} \sum_{j \sim i} K_{ij} (\chi_i - \chi_j)^2}
      \label{eq:discrete-dynamics}
    \end{equation}
    where $K_{ij}$ is the connectivity strength defined in Equation~\eqref{eq:Kij-def}, and the sum runs over neighboring nodes $j$.
    This equation is \textbf{metric-independent} and defines the fundamental dynamics of $\chi$ in terms of its local correlations.

    To illustrate the physical implications, consider a \textbf{homogeneous background} where $\chi_i = \chi_0 + \delta \chi_i$ with $|\delta \chi_i| \ll \chi_0$. Expanding $K_{ij}$ to first order in $\delta \chi_i$:
    \begin{equation}
      K_{ij} \approx K_0 \left(1 - \frac{|\delta \chi_i - \delta \chi_j|^2}{\chi_c^2}\right)
    \end{equation}
    Substituting into the evolution equation yields:
    \begin{equation}
      \frac{d \chi_i}{d \lambda} \approx c \left[1 - \frac{K_0}{2 c^2 \chi_c^2} \sum_{j \sim i} (\delta \chi_i - \delta \chi_j)^2 \right]
    \end{equation}
    This shows that \textbf{localized perturbations} (e.g., particles) induce a \textbf{slowdown in the relaxation of $\chi$}, which manifests macroscopically as \textbf{gravitational time dilation} (see Section~\ref{sec:gravity-emergence}).

  \subsubsection*{Continuum Limit}
    In the limit where the network becomes dense (i.e., the distance between nodes approaches zero), the discrete sum can be approximated by a continuous Laplacian:
    \begin{equation}
      \sum_{j \sim i} K_{ij} (\chi_i - \chi_j)^2 \approx \int |\nabla \chi|^2 \, dV
    \end{equation}
    This yields the effective evolution equation in the continuum:
    \begin{equation}
      \partial_t \chi = c \sqrt{1 - \frac{|\nabla \chi|^2}{c^2}}
      \label{eq:chi_dynamics}
    \end{equation}
    where $\nabla$ is now defined with respect to the emergent metric $g_{\mu\nu}$, derived from the network's connectivity (see Section~\ref{subsec:geometric-action}).

  \subsubsection*{Continuum Limit and Newtonian Gravity}
    In the continuum limit, the discrete sum in Equation~\eqref{eq:discrete-dynamics} becomes a spatial integral, and the evolution equation reduces to:
    \begin{equation}
      \partial_t \chi = c \sqrt{1 - \frac{|\nabla \chi|^2}{c^2}}
    \end{equation}
    The \textbf{source term} $S[\chi, \rho]$ in the effective wave equation (Section~\ref{subsec:effective-evolution-equation}) can now be derived explicitly from the discrete coupling $K_{ij}$.
    For a localized excitation (e.g., a particle of mass $m$), the spatial variation of $\chi$ induces a \textbf{modulation of $K_{ij}$}, which in turn slows the relaxation rate.
    This effect is captured by the \textbf{nonlinear Poisson equation}:
    \begin{equation}
      \nabla \cdot \left( \frac{\nabla \chi}{\sqrt{1 - |\nabla \chi|^2 / c^2}} \right) = \frac{4 \pi G}{c^2} \rho
    \end{equation}
    where $G$ emerges as an \textbf{effective coupling constant} related to $K_0$ and $\chi_c$. In the weak-field limit ($|\nabla \chi| \ll c$), this reduces to the standard Poisson equation:
    \begin{equation}
      \nabla^2 \Phi = 4 \pi G \rho
    \end{equation}
    with the gravitational potential $\Phi$ identified as:
    \begin{equation}
      \Phi = c^2 \ln \left( \frac{\partial_t \chi}{c} \right)
    \end{equation}
    This derivation shows that \textbf{Newtonian gravity} emerges naturally from the \textbf{collective coupling $K_{ij}$} without postulating a fundamental metric or curvature.
    The full derivation, including the mapping between $K_0$, $\chi_c$, and $G$, is provided in Appendix~\ref{subsec:collective-coupling}.

  \subsubsection*{Hamiltonian Constraint}
    While the dynamics of $\chi$ can be viewed as a minimal relaxation principle, it can be more rigorously derived from
    a Hamiltonian constraint.
    We postulate that the dynamics of $\chi$ are governed by a Dirac-type kinematic constraint in phase space, analogous
    to the mass-shell condition for a massless relativistic particle:
    \begin{equation}
    (\partial_t \chi)^2 + |\nabla \chi|^2 = c^2
      \label{eq:hamiltonian_constraint}
    \end{equation}
    where $c$ is the fundamental velocity scale.
    Combined with the arrow of time postulate ($\partial_t \chi \geq 0$), which reflects the irreversible relaxation of
    the Cosmochron, this leads uniquely to the first-order evolution equation:
    \begin{equation}
      \partial_t \chi = c \sqrt{1 - \frac{|\nabla \chi|^2}{c^2}}
    \end{equation}
    This derivation grounds the ``minimal principle'' in the symplectic structure of the field's phase space, ensuring
    that $\chi$ acts as an intrinsic time coordinate.
