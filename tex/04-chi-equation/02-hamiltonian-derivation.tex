\subsection{Hamiltonian Derivation of the Evolution Equation}
  \label{subsec:hamiltonian-derivation}

  While the dynamics of $\chi$ can be viewed as a minimal relaxation principle, it can be more rigorously derived from a
  Hamiltonian constraint.
  We postulate that the dynamics of $\chi$ are governed by a Dirac-type kinematic constraint in phase space, analogous
  to the mass-shell condition for a massless relativistic particle:
  \begin{equation}
  (\partial_t \chi)
    ^2 + |\nabla \chi|^2 = c^2,
    \label{eq:hamiltonian_constraint}
  \end{equation}
  where $c$ is the fundamental velocity scale.
  Combined with the \textit{arrow of time} postulate ($\partial_t \chi \geq 0$), which reflects the irreversible
  relaxation of the Cosmochron, this leads uniquely to the first-order evolution equation:
  \begin{equation}
    \partial_t \chi = c \sqrt{1 - \frac{|\nabla \chi|^2}{c^2}}.
    \label{eq:chi_dynamics}
  \end{equation}

  This derivation grounds the ``minimal principle'' in the symplectic structure of the field's phase space, ensuring
  that $\chi$ acts as an intrinsic time coordinate.

  The mathematical stability of this equation is demonstrated in~\ref{subsec:stability_chi}, while explicit analytical
  solutions are derived in~\ref{subsec:analytical_solutions_chi}.
  The coupling of the $\chi$ field with matter, extending this kinematic backbone to a dynamical theory, is further
  discussed in~\ref{subsec:coupling_matter_chi} and Section~\ref{subsec:variational-formulation}.
