\subsection{Microscopic Origin of the Coupling Tensor and the Poisson Equation}
  \label{subsec:microscopic-origin-of-the-coupling-tensor-and-the-poisson-equation}

  For internal consistency, the effective coupling governing the relaxation of $\chi$
  cannot be treated as a fixed universal constant.
  Instead, it must depend on the internal structural state of the field, reflecting how
  local configurations resist or facilitate relaxation.
  In Cosmochrony, this dependence is captured through a constitutive relation linking
  the effective coupling strength to internal variations of $\chi$, without invoking
  any underlying spatial substrate.

  A convenient phenomenological parametrization of this dependence is given by
  \begin{equation}
    K_{\mathrm{eff}}
    = K_0 \exp\!\left(-\frac{(\Delta \chi)^2}{\chi_c^2}\right),
    \label{eq:effective_coupling_tensor}
  \end{equation}
  where $\Delta\chi$ denotes a measure of internal variation of $\chi$ between correlated
  configurations, $K_0$ characterizes the maximal relaxation conductivity in a homogeneous
  background, and $\chi_c$ sets the characteristic scale beyond which structural
  inhomogeneities significantly reduce the effectiveness of relaxation.

  Configurations exhibiting strong internal variation of $\chi$, such as stable
  solitonic excitations, therefore reduce the effective coupling and locally slow the
  relaxation process.
  This reduction does not represent an additional interaction, but reflects the
  intrinsic resistance of structured configurations to further relaxation.
  The resulting slowdown constitutes the microscopic origin of the emergent
  gravitational phenomenology discussed in the previous section.

  In regimes where a spacetime description becomes applicable, the local relaxation rate
  $\mathcal{D}_{\mathrm{loc}}\chi$ differs from its asymptotic value
  $\mathcal{D}_0$ far from localized configurations.
  An effective gravitational potential $\Phi$ may then be introduced as a descriptive
  parameter through the relation
  \begin{equation}
    \frac{\mathcal{D}_{\mathrm{loc}}\chi}{\mathcal{D}_0}
    \simeq 1 + \frac{\Phi}{c^2},
    \label{eq:relaxation_potential_relation}
  \end{equation}
  which summarizes the relative slowdown of relaxation in a form familiar from
  classical gravitational phenomenology.

  In the weak-structure regime, where internal variations of $\chi$ remain small
  compared to $\chi_c$, the distribution of $\Phi$ admits a simplified elliptic
  description.
  At this coarse-grained level, the effective dynamics reduce to a Poisson-type
  relation,
  \begin{equation}
    \nabla^2 \Phi \simeq 4\pi G_{\mathrm{eff}} \rho,
    \label{eq:effective_poisson_equation}
  \end{equation}
  where $\rho$ denotes the density of localized, relaxation-resistant configurations and
  $G_{\mathrm{eff}}$ is an emergent coupling parameter encoding the collective response
  of the $\chi$ relaxation dynamics.

  This Poisson equation is not fundamental.
  It represents the weak-field, macroscopic limit of the constrained relaxation
  dynamics of $\chi$, expressed in a form adapted to effective geometric description.
  Gravitation therefore appears not as an independent interaction, but as a descriptive
  manifestation of reduced relaxation conductivity induced by structured $\chi$
  configurations.

  A fully relational formulation, consistent with but not required for the effective
  description adopted here, is provided in
  Appendix~\ref{app:relational_formulation}.
