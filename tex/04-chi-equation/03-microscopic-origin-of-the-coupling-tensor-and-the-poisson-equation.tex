\subsection{Microscopic Origin of the Coupling Tensor and the Poisson Equation}
  \label{subsec:microscopic-origin-of-the-coupling-tensor-and-the-poisson-equation}

  To achieve self-consistency, the coupling tensor $K_{ij}$ must not be a mere constant but a dynamical measure of the
  field's local stress.
  We propose that $K_{ij}$ depends on the local gradient of $\chi$ through a non-linear constitutive relation.
  A physically motivated form for this coupling is:

  \begin{equation}
    K_{ij} = K_0 \exp\left( -\frac{(\chi_i - \chi_j)^2}{\chi_c^2} \right)
  \end{equation}

  where $K_0$ is the vacuum coupling constant and $\chi_c$ is a characteristic field scale.
  This scenario implies that regions of high field gradients (solitons) locally weaken the coupling, thereby slowing
  down the relaxation rate.

  To derive the Newtonian limit, consider the continuous limit of the evolution equation (Eq. 34).
  In the presence of a localized excitation (soliton) of mass $M$, the local relaxation rate $v = \partial_t \chi$ is
  modified.
  Let $v_0$ be the global relaxation rate far from any mass.
  The local potential $\Phi$ can be identified through the relative shift in the relaxation velocity:

  \begin{equation}
    \frac{v(r)}{v_0} \approx 1 + \frac{\Phi(r)}{c^2}
  \end{equation}

  By applying the Taylor expansion to the discrete sum $\sum_j K_{ij}(\chi_i - \chi_j)^2$ and assuming the form of
  $K_{ij}$ from our scenario, the discrete evolution equation converges to a non-linear Poisson-like equation:

  \begin{equation}
    \nabla^2 \Phi - \frac{1}{\chi_c^2} (\nabla \Phi)^2 = 4\pi G \rho
  \end{equation}

  In the weak-field limit ($\Phi \ll c^2$), the second-order term becomes negligible, and we recover the standard
  Poisson equation $\nabla^2 \Phi = 4\pi G \rho$.
  This derivation proves that gravitation in Cosmochrony is not an external force but a direct consequence of the local
  reduction in the $\chi$ field's ``relaxation conductivity'' induced by the presence of solitonic configurations.
