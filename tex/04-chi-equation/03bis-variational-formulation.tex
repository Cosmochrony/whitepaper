\subsection{Variational Formulation and Born-Infeld Action}
  \label{subsec:variational-formulation}

  To extend the kinematic constraint introduced above into an effective dynamical description
  that incorporates matter sources, we introduce a variational framework inspired by
  Born--Infeld--type non-linear actions, originally developed to regularize field singularities
  in classical field theories~\cite{BornInfeld1934,DeserGibbons1998}.
  \begin{equation}
    \mathcal{L} = -c^2 \sqrt{1 - \frac{|\nabla \chi|^2}{c^2}} + \partial_t \chi - \frac{4\pi G}{c^2} \rho \chi ,
  \end{equation}
  where $\rho$ represents the matter density. The presence of the term $\partial_t \chi$ linear in the first-order
  temporal derivative is crucial: it ensures that the momentum conjugate to $\chi$, defined as
  $\Pi_\chi = \frac{\partial \mathcal{L}}{\partial (\partial_t \chi)}$, is a non-vanishing constant ($\Pi_\chi = 1$).

  In the present framework, the Born--Infeld--type square-root structure does not introduce additional propagating
  degrees of freedom.
  Instead, it acts as a non-linear regulator that enforces an upper bound on spatial gradients, ensuring consistency
  with the fundamental kinematic constraint while avoiding singular field configurations, in direct analogy with
  the original motivation of Born--Infeld electrodynamics~\cite{BornInfeld1934}.

  In the Hamiltonian formalism, this constant momentum acts as a primary constraint that effectively enforces the
  unit-velocity evolution of the field.
  This structure ensures that the field dynamics remain locked onto the Hamiltonian
  constraint~\eqref{eq:hamiltonian_constraint} while the square-root term acts as a non-linear regularizer for spatial
  gradients.
  The variation with respect to $\chi$ yields a non-linear Poisson equation:
  \begin{equation}
    \nabla \cdot \left( \frac{\nabla \chi}{\sqrt{1 - |\nabla \chi|^2/c^2}} \right) = \frac{4\pi G}{c^2} \rho .
    \label{eq:nonlinear_poisson}
  \end{equation}
  This formulation naturally recovers the Newtonian limit for weak gradients ($|\nabla \chi| \ll c$) while preventing
  gravitational singularities as the gradient magnitude is bounded by $c$.
