\subsection{Structural Interpretation of CPT Symmetry}
  \label{subsec:structural-interpretation-of-cpt-symmetry}

  Let \(\Pi\) denote the non-injective projection from the \(\chi\)-substrate to effective
  descriptions.
  Projected configurations may carry signed invariants \(Q\) associated with relational
  orientation, chirality, or phase winding.
  These invariants are not defined at the fundamental level but emerge through projection.

  Consider the combined transformation
  \[
    (Q,\;\tau,\;\mathbf{x}) \;\longrightarrow\; (-Q,\;-\tau,\;-\mathbf{x}),
  \]
  where \(\tau\) denotes the effective ordering parameter and \(\mathbf{x}\) the effective
  spatial localization.
  This transformation leaves the admissibility conditions invariant.

  Under admissible factorization of a metastable configuration \(\chi_A\), conservation of
  signed invariants requires
  \[
    Q(\chi_A) = \sum_i Q(\chi_i).
  \]
  If local configurations carry nonzero signed contributions while the global invariant
  vanishes, admissibility enforces the appearance of pairs with opposite signs.

  These paired configurations are interpreted as particle–antiparticle pairs.
  CPT symmetry thus emerges as an invariance of the admissible projection structure,
  rather than as a fundamental symmetry imposed on the \(\chi\)-substrate.

  \begin{figure}[t]
    \centering
    \begin{tikzpicture}[
      box/.style={draw, rounded corners, align=center, inner sep=6pt},
      arr/.style={->, thick},
      lab/.style={font=\small, align=center}
    ]

% Chi substrate
      \node[box] (chi) {$\chi$\\[-2pt]\footnotesize relational configuration};

% Projection
      \node[box, below=1.6cm of chi] (proj)
      {$\Pi(\chi)$\\[-2pt]\footnotesize non-injective\\[-2pt]\footnotesize projected description};

% Branches
      \node[box, below left=2.2cm and 2.4cm of proj] (ent)
      {Entanglement\\[-2pt]\footnotesize non-factorizable\\[-2pt]\footnotesize stable};

      \node[box, below=2.2cm of proj] (meas)
      {Measurement / decoherence\\[-2pt]\footnotesize selection\\[-2pt]\footnotesize without fragmentation};

      \node[box, below right=2.2cm and 2.4cm of proj] (dec)
      {Decay\\[-2pt]\footnotesize factorization\\[-2pt]\footnotesize required};

% Decay products
      \node[box, below=1.8cm of dec, xshift=-2.0cm] (p)
      {Particle\\[-2pt]\footnotesize $+Q$};

      \node[box, below=1.8cm of dec, xshift=0.0cm] (ap)
      {Antiparticle\\[-2pt]\footnotesize $-Q$};

      \node[box, below=1.8cm of dec, xshift=2.0cm] (rad)
      {Light modes\\[-2pt]\footnotesize dissipation};

% Arrows
      \draw[arr] (chi) -- node[lab, right] {projection} (proj);

      \draw[arr] (proj) -- (ent);
      \draw[arr] (proj) -- (meas);
      \draw[arr] (proj) -- (dec);

      \draw[arr] (dec) -- (p);
      \draw[arr] (dec) -- (ap);
      \draw[arr] (dec) -- (rad);

% Notes
      \node[lab, below=0.4cm of ent] {unity preserved};
      \node[lab, below=0.4cm of meas] {stabilization};
      \node[lab, below=0.4cm of dec] {invariants redistributed};

    \end{tikzpicture}
    \caption{Unified structural interpretation of quantum entanglement, measurement, and
    particle decay in Cosmochrony.
    All phenomena originate from the non-injective projection of a single relational
    configuration of \(\chi\).
    Antiparticles emerge when admissible factorization requires the redistribution of
    signed structural invariants.}
    \label{fig:projection-entanglement-decay}
  \end{figure}
