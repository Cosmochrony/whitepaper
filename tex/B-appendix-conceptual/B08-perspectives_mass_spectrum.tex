\subsection{Perspectives: Towards a Derivation of the Proton-to-Electron Mass Ratio}
  \label{subsec:perspectives_mass_spectrum}

  The proton-to-electron mass ratio is one of the most precisely measured dimensionless quantities in physics.
  Within the Cosmochrony framework, the purpose of this section is not to derive this
  value from first principles, but to clarify how such a ratio could emerge
  \emph{structurally} from the spectral and topological organization of localized
  solitonic excitations of the projected field \(\chi_{\mathrm{eff}}\).

  The discussion should therefore be understood as a minimal and exploratory spectral ansatz.
  Its aim is to identify the relevant mechanisms, constraints, and scaling relations
  that any successful derivation would have to satisfy, rather than to provide a complete microscopic calculation.

  \subsubsection*{Spectral Stability Hypothesis}

    Let \(\chi_{\mathrm{sol}}\) denote a stationary localized configuration arising in
    the projectable regime of the \(\chi\) dynamics.
    Small perturbations \(\delta\chi_{\mathrm{eff}}\) around this background are
    governed, at the coarse-grained level, by a linear stability operator
    \(\mathcal{L}_{\mathrm{sol}}\), defined as the second variation of an effective localization functional.

    Normal modes satisfy the eigenvalue problem
    \begin{equation}
      \mathcal{L}_{\mathrm{sol}} \psi_n = \lambda_n \psi_n .
    \end{equation}

    The eigenvalues \(\lambda_n\) characterize the resistance of the soliton to
    localized deformations.
    They encode intrinsic stiffness scales associated with the internal organization
    of the solitonic configuration.

    \paragraph{Spectral mass scaling.}
      In regimes where an effective wave description applies, the normal modes exhibit
      characteristic oscillation frequencies
      \begin{equation}
        \omega_n = c \sqrt{\lambda_n} .
      \end{equation}

      Identifying the lowest nontrivial frequency with the rest energy of the excitation
      leads to the effective scaling relation
      \begin{equation}
        m_n \;\propto\; \sqrt{\lambda_n}\,\chi_c ,
      \end{equation}
      where \(\chi_c\) denotes a characteristic geometric scale associated with the
      spatial extension of the solitonic configuration in the projected regime.

      This relation does not define a fundamental mass formula.
      It provides a coarse-grained link between spectral stability and inertial mass,
      consistent with the interpretation of mass as integrated resistance to relaxation.

    \paragraph{Dimensional interpretation.}
      The eigenvalues \(\lambda_n\) carry dimensions of inverse length squared,
      reflecting the restoring stiffness of the soliton per unit deformation.
      The scale \(\chi_c\) has dimensions of length and sets the geometric extension over
      which this stiffness is distributed.

      The combination \(\lambda_n \chi_c^2\) therefore defines a characteristic energy scale,
      \begin{equation}
        E_n \sim \lambda_n \chi_c^2 ,
      \end{equation}
      which is identified with a rest energy through the effective relativistic matching \(E = mc^2\) once a spacetime
      description becomes applicable.

      This identification does not invoke a fundamental quantum constant and remains
      valid independently of the emergence of \(\hbar_{\mathrm{eff}}\).

  \subsubsection*{Projection Scale and Effective Normalization}

    The fundamental description of the \(\chi\) field is formulated in terms of
    relational relaxation rules rather than a spacetime action with fixed physical units.
    When a continuum approximation applies, an effective action for perturbations
    around a stable soliton may be introduced as a bookkeeping device.

    In this regime, the effective action for perturbations
    \(\delta\chi_{\mathrm{eff}}\) may be written schematically as
    \begin{equation}
      S_{\mathrm{eff}}[\delta\chi]
      =
      \int d^4x \,
      \frac{1}{2}
      \left( \frac{\chi_c}{c} \right)^2
      \left[
        (\partial_t \delta\chi)^2
      -
      c^2 (\nabla \delta\chi)^2
      \right] .
    \end{equation}

    Expressing this action in emergent spacetime coordinates introduces a geometric
    rescaling factor linking \(\chi\)-space and spacetime lengths.
    As a result, the canonical normalization of localized modes involves a quadratic
    scaling factor of the form
    \begin{equation}
      \left( \frac{\chi_c}{\ell_{\mathrm{spacetime}}} \right)^2 ,
    \end{equation}
    which controls the effective normalization of spectral quantities.

    This factor reflects the geometric projection from the relational \(\chi\)
        structure to emergent spacetime observables.
        It does not represent a fundamental coupling constant.

  \subsubsection*{Energy Levels from Spectral Stability}

    The discrete energy levels associated with solitonic excitations follow from the
    spectral properties of the stability operator \(\mathcal{L}_{\mathrm{sol}}\), not
    from canonical quantization.

    For a soliton labeled by \(n\), the gradient contribution to the effective energy scales as
    \begin{equation}
      E_{\mathrm{grad}}^{(n)}
      \sim
      c^2 \lambda_n \mathcal{N}_n ,
    \end{equation}
    where \(\mathcal{N}_n\) denotes a normalization factor determined by the spatial profile of the mode.

    In the spacetime-based description, this energy is identified with the rest-mass energy,
    \begin{equation}
      E_n \equiv m_n c^2 .
    \end{equation}

    The discretization of \(E_n\) arises from topological classification and spectral
    stability, not from postulated quantum operators.
    The role of \(\hbar_{\mathrm{eff}}\) appears only when matching this description to
    quantum observables.

  \subsubsection*{Elementary versus Composite Spectral Structures}

    A key distinction must be drawn between elementary and composite solitonic
    excitations.
    Elementary particles, such as leptons, are expected to correspond to topologically
    elementary solitons whose inertial mass is dominated by a single lowest stability
    eigenvalue.

    By contrast, baryonic excitations are composite configurations.
    Their mass reflects the combined contribution of several coupled stability modes
    associated with a bound structure.
    Mass ratios therefore take the schematic form
    \begin{equation}
      \frac{m_{\mathrm{comp}}}{m_{\mathrm{elem}}}
      \;\sim\;
      \frac{\sum_k \sqrt{\lambda^{(\mathrm{comp})}_k}}
      {\sqrt{\lambda^{(\mathrm{elem})}_0}} ,
    \end{equation}
    rather than the ratio of two isolated eigenvalues.

  \subsubsection*{Ansatz for the Proton as a Composite Soliton}

    As an exploratory working hypothesis, the proton is modeled as a composite solitonic excitation.
    Specifically:
    \begin{itemize}
      \item the electron corresponds to a topologically elementary soliton with a
      fundamental stability eigenvalue \(\lambda_e\),
      \item the proton corresponds to a bound configuration involving three such
      elementary solitons, supplemented by an additional collective binding mode with
      eigenvalue \(\lambda_{\mathrm{bind}}\).
    \end{itemize}

    The choice of a three-soliton composite is motivated by stability considerations
    observed in a wide class of nonlinear field theories admitting topological solitons,
    where three-body bound states often exhibit enhanced stability due to geometric
    phase locking~\cite{BattyeSutcliffe2022}.
    This choice is not derived here from a classification of \(\chi\)-soliton sectors
    and is not postulated as fundamental.

    Skyrmion models in QCD provide an instructive analogy, but no dynamical equivalence is assumed.
    The relevance of this analogy lies in the universality of topological stabilization
    mechanisms, which do not depend on the presence of a non-Abelian gauge symmetry~\cite{MantonSutcliffe2004}.

  \subsubsection*{Mass Ratio from Spectral Scaling}

    Under these assumptions, the effective eigenvalue associated with the proton may be
    written schematically as
    \begin{equation}
      \lambda_p \;\approx\; \lambda_{\mathrm{bind}} + 3\lambda_e ,
    \end{equation}
    leading to the mass ratio
    \begin{equation}
      \frac{m_p}{m_e}
      \;\approx\;
      \sqrt{\frac{\lambda_{\mathrm{bind}} + 3\lambda_e}{\lambda_e}} .
    \end{equation}

    In the binding-dominated regime \(\lambda_{\mathrm{bind}} \gg \lambda_e\), this
    reduces to
    \begin{equation}
      \frac{m_p}{m_e}
      \;\approx\;
      \sqrt{\frac{\lambda_{\mathrm{bind}}}{\lambda_e}} .
    \end{equation}

    Matching the observed ratio \(m_p/m_e \simeq 1836\) therefore imposes the spectral
    constraint
    \begin{equation}
      \frac{\lambda_{\mathrm{bind}}}{\lambda_e}
      \;\sim\; 3.4 \times 10^{6}.
    \end{equation}

    This relation is not derived here.
    It is identified as a consistency condition constraining the relative spectral
    organization of elementary and composite solitonic sectors.

    We interpret this large spectral hierarchy as defining a dimensionless
    \emph{spectral packing fraction} $\alpha$, characterizing the relative
    density of admissible stability modes in composite versus elementary
    solitonic sectors.
    Specifically, we define
    \begin{equation}
      \alpha \;\equiv\; \frac{\lambda_e}{\lambda_{\mathrm{bind}}}
      \;\sim\; 3 \times 10^{-7}.
    \end{equation}
    This quantity does not represent a coupling constant, but a structural
    measure of spectral compression induced by topological binding.

    \paragraph{Topological Interpretation of the Spectral Hierarchy}

      Although the ratio $\lambda_{\mathrm{bind}}/\lambda_e$ is introduced here as a spectral
      consistency condition, it is natural to seek a geometric or topological interpretation
      of this large hierarchy.

      In particular, the composite nature of the proton ($Q=3$) suggests that the associated
      binding modes may correspond to configurations of increased topological complexity.
      If the stability spectrum of $L_{\mathrm{sol}}$ is controlled by the effective multiplicity
      of internal configurations admitted under the non-injective projection $\Pi$,
      then $\lambda_{\mathrm{bind}}$ may be interpreted as a coarse-grained measure of the
      volume of the corresponding projection fiber.

      From this perspective, the large ratio
      $\lambda_{\mathrm{bind}}/\lambda_e \sim 10^{6}$
      reflects not an arbitrary energy scale separation, but the rapid growth of internal
      configuration space associated with topologically composite solitons.

    \paragraph{Indicative Geometric Scale}

      Although no explicit geometric or topological model is developed at this stage,
      it is useful to translate the observed spectral hierarchy into a characteristic
      dimensionless scale.

      At a purely heuristic level, one may assume that the effective spectral weight of a
      composite soliton grows quadratically with a characteristic internal scale $\chi_c$,
      so that
      \begin{equation}
        \frac{\lambda_{\mathrm{bind}}}{\lambda_e} \sim \chi_c^{2}.
      \end{equation}
      Under this assumption, the empirical constraint
      $\lambda_{\mathrm{bind}}/\lambda_e \sim 3.4 \times 10^{6}$
      corresponds to a scale of order
      \begin{equation}
        \chi_c \sim \mathcal{O}(10),
      \end{equation}
      with a representative numerical value
      \begin{equation}
        \chi_c \approx 8.3.
      \end{equation}

      Both expressions should be regarded as indicative rather than derived.
      They simply emphasize that the required spectral hierarchy corresponds to a modest
      geometric amplification, not to an extreme or finely tuned parameter choice.

  \subsubsection*{An Explicit Working Ansatz for \(V(\chi)\)}
    \label{subsec:explicit_Vchi_ansatz}

    In the present appendix, the primary driver of mass hierarchies is the spectral organization
    of the solitonic stability operator. Nevertheless, turning this program into a falsifiable
    computational scheme requires an explicit \emph{working} form for the effective potential
    \(V(\chi)\), not as a fundamental source of masses, but as a controlled perturbation that:
    (i) stabilizes localized sectors, (ii) selects admissible core amplitudes, and (iii) produces
    fine splittings within a given topological class.

    A minimal two-scale ansatz compatible with these roles is a \emph{multi-well} (or weakly periodic)
    potential with a characteristic amplitude scale \(\eta\) and stiffness scale \(\lambda\):
    \begin{equation}
      V(\chi)
      \;=\;
      \frac{\lambda}{4}\,\big(\chi^2-\eta^2\big)^2
      \;+\;
      \varepsilon\,\eta^4\Big[1-\cos\!\Big(\frac{\chi}{\eta}\Big)\Big],
      \label{eq:Vchi_multiwell_periodic}
    \end{equation}
    where \(\varepsilon \ll 1\) is a dimensionless modulation parameter.
    The first term provides a robust double-well localization mechanism; the second term introduces
    a gentle quasi-periodic micro-structure capable of generating controlled intra-sector splittings
    without reparameterizing the global mass scale.

    The interpretation in Cosmochrony is strictly \emph{effective}: the coefficients in
    Eq.~\eqref{eq:Vchi_multiwell_periodic} are not fundamental constants, but phenomenological
    descriptors of how coarse-grained projectability constraints reshape the admissible configurations
    of \(\chi_{\mathrm{eff}}\).


  \subsubsection*{Linking \((\lambda,\eta)\) to Observables Without Making Mass Fundamental}
    \label{subsec:lambda_eta_to_observables}

    The parameters \(\eta\) and \(\lambda\) are introduced only to control the \emph{shape} and
    \emph{stiffness} of admissible localized sectors in the projected description. Their observable
    imprint is therefore indirect: they enter through how they shift the stability spectrum
    \(\{\lambda_n\}\) of \(\mathcal{L}_{\mathrm{sol}}\), and how robustly a given topological sector
    remains projectable under perturbations.

    \paragraph{Dimensionless control combinations.}
      For a localized profile with characteristic extension \(\chi_c\), the potential introduces
      two natural dimensionless combinations,
    \begin{equation}
      g \;\equiv\; \lambda\,\chi_c^2\,\eta^2,
      \qquad
      u \;\equiv\; \varepsilon\,,
      \label{eq:dimensionless_controls}
      \end{equation}
      which govern (i) the curvature scale of \(V(\chi)\) near admissible minima, and (ii) the magnitude
      of sub-structure corrections. The spectral hierarchy derived above is then phrased as the statement
      that the \emph{ratio} \(m_p/m_e\) is predominantly controlled by topological/composite spectral packing,
      while \((g,u)\) control the \emph{stability} and \emph{splittings} of the low-lying spectrum.

    \paragraph{Matching strategy using the proton-to-electron ratio.}
      Denote by \(\lambda_e(g,u)\) the fundamental stability eigenvalue of the \(Q=1\) sector, and by
      \(\lambda_{\mathrm{bind}}(Q=3;g,u)\) the characteristic binding-band scale of the composite sector.
      The empirical constraint
      \(\lambda_{\mathrm{bind}}/\lambda_e \sim 3.4\times 10^6\)
      derived in Eq.~(the spectral constraint above) is then reinterpreted as a \emph{feasibility condition}:
    \begin{equation}
      \exists\,(g,u)\ \text{s.t.}\quad
      \frac{\lambda_{\mathrm{bind}}(Q=3;g,u)}{\lambda_e(g,u)}
      \;\approx\; 3.4\times 10^6,
      \label{eq:feasibility_condition_lambda_eta}
      \end{equation}
      while remaining stable under small variations of \((g,u)\).
      In other words, \((\lambda,\eta)\) are not tuned to \emph{set} the mass ratio, but to ensure that the
      \emph{topological spectral mechanism} can realize the required hierarchy in a broad basin of effective
      parameters.

    \paragraph{Secondary observables: fine splittings as diagnostics.}
      Once a viable region in \((g,u)\) exists, the same potential ansatz predicts that small
      intra-sector differences (e.g. neutron--proton splitting, excited baryonic resonances, or
      generational splittings) arise from:
    \begin{itemize}
      \item perturbative eigenvalue shifts \(\delta\lambda_n(g,u)\) induced by local curvature variations of \(V\),
      \item weak breaking of idealized symmetries in composite sectors,
      \item environment-dependent dressing of the effective coefficients through projectability constraints.
      \end{itemize}
      These effects are conceptually aligned with the claim that \(V(\chi)\) controls fine structure rather
      than the global mass scale.


  \subsubsection*{Numerical Program: From \(V(\chi)\) to Spectral Hierarchies}
    \label{subsec:numerical_program_mass_ratio}

    The numerical goal is not to simulate QCD, but to test a \emph{structural} claim:
    whether bounded relaxation dynamics plus a controlled effective potential admits stable localized
    sectors whose stability spectra exhibit (i) a robust elementary mode \(\lambda_e\), and
    (ii) a dense binding band \(\lambda_{\mathrm{bind}}\) in a composite \(Q=3\) sector, separated by a large gap.

    A minimal computational pipeline is:
    \begin{enumerate}
      \item \textbf{Dynamics and formation.} Implement the bounded relaxation update rule for \(\chi\)
      in a discretized representation (spectral/finite-element basis or lattice proxy), including
      the effective potential term Eq.~\eqref{eq:Vchi_multiwell_periodic} as a controlled perturbation.
      \item \textbf{Soliton harvesting.} Identify long-lived localized configurations and classify them
      by topological diagnostics (winding/charge proxies, knot-like invariants when available, or
      stability under deprojection/reprojection cycles).
      \item \textbf{Stability operator extraction.} For each harvested configuration, compute the
      linearized stability operator \(\mathcal{L}_{\mathrm{sol}}\) (second variation of the effective
      localization functional) and extract its low-lying eigen-spectrum.
      \item \textbf{Spectral ratio test.} Evaluate whether the emergent spectra support a regime where
      \(\lambda_{\mathrm{bind}}/\lambda_e \sim 10^6\) arises \emph{without fine tuning}, and whether the
      ratio remains stable under moderate variation of \((g,u)\).
      \item \textbf{Fine-structure diagnostics.} Measure the sensitivity of subleading splittings
      \(\delta\lambda_n\) to \((g,u)\), providing a concrete handle for how \(V(\chi)\) affects
      intra-sector structure while leaving the leading hierarchy topologically controlled.
    \end{enumerate}

    This numerical program connects directly to the broader simulation framework described in the
    technical appendix on simulation algorithms and spectral extraction, and provides a clear set of
    falsifiable diagnostics: either large, robust spectral gaps appear generically in composite sectors,
    or the proposed topological-spectral mechanism fails to reproduce the required hierarchy.

  \subsubsection*{Transition}
    The role of \(V(\chi)\) is therefore operational: it stabilizes and perturbs admissible projected sectors
    while the \emph{origin} of the mass hierarchy remains spectral and topological. This separation is
    developed further in the subsequent appendices on spectral ontology and on the secondary role of \(V(\chi)\).
