\subsection{Perspectives: Towards a Derivation of the Mass Spectrum}
\label{subsec:perspectives_mass_spectrum}

While the identification of particles as topological solitons (Skyrmions, vortices) provides a qualitative mechanism for mass generation via the configuration energy of the $\chi$ field, the explicit derivation of the Standard Model mass spectrum—specifically the hierarchy of lepton and quark flavors—remains an open challenge. In the Cosmochrony framework, this spectrum should not be tuned by arbitrary coupling constants but should emerge from the intrinsic geometry of the network $G(V,E)$.

\subsubsection{The Geometric Resonance Hypothesis}
  We conjecture that observed masses correspond to the eigenvalues of a transfer operator on the discrete network, where the mass $m_n$ of a configuration $n$ follows a scaling law linked to the local curvature induced by the soliton:
  \begin{equation}
    m_n c^2 \approx E_{\text{fund}} \cdot \Lambda(Q_n, \mathcal{K})
  \end{equation}
  where $E_{\text{fund}}$ is a fundamental energy scale (potentially linked to the Planck scale or the global relaxation density), $Q_n$ is the topological charge (winding number), and $\mathcal{K}$ represents a curvature invariant of the network.

\subsubsection{Future Research Program}
  Transitioning toward a predictive theory of the mass spectrum requires:
  \begin{enumerate}
    \item \textbf{Discretization of the Potential $V(\chi)$}: Demonstrating how the minima of the potential on the network favor specific mass scales over others.
    \item \textbf{Network Eigenmode Analysis}: Investigating whether the flavor hierarchy (the three generations of particles) can be interpreted as higher-order harmonic modes of a single fundamental topological structure.
    \item \textbf{Numerical Simulations}: Implementing relaxation algorithms on large-scale graphs to verify if stable configurations spontaneously emerge with mass ratios corresponding to physical constants (e.g., the proton-to-electron mass ratio).
  \end{enumerate}

  This approach aims to transform the ``magic numbers'' of the Standard Model into geometric properties derivable from
  the first principles of relaxation dynamics.
