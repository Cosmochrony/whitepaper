\subsection{Perspectives: Towards a Derivation of the Proton-to-Electron Mass Ratio}
  \label{subsec:perspectives_mass_spectrum}

  The proton-to-electron mass ratio ($m_p/m_e \simeq 1836$) is one of the most
  precisely measured dimensionless constants in physics.
  Within the Cosmochrony framework, the aim of this section is not to derive this
  value from first principles, but to clarify how such a ratio could emerge from
  the spectral and topological structure of localized $\chi$-solitons.
  The discussion below should therefore be understood as a minimal spectral
  ansatz, intended to identify the relevant mechanisms and constraints rather than
  to provide a complete microscopic calculation.

  \subsubsection{Spectral Stability Hypothesis}

    Let $\chi_{\mathrm{sol}}$ denote a stationary localized configuration of the
    $\chi$ field.
    Small perturbations $\delta\chi$ around this background are governed, at the
    coarse-grained level, by a linear stability operator $\mathcal{L}_{\mathrm{sol}}$,
    defined as the second variation of an effective localization functional.
    Normal modes satisfy the eigenvalue problem
    \begin{equation}
      \mathcal{L}_{\mathrm{sol}} \psi_n = \lambda_n \psi_n .
    \end{equation}

    In regimes where an effective wave description applies, these modes exhibit
    oscillation frequencies $\omega_n = c\sqrt{\lambda_n}$.
    Identifying the lowest characteristic frequency with the rest energy of the
    excitation leads to the effective relation
    \begin{equation}
      m_n = \sqrt{\lambda_n}\,\chi_c ,
    \end{equation}
    where $\chi_c$ denotes a characteristic length scale associated with the spatial
    extent of the solitonic configuration.

    \paragraph{Dimensional interpretation of the spectral mass relation.}
      The relation
      \begin{equation}
        m_n = \sqrt{\lambda_n}\,\chi_c
      \end{equation}
      is dimensionally consistent but warrants a clarification of the physical origin of its units.
      The eigenvalues $\lambda_n$ arise from the linearized stability operator governing small
      deformations of a localized $\chi$-soliton.
      As such, they carry units of inverse length squared, $[\lambda_n] = L^{-2}$, reflecting a
      restoring stiffness per unit $\chi$-field deformation.

      The characteristic scale $\chi_c$ has dimensions of length and represents the effective
      geometric extension of the solitonic configuration along the $\chi$ direction.
      It therefore sets the spatial scale over which the deformation energy is distributed.

      Multiplying $\lambda_n$ by $\chi_c^2$ yields a quantity with dimensions of energy,
      \begin{equation}
        E_n \sim \lambda_n\,\chi_c^2,
      \end{equation}
      which can be interpreted as the characteristic energy stored in the corresponding
      eigenmode of the soliton.
      Using the relativistic identification $E = mc^2$, the associated inertial mass scales as
      \begin{equation}
        m_n \sim \frac{\lambda_n\,\chi_c^2}{c^2}
        \;\;\propto\;\; \sqrt{\lambda_n}\,\chi_c,
      \end{equation}
      up to numerical factors absorbed in the effective normalization of the relaxation operator.

      Thus, $\lambda_n$ encodes the energetic cost of deforming the soliton per unit length,
      while $\chi_c$ provides the geometric scale that converts this stiffness into a finite
      rest energy.
      The spectral relation $m_n = \sqrt{\lambda_n}\,\chi_c$ therefore reflects a balance between
      local resistance to $\chi$-field relaxation and the spatial extent of the localized
      configuration, consistent with the soliton energy functional discussed in
      Section~5.3.

    \paragraph{Dimensional consistency and effective scales.}

      The relation above should be understood as an effective parametrization at the
      coarse-grained level.
      Dimensional consistency follows directly from the geometric properties of the
      $\chi$ field and the invariant propagation speed $c$, without invoking any
      fundamental quantum constant.

      In this interpretation, $\lambda_n$ encodes a curvature or stiffness scale of the
      localized $\chi$ configuration, while $\chi_c$ sets the corresponding spatial
      extension over which this curvature is integrated.
      Their combination determines a characteristic rest energy via $E \sim \lambda_n
  \chi_c^2$, and hence an inertial mass through $E = mc^2$.

      Throughout the remainder of this section, the relation
      $m_n=\sqrt{\lambda_n}\,\chi_c$ is used as a convenient coarse-grained description,
      without implying that $\chi_c$ is a universal fundamental constant or that any
      quantum scale is postulated at the microscopic level.

  \subsubsection{Elementary versus Composite Spectral Structures}

    A crucial distinction must be made between elementary and composite excitations
    within the spectral stability framework.
    Elementary particles, such as leptons, are expected to correspond to
    topologically elementary solitonic configurations whose inertial mass is
    dominated by a single lowest stability eigenvalue.
    By contrast, baryonic excitations are composite objects, whose mass reflects the
    combined contribution of several coupled stability modes associated with a bound
    configuration.

    In this view, mass ratios between elementary and composite particles cannot, in
    general, be expressed as the ratio of two single eigenvalues of the same operator.
    Rather, they take the schematic form
    \begin{equation}
      \frac{m_{\mathrm{comp}}}{m_{\mathrm{elem}}}
      \;\sim\;
      \frac{\sum_k \sqrt{\lambda^{(\mathrm{comp})}_k}}
      {\sqrt{\lambda^{(\mathrm{elem})}_0}},
      \label{eq:mass_ratio_schematic}
    \end{equation}
    where $\lambda^{(\mathrm{elem})}_0$ denotes the fundamental stability mode of an
    elementary soliton, and $\{\lambda^{(\mathrm{comp})}_k\}$ label the low-lying modes
    contributing to a composite bound structure.

  \subsubsection{Ansatz for the Proton as a Composite Soliton}

    As an exploratory working hypothesis, inspired by but not equivalent to Skyrme-type
    models, we consider the proton as a composite solitonic excitation.
    Specifically, we assume:
    \begin{itemize}
      \item The electron corresponds to a fundamental soliton with topological charge
      $Q_e = 1$ and eigenvalue $\lambda_e$.
      \item The proton corresponds to a bound state of three such elementary solitons,
      with total topological charge $Q_p = 3$, supplemented by an additional
      collective binding mode with eigenvalue $\lambda_{\mathrm{bind}}$.
    \end{itemize}

    The choice $Q_p=3$ is motivated by the observed threefold constituent structure
    of baryons, but is not derived here from a classification of solitonic topological
    sectors.
    Other composite configurations are not excluded by the present framework.

  \subsubsection{Mass Ratio from Spectral Scaling}

    Under the above assumptions, the effective eigenvalue associated with the proton
    may be written schematically as
    \begin{equation}
      \lambda_p \;\approx\; \lambda_{\mathrm{bind}} + 3\lambda_e ,
    \end{equation}
    leading to the mass ratio
    \begin{equation}
      \frac{m_p}{m_e}
      \;\approx\;
      \sqrt{\frac{\lambda_{\mathrm{bind}} + 3\lambda_e}{\lambda_e}} .
    \end{equation}

    In the binding-dominated regime $\lambda_{\mathrm{bind}} \gg \lambda_e$, this
    expression reduces to
    \begin{equation}
      \frac{m_p}{m_e} \;\approx\; \sqrt{\frac{\lambda_{\mathrm{bind}}}{\lambda_e}} .
    \end{equation}
    Matching the observed value $m_p/m_e \simeq 1836$ therefore imposes the spectral
    constraint
    \begin{equation}
      \frac{\lambda_{\mathrm{bind}}}{\lambda_e}
      \;\sim\; 3.4 \times 10^{6}.
    \end{equation}

    This relation is not derived here but identified as a target condition on the
    relative spectral scales of elementary and composite solitonic sectors.
    Whether such a hierarchy can arise naturally from specific topological
    connectivities and stability operators remains an open problem.

    \paragraph{On the emergence of three-soliton bound states.}

      The analogy with quark confinement motivates considering composite solitons carrying
      a total topological charge $Q_p = 3$, but this choice is not assumed ad hoc.
      In non-linear field theories admitting topological solitons, extensive numerical
      studies have shown that multi-soliton bound states do not exhibit uniform stability
      across all charges.
      In particular, configurations composed of three elementary solitons often display
      enhanced stability due to geometric phase locking and symmetric packing constraints.

      This behavior is well documented in the context of Skyrme-type models, where
      three-soliton bound states form particularly robust minima of the energy functional,
      while two- or four-soliton configurations are either less tightly bound or prone to
      fragmentation (see, e.g.,~\cite{MantonSutcliffe2004}).

      While the present work does not claim a direct dynamical equivalence between Skyrmions
      and $\chi$-field solitons, the analogy suggests that the relaxation dynamics of the
      $\chi$ field may naturally favor composite configurations with $Q_p = 3$.
      The emergence of this preferred topological charge is therefore interpreted as a
      stability selection effect rather than a fundamental postulate.
      A first-principles derivation of the allowed composite charges in Cosmochrony remains
      an open problem and is deferred to future work (see Section~B.9.2).

  \subsubsection{Open Questions and Research Directions}

    Several key questions must be addressed to turn this ansatz into a predictive
    framework:
    \begin{itemize}
      \item What topological features of composite solitons determine the magnitude
      of $\lambda_{\mathrm{bind}}$?
      \item Does a universal scaling law $\lambda_{\mathrm{bind}} =
      f(Q_p,Q_e,\chi_c)$ exist?
      \item Is the ratio $m_p/m_e$ stable under perturbations of the effective
      potential $V(\chi)$?
      \item Can the choice $Q_p=3$ be derived from a systematic classification of
      solitonic topological sectors?
    \end{itemize}

  \subsubsection{Summary}

    Within the Cosmochrony framework, the proton-to-electron mass ratio is interpreted
    not as a fundamental input, but as an emergent constraint on the relative spectral
    organization of elementary and composite solitonic excitations.
    The present analysis provides a consistent toy model that identifies the
    conditions such a framework must satisfy, while leaving their explicit
    realization to future analytical and numerical work.

\subsection{Role of $V(\chi)$ and Outlook}
  \label{subsec:role_vchi_mass}

  The effective potential $V(\chi)$ is expected to play a secondary role in mass
  generation, primarily by controlling fine splittings within a given solitonic
  sector rather than by setting the overall mass scale.

  \subsubsection{Eigenvalue Splittings and Fine Structure}

    In general, one may write
    \begin{equation}
      V(\chi) = \sum_n \lambda_n (\chi - \chi_c)^n ,
    \end{equation}
    where the coefficients $\lambda_n$ encode nonlinear interactions between
    solitonic modes.
    Differences such as the neutron--proton mass splitting could arise from small
    electromagnetic or topological corrections to this potential.
    No quantitative prediction is attempted here in the absence of an explicit form
    for $V(\chi)$.

  \subsubsection{Summary of Testable Predictions}
    \label{subsec:testable_predictions}

    \begin{table}[h!]
      \centering
      \caption{Testable predictions arising from the soliton spectral scaling framework.}
      \label{tab:testable_predictions}
      \begin{tabular}{|l|l|l|l|}
        \hline
        \textbf{Prediction} &
        \textbf{Expected Value} &
        \textbf{Observational Probe} &
        \textbf{Status} \\
        \hline
        Proton-to-electron mass ratio
        & $m_p/m_e \approx 1836$
        & High-precision mass spectrometry
        & Consistent with SM \\
        \hline
        Binding-mode eigenvalue ratio
        & $\lambda_{\mathrm{bind}}/\lambda_e \approx 3.4 \times 10^{6}$
        & Lattice simulations of $\chi$ solitons
        & To be tested \\
        \hline
        Neutron--proton mass difference
        & $m_n - m_p \approx 1.3\,\mathrm{MeV}$
        & Nuclear spectroscopy
        & Consistent with SM \\
        \hline
        Low-$\ell$ CMB power suppression
        & $\Delta C_\ell / C_\ell \approx 10\%$ \\
        (for $\ell \lesssim 10$)
        & CMB (Planck, CMB-S4)
        & Anomaly explained \\
        \hline
        Gravitational wave attenuation near BHs
        & $\Delta A/A \sim 10^{-2}$ \\
        (for $r \lesssim 10GM/c^2$)
        & LISA ringdown analysis
        & Future test \\
        \hline
      \end{tabular}
    \end{table}

    Table~\ref{tab:testable_predictions} summarizes the main testable predictions of the
    soliton spectral scaling approach.
    Several observables, such as the proton-to-electron mass ratio and the neutron--proton
    mass difference, are consistent with the Standard Model and therefore serve primarily
    as consistency checks rather than discriminating tests.

    By contrast, the predicted hierarchy between binding and elementary stability
    eigenvalues, as well as the low-$\ell$ CMB suppression and potential gravitational-wave
    attenuation near compact objects, provide concrete avenues for falsification.
    These signatures are not generic consequences of standard particle physics or
    cosmology and can, in principle, be tested through future lattice simulations and
    high-precision cosmological or gravitational-wave observations.

  \subsubsection{Future Work}

    Key directions for future investigation include:
    \begin{itemize}
      \item Deriving the effective potential $V(\chi)$ from the underlying relaxation
      dynamics of the $\chi$ field.
      \item Constructing and classifying composite solitonic configurations and their
      associated stability operators.
      \item Performing numerical simulations to test whether large spectral
      hierarchies can arise without fine tuning.
    \end{itemize}

    In this sense, topology constrains the structure of the stability spectrum, while
    $V(\chi)$ controls fine splittings.
    The emergence of observed mass hierarchies is thus framed as a concrete but open
    spectral-geometric problem within the Cosmochrony framework.
