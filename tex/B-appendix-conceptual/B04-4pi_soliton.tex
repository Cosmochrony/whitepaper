\subsection{Example: \(4\pi\)-Periodic Soliton and Spinorial Behavior}
  \label{subsec:4pi_soliton}

  This subsection presents an \emph{illustrative geometric construction} showing how
  spin-\(\tfrac{1}{2}\)--like transformation behavior may emerge from a localized
  configuration of the \(\chi\) field, without introducing a fundamental spinor
  degree of freedom.

  The construction is intentionally minimal and purely effective.
  It is not intended as a microscopic derivation of fermions, but as a demonstration
  of \emph{topological plausibility}: namely, that spinorial behavior can arise from
  nontrivial structure in the configuration space of scalar excitations once a
  projectable regime is reached.

  All geometric notions used below refer exclusively to the effective projected field
  \(\chi_{\mathrm{eff}}\).
  The fundamental \(\chi\) field itself does not admit spatial localization,
  complex structure, or phase.

  \subsubsection{Phase-Twisted Effective Solitonic Configuration}

    In the projectable regime, certain localized excitations of \(\chi\) admit an
    effective internal organization that can be parametrized by an angular variable.
    For illustrative purposes only, such configurations may be represented using a
    complex-valued proxy field,
    \begin{equation}
      \chi_{\mathrm{eff}}(x)
      =
      \eta \tanh(\kappa x)\, e^{i\theta(x)} ,
    \end{equation}
    where:
    \begin{itemize}
      \item the underlying physical field remains real,
      \item the complex phase does \emph{not} represent an independent internal degree
      of freedom,
      \item the phase \(\theta\) parametrizes the internal cyclic structure of the
      effective solitonic configuration.
    \end{itemize}

    This representation should be understood purely as a convenient encoding of the
    internal organization of the excitation in effective space.

    Choosing
    \begin{equation}
      \theta(x) = \frac{x}{2}
    \end{equation}
    implies that the configuration returns to an equivalent internal state only after
    a \(4\pi\) variation of the effective rotation parameter,
    \begin{equation}
      \theta(x + 4\pi) = \theta(x) + 2\pi ,
    \end{equation}
    whereas a \(2\pi\) variation produces a configuration that is locally identical in
    effective space but globally inequivalent in internal structure.

  \subsubsection{Topological Interpretation}

    The \(4\pi\)-periodicity does not originate from the introduction of a complex
    field or an intrinsic phase.
    It reflects a nontrivial topology of the configuration space of the soliton.

    Although the spatial projection of the excitation may appear unchanged after a
    \(2\pi\) rotation, the internal organization of the configuration is not.
    Only a \(4\pi\) rotation restores full equivalence in the space of effective
    configurations.

    This behavior mirrors the double-cover structure
    \(\mathrm{SU}(2)\!\to\!\mathrm{SO}(3)\)
    characteristic of spinorial representations.
    In the present framework, however, this structure arises from the topology of
    solitonic configurations of \(\chi_{\mathrm{eff}}\), rather than from a fundamental
    spinor ontology.

  \subsubsection{Relation to Fermionic Transformation Properties}

    At the effective level, \(4\pi\)-periodic excitations naturally acquire a sign
    change under \(2\pi\) rotations.
    In multi-excitation configurations, this topological property implies that the
    exchange of two identical excitations cannot be continuously deformed into the
    identity without crossing a topologically nontrivial sector.

    This provides a geometric basis for fermion-like transformation behavior and
    antisymmetric exchange properties.
    The construction does not constitute a proof of the spin--statistics theorem;
    rather, it demonstrates that fermionic characteristics can emerge consistently
    from topologically constrained scalar-field excitations.

  \subsubsection{Conceptual Scope and Limitations}

    The purpose of this example is conceptual.
    It illustrates that:
    \begin{itemize}
      \item spinorial behavior does not require a fundamental spinor field,
      \item \(4\pi\)-periodicity can arise from topological obstructions,
      \item fermion-like exchange behavior may emerge from scalar excitations with
      nontrivial configuration space.
    \end{itemize}

    No claim is made that this construction reproduces the full dynamics, interactions,
    or statistics of Standard Model fermions.
    A fully relational formulation of these topological properties, independent of any
    embedding geometry or auxiliary representation, is discussed in
    Appendix~\ref{app:relational_formulation}.
