\subsection{Example: \texorpdfstring{$4\pi$}{4π}-Periodic Soliton and Spinorial Behavior}
  \label{subsec:4pi_soliton}

  This subsection provides an \emph{explicit illustrative construction} supporting the
  topological interpretation of spin and statistics presented in Section~\ref{subsec:spin-statistics}.
  Its purpose is not to restate the phenomenological conclusions established there,
  but to demonstrate, through a minimal effective example, the \emph{topological plausibility} of spinorial behavior
  emerging from localized scalar excitations.

  The construction is intentionally minimal and purely effective.
  It does not constitute a microscopic derivation of fermions, nor does it introduce
  a fundamental spinor degree of freedom.
  Rather, it illustrates how $4\pi$-periodic transformation behavior may arise from
  nontrivial topology in the configuration space of admissible projected descriptions
  once a projectable regime is reached.

  All geometric notions used below refer exclusively to the effective projected field $\chi_{\mathrm{eff}}$.
  The fundamental $\chi$ field itself does not admit spatial localization, complex structure, or intrinsic phase.

  \subsubsection*{Phase-Twisted Effective Solitonic Configuration}

    In the projectable regime, certain localized excitations of \(\chi\) admit an
    effective internal organization that can be parametrized by an angular variable.
    For illustrative purposes only, such configurations may be represented using a
    complex-valued proxy field,
    \begin{equation}
      \chi_{\mathrm{eff}}(x)
      =
      \eta \tanh(\kappa x)\, e^{i\theta(x)} ,
    \end{equation}
    where:
    \begin{itemize}
      \item the underlying physical field remains real,
      \item the complex phase does \emph{not} represent an independent internal degree
      of freedom,
      \item the phase \(\theta\) parametrizes the internal cyclic structure of the
      effective solitonic configuration.
    \end{itemize}
    Equivalently, this is an embedding of a cyclic internal label into a complex notation;
    no $U(1)$ symmetry or conserved phase current is assumed.

    This representation should be understood purely as a convenient encoding of the
    internal organization of the excitation in effective space.

    Choosing the effective rotation parameter $\alpha$ and defining
    \begin{equation}
      \theta(\alpha)=\frac{\alpha}{2},
    \end{equation}
    implies the $4\pi$ periodicity
    \begin{equation}
      \theta(\alpha+4\pi)=\theta(\alpha)+2\pi,
    \end{equation}
    whereas a $2\pi$ cycle yields a globally inequivalent internal organization.

  \subsubsection*{Topological Interpretation}

    The $4\pi$ periodicity does not originate from the introduction of a complex field
    or an intrinsic phase.
    It reflects a nontrivial topology of the configuration space of admissible projected
    configurations.

    Although the spatial projection of the excitation may appear unchanged after a
    $2\pi$ rotation, the internal organization of the configuration is not.
    Only a $4\pi$ rotation restores full equivalence in the space of effective descriptions.

    This behavior mirrors the double-cover structure
    $\mathrm{SU}(2)\!\to\!\mathrm{SO}(3)$
    characteristic of spinorial representations.
    In the present framework, however, this structure arises from the topology of
    solitonic configurations of $\chi_{\mathrm{eff}}$, rather than from a fundamental
    spinor ontology.

  \subsubsection*{Relation to Fermionic Transformation Properties}

    At the effective level, \(4\pi\)-periodic excitations naturally acquire a sign
    change under \(2\pi\) rotations.
    In multi-excitation configurations, this topological property implies that the
    exchange of two identical excitations cannot be continuously deformed into the
    identity without crossing a topologically nontrivial sector.

    This provides a geometric basis for fermion-like transformation behavior and
    fermion-like exchange phase (a sign change in the simplest sector).
    In full generality this requires the multi-excitation configuration space and its braid structure,
    which is beyond the scope of this illustrative example.
    The construction does not constitute a proof of the spin--statistics theorem;
    rather, it demonstrates that fermionic characteristics can emerge consistently
    from topologically constrained scalar-field excitations.

    \paragraph{Explicit half-angle map and sign inversion.}
      To make the spin-$\tfrac12$ correspondence explicit, let $\alpha\in[0,2\pi]$ denote an
      \emph{effective} rotation parameter acting on the localized configuration in the projectable regime.
      The defining topological feature is that the relevant loop in configuration space is
      non-contractible at $2\pi$ but becomes contractible at $4\pi$.

      A minimal way to encode this double-cover behavior is to introduce an effective
      spinorial descriptor $\psi(\alpha)$ whose phase advances by a \emph{half-angle}:
    \begin{equation}
      \psi(\alpha) \;\equiv\; \psi_0\,e^{i\alpha/2}.
      \label{eq:B4-half-angle-map}
      \end{equation}
      Then a $2\pi$ cycle produces a sign inversion,
    \begin{equation}
      \psi(\alpha+2\pi)=\psi_0\,e^{i(\alpha+2\pi)/2} = -\,\psi_0\,e^{i\alpha/2} = -\psi(\alpha),
      \label{eq:B4-2pi-sign}
      \end{equation}
      while a $4\pi$ cycle restores the original state,
    \begin{equation}
      \psi(\alpha+4\pi)=\psi(\alpha).
      \label{eq:B4-4pi-identity}
      \end{equation}

    \begin{equation}
      \begin{aligned}
        \alpha: 0 \to 2\pi &\Rightarrow \text{nontrivial loop in } \mathcal{C}_{\mathrm{eff}} \ (\mathbb{Z}_2)
        \Rightarrow \psi \mapsto -\psi,\\
        \alpha: 0 \to 4\pi &\Rightarrow \text{trivial loop (homotopic to identity)}
        \Rightarrow \psi \mapsto \psi.
      \end{aligned}
      \label{eq:B4-schematic-doublecover}
      \end{equation}

      In this construction, $\psi$ is not a fundamental field: it is a compact
      \emph{collective label} for the topological class of the solitonic configuration in the space of
      admissible projected descriptions. The sign change at $2\pi$ is therefore not imposed as a quantum
      postulate; it is an effective encoding of the $\mathbb{Z}_2$ obstruction associated with the
      $4\pi$ periodicity of the configuration.

      (See Sec.~\ref{subsec:status-formulation} for the configuration-space statement
      $\pi_1(\mathcal{C}_{\mathrm{eff}})=\mathbb{Z}_2$
      and its effective implication $\psi\mapsto-\psi$ under a $2\pi$ loop.)

  \subsubsection*{Conceptual Scope and Limitations}\subsubsection*{Conceptual Scope and Limitations}

    The purpose of this example is strictly illustrative.
    It demonstrates that:
    \begin{itemize}
      \item spinorial transformation behavior does not require a fundamental spinor field,
      \item $4\pi$-periodicity may arise from topological obstructions in configuration space,
      \item fermion-like exchange behavior can emerge from scalar excitations with
      nontrivial topology.
    \end{itemize}

    No claim is made that this construction reproduces the full dynamics, interactions,
    or statistics of Standard Model fermions.
    It provides a concrete realization supporting the unified interpretation of spin and
    statistics developed in Section~\ref{subsec:spin-statistics}.
    A fully relational formulation of these topological properties, independent of any
    auxiliary geometric representation, is discussed in
    Appendix~\ref{app:relational_formulation}.
