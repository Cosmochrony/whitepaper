\subsection{Interpretative Status of the $\chi$ Field}
  \label{subsec:nature-chi}

  This subsection does not introduce new postulates regarding the $\chi$ field.
  Its purpose is purely interpretative: to clarify how $\chi$ should be understood
  \emph{once the framework has been established}, and to prevent several common
  misreadings suggested by conventional field-theoretic language.

  In the main text, $\chi$ is often written in the form $\chi(x^\mu)$ and manipulated
  using continuous differential operators.
  This notation should not be taken to imply that $\chi$ is a physical field
  propagating \emph{within} a pre-existing spacetime manifold.
  Rather, spacetime coordinates serve only as convenient labels for organizing
  relational information in regimes where a stable geometric description has emerged.

  Fundamentally, $\chi$ encodes a relational scale of relaxation from which notions
  such as duration, distance, and causal ordering are reconstructed.
  The apparent embedding of $\chi$ in spacetime is therefore representational,
  not ontological.
  The manifold description is a secondary construct, introduced only after the
  relaxation dynamics of $\chi$ has reached sufficient regularity to admit a
  geometric interpretation.

  This distinction mirrors the use of continuum variables in hydrodynamics or
  elasticity theory.
  Just as a velocity field does not exist independently of the underlying molecular
  interactions, $\chi(x^\mu)$ does not represent a fundamental spacetime field.
  It summarizes collective relational properties of the substrate once coarse
  graining becomes meaningful.

  In this sense, $\chi$ should not be interpreted as:
  \begin{itemize}
    \item a matter field living on spacetime,
    \item a dynamical scalar coupled to a pre-existing metric,
    \item or a hidden-variable replacement for the quantum wavefunction.
  \end{itemize}

  Instead, $\chi$ constitutes the pre-geometric quantity from which spacetime
  structure, effective fields, and physical observables emerge through projection
  and coarse graining.
  The use of continuous fields, Lagrangians, and differential equations throughout
  this work reflects practical representational choices rather than fundamental
  commitments.

  This interpretative clarification is particularly important for understanding the
  role of localized excitations, solitonic structures, and effective fields discussed
  in the remainder of this appendix.
  These constructions should be read as regime-dependent invariants of the underlying
  $\chi$ dynamics, not as evidence that $\chi$ itself decomposes into independently
  propagating physical entities.

  In summary, the $\chi$ field is not a field \emph{in} spacetime.
  Spacetime is an emergent bookkeeping structure \emph{for} $\chi$ once its relational
  dynamics becomes sufficiently regular.
  This asymmetry is essential to the ontological parsimony of the Cosmochrony
  framework and underlies its reinterpretation of geometry, matter, and quantum
  phenomena.
