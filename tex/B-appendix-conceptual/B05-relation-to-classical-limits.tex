\subsection{Relation to Classical Limits}
  \label{subsec:classical-limits}

  In Cosmochrony, the emergence of classical behavior does not correspond to the
  introduction of an independent theoretical layer.
  Instead, classical physics arises as a \emph{dynamical regime} of the same underlying
  scalar structure, characterized by smoothness, dilution of localized excitations,
  and the suppression of topological and relational effects.

  \paragraph{Weakly structured regime and effective linearization.}
    In regimes where the fundamental field \(\chi\) admits a stable projective
    representation and where its effective projection \(\chi_{\mathrm{eff}}\) varies
    slowly over large scales, localized excitations become dilute and weakly interacting.
    Under these conditions, the dynamics of \(\chi_{\mathrm{eff}}\) can be linearized
    around a quasi-homogeneous background configuration.

    In this regime, small perturbations propagate as weak disturbances on an effectively
    flat geometric background.
    Superposition, approximate locality, and linear wave propagation emerge as effective
    properties of the coarse-grained relaxation dynamics.
    This reproduces the operational content of classical field theories and of quantum
    field theories formulated on Minkowski spacetime.

    This correspondence should be understood as an \emph{effective recovery} rather than
    as an ontological reduction.
    Cosmochrony does not reduce to standard quantum field theory; rather, standard field
    theories appear as limiting descriptions valid when relational structure becomes
    dynamically inert.

  \paragraph{Suppression of relational and topological effects.}
    In the weakly structured regime, topological constraints associated with solitonic
    configurations are either absent or dynamically irrelevant.
    The configuration space effectively factorizes, and collective relaxation dominates
    over localized structural organization.
    As a result, particle-like excitations behave as approximately independent degrees
    of freedom, and classical intuition becomes applicable.

    The classical limit therefore corresponds to a regime in which the relational content
    of \(\chi\) is present but operationally inaccessible, masked by coarse-graining and
    scale separation.

  \paragraph{Nonlinear regime and effective curvature.}
    Conversely, in regimes of strong spatial variation of \(\chi_{\mathrm{eff}}\) or high
    density of localized excitations, nonlinear effects dominate the dynamics.
    Large gradients locally constrain relaxation, inducing effective curvature, time
    dilation, and horizon-like behavior in the emergent geometric description.

    These regimes reproduce the phenomenology associated with curved spacetime,
    gravitational collapse, and strong-field effects, while remaining governed by the same
    underlying scalar dynamics.
    No additional gravitational degrees of freedom are introduced; curvature emerges as
    a collective response of the relaxation structure of \(\chi_{\mathrm{eff}}\).

  \paragraph{Meaning of the classical limit in Cosmochrony.}
    The classical limit in Cosmochrony is therefore not defined by \(\hbar \to 0\), nor by
    the suppression of quantum postulates.
    It corresponds to a regime in which:
    \begin{itemize}
      \item the effective projection \(\chi_{\mathrm{eff}}\) is smooth and slowly varying,
      \item localized excitations are dilute and weakly correlated,
      \item topological and relational constraints are dynamically suppressed,
      \item coarse-graining yields stable geometric descriptions.
    \end{itemize}

    In this regime, classical spacetime and standard field dynamics emerge as reliable,
    approximate descriptions.
    Their validity reflects not fundamental structure, but the stability of a particular
    relaxation regime of the underlying \(\chi\) field.
