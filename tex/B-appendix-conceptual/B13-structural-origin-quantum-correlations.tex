\subsection{Structural Origin of Quantum Correlations and Non-Locality}
  \label{app:structural-origin-quantum-correlations}

  This section provides a conceptual extension of the Cosmochrony framework,
  illustrating how quantum correlations and spin may be interpreted geometrically
  within a strictly monistic and non-injective projection ontology.
  The discussion is interpretative in nature and does not introduce additional
  dynamical postulates.

  The non-injective nature of the projection operator $\Pi$, which maps the relational substrate $\chi$
      onto the effective 4D manifold, provides a structural reinterpretation of quantum non-locality and entanglement.
      In this framework, EPR-type correlations are not viewed as the result of superluminal signaling, but as a direct
      consequence of the \textbf{shared ontological source} of projected observables.

  \subsubsection*{Non-injectivity and Structural Identity.}
    Within Cosmochrony, what is effectively perceived as two spatially separated particles may correspond to a single,
    unified relational configuration in $\chi$.
    \begin{itemize}
      \item \textbf{Geometric Separation vs. Relational Unity:} While the emergent metric $g_{\mu\nu}$
      assigns a large spatial distance between two detectors, the underlying $\chi$
      -excitation remains a single connected entity in the pre-geometric substrate.
      \item \textbf{The Shared Projection:} Entanglement is thus defined as the manifestation of a single $\chi$
      -source through multiple, non-injective projective ``images''.
      The perceived ``spooky action at a distance'' is an artifact of the metric description, which fails to capture the
      \item underlying relational unity.
    \end{itemize}

  \subsubsection*{Torsional Conservation and the Origin of Spin Correlations.}
    This hypothesis extends naturally to the geometric origin of spin.
    If spin is interpreted as the projection of the internal degrees of freedom of the relational fiber
    (e.g., within the Hopf fibration $S^3 \to S^2$), then:
    \begin{itemize}
      \item A measurement at location $A$ corresponds to a local stabilization of the projection's torsional phase.
      \item Because the underlying configuration in $\chi$ is a unified structure, this local stabilization
      \textit{structurally constrains} the admissible projective states available at any other location $B$
      originating from the same source.
      \item This mechanism ensures the conservation of global topological invariants across the shared projection
      without violating the causal bounds of the relaxation dynamics.
    \end{itemize}

  \subsubsection*{Relationship with Bell's Theorem.}
    Cosmochrony addresses the constraints of Bell's theorem by shifting the locus of reality. The framework remains
        \textbf{realistic}, as the substrate $\chi$ possesses definite relational states, but it is
        \textbf{structurally non-local} with respect to the emergent spacetime.
    The violation of Bell inequalities is not seen as a failure of realism, but as a signature of
        \textbf{metric emergence}
        : the metric distance, used to define ``locality'' in the theorem, is not a fundamental property of the level
        where the correlations are established.
