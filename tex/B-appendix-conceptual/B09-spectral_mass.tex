\subsection{Spectral Characterization of Mass and the Secondary Role of \(V(\chi)\)}
  \label{subsec:spectral_mass}

  This appendix clarifies the conceptual status of inertial mass in Cosmochrony.
  While mass originates physically from the resistance of solitonic configurations to the relaxation of the \(\chi\) field, this resistance admits a quantitative characterization in terms of the spectral properties of an associated stability operator.

  In this context, spectral analysis does not redefine the physical origin of inertial mass—which remains the resistance of localized configurations to \(\chi\) relaxation—but provides a structured and potentially calculable description of this resistance.

  A central conjecture of the Cosmochrony framework is that particle masses are not fundamental parameters encoded in the nonlinear potential \(V(\chi)\).
  Instead, they emerge as spectral properties of a relaxation operator defined on a relational substrate, which may be represented, for calculational purposes, by a discrete graph structure.

  \paragraph{Mass spectrum as eigenmodes of a relaxation operator.}
    Localized particle-like excitations are identified with normal modes of an effective Laplace--Beltrami operator acting on a graph \(G(V,E)\),
    \begin{equation}
      \Delta_G \psi_n = -\lambda_n \psi_n,
    \end{equation}
    where \(\psi_n\) are eigenmodes characterizing the stability of localized configurations.
    The associated inertial masses are conjectured to scale as
    \begin{equation}
      m_n c^2 \propto \sqrt{\lambda_n},
    \end{equation}
    in agreement with the effective spectral relations introduced in Section~\ref{subsec:perspectives_mass_spectrum}.
    Physically, this scaling reflects the fact that inertial mass measures the characteristic frequency associated with the resistance of a localized configuration to the relaxation of the \(\chi\) field.

    This relation is analogous to the emergence of discrete vibrational frequencies in bounded elastic systems, where spectral values are fixed by geometry and connectivity rather than by adjustable parameters.
    Within Cosmochrony, mass hierarchies are therefore interpreted as geometric and topological properties of the underlying relational structure.

    A decisive test of this conjecture would consist in computing the low-lying spectrum of \(\Delta_G\) on large but finite networks with physically motivated connectivity rules.
    Even approximate agreement with observed mass ratios would strongly support the spectral origin of inertial mass and the non-fundamental role of \(V(\chi)\).

  \paragraph{Spectral structure and separation of descriptive levels.}
    To avoid circular dependencies between geometry, dynamics, and emergent particle properties, Cosmochrony distinguishes three conceptual levels.
    This separation is essential to avoid any circular definition in which emergent geometric notions would feed back into the operator responsible for mass generation.

    At the fundamental level, inertial masses are associated with the spectral properties of a background-independent relaxation operator \(\Delta_G^{(0)}\), defined by the intrinsic relational connectivity of the substrate.
    This operator is not tied to any specific spacetime geometry or instantaneous \(\chi\) configuration and provides a stable spectral structure.

    At the emergent geometric level, coarse-grained configurations of \(\chi\) give rise to effective notions of spacetime, including gravitational time dilation and cosmological expansion.
    These geometric effects influence propagation and interaction, but do not modify the underlying spectral operator responsible for mass generation.

    Finally, fast dynamical processes such as radiation, scattering, and decoherence correspond to interaction-induced redistributions of relaxation potential within the \(\chi\) field.
    These processes affect observables without redefining the fundamental spectral structure.

  \paragraph{Residual role of the potential \(V(\chi)\).}
    Within this spectral picture, the nonlinear potential \(V(\chi)\) plays a secondary and effective role.
    It provides a local, coarse-grained description of localization mechanisms associated with low-lying spectral modes, but does not independently set the overall mass scale.
    Its admissible form is constrained by the requirement that it support stable solitonic configurations compatible with the spectral structure.

    To clarify in what sense the potential \(V(\chi)\) may influence observable masses without redefining their spectral origin, we briefly illustrate how it can induce small corrections to the stability eigenvalues.

  \paragraph{Example: potential-induced corrections to stability eigenvalues.}
    To make explicit how the effective potential \(V(\chi)\) can modify the stability eigenvalues \(\lambda_n\) without altering the underlying topological structure, consider a simple illustrative example.
    Let
    \begin{equation}
      V(\chi) = \lambda \left( \chi^2 - \chi_c^2 \right)^2,
    \end{equation}
    where \(\chi_c\) denotes the equilibrium value of the \(\chi\) field in the relaxed background.

    Expanding \(V(\chi)\) around \(\chi = \chi_c\) yields a quadratic contribution for small fluctuations \(\delta\chi = \chi - \chi_c\),
    \begin{equation}
      V(\chi_c + \delta\chi) \simeq
      \frac{1}{2}
      \left.
        \frac{d^2 V}{d\chi^2}
      \right|_{\chi=\chi_c}
      (\delta\chi)^2 + \cdots,
    \end{equation}
    with
    \begin{equation}
      \left.
        \frac{d^2 V}{d\chi^2}
      \right|_{\chi=\chi_c}
      \;\propto\;
      \lambda\,\chi_c^2.
    \end{equation}

    This term contributes additively to the linearized stability operator \(\mathcal{L}_{\mathrm{sol}}\), effectively shifting the eigenvalues as
    \begin{equation}
      \lambda_n \;\longrightarrow\;
      \lambda_n^{(0)} + \Delta\lambda_n^{(V)},
    \end{equation}
    where \(\lambda_n^{(0)}\) encodes the geometric and topological stiffness of the soliton, and \(\Delta\lambda_n^{(V)}\) arises from the local curvature of the potential.

    For composite solitons such as baryons, this potential-induced correction can differ slightly between closely related configurations (e.g., neutron versus proton), thereby generating small mass splittings.
    By contrast, ratios controlled primarily by the number and organization of elementary solitonic constituents (such as \(m_p/m_e\)) remain dominated by the topological structure and are only weakly affected by \(V(\chi)\).

    While \(V(\chi)\) provides a convenient effective description, its detailed form is expected to emerge from the nonlinear dynamics of the \(\chi\) field (see Appendix~\ref{subsec:analytical-solutions}).

  \paragraph{Supporting perspectives.}
    Additional constraints arising from discrete symmetries, information-theoretic considerations, or dynamical consistency may further restrict admissible connectivity structures.
    However, these considerations remain secondary to the central spectral hypothesis and are not required for its internal coherence.

    Taken together, these arguments suggest that a substantial part of the explanatory burden for mass generation in Cosmochrony lies in the spectral properties of the underlying relaxation dynamics, with \(V(\chi)\) serving as a derived effective descriptor rather than a fundamental source of mass.

    Extending this spectral characterization toward concrete mass predictions requires specifying the relaxation operator and its boundary conditions, particularly for composite solitonic sectors, as discussed in Section~\ref{subsec:perspectives_mass_spectrum}.

\subsection{Spectral Stability and Emergence of \(\hbar_{\text{eff}}\)}
\label{sec:hbar_eff_derivation}

In Cosmochrony, the effective Planck constant \(\hbar_{\text{eff}}\) emerges from the **spectral stability** of \(\chi\)-field solitons, without invoking external quantum postulates.
This section derives \(\hbar_{\text{eff}}\) purely from the fundamental parameters of \(\chi\) and clarifies its role in connecting geometric and quantum descriptions.

\subsubsection{Fundamental Scales of the \(\chi\) Field}
  \label{sec:fundamental_scales}

  The \(\chi\) field is characterized by three independent scales:
  \begin{itemize}
    \item \(K_0\): Maximal coupling strength (units: \([L^{-2}]\)), setting the stiffness of the relaxation network.
    \item \(\chi_c\): Correlation length (units: \([L]\)), defining the scale at which solitonic configurations become stable.
    \item \(c\): Maximal relaxation speed (units: \([L/T]\)), bounding the propagation of \(\chi\) perturbations.
  \end{itemize}

  From these, we define the **natural unit of action** for the \(\chi\) field:
  \[
    \boxed{
      \hbar_{\chi} \equiv \frac{c^3}{K_0 \chi_c}
    }
    \quad \text{(units: } [L^2 T^{-1}] = [\text{Action}] \text{)}
  \]
  This quantity is \textbf{independent of \(\hbar\)} and emerges purely from the relaxation dynamics of \(\chi\).

\subsubsection{Topological Origin of Quantization}
  \label{sec:topological_quantization}

  Quantization in Cosmochrony follows from the **discrete spectrum of \(\Delta_G^{(0)}\)**, where eigenvalues \(\lambda_n\) correspond to stable solitonic configurations (e.g., vortices, skyrmions).
  The energy of a soliton is:
  \[
    E_n = \frac{c^2}{2} \lambda_n \mathcal{N}_n,
  \]
  where \(\mathcal{N}_n\) is the norm of the soliton's wavefunction.
  For a soliton with topological charge \(Q\) (e.g., winding number), the frequency of small oscillations is:
  \[
    \nu_n \sim \frac{c Q}{\chi_c}.
  \]
  Combining these with the action quantization condition \(E_n = \hbar_{\text{eff}} \nu_n\) yields:
  \[
    \boxed{
      \hbar_{\text{eff}} = \frac{c^2 \lambda_n \mathcal{N}_n}{2 \nu_n} \sim K_0 \chi_c^2.
    }
  \]
  Here, \(\hbar_{\text{eff}}\) emerges as a **geometric property** of solitons, not an external constant.

\subsubsection{Regime-Dependent Scaling of \(\hbar_{\text{eff}}\)}
  \label{sec:regime_dependent_hbar}

  The value of \(\hbar_{\text{eff}}\) depends on the **spacetime regime**:
  \begin{itemize}
    \item \textbf{Quantum regime} (\(\ell_{\text{spacetime}} \sim \chi_c\)):
    \[
      \hbar_{\text{eff}} \approx \hbar_{\chi} \sim \hbar.
    \]
    \item \textbf{Cosmological regime} (\(\ell_{\text{spacetime}} \gg \chi_c\)):
    \[
      \hbar_{\text{eff}} \approx \hbar_{\chi} \left( \frac{\chi_c}{\ell_{\text{spacetime}}} \right)^2 \ll \hbar.
    \]
  \end{itemize}
  This explains the **absence of quantum effects at macroscopic scales**.

\subsubsection{Consistency with Standard Quantum Mechanics}
  \label{sec:consistency_qm}

  In the quantum regime, \(\hbar_{\text{eff}} \approx \hbar\) reproduces standard quantization:
  \[
    E_n = \hbar_{\text{eff}} \nu_n \approx \hbar \nu_n.
  \]
  This is \textbf{not a postulate} but a consequence of the soliton's topological stability and the scaling of \(\hbar_{\text{eff}}\) with \(\chi_c\).

\subsubsection{Numerical Estimates and Constraints}
  \label{sec:hbar_numerical}

  Using typical values for particle physics:
  \begin{itemize}
    \item For an electron-like soliton (\(m_e \approx 0.5\) MeV), we require:
    \[
      K_0 \chi_c^2 \approx \hbar \approx 10^{-34} \, \text{J} \cdot \text{s}.
    \]
    \item With \(\chi_c \sim 10^{-18}\) m (electroweak scale), this implies:
    \[
      K_0 \approx 10^{50} \, \text{m}^{-2}.
    \]
  \end{itemize}
  These values are \textbf{consistent with soliton stability} and the emergence of the Standard Model mass spectrum.
