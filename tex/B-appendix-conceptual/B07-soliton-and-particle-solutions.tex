\subsection{Soliton and Particle Solutions}
  \label{subsec:soliton-and-particle-solutions}

  Within the Cosmochrony framework, elementary particles are interpreted as stable
  or metastable localized configurations arising in the \emph{projectable regime}
  of the scalar field \(\chi\).
  These configurations, hereafter referred to as \(\chi\)-solitons, emerge from
  nonlinear self-organization of the relaxation dynamics and persist as localized
  resistances to global relaxation.

  This interpretation does not rely on the postulation of additional fundamental
  degrees of freedom.
  The only fundamental entity is the scalar field \(\chi\), which is not defined on
  spacetime.
  Spatial localization, energy, and particle-like persistence arise only once an
  effective geometric projection \(\chi_{\mathrm{eff}}\) becomes applicable.

  While the fundamental field is scalar, certain solitonic configurations of
  \(\chi_{\mathrm{eff}}\) possess a nontrivial internal organization that cannot be
  faithfully encoded by a single real scalar variable.
  In particular, configurations characterized by internal cyclic structure,
  nontrivial winding, and \(4\pi\)-periodicity exhibit transformation properties
  that require a double-valued representation under effective rotations.

  In such cases, an effective spinorial description becomes unavoidable.
  This does not imply the existence of fundamental spinor fields.
  Rather, spinorial variables arise as \emph{collective descriptors} encoding the
  internal topology and spectral structure of fermionic \(\chi\)-solitons.

  At the phenomenological level, these excitations admit a representation in terms
  of Dirac spinors.
  This representation should be understood as an emergent and coarse-grained
  description of the internal degrees of freedom of \(\chi\)-solitons, not as an
  ontological extension of the theory.
  Within this regime, the Dirac equation appears as the minimal effective dynamical
  structure compatible with:
  \begin{itemize}
    \item approximate locality in the projected geometric description,
    \item effective relativistic covariance,
    \item and the topological constraints associated with \(4\pi\)-periodic
    configurations.
  \end{itemize}

  From this perspective, the Dirac structure does not introduce new fundamental
  entities.
  It provides a compact and universal encoding of the internal topology, spectral
  stability, and transformation behavior of fermionic solitons.
  Spin, fermionic statistics, and exclusion behavior arise as effective consequences
  of the nontrivial configuration space of these scalar-field excitations, rather
  than as independent postulates.

  The existence and stability of \(\chi\)-solitons impose structural constraints on
  the effective self-interaction functional governing the projected dynamics.
  Although the explicit form of this functional remains undetermined, it must satisfy
  the following minimal requirements:
  \begin{enumerate}
    \item the support of localized configurations with finite effective mass,
    \item dynamical stability under small perturbations,
    \item and the existence of topologically inequivalent sectors corresponding to
    distinct classes of particle-like excitations.
  \end{enumerate}

  The detailed derivation of effective Dirac dynamics from fluctuations around
  \(\chi\)-soliton backgrounds, as well as the emergence of a realistic mass
  spectrum, remains an open mathematical problem.
  These issues are addressed at a programmatic and illustrative level in
  Sections~\ref{app:topological_solitons}--\ref{subsec:4pi_soliton} and further
  discussed in Appendix~\ref{subsec:perspectives_mass_spectrum}.
