\subsection{Spectral Characterization of Mass and the Secondary Role of \(V(\chi)\)}
  \label{subsec:spectral_mass}

  This appendix clarifies the conceptual status of inertial mass in the Cosmochrony
  framework.
  Physically, mass originates from the resistance of localized configurations to the
  relaxation of the fundamental \(\chi\) field.
  This resistance, however, admits a quantitative and structurally organized
  description in terms of the spectral properties of an associated stability
  operator defined in the projectable regime.

  Spectral analysis therefore does not redefine the physical origin of inertial mass.
  Rather, it provides a coherent and potentially calculable characterization of how
  resistance to relaxation is distributed among stable and metastable configurations.

  A central conjecture of Cosmochrony is that particle masses are not fundamental
  parameters encoded in the nonlinear potential \(V(\chi)\).
  Instead, they emerge as spectral properties of a background-independent relaxation
  operator defined on a relational substrate, which may be represented, for
  calculational purposes, by a discrete graph structure.

  \paragraph{Mass spectrum as eigenmodes of a relaxation operator.}

    Localized particle-like excitations are identified with normal modes of an
    effective relaxation operator \(\Delta_G^{(0)}\), which may be represented as a
    Laplace--Beltrami operator acting on a graph \(G(V,E)\),
    \begin{equation}
      \Delta_G^{(0)} \psi_n = -\lambda_n \psi_n .
    \end{equation}
    The eigenmodes \(\psi_n\) characterize the stability of localized solitonic
    configurations, while the eigenvalues \(\lambda_n\) encode their intrinsic
    resistance to deformation.

    In regimes where an effective spacetime description applies, the inertial masses
    associated with these modes scale as
    \begin{equation}
      m_n c^2 \;\propto\; \sqrt{\lambda_n}\,\chi_c ,
    \end{equation}
    in agreement with the spectral relations introduced in
    Section~\ref{subsec:perspectives_mass_spectrum}.
    This scaling reflects the fact that inertial mass measures the characteristic
    frequency associated with the resistance of a localized configuration to
    \(\chi\)-field relaxation.

    The situation is analogous to bounded elastic systems, where discrete vibrational
    frequencies arise from geometry and connectivity rather than from adjustable
    material parameters.
    Within Cosmochrony, mass hierarchies are therefore interpreted as geometric and
    topological properties of the underlying relational structure.

    A decisive test of this conjecture would consist in computing the low-lying
    spectrum of \(\Delta_G^{(0)}\) on large but finite networks with physically
    motivated connectivity rules.
    Even approximate agreement with observed mass ratios would strongly support the
    spectral origin of inertial mass and the non-fundamental role of \(V(\chi)\).

  \paragraph{Separation of descriptive levels.}

    To avoid circular dependencies between geometry, dynamics, and emergent particle
    properties, Cosmochrony distinguishes three conceptual levels.

    At the fundamental level, inertial masses are associated with the spectral
    properties of a background-independent relaxation operator
    \(\Delta_G^{(0)}\), defined solely by the intrinsic relational connectivity of the
    substrate.
    This operator is not tied to any spacetime geometry or instantaneous \(\chi\)
    configuration and provides a stable spectral backbone.

    At the emergent geometric level, coarse-grained configurations of
    \(\chi_{\mathrm{eff}}\) give rise to effective notions of spacetime, including
    curvature, gravitational time dilation, and cosmological expansion.
    These geometric effects influence propagation and interaction, but do not redefine
    the underlying spectral operator responsible for mass generation.

    Finally, fast dynamical processes such as radiation, scattering, and decoherence
    correspond to interaction-induced redistributions of relaxation potential within
    the \(\chi\) field.
    These processes affect observables without modifying the fundamental spectral
    structure.

  \paragraph{Residual role of the potential \(V(\chi)\).}

    Within this spectral picture, the nonlinear potential \(V(\chi)\) plays a
    secondary and effective role.
    It does not set the overall mass scale.
    Instead, it provides a local coarse-grained description of nonlinear stabilization
    mechanisms associated with low-lying spectral modes.

    The admissible form of \(V(\chi)\) is constrained by the requirement that it
    support stable solitonic configurations compatible with the pre-existing spectral
    structure.
    It encodes neither an independent interaction nor a fundamental energy density.

  \paragraph{Origin of the effective potential \(V(\chi)\).}

    Characterizing \(V(\chi)\) as secondary does not imply arbitrariness.
    Rather, \(V(\chi)\) should be understood as an effective descriptor of the local
    nonlinear response of the relaxation dynamics in the vicinity of a stable
    configuration.

    At the fundamental level, the dynamics of \(\chi\) are governed by bounded
    relaxation rules and their associated spectral structure.
    When this dynamics is projected onto a reduced functional subspace associated with
    a localized soliton, nonlinear self-consistency constraints induce an effective
    local restoring structure.
    In this reduced description, these constraints may be summarized by an effective
    potential \(V(\chi)\).

    Different admissible forms of \(V(\chi)\) correspond to different coarse-graining
    choices, while leaving invariant the underlying spectral origin of mass and
    stability.

  \paragraph{Potential-induced corrections to stability eigenvalues.}

    To illustrate how \(V(\chi)\) can modify stability eigenvalues without altering
    their spectral origin, consider the illustrative form
    \begin{equation}
      V(\chi) = \lambda \left( \chi^2 - \chi_c^2 \right)^2 .
    \end{equation}

    Expanding around the relaxed background \(\chi=\chi_c\) yields a quadratic
    contribution for small fluctuations \(\delta\chi\),
    \begin{equation}
      V(\chi_c + \delta\chi)
      \simeq
      \frac{1}{2}
      \left.
        \frac{d^2 V}{d\chi^2}
      \right|_{\chi_c}
      (\delta\chi)^2 + \cdots ,
    \end{equation}
    with
    \begin{equation}
      \left.
        \frac{d^2 V}{d\chi^2}
      \right|_{\chi_c}
      \propto \lambda\,\chi_c^2 .
    \end{equation}

    This term contributes additively to the linearized stability operator, shifting the
    eigenvalues as
    \begin{equation}
      \lambda_n
      \;\longrightarrow\;
      \lambda_n^{(0)} + \Delta\lambda_n^{(V)} .
    \end{equation}

    For composite solitons, such corrections may differ slightly between closely
    related configurations (e.g., neutron versus proton), generating small mass
    splittings.
    By contrast, ratios dominated by topological organization (such as
    \(m_p/m_e\)) remain largely insensitive to the detailed form of \(V(\chi)\).

  \paragraph{Summary.}

    In Cosmochrony, inertial mass is fundamentally a spectral property of the
    relaxation dynamics of the \(\chi\) field.
    The potential \(V(\chi)\) serves as a derived, effective descriptor controlling
    fine structure, not as a primary source of mass.
    Extending this spectral characterization toward quantitative mass predictions
    requires specifying the relaxation operator and its boundary conditions,
    particularly for composite solitonic sectors.
