\subsection{Status of the Formulation}
  \label{subsec:status-formulation}

  The formulation presented in this work should be understood as a
  \emph{minimal yet structurally complete} theoretical framework.
  Its ontological commitments, dynamical principles, and interpretative structure
  are fully specified at the conceptual level, even though several technical
  developments remain open.

  In particular, a fully covariant action principle formulated solely in terms of
  the fundamental \(\chi\) dynamics, as well as a systematic quantization procedure,
  have not yet been derived in their final and definitive form.
  These missing elements should not be interpreted as conceptual deficiencies.
  Rather, they correspond to technical extensions required to interface a
  fundamentally relational and pre-geometric framework with conventional
  variational and quantum formalisms that presuppose spacetime structure.

  Crucially, the absence of a finalized action or quantization scheme does not
  obstruct the recovery of known physical phenomenology.
  Throughout this work, general relativity and quantum field theory emerge as
  \emph{effective, coarse-grained descriptions} valid within specific dynamical
  regimes of the projected field \(\chi_{\mathrm{eff}}\).
  They are not introduced as independent axioms, but arise as stable limits of the
  underlying relaxation dynamics.

  The present formulation therefore occupies a well-defined intermediate status.
  It is not intended as a closed or final theory, nor as a phenomenological model
  tuned to reproduce specific experimental data.
  Instead, it provides a coherent ontological and dynamical foundation from which
  both geometric and quantum structures can emerge, while remaining open to future
  refinements that may enhance its mathematical completeness, formal elegance, and
  predictive scope.

  In this sense, Cosmochrony should be viewed as a foundational framework rather
  than as a fully developed effective field theory: its primary contribution lies
  in clarifying \emph{what is fundamental} and \emph{how known physical structures
can arise}, rather than in prescribing their final mathematical implementation.
