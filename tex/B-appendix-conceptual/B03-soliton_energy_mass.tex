\subsection{Soliton Energy and Structural Mass Scaling}
  \label{subsec:soliton_energy_mass}

  \paragraph{Status and scope of this analysis.}
    This subsection presents a \emph{quantitative but non-numerical} analysis of the
    effective mass associated with localized solitonic configurations arising in the
    \emph{projectable regime} of Cosmochrony.
    The objective is not to reproduce the observed particle mass spectrum, but to
    identify robust scaling relations, hierarchy constraints, and structural dependencies
    that emerge independently of microscopic details.

    No claim is made that the expressions introduced below define a fundamental
    Hamiltonian or Lagrangian for the \(\chi\) field.
    A fully predictive derivation of particle masses would require a complete effective
    theory incorporating projection dynamics, interaction channels, and renormalization
    effects, which lies beyond the scope of the present work.

    All energetic and spectral quantities discussed in this section refer exclusively
    to the effective projected field \(\chi_{\mathrm{eff}}\).
    The fundamental \(\chi\) field itself does not admit an energy functional or mass
    interpretation.

  \paragraph{Mass as integrated resistance to relaxation.}
    Within Cosmochrony, the mass of a localized excitation is interpreted as a measure of
    the total \emph{resistance to relaxation} imposed by the configuration on the
    surrounding \(\chi_{\mathrm{eff}}\) field.

    Once an effective geometric description applies, this resistance can be summarized
    by an effective diagnostic functional
    \begin{equation}
      M_{\mathrm{eff}} \;\propto\;
      \int_{\mathcal{V}}
      \left[
        \mathcal{T}\!\left(\nabla \chi_{\mathrm{eff}}\right)
        + \mathcal{U}\!\left(\chi_{\mathrm{eff}}\right)
      \right]
      \, d^3x ,
    \end{equation}
    where:
    \begin{itemize}
      \item \(\mathcal{T}\) encodes gradient-induced resistance associated with spatial
      inhomogeneities of \(\chi_{\mathrm{eff}}\),
      \item \(\mathcal{U}\) represents effective nonlinear stabilization terms arising
      from collective relaxation constraints.
    \end{itemize}

    This expression should be understood as a \emph{coarse-grained measure} of structural
    complexity rather than as a fundamental energy density.
    It quantifies how strongly a localized configuration delays or distorts the global
    relaxation of the projected field.

  \paragraph{Scaling with soliton size and internal structure.}
    Consider a localized solitonic configuration characterized by:
    \begin{itemize}
      \item a typical spatial extent \(\ell\),
      \item a characteristic deformation amplitude \(\Delta \chi_{\mathrm{eff}}\).
    \end{itemize}

    Dimensional analysis then yields the generic scaling
    \begin{equation}
      M_{\mathrm{eff}} \;\sim\;
      \ell^3
      \left[
        \frac{(\Delta \chi_{\mathrm{eff}})^2}{\ell^2}
        + V_{\mathrm{eff}}(\Delta \chi_{\mathrm{eff}})
      \right],
    \end{equation}
    where \(V_{\mathrm{eff}}\) denotes an effective nonlinear stabilization potential
    generated by projection and relaxation constraints.

    For simple kink-like configurations, the balance between gradient resistance and
    nonlinear stabilization dynamically fixes the soliton width \(\xi\).
    In this regime, the effective mass scale may be written schematically as
    \begin{equation}
      M_{\mathrm{eff}} \;\sim\;
      \sqrt{\lambda_{\mathrm{eff}}}\,\xi\,\chi_c^2 ,
    \end{equation}
    where:
    \begin{itemize}
      \item \(\chi_c\) denotes the characteristic local relaxation scale,
      \item \(\lambda_{\mathrm{eff}}\) is an emergent, configuration-dependent stiffness
      parameter.
    \end{itemize}

    Neither \(\chi_c\) nor \(\lambda_{\mathrm{eff}}\) should be interpreted as
    fundamental constants.
    They summarize collective properties of the projected relaxation regime.

  \paragraph{Structural stabilization and finite mass.}
    When stabilization arises from a balance between gradient-induced resistance and
    nonlinear relaxation constraints, the soliton size \(\ell\) is dynamically fixed.
    This mechanism ensures that localized excitations possess a finite and stable
    effective mass without the need for fine-tuning.

    Different classes of solitonic configurations (kinks, vortices, knotted or linked
    structures) involve distinct internal organizations of \(\chi_{\mathrm{eff}}\).
    As a result, their effective masses exhibit different scaling behaviors with respect
    to \(\ell\) and \(\Delta \chi_{\mathrm{eff}}\).

    This implies that mass hierarchies arise \emph{structurally} rather than through
    arbitrary parameter choices.

  \paragraph{Topological classes and mass hierarchy.}
    The effective mass depends not only on the spatial extent of a soliton but also on
    its topological class.
    Configurations characterized by higher winding, linking, or covering indices
    necessarily involve increased internal gradients and more complex relaxation
    constraints.

    Consequently, masses associated with different topological families obey ordering
    relations of the form
    \begin{equation}
      M_{n+1} \;>\; M_n ,
    \end{equation}
    where \(n\) labels an effective topological invariant.
    This establishes a natural mechanism for discrete mass hierarchies without introducing
    ad hoc mass parameters.

  \paragraph{Spectral interpretation.}
    From a spectral perspective, localized excitations correspond to bound modes of the
    linearized relaxation operator around a solitonic background configuration of
    \(\chi_{\mathrm{eff}}\).

    The effective mass is then controlled by the lowest nontrivial eigenvalue of this
    operator,
    \begin{equation}
      M_{\mathrm{eff}} \;\sim\; \lambda_{\min}^{-1},
    \end{equation}
    where \(\lambda_{\min}\) denotes the smallest positive eigenvalue governing the
    stability of the configuration.

    This formulation emphasizes that mass is fundamentally a \emph{spectral property}
    of the relaxation dynamics rather than an intrinsic attribute of a particle-like
    object.

  \paragraph{Robustness and universality.}
    The scaling relations derived above depend only on generic features of the projected
    \(\chi\) dynamics—locality, monotonic relaxation, and nonlinear stabilization—and
    are therefore expected to be robust against modifications of microscopic details.

    While specific numerical values of particle masses cannot be fixed at this level,
    the existence of discrete, ordered, and stable mass scales emerges as a structural
    prediction of the framework.

  \paragraph{Order-of-magnitude consistency.}
    Although the present analysis does not aim to reproduce the observed particle mass
    spectrum, it is instructive to examine whether the structural parameters entering the
    solitonic energy scale admit values compatible with known masses.

    For a simple kink-like configuration with characteristic width
    \(\lambda_{\mathrm{eff}}^{-1}\) and amplitude set by the local relaxation scale
    \(\chi_c\), the effective rest energy scales as
    \begin{equation}
      E_{\mathrm{sol}} \;\sim\; \chi_c^2 \lambda_{\mathrm{eff}},
    \end{equation}
    up to dimensionless shape-dependent factors of order unity.

    Identifying this energy with the electron rest mass,
    \(E_{\mathrm{sol}} \sim m_e c^2 \approx 0.511\,\mathrm{MeV}\),
    and expressing all quantities in natural units (\(\hbar = c = 1\)),
    one finds that reproducing the electron mass requires
    \begin{equation}
      \lambda_{\mathrm{eff}} \;\sim\; 10^{-44},
    \end{equation}
    for \(\chi_c\) normalized near the Planck scale.

    Such an extremely small value should not be interpreted as a fundamental coupling.
    Rather, it strongly suggests that \(\lambda_{\mathrm{eff}}\) is dynamically generated
    through collective relaxation, projection effects, and topological constraints of
    the \(\chi\) field.

  \paragraph{Summary.}
    Localized solitonic configurations of the projected field \(\chi_{\mathrm{eff}}\)
    naturally possess finite effective masses determined by their size, internal
    organization, and topological class.
    Rather than predicting specific numerical values, Cosmochrony constrains the
    \emph{scaling, ordering, and stability} of masses through geometric and spectral
    principles.

    This structural quantitativity provides a coherent foundation for future extensions
    toward a fully predictive effective theory, without compromising the pre-geometric
    nature of the fundamental \(\chi\) field.
