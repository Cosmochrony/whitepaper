\subsection{Vacuum Fluctuations and the Casimir Effect}
  \label{subsec:vacuum-fluctuations-and-the-casimir-effect}

  In the Cosmochrony framework, vacuum fluctuations do not correspond to physical
  oscillations of a background field nor to spontaneous particle--antiparticle
  creation.
  Instead, they reflect the intrinsic structural indeterminacy of the $\chi$ substrate
  in regimes where no stable localized excitations are present.

  In the absence of matter-induced constraints, the relaxation of $\chi$ admits a wide
  range of locally compatible projective descriptions.
  These fluctuations are not dynamical events occurring \emph{in} spacetime, but
  expressions of the fact that the underlying relational structure of $\chi$ does not
  select a unique effective configuration when projected.
  They therefore represent variability of effective descriptions rather than physical
  energy stored in the vacuum.

  When material boundaries are introduced, they impose structural constraints on the
  local projectability of $\chi$.
  Certain effective descriptions become incompatible with the imposed relational
  conditions, reducing the set of admissible projective configurations between the
  boundaries compared to the exterior region.

  The Casimir effect arises from this asymmetry.
  It reflects a difference in the density of admissible effective reprojections
  compatible with the boundary conditions, which manifests in spacetime descriptions
  as a pressure acting on the confining surfaces.
  No fundamental vacuum energy density or propagating vacuum modes are required.

  In this sense, the Casimir effect probes the relational relaxation capacity of the
  $\chi$ substrate under imposed constraints, rather than revealing the presence of a
  physical zero-point energy filling space.
