\subsection{Geometric Origin of $E = h\nu$}
  \label{subsec:energy-frequency-radiation}

  This section develops, in the context of radiative processes, the energy--frequency
  relation introduced earlier in
  Section~\ref{subsec:energy-frequency-solitons}, while remaining fully consistent with
  the non-propagative and pre-geometric nature of the $\chi$ substrate.

  In Cosmochrony, radiative events do not correspond to the emission of physical waves
  or disturbances propagating within the $\chi$ field.
  Instead, they correspond to transitions between localized and delocalized regimes of
  effective projectability.
  During such events, a portion of the relaxation potential stored in a localized
  matter excitation becomes expressible only through an extended, non-localized
  projective description.

  Within effective spacetime representations, these delocalized regimes are described
  using oscillatory field modes characterized by a frequency $\nu$.
  This frequency does not represent a fundamental oscillation of $\chi$, but a
  parameter labeling the internal structural periodicity of the effective description
  required to represent the released relaxation potential.

  The Planck relation
  \begin{equation}
    E = h \nu
  \end{equation}
  thus acquires a geometric interpretation.
  The energy $E$ measures the amount of relaxation potential redistributed during a
  reprojection event, while the frequency $\nu$ characterizes the minimal temporal
  resolution required for a coherent effective description of this redistribution.
  The proportionality constant $h$ does not encode a fundamental quantum postulate, but
  acts as an effective conversion factor linking structural relaxation capacity to
  temporal projective resolution.

  This interpretation does not derive the numerical value of $h$ from first principles.
  Rather, it explains why energy transfer in radiative processes scales linearly with
  frequency across a wide range of phenomena.
  In the photoelectric effect, the threshold frequency $\nu_0$ corresponds to the
  minimal projective resolution required to destabilize a bound electronic soliton.
  Above this threshold, the linear dependence on $\nu$ reflects the additional
  relaxation capacity made accessible through the reprojection process.

  In this sense, quantization of radiative energy does not arise from discretized
  propagation, but from the discrete nature of local reprojection events, which impose
  a minimal unit of effective relaxation transfer.
