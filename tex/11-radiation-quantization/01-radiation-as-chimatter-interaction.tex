\subsection{Radiation as $\chi$--Matter Interaction}
  \label{subsec:radiation-as-chimatter-interaction}

  Within the Cosmochrony framework, radiation does not correspond to the emission or
  propagation of fundamental particle entities.
  It arises as an effective phenomenon associated with the reconfiguration of
  localized matter descriptions and their relational coupling to the surrounding
  $\chi$ substrate.

  When a localized, relaxation-resistant configuration undergoes a transition toward
  a less constrained state, part of the relational structure that sustained its
  previous persistence becomes incompatible with continued localization.
  This excess relational content ceases to admit a particle-like projected
  description and instead becomes expressible only through delocalized projected
  modes.

  In effective spacetime descriptions, this redistribution appears as radiative
  emission.
  Radiation thus represents the loss of localized projectability and the transfer of
  descriptive weight from particle-like configurations to propagating field-like
  descriptions, without invoking the transport of discrete objects or underlying
  stochastic processes.

  From this perspective, radiative phenomena reflect a change in the organization and
  projectability of relational structure within $\chi$, rather than the emission of
  pre-existing quanta or the manifestation of fundamental fluctuations.
