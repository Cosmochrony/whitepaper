\section{Radiation and Quantization}
  \label{sec:radiation-and-quantization}

  \subsection{Radiation as $\chi$--Matter Interaction}
    \label{subsec:radiation-as-$chi$matter-interaction}

    In Cosmochrony, radiation does not correspond to the emission of pre-existing
    particle entities.
    Instead, it arises from the interaction between localized excitations (matter)
    and the surrounding $\chi$ field.

    When an excited configuration interacts with $\chi$, part of the stored
    relaxation potential may be released into the surrounding field as a propagating
    disturbance.
    This process is intrinsically stochastic, reflecting local fluctuations and
    instabilities in $\chi$, and gives rise to radiative phenomena.

    Radiation therefore represents a redistribution of $\chi$-structure rather than
    the transport of discrete objects through space.

  \subsection{Emergence of Photons}
    \label{subsec:emergence-of-photons}

    Photons are not fundamental entities in this framework.
    They correspond to transient, propagating disturbances of $\chi$ generated during
    interactions with matter.

    Prior to emission or detection, no localized photon exists as an independent
    object.
    Quantization appears only at the moment of interaction, when continuous $\chi$
    dynamics produces a discrete transfer of relaxation potential.

    Although propagating electromagnetic waves correspond to continuous $\chi$
    disturbances, photon-like events emerge only when such disturbances interact with
    localized excitations.
    In a double-slit experiment, the interference pattern arises from the continuous
    wave dynamics of $\chi$, while individual detection events correspond to
    interaction-induced localization.
    This naturally explains wave--particle duality without invoking fundamental
    wavefunction collapse.

  \subsection{Geometric Origin of $E = h\nu$}
    \label{subsec:energy-frequency-radiation}

    This section develops, in the context of radiative processes, the energy--frequency
    relation introduced earlier in
    Section~\ref{subsec:energy-frequency-solitons} for localized excitations of the
    $\chi$ field.

    In Cosmochrony, radiative events correspond to the partial release of relaxation
    potential stored in localized matter excitations into propagating disturbances of
    the $\chi$ field.
    The energy transferred during such an event is associated with the internal
    oscillatory structure of the emitted $\chi$ disturbance.
    Higher-frequency disturbances correspond to tighter curvature and therefore to a
    larger amount of relaxation potential being redistributed.

    The Planck relation
    \begin{equation}
      E = h \nu
    \end{equation}
    is thus interpreted as an effective geometric proportionality between the
    frequency of a propagating $\chi$ disturbance and the amount of relaxation
    potential released during an interaction.
    In this framework, the constant $h$ does not represent a fundamental quantum
    postulate, but an effective conversion factor relating oscillation frequency to
    curvature-based energy within the $\chi$ field.

    This interpretation does not constitute a derivation of $h$ from first principles.
    Rather, it provides a geometric explanation for why energy transfer in radiative
    processes scales linearly with frequency across a wide range of phenomena.
    In the photoelectric effect, the threshold frequency $\nu_0$ corresponds to the
    minimal curvature required to liberate an electron soliton from its binding
    configuration, while the linear dependence on $\nu$ reflects the relaxation
    potential carried by the emitted $\chi$ disturbance.

  \subsection{Vacuum Fluctuations and the Casimir Effect}
    \label{subsec:vacuum-fluctuations-and-the-casimir-effect}

    Vacuum fluctuations correspond to stochastic variations of $\chi$ in the absence
    of localized excitations.
    These fluctuations are not interpreted as particle--antiparticle creation events,
    but as intrinsic variability of the continuously relaxing field.

    Boundary conditions imposed by matter constrain these fluctuations, altering the
    local spectrum of allowed modes.
    The Casimir effect arises naturally as a pressure difference resulting from
    modified $\chi$ dynamics between closely spaced boundaries, without requiring a
    fundamental vacuum energy density.

    In this sense, the Casimir effect probes the relational relaxation capacity of
    the $\chi$ field rather than a fundamental vacuum energy density.

  \subsection{Weakly Interacting Radiation}
    \label{subsec:weakly-interacting-radiation}

    Disturbances with minimal curvature, such as low-frequency electromagnetic waves
    or weakly coupled excitation modes, interact only weakly with matter.
    Their near-planar or low-contrast structure reduces the probability of inducing
    localized energy transfer.

    This explains both the transparency of the vacuum to most radiation and the small
    interaction cross sections of certain weakly interacting modes.

  \subsection{Summary}
    \label{subsec:summary3}

    Radiation and quantization arise from interactions between localized matter
    excitations and the $\chi$ field.
    Photons emerge during interactions rather than existing as independent entities,
    and quantization reflects geometric constraints of $\chi$ dynamics rather than
    fundamental discreteness.
