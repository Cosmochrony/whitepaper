\subsection{Summary}
  \label{subsec:summary-quantum}

  Within the Cosmochrony framework, the phenomenology conventionally described by the
  Standard Model—from gauge interactions to quantum correlations—emerges as a
  consequence of the spectral organization of the relational substrate $\chi$ and of
  the constraints imposed by its effective projection $\Pi$.
  No particle species, coupling constant, or interaction field is introduced as a
  fundamental ontological element.

  \begin{itemize}
    \item \textbf{Gauge Interactions as Projection Dynamics:}
    Interactions are not mediated by autonomous fields propagating on spacetime, but by
    admissible modes of the projection process itself.
    The photon corresponds to scalar transmission modes, while the $W^\pm$ and $Z^0$
    bosons arise as shear-like spectral modes whose effective masses reflect the
    spectral rigidity of the projection fiber.

    \item \textbf{Topological Origin of the Strong Sector:}
    Strong interactions and confinement are reinterpreted in terms of topological
    stability.
    Color charge does not represent an internal degree of freedom, but the energetic
    cost required to preserve the coherence of knotted solitonic configurations—such
    as the $Q=3$ proton—under admissible deformations.

    \item \textbf{Mass as Spectral Overlap:}
    The Higgs mechanism is replaced by the principle of spectral overlap.
    Mass is not an intrinsic coupling parameter but an emergent measure of resonance
    between the stability spectrum of a localized configuration and the global
    relaxation flux of the $\chi$ substrate.

    \item \textbf{Quantum Phenomena as Limits of Projectability:}
    Entanglement and nonlocal correlations do not require superluminal signaling or
    hidden variables.
    They reflect the persistence of non-factorizable admissible configurations across
    projection.
    Quantum mechanics thus appears as an effective statistical framework describing
    the limits of local projectability for globally consistent spectral descriptions.
  \end{itemize}

  In this perspective, the Standard Model is not a collection of fundamental particles
  and forces, but an effective theory describing the \textbf{harmonics of relaxation}
  selected by the projection from the relational substrate into spacetime.
  Discrete symmetries, coupling structures, and apparent constants arise as organized
  spectral features of this filtering process, rather than as independent postulates.
