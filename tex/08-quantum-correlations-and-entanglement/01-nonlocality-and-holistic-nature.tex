\subsection{Nonlocality and the Holistic Character of Projected Descriptions}
  \label{subsec:nonlocality-and-holistic-nature}

  In the Cosmochrony framework, quantum nonlocality does not arise from superluminal
  interactions or from violations of relativistic causality~\cite{Bell1964}.
  Instead, it reflects the intrinsically non-factorizable character of certain
  admissible projected descriptions.

  Entangled systems correspond, at the level of effective descriptions, to single
  projected configurations that cannot be decomposed into independent subsystems
  without loss of admissibility.
  Once such configurations have formed through interaction, their subsequent
  descriptions remain globally constrained, even when effective spacetime language
  assigns them to spatially separated regions.

  The persistence of quantum correlations across spatial separation therefore follows
  from the relational structure of admissible projected descriptions, rather than from
  any spatial connectivity or signal exchange.
  Although effective geometric descriptions may associate distant locations with
  different parts of an entangled system, these locations correspond to correlated
  aspects of a single non-factorizable descriptive configuration.

  In this sense, quantum nonlocality in Cosmochrony is structural rather than dynamical.
  The correlations are constrained by the global consistency conditions of admissible
  descriptions, while all local physical processes remain compatible with relativistic
  causality.

  This non-factorizable character plays a crucial role in quantum measurement.
  Because entangled systems correspond to a single admissible projected configuration,
  measurement outcomes cannot be interpreted as revealing pre-existing local
  properties of the projected description.
  Instead, decoherence suppresses relational alternatives within the space of
  admissible descriptions, yielding effectively independent local projections.

  Local measurement outcomes correspond to particular reprojections selected from a
  space of structurally compatible descriptions.
  This selection does not involve nonlocal influence or hidden communication, but
  reflects the loss of access to global relational coherence within effective
  descriptions.

  In this context, the Born rule does not encode a dynamical nonlocal mechanism.
  It reflects the statistical distribution of locally accessible outcomes compatible
  with a single non-factorizable descriptive structure once decoherence has occurred.
  Nonlocal correlations therefore arise from global descriptive consistency, while
  measurement statistics remain fully compatible with relativistic causality.

  Crucially, admissible projected descriptions do not encode predetermined measurement
  outcomes.
  They define a space of relationally compatible realizations, whose effective
  selection occurs through decoherence and reprojection, within the limits imposed
  by spacetime representability.
