\section{Quantum Phenomena and Entanglement}
  \label{sec:quantum-phenomena-and-entanglement}

  \subsection{Nonlocality and the Holistic Nature of the $\chi$ Field}
    \label{subsec:nonlocality-and-holistic-nature}

    In the Cosmochrony framework, quantum nonlocality does not arise from superluminal
    interactions or from violations of relativistic causality~\cite{Bell1964}.
    Instead, it reflects the intrinsically holistic nature of the $\chi$ field.
    Entangled systems correspond to single, extended configurations of $\chi$ that
    cannot be factorized into independent subsystems once they have interacted.

    The persistence of quantum correlations across spatial separation follows from
    the internal relational structure of $\chi$, rather than from spatial connectivity
    or signal exchange.
    Although effective geometric descriptions may assign distant locations to parts
    of an entangled system, these locations correspond to different manifestations of
    a single underlying field configuration.

    In this sense, quantum nonlocality in Cosmochrony is ontological rather than
    dynamical: the field configuration is globally defined, while its evolution
    remains locally governed by the relaxation dynamics of $\chi$.

    This holistic character of $\chi$ plays a crucial role in quantum measurement.
    Because entangled systems correspond to a single, non-factorizable configuration,
    measurement outcomes cannot be understood as revealing pre-existing local
    properties.
    Instead, decoherence acts to suppress relational alternatives within a globally
    defined configuration, while local measurement outcomes correspond to effective
    reprojections selected by fluctuations.

    In this context, the Born rule does not encode nonlocal influence or hidden
    communication.
    It reflects the statistical distribution of locally accessible outcomes arising
    from a single holistic configuration of $\chi$, once relational coherence has been
    lost.
    Nonlocal correlations therefore arise from global structural consistency, while
    measurement statistics remain compatible with relativistic causality.

    Crucially, this global configuration does not encode predetermined measurement
    outcomes, but only a space of structurally compatible relational realizations,
    whose effective selection occurs through decoherence and reprojection.

  \subsection{Shared Configurations and Correlation Structure}
    \label{subsec:shared-configurations}

    When two particle-like excitations interact, they may form a composite
    configuration of the $\chi$ field that remains partially unified even after
    spatial separation in an effective geometric description.
    Such configurations give rise to persistent correlations between measurement
    outcomes.

    In an effective spacetime representation, this shared configuration may be
    schematically modeled as
    \begin{equation}
      \chi(x) = \chi_0 \exp\!\left(
                               -\frac{|x - x_1|^2}{\xi^2}
                               -\frac{|x - x_2|^2}{\xi^2}
      \right),
    \end{equation}
    where $x_1$ and $x_2$ label the effective locations of the two excitations.
    This expression is not fundamental, but serves as an illustrative coarse-grained
    representation of a single extended $\chi$ excitation whose internal structure
    spans multiple regions.

    The observed quantum correlations arise because measurements performed on
    different parts of this unified configuration probe the same underlying relational
    state, rather than because of any exchange of signals at the time of measurement.

  \subsection{Nonlocal Correlations Without Superluminality}
    \label{subsec:nonlocal-correlations-without-superluminality}

    Because the $\chi$ field evolves locally according to its relaxation dynamics,
    no superluminal propagation of information occurs.
    Measurement outcomes at spacelike separated regions do not influence one another
    through causal signals.

    Instead, correlated outcomes arise because both measurements sample the same
    pre-existing relational structure of $\chi$.
    This violates the factorization assumptions underlying Bell-type inequalities,
    while preserving dynamical locality and relativistic causality.

    In this way, Cosmochrony naturally accounts for experimentally observed violations
    of Bell inequalities without invoking nonlocal forces, retrocausality, or hidden
    signal channels.

  \subsection{Measurement, Decoherence, and Apparent Collapse}
    \label{subsec:measurement-and-decoherence}

    Within Cosmochrony, quantum measurement does not involve a fundamental wavefunction
    collapse.
    The $\chi$ field evolves continuously according to its intrinsic dynamics, and no
    discontinuous update of the underlying configuration occurs.

    What is conventionally interpreted as wavefunction collapse corresponds to the
    loss of coherence between different components of a $\chi$ configuration due to
    interaction with an environment.
    This process dynamically suppresses interference between alternative relational
    branches, yielding effectively classical outcomes.

    Decoherence therefore arises as a physical process rooted in the coupling between
    localized excitations and the broader $\chi$ field, rather than as a postulated
    measurement axiom~\cite{Zurek2003}.

    Importantly, decoherence does not destroy information at the level of the $\chi$
    substrate.
    Rather, it suppresses the relational accessibility of phase information within
    emergent spacetime.
    In this sense, decoherence may be viewed as a local and partial form of
    deprojection: relational histories become dynamically inaccessible while the
    underlying structural configuration of $\chi$ remains unchanged.

    More extreme regimes, such as those associated with strong gravitational
    confinement, represent a limiting case of this mechanism.
    There, relational and spatiotemporal descriptions themselves cease to be
    maintainable, and information undergoes full deprojection into the $\chi$
    substrate, beyond the domain where decoherence is defined.

  \subsection{Temporal Ordering and Relativistic Consistency}
    \label{subsec:temporal-ordering-and-relativistic-consistency}

    Because temporal ordering in Cosmochrony is defined by the monotonic relaxation of
    $\chi$, it does not rely on a global simultaneity structure.
    Different observers may assign different temporal orderings to spacelike separated
    events in effective geometric descriptions, without affecting the underlying
    dynamics of the field.

    Entanglement correlations are therefore compatible with relativistic causality:
    they do not depend on any preferred reference frame or absolute notion of time.
    The relational structure of $\chi$ remains invariant under changes of effective
    spacetime slicing.

  \subsection{Limits of Entanglement and Environmental Effects}
    \label{subsec:limits-of-entanglement}

    Entanglement is not a generic or permanent feature of all $\chi$ configurations.
    Environmental interactions, noise, and continued relaxation progressively degrade
    the coherence of shared configurations.

    As a result, entanglement is most robust for isolated systems and diminishes in
    macroscopic or strongly interacting environments.
    This explains the emergence of classical behavior at large scales without
    requiring a fundamental quantum-to-classical transition.

  \subsection{Summary}
    \label{subsec:summary8}

    Entanglement emerges in Cosmochrony as the persistence of shared topological
    configurations within the $\chi$ field.
    Quantum correlations arise without superluminal signaling, fundamental wavefunction
    collapse, or violations of relativistic causality.

    Within this framework, quantum phenomena reflect the holistic yet dynamically local
    structure of $\chi$, from which the standard quantum-mechanical formalism emerges
    as an effective statistical description rather than a fundamental ontology.
