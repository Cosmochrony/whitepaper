\subsection{Temporal Ordering and Relativistic Consistency}
  \label{subsec:temporal-ordering-and-relativistic-consistency}

  Within the Cosmochrony framework, temporal ordering is not defined by a global notion of simultaneity.
  It arises only at the level of effective descriptions, as an ordering of admissible
  projected events induced by constrained relaxation ordering, rather than by any
  fundamental or absolute time parameter.

  This effective character of temporal ordering follows directly from the
  non-injective nature of the projection from the underlying $\chi$ substrate,
  for which no unique global temporal parametrization exists at the fundamental
  level.

  Different observers may therefore assign different temporal orderings to spacelike
  separated events within effective geometric descriptions.
  Such differences reflect the observer-dependence of spacetime slicing and have no
  impact on the underlying relational consistency of admissible projected descriptions.
  No preferred foliation, global clock, or absolute temporal structure is selected at the fundamental level.

  Entanglement correlations are thus fully compatible with relativistic causality.
  Because entangled correlations originate from a single underlying relational
  configuration, their consistency does not depend on any particular temporal
  ordering assigned within effective descriptions.
  They do not rely on any privileged reference frame or on a globally defined temporal ordering.
  Instead, they arise from non-factorizable admissible projected descriptions whose
  internal relational consistency is preserved under changes of effective spacetime coordinates and temporal slicings.

  In this sense, relativistic covariance is maintained because the physical content of
  the theory resides in relational consistency conditions rather than in observer-dependent spatiotemporal labels.
  Temporal ordering remains an effective, observer-relative notion, while admissible
  correlations and their consistency relations remain invariant across all equivalent projected descriptions.

  Relativistic covariance is therefore preserved not by enforcing a specific
  temporal structure, but by the invariance of the underlying non-injective
  relational configuration across all admissible projected descriptions.

  \begin{tcolorbox}[
    colback=white,
    colframe=black!70,
    title={Relation to Time-Symmetric and Two-State Vector Formulations},
    fonttitle=\bfseries
  ]
    Several time-symmetric formulations of quantum mechanics, such as the
    two-state vector formalism, account for delayed-choice and post-selection
    effects by introducing both forward- and backward-evolving quantum states.

    In the Cosmochrony framework, this apparent temporal symmetry is a consequence of
    \textbf{Ontological Monism}.
    All correlated outcomes originate from a single underlying $\chi$ configuration,
    rather than from dynamically interacting subsystems.
    What appears as spatial or temporal separation in effective descriptions does not
    correspond to a fundamental ontological separation.
    Consequently, no notion of information transmission—whether forward or backward
    in time—is required to account for quantum correlations.

    What time-symmetric approaches encode through boundary conditions imposed at
    both initial and final times is here recovered as a purely structural feature
    of admissible projected descriptions.
    The future does not influence the past; rather, effective temporal ordering
    does not exhaust the relational information contained in the underlying
    configuration.

    Time-symmetric formalisms thus appear as efficient descriptive tools within
    standard quantum mechanics, but they are not ontologically required once
    non-injective projection and irreversible relaxation are taken as fundamental.
  \end{tcolorbox}
