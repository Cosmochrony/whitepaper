\subsection{Entanglement as a Critical Regime of Projective Compression}
  \label{subsec:entanglement-as-a-critical-regime-of-projective-compression}

  Within the Cosmochrony framework, quantum entanglement is not introduced as a primitive
  feature of physical systems, nor as a purely formal property of Hilbert space states.
  Instead, it arises as a structural consequence of the non-injective projection
  $\Pi:\chi \rightarrow \chi_{\mathrm{eff}}$ that maps the relational substrate to effective descriptions.

  \paragraph{Projection as an information-compressive process.}
    The projection $\Pi$ reduces a high-dimensional relational configuration of $\chi$ to a
    lower-dimensional effective description by discarding unresolved internal degrees of freedom.
    As a result, a single effective configuration $\chi_{\mathrm{eff}}$ generally corresponds
    to an equivalence class of admissible underlying configurations, forming a projection
    fiber $\Pi^{-1}(\chi_{\mathrm{eff}})$.
    This fiber may be understood as an information-theoretic channel whose effective bandwidth
    is determined by the number and structure of unresolved modes of $\chi$.
    The non-injectivity of $\Pi$ thus corresponds to an intrinsic compression of relational
    information, rather than to epistemic ignorance or hidden variables.

  \paragraph{Compression and effective separability.}
    The degree of compression induced by $\Pi$ controls the structure of admissible projected descriptions.
    In the limit of negligible compression, the effective description retains too much
    microscopic relational detail to admit a stable decomposition into subsystems, and no
    robust notion of separability arises.
    Conversely, in the limit of extreme compression, most relational information is erased,
    and projected descriptions become effectively factorized, recovering classical statistical behavior.

    Crucially, non-factorizable correlations do not increase monotonically with the strength of compression.
    Instead, they emerge only within an intermediate regime in which the effective description
    is sufficiently coarse-grained to permit subsystem identification, yet retains enough
    global relational structure to prevent full factorization.

  \paragraph{Entanglement as a critical regime.}
    Quantum entanglement corresponds precisely to this intermediate, critical regime of projective compression.
    In this regime, distinct effective subsystems are well defined, but remain globally
    constrained by compatibility conditions inherited from the underlying relational configuration.
    As a result, joint outcome statistics fail to admit an ontologically factorizable
    representation, even though no dynamical interaction or information exchange occurs
    between spatially separated subsystems.

    This interpretation naturally explains why entanglement correlations are both robust and bounded.
    If compression is increased beyond the critical regime—through environmental coupling,
    decoherence, or coarse-graining—the effective description becomes over-compressed and
    correlations are suppressed, leading to classical behavior.
    If compression is reduced below the critical regime, effective subsystem separation
    breaks down and no stable notion of entanglement applies.

  \paragraph{Structural role of entanglement.}
    From this perspective, entanglement is neither a consequence of maximal information
    preservation nor of maximal information loss.
    Rather, it is a structural feature that emerges at the boundary between the two, as a
    manifestation of residual global constraints surviving projection.
    This view unifies the appearance of entanglement, its sensitivity to environmental
    effects, and its disappearance in the classical limit within a single relational and
    information-theoretic mechanism.

    Bell inequality violations, discussed in Section~\ref{subsec:relation-to-bell-inequalities}, follow necessarily from
    the failure of ontological factorization in this critical regime, and do not require the introduction of
    superluminal influences or hidden variables.

  \paragraph{Intermittent character of the critical regime.}
    Importantly, the critical regime associated with entanglement need not be realized
    as a single continuous interval of projective compression.
    Because the structure of the projection fiber depends on the detailed spectral
    organization of admissible $\chi$-configurations, non-factorizable correlations may
    emerge only at specific points where spectral reorganization occurs.

    In this sense, entanglement is not a generic property of all moderately compressed
    descriptions, but a critically intermittent phenomenon.
    Non-factorizable correlations may appear, disappear, and reappear as the relational
    spectrum reorganizes under relaxation or external constraints, even when global
    compression parameters vary monotonically.

  \paragraph{Activation versus admissibility.}
    Bell inequalities constrain the logical structure of admissible effective descriptions,
    but do not determine when entanglement is physically activated.
    Within Cosmochrony, admissibility and activation are distinct notions:
    non-injective projection makes entanglement \emph{possible}, while its actual
    manifestation depends on whether the projection fiber retains sufficient internal
    freedom before saturation effects suppress relational mobility.

    Numerical investigations reported in Appendix~\ref{sec:appendix-simulation} support this distinction, showing
    that entanglement-related diagnostics peak only during specific spectral
    reconfiguration events, and are suppressed both in weakly constrained
    (under-compressed) and strongly saturated (over-compressed) regimes.
