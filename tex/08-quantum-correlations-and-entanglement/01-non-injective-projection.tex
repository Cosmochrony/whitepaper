\subsection{Non-Injective Projection as the Origin of Quantum Correlations}
  \label{subsec:non-injective-projection}

  The emergence of quantum correlations and entanglement in the Cosmochrony
  framework is rooted in a single structural property of the projection from
  the fundamental $\chi$ substrate to effective physical descriptions:
  this projection is \emph{non-injective}.

  A single admissible configuration of $\chi$ may correspond to multiple
  distinct effective degrees of freedom.
  What appear, at the effective level, as separate particles or subsystems
  are therefore multiple projective manifestations of a single underlying
  ontological configuration.

  This non-injectivity implies that such degrees of freedom cannot be assigned
  independent states.
  Any admissible effective description must be globally consistent with the
  underlying $\chi$ configuration, leading to persistent correlations that
  do not rely on signal exchange, causal influence, or spacetime proximity.

\subsubsection*{Ontological Monism and the Shared Projection Hypothesis}
  \label{subsec:ontological-monism-shared-projection}

  In Cosmochrony, the apparent multiplicity of effective particles does not imply a multiplicity of
  ontological entities.
  The framework adopts \textbf{Ontological Monism}: there is a single ontological source, the substrate $\chi$.
  What appears as ``two systems'' in spacetime may correspond to a single underlying relational configuration.

  \paragraph{Shared projection hypothesis.}
    Because the projection $\Pi$ is non-injective and may admit multiple effective images,
    a single connected excitation in $\chi$ can be represented as several spatially separated effective
    excitations in the emergent description.
    The observed separation is therefore a property of the projected metric representation,
    not a fundamental separation of the underlying entity.

  \paragraph{Shared Fiber Phase and a Geometric Reading of Spin Correlations}
    \label{subsec:shared-fiber-phase-spin}

    In the Cosmochrony interpretation, spin correlations can be read as correlations of a \emph{shared}
    internal degree of freedom of the projection fiber, rather than as correlations between independent
    spacetime-local properties.
    When an entangled pair admits a description as multiple effective images of a single underlying
    $\chi$-configuration (Section~\ref{subsec:ontological-monism-shared-projection}), a measurement at one location
    selects a locally stable reprojection of that shared fiber degree of freedom.
    The correlated statistics at the distant location then reflect the restricted set of reprojections
    still compatible with the same underlying configuration, without invoking any signal exchange.

    This viewpoint is compatible with the emergence of half-integer spin from nontrivial fiber topology
    (e.g.\ Hopf-type structures), while keeping the explanation at the structural level: the relevant
    degrees of freedom are global properties of the shared configuration, not attributes of the two
    effective particles taken separately.

  \paragraph{Consequence for nonlocal correlations.}
    Correlations between measurements performed at spacelike separated locations do not require any
    superluminal influence.
    They follow from the fact that the correlated outcomes are local reprojections of the \emph{same}
    underlying configuration.
    In this sense, the ``Bell tension'' is not resolved by a faster-than-light mechanism, but by denying
    the premise that two independent ontological subsystems existed in the first place.
