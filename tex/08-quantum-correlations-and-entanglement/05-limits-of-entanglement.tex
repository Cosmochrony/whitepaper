\subsection{Limits of Entanglement and Environmental Effects}
  \label{subsec:limits-of-entanglement}

  Entanglement is not a generic or permanent feature of admissible projected
  descriptions.
  It arises only within restricted regimes in which a non-factorizable global
  description remains jointly projectable into an effective spacetime representation.

  Interactions with an environment progressively restrict the set of admissible
  projected descriptions.
  As additional degrees of freedom become dynamically coupled, global relational
  coherence ceases to be representable within a single effective description.
  The system then admits only effectively factorized local projections, and
  entanglement is no longer accessible at the level of projected descriptions.

  As a result, entanglement is most robust for effectively isolated systems and
  becomes increasingly fragile in macroscopic or strongly interacting environments.
  This transition does not correspond to a physical degradation of an underlying
  substrate, but to a progressive loss of projectability of non-factorizable
  descriptions.

  In this sense, the emergence of classical behavior at large scales reflects a
  descriptive limitation rather than a fundamental quantum-to-classical transition.
  Classicality arises when only factorized projected descriptions remain admissible,
  without requiring any modification of the underlying relational structure or the
  introduction of additional postulates.
