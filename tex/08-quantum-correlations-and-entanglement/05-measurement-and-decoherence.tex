\subsection{Measurement, Decoherence, and Apparent Collapse}
  \label{subsec:measurement-and-decoherence}

  Within the Cosmochrony framework, quantum measurement does not involve a fundamental wavefunction collapse.
  At the fundamental level, no discontinuous update of the underlying relational substrate occurs.
  What is conventionally described as collapse arises entirely within the domain of effective projected descriptions.

  This is a direct consequence of the non-injective nature of the projection from the
  underlying $\chi$ substrate: multiple effective descriptions may correspond to a
  single relational configuration, with no uniquely defined inverse mapping.

  Measurement corresponds to the transition from a non-factorizable admissible
  projected description to a set of effectively factorized local projections.
  This transition is induced by interactions with an environment that progressively
  eliminate the accessibility of global relational coherence.
  As a result, alternative relational components cease to admit a joint effective
  description within a single spacetime representation.

  From this perspective, decoherence corresponds to an effective reduction of the
  multiplicity of admissible non-injective projections to a set of locally injective
  descriptions.

  Decoherence therefore does not represent a postulated measurement axiom, but a
  dynamical restriction on admissible projected descriptions~\cite{Zurek2003}.
  It suppresses interference between incompatible descriptive branches by rendering
  their relative phase information inaccessible within spacetime representations.
  The underlying relational structure remains globally well defined, even though it
  can no longer be represented coherently at the effective level.

  Importantly, decoherence does not destroy information.
  It limits the projectability of relational correlations into spacetime descriptions.
  In this sense, decoherence may be understood as a local and partial loss of
  projectability: certain relational distinctions persist structurally but cease to be
  jointly representable within an effective geometric description.

  More extreme regimes, such as those associated with strong gravitational confinement,
  represent a limiting case of this mechanism.
  In such regimes, not only coherence but spacetime representability itself breaks down.
  Relational information remains globally encoded, but undergoes a complete loss of
  spatiotemporal projectability, beyond the domain in which decoherence and effective
  Hilbert-space descriptions can be meaningfully defined.

  The apparent collapse observed in quantum measurement is therefore not a physical
  event, but the effective manifestation of a non-injective relational structure becoming
  only partially projectable into spacetime descriptions.

  \begin{tcolorbox}[
    colback=white,
    colframe=black!70,
    title={Do Quantum Particles Modify Their Past?},
    fonttitle=\bfseries
  ]
    It is sometimes claimed that quantum mechanics allows particles to
    \emph{modify their past}, particularly in delayed-choice or quantum eraser experiments.
    This statement is misleading.

    In quantum mechanics, physical descriptions are constrained globally by consistency conditions.
    Certain properties—such as path information or temporal ordering—do not possess
    well-defined values independently of the measurement context.
    A later measurement choice does not alter a previously existing physical fact,
    but rather restricts the set of admissible descriptions compatible with the entire experimental configuration.

    In the Cosmochrony framework, this behavior follows directly from the non-injective character of projection.
    A single underlying $\chi$ configuration may admit multiple effective
    descriptions, none of which defines a unique local past prior to measurement.
    The act of measurement does not modify the underlying relational structure;
    it reduces the multiplicity of admissible projections, rendering certain past properties effectively definable.

    Thus, quantum mechanics does not imply retrocausal influence or violations of relativistic causality.
    Instead, it reveals that some features commonly attributed to the past are not
    fundamental ontological facts, but emergent properties of admissible projected descriptions.
  \end{tcolorbox}
