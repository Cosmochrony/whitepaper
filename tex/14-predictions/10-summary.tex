\subsection{Summary}
  \label{subsec:summary-predictions}

  Cosmochrony yields a set of observationally testable phenomenological signatures
  spanning cosmology, gravitation, galactic dynamics, and particle phenomena, all
  interpreted as emergent consequences of non-injective projection and structural relaxation.
  These signatures arise across widely separated physical scales but share a common
  origin in the relational dynamics of the \(\chi\) substrate.

  While all predicted effects remain compatible with current observations, the framework
  generically allows for correlated departures from standard effective predictions.
  Such departures are not expected to be isolated anomalies, but structurally related
  features that may become accessible to future high-precision measurements across multiple observational domains.
  Across these domains, Cosmochrony predicts quantitative deviations from $\Lambda$CDM
  at the percent level, including a $\sim 10\%$ suppression of low-$\ell$ CMB power,
  a mild redshift-dependent enhancement of $H(z)$ at intermediate redshifts, and
  coherence-modifying effects in gravitational-wave propagation near compact objects.
  These deviations arise from a single underlying mechanism—the irreversible relaxation
  and partial projectability of the $\chi$ substrate—and define a coherent set of discriminating observational tests.

  Taken together, these signatures provide concrete and diverse avenues for empirical scrutiny.
  They are not introduced as ad hoc modifications, but arise generically from the
  relaxation dynamics of the $\chi$ substrate and from the structural constraints of non-injective projection.
  Their role is to delineate potential observational discriminants of the framework,
  rather than to provide precision predictions at the current stage.
  The confirmation or falsification of any subset of these effects would therefore
  directly constrain the viability of Cosmochrony as a physical framework and, more
  importantly, test the hypothesis that a single underlying relaxation mechanism can
  account for phenomena ranging from cosmological expansion to particle decay.
