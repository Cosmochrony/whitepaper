\subsection{Galaxy Rotation Curves from Structural Relaxation}
  \label{subsec:galaxy-rotation-chi}

  As shown in Section~\ref{subsec:phi-eff-galaxies}, the Cosmochrony framework
  predicts modifications of galactic rotation curves arising from the structural
  relaxation of the projected \(\chi_{\mathrm{eff}}\) field, without introducing
  additional dark matter components.

  The Cosmochrony framework predicts modifications of galactic rotation curves arising
  from the structural relaxation of the projected \(\chi_{\mathrm{eff}}\) field, without
  introducing additional dark matter components.

  In regions where localized matter configurations induce persistent relaxation
  gradients in the \(\chi\)-substrate, the effective inertial response of orbiting matter
  is modified.
  This leads to an enhancement of tangential velocities at large radii, producing
  approximately flat rotation curves.

  Unlike modified gravity scenarios, this effect does not require an explicit change of
  the gravitational force law.
  It arises from a non-local redistribution of relaxation capacity in the projected
  description, tied to the large-scale coherence of the underlying \(\chi\)-configuration.

  Observable consequences include:
  \begin{itemize}
    \item a correlation between rotation curve flattening and indicators of structural
    relaxation activity rather than baryonic mass alone,
    \item deviations from simple baryonic scaling relations in dynamically young or
    disturbed galaxies,
    \item a reduced need for fine-tuned dark matter profiles in low-surface-brightness
    systems.
  \end{itemize}

  These signatures provide a discriminant between Cosmochrony, particle dark matter
  models, and purely phenomenological modified gravity approaches.
