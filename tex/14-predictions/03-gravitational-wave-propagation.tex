\subsection{Gravitational Wave Propagation}
  \label{subsec:gravitational-wave-propagation}

  In the Cosmochrony framework, gravitational waves correspond to propagating
  collective modulations of the $\chi$ field in regimes where a spacetime
  description is applicable.
  They do not constitute independent propagating degrees of freedom, but reflect
  time-dependent redistributions of relaxation constraints within the field.

  In regions of high excitation density, such as near compact objects, the local
  slowdown of $\chi$ relaxation is expected to modify the propagation of these
  modulations.
  In particular, partial decoherence or attenuation may arise due to the coupling
  of propagating modulations to strongly constrained relaxation regions.
  These effects originate from the same collective relaxation constraints
  responsible for gravitational time dilation and horizon formation, and do not
  require the introduction of additional dynamical fields.

  \paragraph{Order-of-magnitude attenuation estimate.}
    Consider a compact object of mass $M$, characterized in effective geometric
    descriptions by a Schwarzschild radius
    \[
      r_s = \frac{2GM}{c^2}.
    \]
    Gravitational-wave modulations of the $\chi$ field propagating through regions
    where the effective relaxation rate is significantly reduced are expected to lose
    coherence through partial redistribution into non-propagating relaxation modes.

    For waves traversing regions within a characteristic distance
    \[
      r \lesssim 10\,\frac{GM}{c^2},
    \]
    the cumulative reduction of effective relaxation conductivity suggests an
    attenuation factor that may be parametrized, at the level of order-of-magnitude
    estimates, as
    \[
      \frac{\Delta A}{A} \sim \mathcal{O}(10^{-2} - 10^{-1}),
    \]
    where the precise magnitude depends on the local $\chi$ correlation length $\xi$
    and on the effective relaxation fraction $\Omega_\chi$ in the vicinity of the
    source.
    This attenuation should be interpreted as a redistribution of wave coherence
    within the $\chi$ relaxation dynamics rather than as dissipative energy loss in
    the conventional field-theoretic sense.

  \paragraph{Observational signature.}
    Such effects are expected to manifest most clearly during the late-time ringdown
    phase of binary black hole mergers, where gravitational-wave signals probe the
    strongly constrained relaxation regime near the effective horizon.
    The resulting signature would appear as a frequency-dependent deviation from
    general relativistic ringdown templates, potentially mimicking anomalous damping
    or mode-dependent quality factors.

    While current ground-based detectors do not yet achieve the signal-to-noise ratios
    required to resolve attenuation at the few-percent level, future space-based
    observatories operating in the LISA band, with expected signal-to-noise ratios
    exceeding $\sim 100$ for massive black hole mergers, may provide sufficient
    sensitivity to test this prediction.

  \paragraph{Semi-quantitative scaling estimate.}
    Within the Cosmochrony framework, attenuation of gravitational-wave amplitudes
    near compact objects arises from the local suppression of $\chi$ relaxation in
    regions of high effective curvature.
    At leading order, the relative amplitude reduction is expected to scale with the
    dimensionless curvature parameter $(r_s / r)$.

    A simple dimensional estimate yields
    \[
      \frac{\Delta A}{A} \sim \left( \frac{r_s}{r} \right)^2 ,
    \]
    indicating that the effect depends explicitly on both the compact object mass and
    the wave trajectory’s impact parameter.
    For propagation at distances $r \approx 10\,r_s$, this scaling gives
    \[
      \frac{\Delta A}{A} \sim 10^{-2},
    \]
    consistent with the order-of-magnitude estimates above and with exploratory
    numerical results obtained from $\chi$-field simulations
    (Appendix~D.3).
