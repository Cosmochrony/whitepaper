\subsection{Gravitational Wave Propagation}
  \label{subsec:gravitational-wave-propagation}

  As discussed in Section~\ref{subsec:gravitational-waves}, in the Cosmochrony framework,
  gravitational waves are not interpreted as fundamental propagating excitations of the
  $\chi$ substrate.
  They correspond instead to coherent, extended projective descriptions of
  large-scale relational variations, admissible only in regimes where an effective
  spacetime representation applies.

  These projective descriptions do not constitute independent dynamical degrees of
  freedom.
  Rather, they reflect time-dependent redistributions of relaxation constraints at
  the level of effective projection.

  In regions of high excitation density, such as near compact objects, the local
  suppression of $\chi$ relaxation is expected to modify the persistence and coherence
  of these projective descriptions.
  Rather than inducing dissipative losses, this effect manifests as frequency-dependent
  phase shifts or dispersion-like behavior in gravitational-wave signals, arising from
  partial loss of projective coherence due to coupling with strongly constrained
  relaxation regions.
  These effects originate from the same collective relaxation constraints responsible
  for gravitational time dilation and horizon formation, and do not require the
  introduction of additional dynamical fields.

  \textbf{LISA signature}: Cosmochrony predicts a $\sim 10\%$ effective reduction of
  coherent gravitational-wave amplitudes near black holes
  (Section~\ref{subsec:strong-gravity-and-black-holes}), distinct from general
  relativity’s purely propagative behavior.

  \paragraph{Order-of-magnitude attenuation estimate.}
    Consider a compact object of mass $M$, characterized in effective geometric
    descriptions by a Schwarzschild radius
    \[
      r_s = \frac{2GM}{c^2}.
    \]
    Gravitational-wave descriptions traversing regions where the effective relaxation
    capacity is significantly reduced are expected to exhibit a partial degradation of
    coherence, due to redistribution of descriptive weight into locally non-projectable
    configurations.

    For trajectories passing within a characteristic distance
    \[
      r \lesssim 10\,\frac{GM}{c^2},
    \]
    the cumulative reduction of effective relaxation conductivity suggests an
    order-of-magnitude suppression of the coherent projective amplitude that may be
    parametrized as
    \[
      \frac{\Delta A}{A} \sim \mathcal{O}(10^{-2} - 10^{-1}),
    \]
    where the precise magnitude depends on the local $\chi$ correlation length $\xi$
    and on the effective relaxation fraction $\Omega_\chi$ in the vicinity of the
    source.
    This effect should be interpreted as a redistribution of projective coherence
    within the relaxation dynamics, rather than as dissipative energy loss.

  \paragraph{Observational signature.}
    Such effects are expected to manifest most clearly during the late-time ringdown
    phase of binary black hole mergers, where gravitational-wave signals probe regions
    of strongly constrained relaxation near the effective horizon.
    The resulting signature would appear as frequency-dependent deviations from
    general relativistic ringdown templates, potentially mimicking anomalous damping
    or mode-dependent quality factors.

    While current ground-based detectors do not yet achieve the signal-to-noise ratios
    required to resolve attenuation at the few-percent level, future space-based
    observatories operating in the LISA band, with expected signal-to-noise ratios
    exceeding $\sim 100$ for massive black hole mergers, may provide sufficient
    sensitivity to test this prediction.

  \paragraph{Semi-quantitative scaling estimate.}
    Within the Cosmochrony framework, the reduction of coherent gravitational-wave
    amplitudes near compact objects arises from the local suppression of effective
    projectability in regions of high curvature.
    At leading order, the relative amplitude reduction is expected to scale with the
    dimensionless curvature parameter $(r_s / r)$.

    A simple dimensional estimate yields
    \[
      \frac{\Delta A}{A} \sim \left( \frac{r_s}{r} \right)^2 ,
    \]
    indicating that the effect depends explicitly on both the compact object mass and
    the wave trajectory's impact parameter.
    For propagation at distances $r \approx 10\,r_s$, this scaling gives
    \[
      \frac{\Delta A}{A} \sim 10^{-2},
    \]
    consistent with the order-of-magnitude estimates above and with exploratory
    numerical results obtained from $\chi$-based simulations
    (Appendix~D.3).

    The same structural relaxation mechanisms that affect cosmological expansion and
    gravitational-wave propagation are expected to leave observable imprints at galactic
    scales, where long-lived projected configurations interact with the surrounding
    \(\chi\)-substrate.
