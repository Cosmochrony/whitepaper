\subsection{Neutrino-Mediated Relaxation and Decay Signatures}
  \label{subsec:neutrino-decay-signatures}

  An implication of the structural interpretation of particle decay
  (Section~\ref{subsec:metastability-and-decay}) together with
  the partial projectability of neutrino modes
  (Section~\ref{subsec:neutrinos-partially-projectable-modes}) is that, in
  Cosmochrony, particle decay and neutrino emission are manifestations of structural
  reorganization rather than independent microscopic processes.
  This interpretation leads to distinct observational signatures.

  Because neutrino-like excitations act as non-local relaxation channels, decay processes
  in the early universe contribute to an irreversible smoothing of admissible configurations.
  This smoothing is expected to leave detectable imprints across multiple observational domains.

  Specifically, the framework predicts:
  \begin{itemize}
    \item an enhanced role of neutrino backgrounds in suppressing large-scale coherence
    without behaving as conventional radiation pressure,
    \item a weak coupling between decay rates and late-time environmental conditions,
    reflecting their origin in early structural metastability,
    \item possible correlations between decay-driven neutrino emission and large-scale
    anisotropies observed in the cosmic microwave background.
  \end{itemize}

  At the particle-physics level, this interpretation suggests that decay lifetimes and
  branching ratios encode information about the stability landscape of admissible
  projected configurations rather than fundamental stochasticity.
  Future precision measurements of rare decays may therefore provide indirect probes of
  the structural relaxation dynamics underlying the Cosmochrony framework.

  \subsubsection*{Environmental Modulation of Particle Stability}
    \label{subsec:environmental-decay-modulation}

    A distinctive prediction of the Cosmochrony framework is that particle stability is not
    strictly universal, but may exhibit a weak dependence on the surrounding structural environment.
    Because particle decay originates from the susceptibility of metastable projected
    configurations to their own admissible fluctuations, any factor that modifies the local
    relaxation landscape can, in principle, affect decay rates.

    In regions characterized by strong gradients of the \(\chi\) substrate—such as galactic
    cores or highly structured gravitational environments—the spectrum and amplitude of
    admissible fluctuations are expected to differ slightly from those in weakly structured
    regions.
    As a result, the effective lifetime of unstable particles may acquire a small environment-dependent modulation.

    This effect is predicted to be extremely weak and therefore compatible with all current
    laboratory and astrophysical constraints.
    Nevertheless, it represents a qualitatively new signature: a violation of strict
    universality of decay rates induced not by local interactions or spacetime curvature,
    but by structural relaxation gradients in the underlying relational substrate.

    Future high-precision measurements of decay processes in environments with strong
    gravitational or structural gradients, as well as dedicated numerical simulations of
    \(\chi\)-field dynamics, may allow quantitative estimates of this effect.
    Its detection or exclusion would provide a direct and stringent test of the Cosmochrony framework.

    \paragraph{Order-of-magnitude estimate.}

      Within the Cosmochrony framework, the effective lifetime of an unstable particle is
      controlled by the susceptibility of a metastable projected configuration to its own
      admissible fluctuations.
      Because the spectrum and density of such fluctuations depend weakly on the surrounding
      structural environment, a small modulation of decay rates is expected in regions
      characterized by strong gradients of the \(\chi\) substrate.

      A minimal parametrization of this effect may be written as
    \begin{equation}
      \frac{\delta \Gamma}{\Gamma}
      \;\simeq\;
      \beta\,\frac{\Delta U}{c^2},
      \end{equation}
      where \(\Gamma\) is the decay rate, \(U\) denotes an effective gravitational or
      structural potential serving as a proxy for the local relaxation gradient, and
      \(\beta\) is a dimensionless sensitivity coefficient encoding the response of the
      projected configuration to environmental variations.

      Existing tests of local position invariance based on high-precision atomic clocks
      strongly constrain any environmental dependence of effective physical rates.
      Requiring compatibility with these constraints suggests a conservative bound
      \(\beta \lesssim 10^{-6}\).

      For typical galactic environments, the difference in effective potential between
      weakly structured regions and galactic cores is of order
      \(\Delta U / c^2 \sim 10^{-7}\)–\(10^{-6}\).
      Combining these estimates yields a fractional modulation of particle lifetimes of
      order
    \begin{equation}
      \frac{\delta \tau}{\tau}
      \;\sim\;
      10^{-13}\text{--}10^{-12}.
      \end{equation}

      Such an effect is far below the sensitivity of current laboratory measurements and
      therefore fully compatible with existing experimental bounds.
      Nevertheless, it represents a qualitatively novel signature: a weak violation of the
      strict universality of decay rates induced not by spacetime curvature or local
      interactions, but by structural relaxation gradients in the underlying relational
      substrate.
      Future high-precision astrophysical observations or dedicated numerical simulations of
      \(\chi\)-field dynamics may allow this prediction to be further quantified or
      constrained.

    \paragraph{Which decay channels are most sensitive?}

      In Cosmochrony, an environmental modulation of decay rates is expected to be largest
      for metastable configurations whose admissible fluctuation spectrum is only weakly
      protected by topology, and whose decay proceeds through a comparatively ``thin''
      projective channel (small number of admissible final factorizable branches).
      This suggests the following qualitative hierarchy of experimental leverage:

    \begin{itemize}
      \item \textbf{Purely leptonic weak decays} (e.g.\ \(\mu^\pm \to e^\pm \nu \bar{\nu}\)):
      theoretically clean (minimal hadronic uncertainty), with extremely well-characterized
      kinematics. The limitation is practical: \(\tau_\mu\) is measured very precisely, but
      changing the \emph{structural environment} sufficiently in the laboratory is difficult.

      \item \textbf{Hadronic weak decays and oscillating neutral mesons}
      (e.g.\ \(K^0\)--\(\bar{K}^0\), \(B^0\)--\(\bar{B}^0\)):
      in standard physics these are exquisitely sensitive to tiny perturbations in the
      effective Hamiltonian. In Cosmochrony terms, they probe whether the projection fiber
      admits a detectable environment-dependent bias between conjugate branches.
      The price is interpretability: hadronic and medium effects require careful control.

      \item \textbf{Nuclear \(\beta\)-decays and long-lived isotopes}:
      exceptionally good metrological stability allows long integration times, but nuclear
      structure systematics are harder to disentangle from any putative projective effect.

      \end{itemize}

      Overall, the cleanest conceptual target is leptonic weak decay, while the most
      ``amplified'' interferometric target is neutral-meson mixing, provided that standard
      environmental systematics are tightly controlled.

    \paragraph{Differential astrophysical signature.}

      Directly comparing lifetimes of unstable particles between ``void'' and typical
      galactic cores is observationally challenging, because decay processes are not
      tagged \emph{in situ}.
      However, Cosmochrony predicts a \emph{differential} effect that can in principle be
      searched for in environments where the effective potential proxy \(\Delta U/c^2\)
      is substantially larger than in ordinary galactic regions.

      In particular, energetic hadronic cascades near compact objects (accretion flows and
      relativistic jets) are controlled by the competition between interaction lengths and
      decay lengths of \(\pi^\pm\), \(K^\pm\), and \(\mu^\pm\), which shapes the emergent
      high-energy \(\gamma\)-ray and neutrino spectra.
      If decay rates acquire a weak environmental modulation,
    \begin{equation}
      \frac{\delta \Gamma}{\Gamma} \simeq \beta\,\frac{\Delta U}{c^2},
      \end{equation}
      then the \emph{effective} critical energy at which decay dominates over interaction
      is shifted by the same fractional amount,
    \begin{equation}
      \frac{\delta E_\ast}{E_\ast} \sim \frac{\delta \tau}{\tau}
      \sim \beta\,\frac{\Delta U}{c^2}.
      \end{equation}

      Near the innermost regions of accretion flows around massive compact objects,
      a characteristic scale \(\Delta U/c^2 \sim 10^{-4}\) is not excluded as an order-of-magnitude proxy,
      leading (for conservative \(\beta \lesssim 10^{-6}\)) to
    \begin{equation}
      \frac{\delta \tau}{\tau} \sim 10^{-10},
      \end{equation}
      still extremely small, but conceptually clean: a systematic, environment-correlated
      spectral bias that cannot be mimicked by late-time cosmological parameter shifts.

      The key discriminant is \emph{correlation with environment}:
      sources with otherwise similar intrinsic properties but different compactness proxies
      (e.g.\ inferred emission radius or gravitational redshift indicators) should exhibit
      a tiny but coherent shift in the decay-limited spectral features of hadronic secondaries.

    \paragraph{Connection to numerical \(\chi\)-field simulations.}

      The sensitivity coefficient \(\beta\) can be operationally defined and extracted from
      \(\chi\)-field simulations by measuring how the \emph{escape time} of a metastable
      localized configuration changes when embedded in a controlled background gradient.

      Let \(\tau_0\) denote the mean first-passage (escape) time of a metastable localized
      configuration in a reference background, estimated from an ensemble of runs with
      different admissible fluctuation seeds.
      Introduce a dimensionless structural gradient proxy \(G\) (computed from the projected
      field) such as
    \begin{equation}
      G \;\equiv\; \frac{\ell^2}{\chi_0^2}\,\langle |\nabla \chi_{\mathrm{eff}}|^2 \rangle_{\mathcal{R}},
      \end{equation}
      where \(\ell\) is a chosen coarse-graining scale, \(\chi_0\) a normalization, and
      \(\langle \cdot \rangle_{\mathcal{R}}\) denotes averaging over a region \(\mathcal{R}\)
      containing the localized excitation.

      Define the empirical susceptibility
    \begin{equation}
      S \;\equiv\; \frac{d\ln \tau}{dG},
      \end{equation}
      estimated numerically by repeating the experiment at slightly different controlled
      background gradients \(G\).
      A minimal mapping to the phenomenological parametrization
      \(\delta\tau/\tau \simeq \beta\,\Delta U/c^2\) is then obtained by specifying an
      effective correspondence between \(\Delta U/c^2\) and \(\Delta G\) in the simulation,
    \begin{equation}
      \frac{\delta \tau}{\tau} \;\simeq\; S\,\Delta G
      \;\equiv\; \beta\,\frac{\Delta U}{c^2}.
      \end{equation}

      In practice, this provides a clear numerical pipeline:
      (i) prepare a metastable localized configuration, (ii) embed it in backgrounds with
      controlled \(G\), (iii) estimate \(\tau(G)\) from ensembles, and (iv) infer \(S\),
      hence \(\beta\), up to the chosen environment proxy mapping.
      The resulting \(\beta\) can then be confronted with the conservative metrological
      requirement \(\beta \lesssim 10^{-6}\).
