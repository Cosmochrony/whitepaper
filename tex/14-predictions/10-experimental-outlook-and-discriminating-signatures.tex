\subsection{Experimental Outlook and Discriminating Signatures}
  \label{subsec:experimental-outlook-and-discriminating-signatures}

  A recurring theme of the Cosmochrony framework is that extreme physical regimes
  are not resolved by unbounded transport or arbitrarily large fields, but by the
  creation of new stable structures that enlarge the space of admissible configurations.
  The Schwinger effect, reinterpreted as a saturation-induced transition of the relaxation
  flux, provides a paradigmatic example of this principle.

  In this perspective, particle creation is not an anomaly of quantum vacuum physics,
  but a universal dissipation channel activated whenever directional relaxation approaches
  its maximal transport capacity.
  This mechanism suggests a unified interpretation of several high-energy phenomena,
  from laboratory experiments to astrophysical and cosmological regimes, while remaining
  fully compatible with established effective descriptions.

  \subsubsection*{Astrophysical Jets: Saturation Clamping Near Compact Objects}

    \paragraph{Astrophysical jets as saturation channels.}
      In the vicinity of rotating compact objects, particularly supermassive black holes,
      the density of directional relaxation flux may reach a quasi-permanent saturation regime.
      Within Cosmochrony, relativistic jets are then interpreted not merely as magnetically
      driven outflows, but as dynamically selected channels through which excess substrate
      tension is discharged by continuous creation of projectable structures.

    \paragraph{Discriminating signature.}
      A characteristic prediction is the existence of a \emph{saturation clamping}:
      beyond a critical stress near the horizon, further increases in the effective field
      do not translate linearly into higher transport, but instead into enhanced pair loading
      of the jet.
      This predicts a non-linear relation, or a sharp change of regime, between near-horizon
      stress indicators and plasma density.
      Such a signature is directly testable with high-resolution observations of jet bases,
      notably by the Event Horizon Telescope.

  \subsubsection*{Primordial Cosmology: Structural Reheating and Discrete Nucleation}

    \paragraph{Structural reheating in the early Universe.}
      At the earliest stages of cosmic history, the principle of ontological poverty implies
      a highly constrained configuration of the $\chi$ substrate.
      In this regime, rapid expansion is interpreted not as a driving force, but as the
      geometric response to the necessity of increasing the admissible configuration space.
      Matter production emerges as a byproduct of this process, as the substrate relaxes by
      nucleating stable projectable modes.

    \paragraph{Discriminating signature.}
      In contrast with standard inflationary scenarios based on smooth scalar fields,
      Cosmochrony predicts that primordial fluctuations may carry imprints of discrete
      nucleation events.
      This suggests specific forms of non-Gaussianity and phase correlations, particularly
      at large angular scales, where observed low-$\ell$ anomalies in the CMB may be
      interpreted as relics of the earliest admissibility transitions.

  \subsubsection*{Ultra-High-Energy Cosmic Rays: Local Production by Saturation Spikes}

    \paragraph{Local nucleation of ultra-high-energy excitations.}
      Ultra-high-energy cosmic rays exceeding the GZK threshold may be reinterpreted as
      locally produced excitations, generated during transient saturation spikes of the
      relaxation flux in regions of extreme curvature or field intensity.
      In this view, the energy of these particles reflects local relaxation constraints
      rather than long-distance acceleration and transport.

    \paragraph{Discriminating signature.}
      This scenario predicts a strong anisotropy correlated with the distribution of nearby
      compact objects capable of sustaining saturation regimes.
      A decisive signature would be the detection of multimessenger correlations, combining
      UHECRs, high-energy neutrinos, gamma-ray activity, and rapid changes in polarization
      during episodes of enhanced relaxation stress.

  \subsubsection{Synthesis: A Multi-Scale Falsification Program}
    Taken together, these perspectives illustrate how Cosmochrony transforms the notion
    of dissipation into a unifying, testable principle across scales.
    Rather than constituting a closed theoretical system, the framework naturally suggests
    a hierarchy of experimental and observational tests:

    \begin{itemize}
      \item \textbf{Atomic scale:} precision tests of Lamb shifts and hyperfine structure
      (Section~\ref{subsec:lamb-shift}).
      \item \textbf{Laboratory scale:} Schwinger threshold and saturation effects in
      ultra-intense laser experiments (Section~\ref{subsec:schwinger-saturation}).
      \item \textbf{Astrophysical scale:} saturation clamping and chiral signatures in
      relativistic jets near black holes.
      \item \textbf{Cosmological scale:} non-Gaussian and large-scale anomalies as relics
      of discrete primordial nucleation.
    \end{itemize}

    In this sense, Cosmochrony does not seek to replace established effective theories,
    but to provide a minimal structural foundation that unifies their extreme regimes
    and exposes new, sharply defined avenues for falsification.
