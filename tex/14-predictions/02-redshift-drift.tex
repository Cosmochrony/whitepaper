\subsection{Redshift Drift}
  \label{subsec:redshift-drift}

  The monotonic relaxation of the $\chi$ field implies a slow temporal evolution of
  cosmological redshifts when described in an effective spacetime parametrization.
  This leads to a redshift drift whose magnitude and redshift dependence differ
  quantitatively from those predicted by the standard $\Lambda$CDM model,
  particularly at intermediate redshifts.

  At the level of order-of-magnitude estimates, the effective drift rate may be
  written as
  \begin{equation}
    \dot{z}_{\mathrm{eff}}
    \;\sim\;
    H_0 (1+z) - \frac{c}{\chi(t)},
  \end{equation}
  where the second term reflects the ongoing relaxation of the $\chi$ field rather
  than a dark-energy-driven acceleration.
  This corresponds to a secular variation of order
  \[
    \Delta z \sim 10^{-10}\,\mathrm{yr}^{-1}
  \]
  at redshift $z \sim 1$, differing from $\Lambda$CDM expectations at the
  $\sim 10\%$ level in this regime.

  Future high-precision spectroscopic facilities, such as extremely large telescopes
  equipped with ultra-stable spectrographs, may be capable of probing this effect.
  A detection of a redshift drift incompatible with $\Lambda$CDM predictions would
  therefore provide a direct observational discriminator between geometric
  relaxation of the $\chi$ field and dark-energy-driven cosmic acceleration.
