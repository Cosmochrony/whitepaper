\subsection{Emergent Phenomenology and Observational Probes}
  \label{subsec:emergent_phenomenology}

  An implication of the monotonic relaxation of the $\chi$ substrate
  (Section~\ref{subsec:monotonicity-and-arrow-of-time}) together with the topological
  organization of localized configurations
  (Section~\ref{subsec:topological-stability}) is that the Cosmochrony framework leads
  to a set of qualitative and semi-quantitative phenomenological signatures that
  distinguish it from standard cosmological and quantum approaches.

  \paragraph{Cosmic Microwave Background.}
    Residual large-scale projective correlations inherited from the pre-geometric
    relaxation of the $\chi$ substrate imprint scale-dependent temperature anisotropies
    in the cosmic microwave background through their modulation of the local relaxation rate.
    Unlike inflationary scenarios, Cosmochrony does not rely on superluminal stretching:
    the relaxation of $\chi$ is locally bounded by the invariant speed $c$.
    As a consequence, correlations at the largest angular scales are naturally suppressed,
    leading to a reduction of power at low multipoles ($\ell \lesssim 10$).

    This mechanism is consistent with several large-angle features reported in CMB data,
    such as hemispherical asymmetry, without requiring fine-tuned initial conditions.
    Quantitative estimates of the resulting low-$\ell$ suppression are discussed in Appendix~\ref{app:lowell_attenuation}.

  \paragraph{Connection with CMB observations.}
    Observationally, the \emph{Planck} 2018 data report a suppression of the CMB
    quadrupole power at the level of $\sim 10\%$ relative to the $\Lambda$CDM best-fit
    expectation, corresponding to the long-standing low-$\ell$ anomaly at $\ell = 2$~\cite{Aghanim2020}.
    Within Cosmochrony, this suppression arises naturally from the pre-geometric
    relaxation dynamics of the $\chi$ substrate, which reduces large-angle correlations
    prior to the emergence of an effective spacetime description.
    Unlike phenomenological explanations relying on specific initial conditions or
    model-dependent modifications of primordial spectra, the effect follows directly
    from the intrinsic relaxation properties of the underlying field.

  \paragraph{Gravitational-wave propagation.}
    In regions of strong structural variation of the $\chi$ substrate, such as near
    compact objects, the local slowdown of $\chi$ relaxation modifies the persistence
    and coherence of gravitational-wave projective descriptions.
    Rather than inducing dissipative losses, this effect manifests as frequency-dependent
    phase shifts or dispersion-like behavior in gravitational wave signals.
    Such modifications could, in principle, affect the ringdown phase of binary black
    hole mergers and may become accessible to next-generation observatories including future space-based observatories
    such as LISA and next-generation ground-based detectors.

  \paragraph{Hubble tension.}
    The modulation of the $\chi$ relaxation rate by large-scale matter inhomogeneities
    provides a natural mechanism for reconciling early-universe and late-time measurements of the Hubble constant.
    Within this framework, the effective Hubble parameter $H(z)$ acquires a mild
    redshift dependence that departs from $\Lambda$CDM at intermediate redshifts ($0.1 \lesssim z \lesssim 10$).
    This behavior is testable through future baryon acoustic oscillation and supernova surveys.

  \paragraph{Particle phenomenology.}
    In Cosmochrony, intrinsic particle properties such as mass and spin originate from
    the topological structure of localized projectable $\chi$ configurations.
    This perspective does not predict violations of the spin--statistics connection,
    but suggests that its origin is geometric rather than axiomatic.
    While the present work does not provide a classification of all possible topological
    excitations, it opens the possibility that additional, non-standard configurations may exist.
    Identifying observable consequences of such configurations remains an open problem
    for future theoretical and experimental investigation.

  \paragraph{Quantitative deviations from \texorpdfstring{$\Lambda$}{Λ}CDM.}
    Quantitative comparisons between Cosmochrony and $\Lambda$CDM predictions are provided
    in Appendix~\ref{app:lowell_attenuation} for cosmic microwave background observables and in
    Appendix~\ref{app:hubble_tension} for the Hubble tension.
    As illustrative examples, the suppression of low-$\ell$ CMB power in Cosmochrony is of
    order $\sim 10\%$ for $\ell \lesssim 10$, exceeding the $\sim 5\%$ level expected from
    cosmic variance within $\Lambda$CDM.
    In addition, gravitational wave propagation near compact objects is predicted to exhibit
    an effective amplitude reduction or coherence attenuation of order $\Delta A / A \sim 10^{-2}$
    for trajectories passing within $r \lesssim 10\,GM/c^{2}$ of a black hole, a magnitude potentially
    accessible to future space-based interferometers such as LISA.

  \paragraph{Status of predictions.}
    The phenomenological signatures discussed above are not introduced as ad hoc
    modifications, but arise generically from the relaxation dynamics of $\chi$.
    Their role is to delineate potential observational discriminants of the framework,
    rather than to provide precision predictions at the current stage.
    Confirmation or falsification of any subset of these effects would therefore
    constitute a critical test of the Cosmochrony approach.
