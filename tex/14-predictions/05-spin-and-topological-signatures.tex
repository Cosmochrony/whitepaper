\subsection{Spin and Topological Signatures}
  \label{subsec:spin-and-topological-signatures}

  An implication of the topological origin of spin developed in
  Section~\ref{subsec:spin_topology} is that, if particle spin originates from
  topologically nontrivial configurations of the $\chi$ substrate, then spin-related
  phenomena may admit geometric signatures not captured by purely algebraic quantum
  descriptions.

  In particular, ultra-high-precision interference experiments sensitive to
  $4\pi$ rotational symmetry may, in principle, probe deviations associated with
  the internal topology of localized projectable $\chi$ configurations.
  Such deviations would not modify standard spin--statistics relations, but could
  appear as extremely small phase shifts or coherence effects under closed
  $2\pi$ versus $4\pi$ rotational cycles.

  These signatures are expected to be strongly suppressed and therefore lie beyond
  current experimental resolution.
  However, their existence would provide a conceptually distinctive test of the
  topological origin of spin proposed in Cosmochrony, as opposed to interpretations
  in which spin is treated as an abstract representation of spacetime symmetry
  groups.
