\subsection{Strong gravitational lensing}
  \label{subsec:strong-gravitational-lensing}

  For comparison with standard analyses, it is convenient to decompose the effective
  lensing potential as
  \begin{equation}
    \Phi_{\mathrm{eff}}
    =\Phi_{\mathrm{bar}}+\Phi_{\chi},
  \end{equation}
  where $\Phi_{\mathrm{bar}}$ is the contribution reconstructed from baryonic matter,
  and $\Phi_{\chi}$ encodes the emergent geometric contribution associated with
  collective relaxation constraints of the $\chi$-substrate.

  The lensing convergence is given by
  \begin{equation}
    \kappa(\boldsymbol{\theta})
    =\frac{D_lD_{ls}}{c^2D_s}
    \int \nabla_\perp^2
    \Phi_{\mathrm{eff}}\!\left(D_l\boldsymbol{\theta},z\right)\,dz ,
  \end{equation}
  leading to
  \begin{equation}
    \kappa=\kappa_{\mathrm{bar}}+\kappa_{\chi}.
  \end{equation}

  In Cosmochrony, $\kappa_{\chi}$ does not correspond to an additional mass density,
  but to a densification of admissible null geodesics induced by non-uniform
  spectral rigidity.
  Strong lensing and giant arcs therefore arise from geometric focusing rather than
  from massive dark halos.

  At leading order, the emergent contribution may be parametrized as
  \begin{equation}
    \Phi_{\chi}(\mathbf{x})
    =\frac{c^2}{2}\,
    \ln\!\left(\frac{K(\bar\chi(\mathbf{x}))}{K_\infty}\right),
  \end{equation}
  where $K(\bar\chi)$ is the effective spectral rigidity controlling relational
  distances in the projected regime.
  This form follows directly from the weak-field expansion of the reconstructed
  effective metric.

  This framework leads to several testable signatures:
  \begin{itemize}
    \item partial decorrelation between baryonic mass and strong-lensing strength
    \item enhanced sensitivity of arc formation to cluster morphology and relaxation state
    \item strong lensing without corresponding dark matter substructure
    \item redshift dependence tracing relaxation rather than mass accretion history
  \end{itemize}

  These effects typically require significant fine tuning within standard dark
  matter halo models.

  These considerations naturally extend to the emergence of massive structures and
  strong lensing at very high redshift, as revealed by recent JWST observations.

\subsubsection*{Early massive structures and strong lensing in the JWST era}
  \label{subsec:jwst-early-structures}

  Recent observations from the \emph{James Webb Space Telescope} have revealed the
  presence of unexpectedly massive galaxies, coherent morphologies, and in some
  cases strong gravitational lensing features at redshifts $z \gtrsim 8$.
  Within the standard $\Lambda$CDM framework, such early structures are challenging
  to accommodate, as hierarchical growth through gravitational collapse typically
  requires longer assembly timescales.

  In the Cosmochrony framework, this apparent tension is naturally alleviated.
  Cosmic expansion is not interpreted as a purely kinematical stretching of an
  otherwise inert spacetime, but as the effective manifestation of an irreversible
  relaxation of the relational $\chi$-substrate.
  As a consequence, large-scale structural constraints and admissible spectral modes
  may become stabilized at very early stages, prior to the formation of mature
  baryonic assemblies.

  Massive galaxies at high redshift are therefore not required to result from rapid
  late-time accretion or finely tuned merger histories.
  Instead, their effective mass reflects a localized resistance to relaxation,
  associated with spectrally robust projected configurations of the $\chi$-substrate.
  In this picture, mass is an emergent descriptor of structural persistence rather
  than a cumulative record of past gravitational growth.

  The same mechanism applies to early gravitational lensing.
  Strong lensing depends on the effective geometry experienced by null geodesics,
  which in Cosmochrony is determined by the collective spectral organization of the
  projected regime.
  Regions exhibiting early structural coherence and non-injective projection may
  therefore generate significant geometric focusing, independently of the maturity
  of their baryonic content.
  Strong lensing features at high redshift thus do not require the prior formation of
  massive dark matter halos.

  From this perspective, the early appearance of massive galaxies and strong lensing
  structures observed by JWST is not anomalous, but reflects the rapid establishment
  of spectrally admissible configurations during the initial relaxation of the
  $\chi$-substrate.
  Cosmochrony predicts that the emergence timescale of effective mass and curvature
  is governed primarily by relaxation ordering rather than by hierarchical
  gravitational assembly, providing a unified explanation for these early structures.

\subsubsection*{``Impossibly early'' galaxies}
  \label{subsec:impossibly-early-galaxies}

  Several recent JWST results have reported the existence of galaxies at
  $z \gtrsim 8$--$12$ whose inferred stellar masses, luminosities, and structural
  coherence appear difficult to reconcile with standard hierarchical formation
  timescales.
  These objects are often referred to as ``impossibly early galaxies'', as their
  properties seem to require either extremely efficient early star formation or
  substantial revisions of the underlying cosmological model.

  Within the $\Lambda$CDM paradigm, the tension arises from the assumption that
  galactic mass and structure must be built progressively through gravitational
  collapse and mergers inside dark matter halos.
  At very high redshift, the available cosmic time is limited, making the emergence
  of massive, morphologically organized galaxies statistically unlikely without
  fine tuning.

  In the Cosmochrony framework, this tension is alleviated at a conceptual level.
  Because cosmic expansion is interpreted as the effective manifestation of an
  irreversible relaxation of the relational $\chi$-substrate, structurally admissible
  configurations may stabilize coherently at very early stages.
  Effective mass and geometric persistence are therefore not required to track a
  long history of hierarchical assembly, but instead reflect the early spectral
  robustness of projected configurations.

  From this perspective, ``impossibly early'' galaxies are not anomalous objects,
  but early manifestations of stable relaxation modes.
  Their large effective masses encode a strong resistance to relaxation rather than
  the cumulative outcome of rapid accretion.
  The apparent discrepancy highlighted by JWST observations thus reflects a mismatch
  between hierarchical growth assumptions and an expansion driven by global
  relaxation ordering.

  Cosmochrony therefore predicts that the onset of massive galactic structures is
  governed primarily by relaxation dynamics rather than by the gradual buildup of
  dark matter halos, naturally accounting for the early appearance of massive and
  coherent galaxies without invoking exotic baryonic efficiencies or modified
  initial conditions.

\subsubsection*{Qualitative prediction: early stabilization of massive galaxies}
  \label{subsec:qualitative-prediction-early-galaxies}

  A qualitative prediction of the Cosmochrony framework concerns the temporal
  evolution of massive galaxies at very high redshift.
  If effective mass and structural coherence primarily reflect early spectral
  stabilization rather than prolonged hierarchical assembly, then galaxies identified
  as massive at $z \gtrsim 8$ are not expected to undergo rapid subsequent mass growth.

  In this picture, such systems should exhibit relatively stable effective masses
  over extended redshift intervals, with evolution dominated by internal reorganization
  and relaxation rather than by continued accretion or frequent major mergers.
  Their apparent maturity at early times is therefore not transient, but indicative
  of an early transition into a spectrally robust projected regime.

  This behavior contrasts with standard hierarchical scenarios, in which early massive
  galaxies are expected to represent rare, rapidly growing outliers that must continue
  to accumulate mass efficiently in order to remain compatible with later populations.
  Cosmochrony instead predicts that a significant fraction of the massive galaxies
  detected by JWST should persist as coherent structures with only moderate mass
  evolution, rather than evolving through extreme growth trajectories.

  Future JWST surveys probing the same galaxy populations across multiple redshift
  bins may therefore distinguish between rapid hierarchical growth and early
  stabilization, providing a qualitative test of relaxation-driven structure
  formation.


  We now illustrate this framework on a well-studied strong-lensing cluster.

\subsubsection*[Strong gravitational lensing in Abell 1689]{Strong gravitational lensing in Abell~1689}
  \label{subsec:a1689-strong-lensing}

  We illustrate the effective lensing formalism on the cluster Abell~1689, for which
  high-quality strong and weak lensing reconstructions as well as X-ray gas profiles
  are available in the literature.
  In the thin-lens approximation, the convergence field is defined by
  \begin{equation}
    \nabla_{\boldsymbol{\theta}}^2\psi(\boldsymbol{\theta})=2\kappa(\boldsymbol{\theta}),
  \end{equation}
  with $\psi$ the lensing potential.
  We define the effective convergence $\kappa_{\mathrm{eff}}$ from lensing reconstruction
  and a baryonic contribution $\kappa_{\mathrm{bar}}=\Sigma_{\mathrm{bar}}/\Sigma_{\mathrm{crit}}$
  from gas (X-ray) and stellar components, and isolate the emergent contribution
  \begin{equation}
    \kappa_{\chi}(\boldsymbol{\theta})
    =\kappa_{\mathrm{eff}}(\boldsymbol{\theta})
    -\kappa_{\mathrm{bar}}(\boldsymbol{\theta}).
  \end{equation}
  The corresponding emergent lensing potential is obtained by solving the 2D Poisson
  equation
  \begin{equation}
    \nabla_{\boldsymbol{\theta}}^2\psi_{\chi}(\boldsymbol{\theta})
    =2\kappa_{\chi}(\boldsymbol{\theta}),
  \end{equation}
  from which deflection, shear, magnification and critical curves follow in the usual
  way.
  In Cosmochrony, $\kappa_{\chi}$ is not interpreted as an additional dark matter
  surface density, but as an emergent geometric focusing induced by collective
  constraints of the projected $\chi$-substrate.
