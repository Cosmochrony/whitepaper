\subsection{Summary}
  \label{subsec:summary-predictions}

  Cosmochrony yields a set of observationally testable phenomenological signatures
  spanning cosmology, gravitation, galactic dynamics, and particle phenomena, all
  interpreted as emergent consequences of non-injective projection and structural
  relaxation.
  These signatures arise across widely separated physical scales but share a common
  origin in the relational dynamics of the $\chi$ substrate.

  While all predicted effects remain compatible with current observations, the framework
  generically allows for correlated departures from standard effective predictions.
  Such departures are not expected to appear as isolated anomalies, but as structurally
  related features reflecting the topology and spectral organization of admissible
  projected configurations.
  Across these domains, Cosmochrony predicts quantitative deviations from $\Lambda$CDM
  at the percent level, including a suppression of low-$\ell$ CMB power, a mild
  redshift-dependent modulation of $H(z)$, and coherence-modifying effects in
  gravitational-wave propagation near compact objects.

  Taken together, these signatures define a coherent phenomenological pattern arising
  from a single underlying mechanism—the irreversible relaxation and partial
  projectability of the $\chi$ substrate.
  Their role is not to provide precision predictions at the present stage, but to
  delineate a consistent space of observational consequences against which the
  viability of the framework can be assessed.
