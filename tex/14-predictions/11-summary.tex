\subsection{Summary}
  \label{subsec:summary-predictions}

  Cosmochrony yields a set of observationally testable phenomenological signatures
  spanning cosmology, gravitation, galactic dynamics, and particle phenomena, all
  interpreted as emergent consequences of non-injective projection and structural
  relaxation.
  These signatures arise across widely separated physical scales but share a common
  origin in the relational dynamics of the \(\chi\) substrate.

  While all predicted effects remain compatible with current observations, the framework
  generically allows for correlated departures from standard effective predictions.
  Such departures are not expected to be isolated anomalies, but structurally related
  features that may become accessible to future high-precision measurements across
  multiple observational domains.

  Taken together, these signatures provide concrete and diverse avenues for empirical
  scrutiny.
  The confirmation or falsification of any subset of them would directly constrain the
  viability of Cosmochrony as a physical framework and, more importantly, test the
  hypothesis that a single underlying relaxation mechanism can account for phenomena
  ranging from cosmological expansion to particle decay.
