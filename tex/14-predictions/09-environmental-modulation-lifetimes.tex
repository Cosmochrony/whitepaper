\subsection{Environmental Modulation of Particle Lifetimes}
  \label{subsec:environmental-modulation-lifetimes}

  Within the Cosmochrony framework, particle decay is not interpreted as a
  fundamental stochastic process, but as the manifestation of the
  \emph{structural susceptibility} of a metastable projected configuration
  to admissible fluctuations of the underlying relational substrate $\chi$.
  As discussed in Section~\ref{subsec:neutrino-decay-signatures}, these
  fluctuations form a bounded background whose spectral density depends
  weakly on the local relaxation structure of $\chi$.

  As a consequence, the decay rate $\Gamma$ of an unstable particle is not
  strictly universal, but may exhibit an extremely small environmental
  dependence reflecting variations in the local relaxation gradient of the substrate.
  At leading order, this effect may be parametrized as
  \begin{equation}
    \frac{\delta \Gamma}{\Gamma}
    \;\simeq\;
    \beta \, \frac{\Delta U}{c^{2}},
    \label{eq:lifetime-modulation}
  \end{equation}
  where $U$ denotes an effective gravitational or structural potential
  serving as a proxy for the local relaxation density, and $\beta$ is a
  dimensionless sensitivity parameter encoding the coupling between the
  projected metastable configuration and the admissible substrate fluctuations.

  Since the particle lifetime is given by $\tau = \Gamma^{-1}$, the relative
  variation of the lifetime satisfies, to first order,
  \begin{equation}
    \frac{\delta \tau}{\tau}
    \;\simeq\;
    - \frac{\delta \Gamma}{\Gamma}.
  \end{equation}

  Existing experimental constraints on local position invariance, in
  particular those derived from high-precision atomic clock comparisons,
  suggest a conservative upper bound $\beta \lesssim 10^{-6}$.
  Typical contrasts in effective potential between weakly structured
  environments and regions of high structural density (e.g.\ galactic
  cores) correspond to $\Delta U / c^{2} \sim 10^{-7}$--$10^{-6}$.
  Under these conditions, Cosmochrony predicts
  \begin{equation}
    \frac{\delta \tau}{\tau}
    \;\sim\;
    10^{-13} \text{--} 10^{-12},
    \label{eq:lifetime-order-of-magnitude}
  \end{equation}
  well below current laboratory sensitivities, but conceptually distinct
  from standard quantum or relativistic effects.

  Importantly, this modulation does not arise from spacetime curvature,
  local interactions, or environmental decoherence, but from the structural
  properties of the non-injective projection from $\chi$ to its effective
  description.
  A future detection of even a minute, reproducible deviation from strict lifetime universality would therefore
  constitute a direct signature of the underlying relational substrate.
