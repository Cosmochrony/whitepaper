\subsection{CMB Polarization Signatures (Outlook)}
  \label{subsec:cmb-polarization-signatures}

  As discussed in Section~\ref{subsec:cmb}, residual large-scale projective correlations
  inherited from the pre-geometric relaxation of the $\chi$ substrate are expected to
  imprint scale-dependent signatures on the Cosmic Microwave Background (CMB).
  In Cosmochrony, these correlations arise from the bounded and irreversible relaxation
  of $\chi$, locally constrained by the invariant speed $c$, without invoking any
  superluminal stretching or inflationary phase.
  As a consequence, correlations at the largest angular scales are naturally suppressed,
  leading to a reduction of power at low multipoles ($\ell \lesssim 10$).
  This mechanism is consistent with several large-angle features reported in CMB data,
  such as hemispherical asymmetry, without requiring fine-tuned initial conditions.
  Quantitative estimates of the resulting low-$\ell$ attenuation are discussed in
  Appendix~\ref{app:lowell_attenuation}.

  Observationally, the \emph{Planck} 2018 data report a suppression of the CMB quadrupole
  power at the level of $\sim 10\%$ relative to the $\Lambda$CDM best-fit expectation,
  corresponding to the long-standing low-$\ell$ anomaly at $\ell = 2$~\cite{Aghanim2020}.
  Within Cosmochrony, this suppression arises naturally from the pre-geometric relaxation
  dynamics of the $\chi$ substrate, which reduces large-angle correlations prior to the
  emergence of an effective spacetime description.
  Unlike phenomenological explanations relying on specific initial conditions or
  model-dependent modifications of primordial spectra, the effect follows directly from
  the intrinsic relaxation properties of the underlying field.

  \paragraph{Cosmological Imprints: The \texorpdfstring{$8/3$}{8/3} Scaling in CMB Polarization}
    \label{subsec:cmb_8_3_scaling}

    Beyond temperature anisotropies, the fundamental spectral ratio
    $\lambda_2/\lambda_1 = 8/3$, which governs the electroweak mass hierarchy at the
    micro-scale (see Appendix~\ref{sec:spectral_ratio_derivation}), is expected to leave a
    structural signature in the polarization sector of the CMB.
    In this framework, primordial scalar and tensor perturbations are reinterpreted as
    dual manifestations of the substrate's relaxation, corresponding respectively to
    base transmittance and fiber shear modes of the projection.

    \subparagraph{Geometric Bound on the Tensor-to-Scalar Ratio (\texorpdfstring{$r$}{r})}

      In Cosmochrony, the tensor-to-scalar ratio $r$ is constrained by the relative spectral
      stiffness of the $\chi$ substrate's projection modes.
      Under the principle of \textbf{Projective Spectral Saturation} at the high-energy limit
      ($k \approx 1/h_\chi$), the relaxation energy $\mathcal{E}$ is distributed according to
      the maximal kinematic capacity of each mode:
    \begin{equation}
      \mathcal{E}_s \propto \lambda_{\text{base}} \Delta_s^2,
      \qquad
      \mathcal{E}_t \propto \lambda_{\text{fiber}} \Delta_t^2 .
      \end{equation}
      The ``bare'' geometric ratio $r_0$ is defined by the saturation of these spectral densities:
    \begin{equation}
      r_0
      = \frac{\Delta_t^2}{\Delta_s^2}
      = \frac{\lambda_{\text{base}}}{\lambda_{\text{fiber}}}
      = \frac{3}{8}
      \simeq 0.375 .
      \end{equation}
      This value does not correspond to an observable tensor-to-scalar ratio at recombination,
      but defines a \emph{geometric upper bound} imposed by the topology of the projection fiber
      at the saturation scale.

    \subparagraph{Topological Decoherence and Parametrization of \texorpdfstring{$r_{\text{obs}}$}{robs}}

      The observed ratio $r_{\text{obs}}$ undergoes \textbf{topological decoherence} as the
      substrate expands.
      Since fiber shear modes are intrinsically more sensitive to losses of projective
      alignment, the cumulative degradation of alignment induces a monotonic suppression
      of tensor modes.
      To leading order, this effect may be effectively parametrized as
    \begin{equation}
      r_{\text{obs}}(t)
      = r_0 \cdot \exp\!\left( -\zeta \frac{\tau_\chi}{t} \right),
      \end{equation}
      where $\tau_\chi$ denotes the characteristic relaxation time of the substrate.
      The precise functional form is not fundamental and merely encodes the fact that fiber
      shear modes decohere faster than base transmittance during cosmic relaxation.
      This decay represents the transition from the primordial saturated state to the
      present large-scale geometric stability, providing a structural explanation for the
      low observed value of $r$ ($r < 0.036$), without invoking slow-roll dynamics or
      fine-tuned inflationary potentials.
