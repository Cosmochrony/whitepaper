\subsection{Strong Gravity and Black Holes}
  \label{subsec:strong-gravity-and-black-holes}

  In regions where excitation density becomes sufficiently high, the relaxation of $\chi$ may approach zero.
  This defines an effective horizon beyond which $\chi$ ceases to evolve relative to external observers.

  Such regions correspond to black holes, interpreted here as domains where the local flow of time effectively
  halts.
  This perspective naturally accounts for extreme time dilation and suggests that black holes act as absorbers of
  $\chi$ disturbances.

  \subsubsection{Gravitational and Temporal Shadows}

    In the Cosmochrony framework, black holes correspond to regions where the relaxation of the
    $\chi$ field becomes extremely constrained.
    As the local energy density increases, spatial gradients of $\chi$ grow and the effective relaxation rate $\partial_t \chi$ is
    progressively reduced, approaching zero in the strong-gravity limit.

    This picture naturally reproduces the notion of a \emph{gravitational shadow}.
    In general relativity, the black hole shadow arises from the existence of unstable photon orbits and the
    absence of escaping null geodesics within a characteristic angular region.
    In Cosmochrony, an equivalent effect emerges because propagating excitations of the $\chi$ field (including
    photonic modes) cannot be sustained in regions where the relaxation dynamics is effectively frozen.
    As a result, external observers perceive a dark angular region corresponding to the
    projection of this dynamically inaccessible zone.

    Beyond this optical manifestation, the framework predicts a deeper and purely geometric
    phenomenon: a \emph{temporal shadow}.
    As $\partial_t \chi \rightarrow 0$, the local unfolding of time asymptotically halts, not as a coordinate artifact but as a physical
    consequence of the field dynamics.
    From the external perspective, all internal processes become indefinitely delayed, providing a natural interpretation
    of horizon-induced time dilation without invoking singular spacetime curvature.

    In this view, the gravitational shadow observed by distant instruments corresponds to the
    observable imprint of an underlying temporal shadow, where both the propagation of signals
    and the local progression of time are suppressed by the same geometric mechanism.
    Unlike general relativity, where these effects arise from tensorial spacetime curvature, Cosmochrony
    attributes them to the scalar relaxation dynamics of the $\chi$ field.
