\subsection{Unified origin of gravitational and electromagnetic effects}
  \label{subsec:unified-origin-of-gravitational-and-electromagnetic-effects}
  Within this framework, gravitational and electromagnetic phenomena are not attributed to distinct
  fundamental interactions, but arise as complementary manifestations of the same underlying $\chi$-dynamics.
  Gravitational effects correspond to a sustained reduction of the local relaxation
  rate of $\chi$, induced by large-scale or persistent spatial gradients, leading to effective time
  dilation and attraction.
  Electromagnetic phenomena, by contrast, emerge from oscillatory or
  phase-dependent modulations of $\chi$, allowing for both attractive and repulsive interactions and
  wave-like propagation.

  In this sense, gravity and electromagnetism differ not by their origin, but by the temporal and
  structural character of the $\chi$ modulations they involve: quasi-static for gravitation, dynamic
  and oscillatory for electromagnetism.
