\subsection{Local Slowdown of $\chi$ Relaxation}
  \label{subsec:local-slowdown-of-chi-relaxation}

  In cosmochrony, gravity does not arise from a fundamental interaction but from the collective influence of
  particle excitations on the dynamics of the $\chi$ field.
  As established in the previous section, localized excitations resist the local relaxation of $\chi$.

  When many such excitations are present, their effects superpose, leading to a macroscopic reduction of the
  relaxation rate:
  \begin{equation}
    \partial_t \chi = c \left( 1 - \alpha \rho \right),
  \end{equation}
  where $\rho$ denotes the effective density of particle excitations and $\alpha$
  is a coupling parameter encoding their influence on $\chi$.

  The coupling parameter $\alpha$ in $\partial_t \chi = c(1 - \alpha \rho)$
  is determined by the interaction strength between $\chi$
  and localized excitations. For a point-like excitation of mass $m$, $\alpha$ scales as $\alpha \sim G m / c^2$
  , where $G$ emerges as an effective coupling constant linking matter density to $\chi$
  -relaxation slowing. This yields the Newtonian potential $\Phi \sim \alpha \rho$ in the weak-field limit.

  This slowdown manifests physically as gravitational time dilation.
