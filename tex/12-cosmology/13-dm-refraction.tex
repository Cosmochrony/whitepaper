\subsection{Dark Matter: Spectral Refraction and Substrate Memory}
  \label{subsec:dm-refraction}

  The dark matter phenomenology is reinterpreted as a direct consequence of the non-linear elastic response of the
  $\chi$ substrate.

  \paragraph{Variable Threshold $\mathcal{K}_c$.}
    The observed flat rotation curves arise when the relaxation flux $\Phi_\chi$ drops below the saturation threshold
    $\mathcal{K}_c$.
    Unlike the universal constant $a_0$ in MOND, $\mathcal{K}_c$ is a local property of the substrate's spectral
    density.
    This explains why the ``dark matter'' fraction appears to vary between galaxies of different spectral ages or
    environments, as the substrate's stiffness is a dynamical state, not a fixed law.

  \paragraph{Gravitational Lensing as Metric Refraction.}
    The displacement in the Bullet Cluster provides evidence for the \textbf{phase lag} of the projection $\Pi$.
    Light deflection is treated as a refraction process within the spectral gradient of the $\chi$ field.
    In high-energy collisions, the dissipative baryonic component (gas) decouples from the primary solitons (
    mass peaks).
    The lensing signal tracks the \textbf{residual geometric deformation} of the substrate, effectively measuring the
    ``wake'' left by the passing mass-solitons in the $\chi$ medium.

  \paragraph{Comparison and Predictions.}
    Cosmochrony predicts that dark matter ``halos'' should exhibit \textbf{spectral echoes}—faint gravitational
    signatures in regions where matter was previously present but has since moved, a phenomenon fundamentally
    incompatible with particulate WIMP models but inherent to a substrate with finite relaxation time.
