\subsection{The Hubble Tension}
  \label{subsec:hubble-tension}

  The discrepancy between early-universe and late-universe determinations of the Hubble constant is now well established
  observationally~\cite{Riess2019,Riess2022,Planck2020,HubbleTensionReview}.
  Standard cosmological models interpret this tension as a potential indication of ``new physics'' beyond the
  $\Lambda$CDM framework.

  Within the Cosmochrony framework, this discrepancy admits a natural qualitative interpretation without introducing new
  fundamental components or modifying the underlying relaxation dynamics.
  The key point is that different observational probes access different regimes of effective projectability of the
  relational $\chi$ substrate.

  Early-universe measurements, such as those inferred from the cosmic microwave background, probe a regime close to the
  transition from maximal constraint to geometric projectability.
  In this regime, relational constraints remain significant, and the effective mapping between $\chi$
  relaxation and spacetime observables differs from that characterizing the late universe.

  Late-time measurements, based on local distance ladders and astrophysical standard candles, probe a regime in
  which $\chi$ has undergone substantial further relaxation.
  In this more weakly constrained regime, effective spacetime descriptions are more fully developed, leading to a
  different inferred relation between relaxation ordering and observational distance–redshift relations.

  The resulting difference in inferred values of $H_0$ does not reflect a change in a fundamental expansion rate.
  It arises from the use of a single spacetime-based parametrization to describe observations sampling distinct stages
  of relational relaxation.
  In this sense, the Hubble tension reflects a limitation of homogeneous effective descriptions when applied across
  regimes of differing projectability.

  \subsubsection*{The Hubble Tension as a Diagnostic of Topological Decoherence}
    \label{subsec:hubble_tension_tau}

    Within the Cosmochrony framework, the Hubble parameter is not interpreted as a fundamental expansion rate of
    spacetime, but as an emergent kinematic proxy for the locally inferred relaxation rate within the effective
    projected description.
    More precisely, the effective Hubble parameter inferred from observations may be expressed as
    \begin{equation}
      H_{\mathrm{eff}}(\mathbf{x}, t) \;\sim\; \frac{1}{\tau_\chi(\mathbf{x}, t)} ,
    \end{equation}
    where $\tau_\chi$ denotes an effective relaxation timescale associated with the projected $\chi$ configuration,
    defined only within spacetime-based descriptions.
    This relation emphasizes that $H_{\mathrm{eff}}$
        characterizes a local relaxation rate rather than a literal velocity of
        metric expansion.

    \paragraph{Topological Frustration and Inhomogeneous Relaxation.}
      The relaxation of the $\chi$ substrate is not globally uniform.
      Regions of high structural complexity, such as galaxy clusters, filaments, and bound halos, correspond to
      \textbf{topologically frustrated} configurations of $\chi$
          , in which knots, defects, and bound spectral modes inhibit
          or delay relaxation.
          As a result, the effective relaxation timescale acquires a spatial dependence:
    \begin{equation}
      \tau_\chi(\mathbf{x}) = \tau_\chi^{(0)} \left[1 + \epsilon\, \mathcal{T}(\mathbf{x})\right] ,
      \end{equation}
      where $\mathcal{T}(\mathbf{x})$ encodes the local topological density and $\epsilon$
          parametrizes the coupling between
          topological complexity and relaxation dynamics.

    \paragraph{Global Averaging vs.\ Local Projection.}
      The Hubble tension may then be understood as a manifestation of a non-commutativity between cosmological averaging
      and local projection.
      Early-universe observables, such as those derived from the cosmic microwave background, probe a nearly
      homogeneous, pre-decoherent phase of the substrate and effectively measure a global average
      $\langle \tau_\chi^{-1} \rangle_{\mathrm{early}}$.
      In contrast, late-time local measurements based on distance ladders or standard candles sample an already
      structured
      universe, in which topological decoherence is fully developed, and therefore probe a biased local relaxation rate
      $\tau_\chi^{-1}(\mathbf{x}_{\mathrm{late}})$.
      This non-commutativity mirrors the breakdown of homogeneous parametrizations discussed in the emergence of cosmic
      acceleration.

    \paragraph{No Additional Dark Sector.}
      This interpretation resolves the Hubble tension without introducing any exotic dark energy component or modifying
      the
      gravitational field equations.
      The discrepancy arises instead as a structural consequence of the projective and relational dynamics of the $\chi$
      substrate itself.

    \paragraph{Testable Prediction.}
      Cosmochrony predicts that locally inferred values of $H_0$
          should exhibit weak but systematic correlations with the
          surrounding topological environment, such as density gradients or void/filament membership.
          From this perspective, the Hubble tension becomes a direct observational signature of the non-uniform
          relaxation of
      $\chi$, rather than a failure of cosmological modeling.

      The present discussion is intended as a qualitative explanation rather than a precision prediction.
      A more detailed analysis, outlining how different observables map onto the relaxation history of $\chi$
          , is provided
          in Appendix~\ref{app:hubble_tension}.
