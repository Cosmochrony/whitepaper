\subsection{Phenomenology of Galactic Dynamics and Lensing}
  \label{subsec:dm-pheno}

  Cosmochrony provides a non-particulate explanation for dark matter phenomena by coupling the effective gravitational
  coupling to the substrate's relaxation density $\Phi_\chi$.

  \paragraph{Effective Force Law.}
    The departure from the inverse-square law at galactic scales is described by a modified stiffness
    $K(r) = K_0 \cdot \mathcal{F}(\Phi_\chi)$.
    In low-density regimes, the relaxation flux $\Phi_\chi$ drops below a critical threshold $\mathcal{K}_c$, inducing
    a transition to a regime where the potential gradient becomes logarithmic, naturally recovering flat rotation curves
    without the need for dark halos.

  \paragraph{Gravitational Lensing as Spectral Refraction.}
    The displacement observed in the Bullet Cluster is interpreted as a \textbf{phase lag} in the substrate's response.
    While baryonic gas dissipates energy through collisions, the geometric deformations of $\chi$ (solitons) maintain
    their momentum.
    Gravitational lensing occurs due to the \textbf{refractive index gradient} of the substrate, which persists along
    the trajectory of the solitons, independent of the slowed-down gas.

  \paragraph{Predictive Distinction from WIMPs.}
    Unlike WIMP models, which predict localized particle scattering, Cosmochrony predicts a
    \textbf{non-local correlation} between the mass discrepancy and the global spectral age of the system.
    A specific signature of this framework is the absence of small-scale dark matter cusps, as the substrate's
    elasticity imposes a minimum smoothing scale (the ``spectral graininess'' $h_\chi$).
