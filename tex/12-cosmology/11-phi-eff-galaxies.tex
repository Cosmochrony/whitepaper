\subsection{Effective Potential for Galactic Dynamics from $\chi$-Relaxation Saturation}
  \label{subsec:phi-eff-galaxies}

  \paragraph{Operational status.}
    In Cosmochrony, gravitation is not introduced as a fundamental interaction.
    In regimes where a smooth geometric description is \emph{operationally admissible},
    the collective slowdown of $\chi$-relaxation induced by localized excitations
    may be summarized by an \emph{effective} potential $\Phi_{\mathrm{eff}}(r)$.
    This potential is defined only through observable kinematics,
    \begin{equation}
      g_{\mathrm{eff}}(r) \equiv -\frac{d\Phi_{\mathrm{eff}}}{dr},
      \qquad
      v^2(r)= r\,\frac{d\Phi_{\mathrm{eff}}}{dr},
    \end{equation}
    and does not represent an additional degree of freedom.

  \paragraph{Emergent acceleration scale.}
    Because the $\chi$-relaxation constraint is nonlinear, cosmological relaxation
    induces an effective background kinematic scale $a_0(t)$, expected to track
    the global relaxation rate through
    \begin{equation}
      a_0(t)\sim c\,H(t),
    \end{equation}
    so that $a_0$ is not postulated as a fundamental constant but is weakly time-dependent.

    Operationally, the galactic manifestation of this cosmological scale is expected
    to be reduced by a dimensionless projection efficiency factor $\eta=O(0.1)$,
    reflecting the partial coupling between global $\chi$-relaxation and local
    projectable galactic configurations.
    Accordingly, the effective scale entering galactic dynamics is
    $a_0(t) = \eta\,cH(t)$.

  \paragraph{Asymptotic regimes and flat rotation curves.}
    Let $g_N(r)=GM_b(r)/r^2$ denote the Newtonian baryonic acceleration inferred from the
    enclosed baryonic mass $M_b(r)$.
    In the high-acceleration regime $g_N\gg a_0$, projective relaxation remains unsaturated
    and one recovers $g_{\mathrm{eff}}\simeq g_N$, hence $\Phi_{\mathrm{eff}}(r)\simeq -GM_b/r$.
    In the low-acceleration regime $g_N\ll a_0$, saturation of the effective relaxation
    constraint yields the deep-saturation scaling
    \begin{equation}
      g_{\mathrm{eff}}(r)\simeq \sqrt{g_N(r)\,a_0(t)}.
    \end{equation}
    For an asymptotically settled baryonic mass $M_b(r)\to M_b$, this implies
    \begin{equation}
      g_{\mathrm{eff}}(r)\simeq \frac{\sqrt{G M_b\,a_0(t)}}{r}
      \quad\Rightarrow\quad
      \Phi_{\mathrm{eff}}(r)\simeq \sqrt{G M_b\,a_0(t)}\,\ln\!\left(\frac{r}{r_s}\right)+\mathrm{const.},
      \label{eq:phi-log}
    \end{equation}
    i.e. a logarithmic potential producing asymptotically flat rotation curves.

  \paragraph{Transition scale.}
    Define the transition radius $r_s$ operationally by $g_N(r_s)=a_0(t)$.
    For $M_b(r)\to M_b$, one finds
    \begin{equation}
      r_s(t)=\sqrt{\frac{G M_b}{a_0(t)}}.
    \end{equation}

  \paragraph{Minimal smooth interpolation (operational fit function).}
    A parsimonious interpolation consistent with both limits is
    \begin{equation}
      g_{\mathrm{eff}}(r)=\sqrt{g_N(r)^2 + a_0(t)\,g_N(r)}.
      \label{eq:geff-interp}
    \end{equation}
    Equation~\eqref{eq:geff-interp} is not postulated as a fundamental law but used as a
    compact operational representation of the saturation crossover in projectable regimes.
    Combined with the definition of $\Phi_{\mathrm{eff}}$ above, it yields direct and testable
    predictions for rotation curves and baryonic scaling relations.

  \paragraph{Baryonic Tully--Fisher scaling.}
    In the deep-saturation regime, Eq.~\eqref{eq:phi-log} implies
    \begin{equation}
      v_\infty^4 \simeq G\,M_b\,a_0(t),
    \end{equation}
    providing a structural origin for a baryonic Tully--Fisher-type relation, while predicting
    a mild redshift dependence through $a_0(t)\sim cH(t)$.

    For clarity, Table~\ref{tab:dm-mond-cosmochrony} summarizes how the explanation of
    flat galactic rotation curves in Cosmochrony compares conceptually with the
    standard $\Lambda$CDM framework and with MOND-like approaches.

    \begin{table}[t]
      \centering
      \small
      \begin{tabular}{p{3.1cm} p{4.2cm} p{4.2cm} p{4.2cm}}
        \hline
        \textbf{Aspect} & \textbf{$\Lambda$CDM (dark matter halo)} & \textbf{MOND (modified dynamics)} &
        \textbf{Cosmochrony (projective saturation)} \\
        \hline
        Ontology
        & Introduces a non-baryonic matter component (halo)
        & Modifies the law of motion / gravitation at low acceleration
        & No new particle; ``dark'' effects arise from relaxation properties of the $\chi$ substrate \\
        \hline
        Origin of flat rotation curves
        & Invisible mass $M_{\mathrm{DM}}(r)$ such that $v(r)\approx \mathrm{const}$
        & Deep-MOND regime $g \simeq \sqrt{g_N a_0}$
        & Saturation and nonlinearity of the relaxation constraint yielding
        $g \simeq \sqrt{g_N a_0(t)}$ \\
        \hline
        Key scale or parameter
        & Halo profile (NFW, cored, etc.) and formation parameters
        & Acceleration scale $a_0$ (often universal)
        & Emergent, slowly evolving scale $a_0(t)\sim cH(t)$; threshold linked to projective regime \\
        \hline
        Large-$r$ effective potential
        & $\Phi \sim v^2 \ln r$ via quasi-isothermal halo (or equivalent)
        & $\Phi \sim \sqrt{GM a_0}\,\ln r$
        & $\Phi_{\mathrm{eff}} \sim \sqrt{G M_b a_0(t)}\,\ln r$ as an operational summary of saturation \\
        \hline
        Baryonic Tully--Fisher relation
        & Emergent from galaxy formation models (feedback, tuning)
        & $v^4 \propto G M_b a_0$ (direct)
        & $v^4 \propto G M_b a_0(t)$ (direct), with possible mild redshift dependence via $H(t)$ \\
        \hline
        Environmental dependence
        & Strong, through halo assembly history and profile diversity
        & Must be treated via external-field effects or extensions
        & Expected through the relaxation state and projectability
        (spectral age, environment), without postulating a halo \\
        \hline
        Gravitational lensing
        & Due to total mass (baryons + dark matter)
        & Possible but depends on relativistic extensions (e.g. TeVeS)
        & Interpreted via effective propagation/refraction in projectable
        $\chi$ regimes \\
        \hline
        Discriminating signature
        & Cusps vs.\ cores, substructure abundance,
        halo--baryon correlations
        & Strict universality of $a_0$ (except in extensions)
        & Slow evolution of $a_0(t)$ and correlations with relaxation history
        (``memory'' / lag effects) \\
        \hline
      \end{tabular}
      \caption{Conceptual comparison of flat-rotation-curve explanations.
      In Cosmochrony, $\Phi_{\mathrm{eff}}$ is an operational summary of saturation in
      projectable regimes, not a fundamental gravitational field.}
      \label{tab:dm-mond-cosmochrony}
    \end{table}

\subsubsection*{Observed rotation curves: multi-galaxy test}

  To assess the phenomenological viability of the effective saturation picture,
  we confront the Cosmochrony prediction with observed rotation curves of three
  representative spiral galaxies, spanning the main empirical classes, thereby probing the flexibility of the saturation
  mechanism beyond asymptotically flat profiles:
  (i) a nearly flat rotation curve (NGC~3198),
  (ii) a rising curve (NGC~2403),
  and (iii) a mildly declining curve (NGC~5055).

  These systems are chosen because they are extensively studied, have well-constrained
  baryonic decompositions, and are commonly used as benchmarks in both
  $\Lambda$CDM and MOND analyses.
  No dark matter halo is introduced at any stage.

  The comparison follows a minimal-parameter philosophy.
  Geometric quantities such as distance, inclination, and position angle are fixed
  to their observationally inferred values.
  The baryonic contributions from gas and stars are taken directly from the literature.

  The only fitted parameter is the stellar mass-to-light ratio $\Upsilon_\star$,
  assumed constant within each galaxy.
  The acceleration scale $a_0(t_0)$ is \emph{not} fitted but fixed by the cosmological
  relation $a_0(t)\sim cH(t)$ evaluated at the present epoch.

  The resulting rotation curves are shown in Fig.~\ref{fig:rotation-curves-comparison}.
  The effective saturation model reproduces:
  (i) flat asymptotic velocities where observed,
  (ii) rising profiles in baryon-dominated outer regions,
  and (iii) mild declines without overshooting.

  This behavior arises naturally from the interplay between baryonic distribution
  and saturation of the $\chi$-relaxation constraint, without invoking
  galaxy-specific tuning or additional degrees of freedom.

  \begin{figure}[t]
    \centering
    \includegraphics[width=\linewidth]{12-cosmology/galaxy_rotcurves_3panel}
    \caption{Observed rotation curves compared with the Cosmochrony saturation
    prediction.
    Left: NGC~3198 (flat).
    Center: NGC~2403 (rising).
    Right: NGC~5055 (mildly declining).
    Points show observational data; solid lines show the Cosmochrony prediction.
    The only fitted parameter is the stellar mass-to-light ratio $\Upsilon_\star$.
    See Appendix~\ref{subsec:rotation-curve-fits} for data sources and fitting protocol.}
    \label{fig:rotation-curves-comparison}
  \end{figure}

  \paragraph{Limitations and scope of the comparison.}
    The reduced $\chi^2$ values associated with the fits should not be interpreted
    as strict goodness-of-fit estimators.
    Rotation-curve data points are affected by correlated uncertainties
    (inclination, non-circular motions, disk thickness, asymmetric drift)
    that are not fully captured by the quoted statistical errors.
    The aim of the present comparison is therefore not to achieve an optimal
    statistical fit, but to test whether the \emph{global radial trends} observed
    across different morphological classes can be reproduced within a single
    saturation-based framework, without introducing dark matter halos or
    galaxy-dependent acceleration scales.
