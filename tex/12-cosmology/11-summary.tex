\subsection{Summary}
  \label{subsec:summary-cosmology}

  Within the Cosmochrony framework, cosmological phenomena do not originate from fundamental spacetime dynamics.
  They emerge from the global relaxation ordering of the relational $\chi$ substrate, from which effective notions of
  space, time, and geometry become progressively admissible.

  Cosmic expansion, large-scale homogeneity, late-time acceleration, and the arrow of time arise naturally from this
  relaxation process, without invoking an inflationary phase, a ``dark energy'' component, or an initial spacetime
  singularity.
  The Big Bang is reinterpreted as a limiting regime of maximal constraint beyond which spacetime descriptions cease to
  be meaningful, while black holes represent localized reapproaches to the same descriptive boundary.


  At the level of effective spacetime descriptions, Cosmochrony reproduces the phenomenological successes of
  standard cosmology, including the Hubble law, the cosmic microwave background structure, and large-scale
  gravitational behavior.
  At the same time, it provides a unified and pre-geometric interpretation of these phenomena, rooted in a single
  relational relaxation process rather than in multiple independent cosmological ingredients.

  In this sense, Cosmochrony does not propose an alternative cosmological model, but a foundational framework
  clarifying the origin, scope, and domain of validity of cosmological descriptions themselves.
