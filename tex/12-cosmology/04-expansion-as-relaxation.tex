\subsection{Cosmic Expansion as $\chi$ Relaxation}
  \label{subsec:expansion-as-relaxation}

  In the Cosmochrony framework, cosmic expansion does not correspond to the motion of matter through a pre-existing
  space~\cite{Friedmann1922}.
  It reflects instead the progressive relaxation of the relational $\chi$
  substrate, from which effective spatial distinctions and separations gradually emerge.

  As the ordering parameter associated with $\chi$
  increases monotonically, projected descriptions admit an ever larger range of mutually distinguishable regions.
  What is described in effective cosmological models as the ``expansion of space'' thus corresponds to the
  increasing projectability of relational differences within $\chi$, rather than to a dynamical stretching of a
  fundamental metric background.

  In this interpretation, expansion is not driven by an external energy component or a specific cosmological fluid.
  It is an intrinsic consequence of the relaxation ordering of the substrate itself.
  Localized matter configurations act as persistent structural constraints on this relaxation, leading to spatially
  inhomogeneous unfolding that later manifests, in effective geometric descriptions, as large-scale structure.


  Cosmic expansion is therefore reinterpreted as a geometric and relational phenomenon, emerging from the intrinsic
  evolution of $\chi$ and acquiring a spacetime interpretation only once a stable geometric regime becomes applicable.
