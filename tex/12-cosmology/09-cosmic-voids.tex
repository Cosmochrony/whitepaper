\subsection{Cosmic Voids as Maximal Relaxation Probes}
  \label{subsec:cosmic-voids}

  Within the Cosmochrony framework, cosmic voids constitute regions where the
  relaxation of the $\chi$ substrate is least frustrated by localized excitations.
  As a result, within the admissible effective description defined in
  Sec.~\ref{subsec:variational-formulation}, regions of near-maximal substrate relaxation
  exhibit enhanced geodesic defocusing, leading to a negative gravitational lensing
  signal and to non-linear peculiar velocity outflows at void boundaries.

  These effects are absent or strongly suppressed in $\Lambda$CDM, making cosmic
  voids a clean discriminant between the two frameworks.

  Cosmic voids act as natural laboratories for maximal substrate relaxation, providing a falsifiable signature of
  Cosmochrony through negative lensing and enhanced boundary outflows.

  \paragraph{Connection to local $H_0$ determinations.}
    If cosmic voids correspond to regions of near-maximal substrate relaxation,
    Cosmochrony predicts enhanced outward peculiar velocities at void boundaries
    together with a more negative void-lensing signal than in $\Lambda$CDM.
    Both effects can bias low-redshift distance--redshift inferences toward higher
    locally inferred expansion rates.
    A decisive test is therefore the cross-correlation between (i) void lensing
    profiles and (ii) locally inferred $H_0$ maps: regions with higher $H_0$
    should statistically coincide with stronger void defocusing signatures.

  \paragraph{Phenomenological void parametrization within the saturation-constrained regime.}
    We model the observable void signals as a $\Lambda$CDM baseline plus a phenomenological
    saturating correction, controlled by a single dimensionless amplitude parameter
    $\beta_{\textrm{void}}$, designed to capture the effects expected within the
    saturation-constrained effective regime defined in
    Sec.~\ref{subsec:variational-formulation}:
    \begin{align}
      \kappa_{\textrm obs}(R) &= \kappa_{\Lambda{\textrm CDM}}(R)
      \left[1+\beta_{\textrm void}\, \mathcal{S}\!\big(\mathcal{A}(R)\big)\right],\\
      v_{\textrm obs}(r) &= v_{\Lambda{\textrm CDM}}(r)
      \left[1+\beta_{\textrm void}\, \mathcal{S}\!\big(\mathcal{B}(r)\big)\right],
    \end{align}
    with a saturating function
    $
    \mathcal{S}(x)=x/\sqrt{1+x^2}.
    $
    The dimensionless activities $\mathcal{A},\mathcal{B}$ can be chosen as
    projected-density and local-gradient proxies, e.g.
    $
    \mathcal{A}(R)=|\Delta(R)|/s_\star
    $
    and
    $
    \mathcal{B}(r)=|d\Phi/dr|/g_\star
    $
    (or $|\delta(r)|/s_\star$), where $s_\star,g_\star$ set the Born--Infeld
    saturation thresholds.

    The specific functional form of $\mathcal{S}(x)$ is chosen for convenience as a smooth
    interpolant between linear and saturated regimes, and does not introduce additional
    dynamical assumptions beyond those already encoded in the effective variational
    formulation of Sec.~\ref{subsec:variational-formulation}.
