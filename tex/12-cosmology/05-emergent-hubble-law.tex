\subsection{Emergent Hubble Law}
  \label{subsec:emergent-hubble-law}

  In homogeneous regimes, the relaxation ordering of the relational $\chi$ substrate is uniform.
  When described using an effective cosmological time parameter $t$, introduced as a convenient label of the relaxation
  ordering, this uniform regime admits the linear representation:
  \begin{equation}
    \chi(t) = \chi_0 + c\, t .
  \end{equation}
  This expression does not define a fundamental time evolution, but provides an effective parametrization of cumulative
  relaxation in a homogeneous cosmological regime.

  Identifying effective spatial scales with accumulated relational differentiation in $\chi$
  leads naturally to a ``Hubble-like law'' relating relative separation rates to separation
  itself~\cite{Hubble1929,Hogg1999}.
  Within this effective description, the Hubble parameter may be written as:
  \begin{equation}
    H(t) \equiv \frac{1}{\chi}\,\frac{d\chi}{dt},
  \end{equation}
  where the derivative is understood as an effective rate with respect to the cosmological time parameter, not as a
  fundamental dynamical derivative.

  In this perspective, no independent scale factor or expansion field is required.
  The Hubble parameter emerges as a dimensionless measure of the global relaxation rate of $\chi$ relative to its
  accumulated value.

  The present-day Hubble constant $H_0$ is therefore interpreted as an effective observable quantifying the current
  state of global relaxation, rather than as a fundamental constant governing the dynamics of spacetime itself.
