\subsection{Emergent Hubble Law}
  \label{subsec:emergent-hubble-law}

  In homogeneous regimes, the relaxation ordering of the relational $\chi$ substrate is uniform.
  When described using an effective cosmological time parameter $t$, introduced solely as a convenient label of the
  relaxation ordering, this uniform regime admits the linear representation:
  \begin{equation}
    \chi(t) = \chi_0 + c\, t .
  \end{equation}
  This expression does not define a fundamental time evolution, but provides an effective parametrization of cumulative
  relaxation in a homogeneous cosmological regime.

  Identifying effective spatial scales with accumulated relational differentiation in $\chi$
  leads naturally to a ``Hubble-like law'' relating relative separation rates to separation
  itself~\cite{Hubble1929,Hogg1999}.
  Within this effective description, the Hubble parameter may be written as:
  \begin{equation}
    H(t) \equiv \frac{1}{\chi}\,\frac{d\chi}{dt},
  \end{equation}
  where the derivative denotes an effective rate with respect to the cosmological parametrization,
  not a fundamental dynamical derivative.

  In this perspective, no independent scale factor or expansion field is required.
  The Hubble parameter emerges as a dimensionless measure of the global relaxation rate of $\chi$ relative to its
  accumulated value.

  The present-day Hubble constant $H_0$ is therefore interpreted as an effective observable quantifying the current
  state of global relaxation, rather than as a fundamental constant governing the dynamics of spacetime itself.
  This interpretation relies implicitly on the validity of a homogeneous effective description, whose limitations
  become relevant once relaxation proceeds unevenly across scales.

  \textbf{Observational discriminant}: Cosmochrony predicts a $\sim 5\%$ higher $H(z)$ at $z\sim 1$ compared to
  $\Lambda$CDM, testable with DESI/Euclid (Section~\ref{subsec:emergent_phenomenology}).

  \paragraph{Cosmic Acceleration Without Dark Energy}
    \label{subsec:acceleration-without-dark-energy}

    Within the Cosmochrony framework, the observed late-time cosmic acceleration does not require the introduction of a
    cosmological constant or a ``dark energy'' component.
    No additional energy density or repulsive interaction is postulated at the fundamental level.

    The apparent acceleration arises as an effective consequence of the cumulative relaxation history of the relational
    $\chi$ substrate.
    As cosmic evolution proceeds, the formation of localized and long-lived structures (such as galaxies and clusters)
    increasingly constrains the local relaxation of $\chi$.
    These constraints introduce growing spatial inhomogeneities in the relaxation process.

    When interpreted within standard spacetime-based cosmological models, which assume a homogeneous and isotropic
    expansion driven by a global scale factor, these inhomogeneities manifest as an apparent acceleration of cosmic
    expansion.
    The effect reflects a mismatch between the underlying relational relaxation dynamics and the assumptions built into
    effective geometric descriptions.

    In this sense, cosmic acceleration is not a dynamical phenomenon requiring a new source of energy.
    It is an emergent, interpretative effect arising from the progressively uneven relaxation of $\chi$ across cosmic
    scales.
    As structure formation proceeds, the effective expansion inferred from observations naturally departs from the
    predictions of homogeneous models, without invoking dark energy.

    This interpretation aligns with approaches that attribute late-time acceleration to ``backreaction'' effects,
    while providing a unified and pre-geometric origin rooted in the relaxation dynamics of the $\chi$ substrate.
