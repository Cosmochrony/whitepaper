\subsection{Dark Matter as Residual Relaxation Effects}
  \label{subsec:dark-matter-phenomenology}

  Cosmochrony addresses dark matter phenomenology not through the introduction of
  hypothetical particles (such as WIMPs or axions), but as a structural consequence
  of the relaxation dynamics of the relational $\chi$ substrate.

  Although the $\chi$ substrate is not a material medium, its projected relaxation
  dynamics can admit effective hydrodynamic descriptions in regimes where large-scale
  relational fluxes dominate.
  These effective descriptions give rise to flow-, lag-, and memory-like phenomena
  without implying the existence of a physical fluid or mechanical elasticity.

  \paragraph{Spectral Origin of Dark Effects.}
    At the fundamental level, dark matter phenomena correspond to configurations of
    the $\chi$ substrate that resist relaxation while failing the projectability
    conditions required for Standard Model interactions.
    These \textbf{non-projected spectral modes} possess inertial mass (resistance to
    relaxation) and contribute to gravitational curvature, while remaining invisible
    to electromagnetic or electroweak probes.

    \subparagraph{Spectral vs.\ Particulate Mass.}
      While baryonic particles correspond to resonant, topologically stable projected
      modes, dark matter consists of sub-threshold harmonics.
      These modes contribute to the global \textit{fiber weight} of the projected
      geometry but lack the spectral signatures required for projection $\Pi$ into
      particle degrees of freedom.

  \paragraph{Galactic Rotation and Effective Spectral Stiffness.}
    The flattening of galactic rotation curves is interpreted as a spatial variation
    of the effective gravitational coupling $G_{\mathrm{eff}}$ induced by the local
    relaxation state of the substrate.
    Near galactic centers, the high density of matter localizes the relaxation flow.
    At larger radii, the effective spectral stiffness $K_0$ of the $\chi$ relaxation
    response undergoes a transition, leading to a logarithmic gravitational potential.
    This reproduces MOND-like behavior as an asymptotic description, without invoking
    a modification of gravity or a universal acceleration scale.

    This stiffness is not mechanical in nature.
    It encodes the density of unresolved relaxation constraints in the projected
    spectral response and reflects the local spectral age and environment of the
    system.

    \subparagraph{Variable Threshold $\mathcal{K}_c$.}
      The transition to the MOND-like regime occurs when the relaxation flux $\Phi_\chi$
      drops below a saturation threshold $\mathcal{K}_c$.
      Unlike the universal constant $a_0$ in MOND, $\mathcal{K}_c$ is a local and
      environment-dependent property of the substrate's spectral density.
      This naturally explains the observed variation of the apparent dark matter
      fraction among galaxies of different masses, morphologies, and environments.

  \paragraph{Gravitational Lensing and Substrate Memory.}
    Gravitational lensing phenomena, particularly in systems such as the Bullet
    Cluster, are interpreted as manifestations of \textbf{relaxation lag} and
    substrate memory.
    In high-velocity collisions, the dissipative baryonic component (gas) rapidly
    loses coherence, while the projective geometry $\Pi$ associated with localized
    mass-solitons persists.

    \subparagraph{Lensing as Spectral Refraction.}
      Light deflection is treated as an effective refraction process within the spectral
      gradient of the projected $\chi$ geometry.
      The lensing signal therefore tracks the residual geometric deformation induced by
      the passage of mass-solitons, rather than the instantaneous distribution of
      baryonic matter.
      This description is purely effective and does not imply the existence of a
      material refractive medium.

  \paragraph{Predictive Distinction from Particulate Dark Matter.}
    Unlike WIMP-based models, which predict localized particle halos and small-scale
    cusps, Cosmochrony predicts a \textbf{non-local correlation} between gravitational
    mass discrepancies and the global spectral age of a system.
    A characteristic signature of this framework is the absence of sharp central
    cusps and the presence of a minimum smoothing scale imposed by the substrate's
    spectral response (the ``spectral graininess'' $h_\chi$).

    \subparagraph{Spectral Echoes.}
      Cosmochrony further predicts the existence of \textbf{spectral echoes}—faint
      gravitational signatures in regions where matter was previously present but has
      since moved.
      Such echoes arise from the finite relaxation time of the substrate and are
      incompatible with particulate dark matter models, while being a natural
      consequence of a relational substrate with memory.
