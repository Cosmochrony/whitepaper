\subsection{Entropy and the Arrow of Time}
  \label{subsec:entropy-arrow}

  Within the Cosmochrony framework, the arrow of time is not a derived or emergent statistical phenomenon.
  It is a fundamental structural feature arising from the intrinsic monotonic relaxation ordering of the relational
  $\chi$ substrate.
  Temporal directionality therefore precedes and grounds all thermodynamic considerations.

  Entropy increase emerges only at the level of effective spacetime descriptions.
  It provides a statistical summary of how macroscopic degrees of freedom evolve under the irreversible relaxation of
  $\chi$ when coarse-grained descriptions become applicable.
  In this sense, entropy growth does not explain the arrow of time; rather, it reflects the underlying temporal
  asymmetry already present in the substrate.

  This reverses the standard explanatory hierarchy of statistical physics.
  Time asymmetry is not attributed to special initial conditions or probabilistic arguments, but is imposed
  intrinsically by the relaxation structure of $\chi$.
  Thermodynamic irreversibility is thus a secondary manifestation of a more fundamental ordering principle.

  It is crucial to note that entropy and the second law are defined only within regimes where a spacetime-based,
  coarse-grained description of physical processes is valid.
  Processes involving deprojection of relational information into the $\chi$ substrate—such as those associated with
  extreme gravitational confinement—do not correspond to entropy decrease or temporal reversal.
  Instead, they represent a transition to a level of description where thermodynamic notions no longer apply.

  From this perspective, entropy increase characterizes the evolution of descriptions within spacetime, while the
  arrow of time itself is rooted in the deeper relational dynamics of $\chi$.
