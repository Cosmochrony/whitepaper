\subsection{Cosmic Expansion Without Inflation}
  \label{subsec:expansion-without-inflation}

  In standard cosmology, an inflationary phase is introduced to account for the observed large-scale homogeneity,
  isotropy, and near-flatness of the universe, as well as to resolve the horizon problem~\cite{Guth1981,Linde1982}.
  Within the Cosmochrony framework, these features do not require a distinct inflationary epoch.

  At the pre-geometric level, the relational structure of the $\chi$ substrate is not organized according to spatial
  separation or causal horizons.
  Prior to the emergence of a stable geometric projection, notions such as distance, light cones, and causal
  disconnection are undefined.
  As a result, the conditions that give rise to the ``horizon problem'' in standard spacetime-based cosmology do not
  apply.

  Large-scale homogeneity and isotropy therefore reflect the global relational coherence of the $\chi$
  substrate in the maximally constrained regime, rather than the outcome of a rapid expansion of spacetime.
  When geometric descriptions become admissible, this coherence is inherited as initial large-scale regularity in the
  emergent spacetime.

  Cosmic expansion itself is interpreted as the progressive relaxation of relational constraints, leading to
  increasing effective separation between projected regions.
  This expansion does not correspond to motion through space, but to the gradual unfolding of geometric distinctions as
  ``projectability'' improves.

  The present framework does not aim to reproduce inflationary scenarios at the level of
  field-driven perturbation spectra.
  Instead, it redefines the origin and interpretation of primordial regularities and
  correlations as structural features of the relaxation and projection process itself.
  While a fully quantitative treatment of emergent anisotropies requires further
  spectral and numerical development, no distinct inflationary phase or additional
  dynamical degrees of freedom are invoked.
