\subsection{Cosmic Microwave Background}
  \label{subsec:cmb}

  Within the Cosmochrony framework, the cosmic microwave background (CMB) does not encode primordial fluctuations
  generated during a distinct inflationary phase.
  It reflects instead the imprint of early relaxation and reprojection processes as the relational $\chi$ substrate
  transitioned toward a regime admitting stable geometric descriptions.

  At this stage, the universe was not yet structured by well-defined spatial separation or causal horizons.
  Large-scale correlations observed in the CMB therefore arise naturally from the global relational coherence of $\chi$
  prior to geometric differentiation, rather than from superluminal expansion within spacetime.

  The acoustic features observed in the temperature power spectrum admit an effective interpretation as resonance
  patterns arising during the transition to a stable geometric regime.
  They reflect the coupling between emerging matter configurations and the relaxation dynamics governing admissible
  projected descriptions, rather than oscillations of a fundamental physical field~\cite{SachsWolfe1967,HuWhite1997}.

  In this perspective, the CMB encodes a ``fossil record'' of the emergence of spacetime itself.
  Its large-angle correlations and statistical properties are consequences of the pre-geometric relational structure of
  $\chi$, inherited by the emergent spacetime description without requiring an inflationary epoch or superluminal causal
  processes~\cite{Planck2020}.

  This suppression may explain the observed $\ell=2$ anomaly in Planck data, without invoking inflationary mechanisms.

  Because the emergence of geometric descriptions and the stabilization of projection
  modes are governed by the same underlying spectral structure of the $\chi$ substrate,
  fundamental ratios characterizing stable configurations at the micro-scale may leave
  scale-independent imprints on cosmological observables.

  \paragraph{Cosmological Imprints: The $8/3$ Scaling in CMB Polarization}
  \label{subsec:cmb_8_3_scaling}

  The fundamental spectral ratio $\lambda_2/\lambda_1 = 8/3$, which governs the electroweak
  mass hierarchy at the micro-scale (see Appendix~\ref{sec:spectral_ratio_derivation}),
  is expected to leave a structural signature on the Cosmic Microwave Background (CMB).
  In this framework, primordial scalar and tensor perturbations are reinterpreted as
  dual manifestations of the substrate's relaxation.

  \subparagraph{Geometric Bound on the Tensor-to-Scalar Ratio ($r$)}

    In Cosmochrony, the tensor-to-scalar ratio $r$ is constrained by the relative spectral
    stiffness of the $\chi$ substrate's projection modes.
    Under the principle of \textbf{Projective Spectral Saturation} at the high-energy limit
    ($k \approx 1/h_\chi$), the relaxation energy $\mathcal{E}$ is distributed according to
    the maximal kinematic capacity of each mode:
    \begin{equation}
      \mathcal{E}_s \propto \lambda_{\text{base}} \Delta_s^2,
      \qquad
      \mathcal{E}_t \propto \lambda_{\text{fiber}} \Delta_t^2 .
    \end{equation}
    The ``bare'' geometric ratio $r_0$ is defined by the saturation of these spectral densities:
    \begin{equation}
      r_0
      = \frac{\Delta_t^2}{\Delta_s^2}
      = \frac{\lambda_{\text{base}}}{\lambda_{\text{fiber}}}
      = \frac{3}{8}
      \simeq 0.375 .
    \end{equation}
    This value does not correspond to an observable tensor-to-scalar ratio at recombination,
    but defines a \emph{geometric upper bound} imposed by the topology of the projection fiber
    at the saturation scale.

  \subparagraph{Topological Decoherence and Parametrization of $r_{\text{obs}}$}

    The observed ratio $r_{\text{obs}}$ undergoes \textbf{topological decoherence} as the
    substrate expands.
    Since fiber shear modes are intrinsically more sensitive to losses of projective alignment,
    the cumulative degradation of alignment induces a monotonic suppression of tensor modes.
    To leading order, this effect may be effectively parametrized as
    \begin{equation}
      r_{\text{obs}}(t)
      = r_0 \cdot \exp\!\left( -\zeta \frac{\tau_\chi}{t} \right),
    \end{equation}
    where $\tau_\chi$ denotes the characteristic relaxation time of the substrate.
    The precise functional form is not fundamental and merely encodes the fact that fiber
    shear modes decohere faster than base transmittance during cosmic relaxation.
    This decay represents the transition from the primordial saturated state to the present
    large-scale geometric stability, providing a structural explanation for the low observed
    value of $r$ ($r < 0.036$), without invoking slow-roll dynamics or fine-tuned inflationary
    potentials.
