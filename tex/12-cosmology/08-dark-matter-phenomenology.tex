\subsection{Dark Matter as Residual Relaxation Effects}
  \label{subsec:dark-matter-phenomenology}

  Cosmochrony addresses dark matter phenomenology not through the addition of hypothetical particles (WIMPs or Axions),
  but as a structural consequence of the substrate's relaxation dynamics.

  \paragraph{Galactic Rotation and Effective Stiffness.}
    The flattening of galactic rotation curves is interpreted as a spatial variation of the effective gravitational
    constant $G_{\mathrm{eff}}$.
    Near galactic centers, the high density of matter localizes the relaxation flow.
    At large radii, the stiffness $K_0$ of the $\chi$ field undergoes a transition, leading to a logarithmic potential
    similar to MOND, but derived from the substrate's elasticity.

  \paragraph{The Bullet Cluster as Relaxation Lag.}
    The observed displacement between baryonic mass and gravitational lensing in systems like the Bullet Cluster is
    reinterpreted as a \textbf{relaxation lag}.
    In high-velocity collisions, the projective geometry $\Pi$ associated with localized solitons (``dark peaks'')
    persists longer than the dissipative gas configurations, manifesting as a ``geometric memory'' in the $\chi$ field.

  \paragraph{Comparison with WIMPs and Axions.}
    While WIMPs require new fundamental fields, Cosmochrony suggests that ``dark'' effects arise from
    \textbf{non-projected spectral modes}—configurations of $\chi$ that possess inertial mass (resistance to relaxation)
    but lack the specific symmetry required for electromagnetic transmittance.

\paragraph{Dark Matter and Energy: Relicts of the Relaxation Flux}
\label{subsec:dark-sector}

A unified framework must address the ``dark sector'' without invoking ad-hoc particles or fields.
In Cosmochrony, these phenomena emerge naturally from the \textbf{spectral density of the relaxation process}.

\subparagraph{}*{Dark Matter as Sub-Threshold Spectral Inertia}

  Dark Matter is reinterpreted here as configurations of the $\chi$ substrate that possess \textbf{spectral mass}
  but fail the \textbf{projectability criteria} for electroweak or electromagnetic interaction.

  \begin{itemize}

    \item \textbf{Spectral vs. Particulate Mass:}
    While baryonic particles are ``resonant notes'' (topologically stable and projected), Dark Matter consists of
    ``sub-threshold harmonics''—modes that contribute to the global
    \textit{fiber weight} (and thus to gravitational curvature)
    but lack the spectral signature required for projection $\Pi$ into the Standard Model.

    \item \textbf{Spectral Rigidity vs. Mechanical Stiffness:}
    In high-density regions, the substrate exhibits an effective \textbf{spectral rigidity}.
    It is crucial to note that this rigidity does not correspond to a mechanical stiffness, but to a concentration
    of unresolved relaxation constraints in the spectral domain.
    This notion of spectral rigidity is not merely conceptual: Appendix~\ref{sec:spectral_ratio_derivation} shows
    that the same invariant ratio is recovered both through stochastic relational sampling
    and through the spectral response of a discrete Laplacian defined on the same
    graph.

    \item \textbf{Asymptotic MOND-like Emergence:}
    While the resulting dynamics may resemble Modified Newtonian Dynamics (MOND) in the low-acceleration regime, the
    origin is fundamentally different.
    MOND-like relations appear only as \textbf{asymptotic descriptions}
    of the projected dynamics, valid in regimes where the spectral constraint density varies slowly.

  \end{itemize}

\subparagraph{}*{Dark Energy as the Global Relaxation Flux}

  Similarly, what is interpreted as \textit{Dark Energy} is not a vacuum energy density ($\Lambda$), but the
  \textbf{global potential of the $\chi$ relaxation flux} $\Phi_\chi$.

  \begin{itemize}

    \item \textbf{Irreversible Approach to Equilibrium:}
    The observed acceleration of galaxies is a direct consequence of the diminishing tempo of relaxation as the
    substrate \textbf{irreversibly approaches a global relaxation equilibrium}.
    This ``cooling'' of the relaxation rhythm induces an apparent stretching of the emergent metric.

    \item \textbf{Ontological Arrow:}
    This expansion is not a dynamical ``push'' within spacetime, but a manifestation of the underlying chrono-genesis:
    the irreversible transition from the substrate's complexity to the projected state's simplicity.

  \end{itemize}
