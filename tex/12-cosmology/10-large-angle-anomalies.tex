\subsection{Large-Angle Temperature Anomalies}
  \label{subsec:large-angle-anomalies}

  Large-angle anomalies observed in the cosmic microwave background, such as the suppression of power at low multipoles
  and the presence of unexpected large-scale alignments, remain only partially explained within the standard $\Lambda$
  CDM framework~\cite{Planck2018}.

  \begin{figure}[htbp]
    \centering
    \includegraphics[width=0.85\textwidth]{12-cosmology/cmb_lowell_lcdm_cosmo}
    \caption{Low-$\ell$ CMB TT power spectrum comparison.
    The Cosmochrony ansatz (green line) shows a natural suppression of power at large angular scales ($\ell < 10$),
      providing a closer fit to the Planck 2018 data points compared to the standard $\Lambda$CDM best-fit
      (dashed orange line).}
    \label{fig:cmb-low-l-anomalies}
  \end{figure}

  Within the Cosmochrony framework, these features are naturally interpreted as residual relational correlations
  inherited from the pre-geometric regime of the $\chi$ substrate.
  Before the emergence of a stable spacetime description, relational structure was not organized
  according to spatial separation or causal horizons.
  As a result, long-range correlations could persist without requiring superluminal processes or inflationary
  amplification.

  When geometric projection becomes admissible, most of these correlations are washed out by subsequent relaxation
  and structure formation.
  However, weak remnants may survive at the largest angular scales, where ``cosmic variance'' is dominant and effective
  geometric descriptions are least constraining.

  In this perspective, large-angle CMB anomalies do not signal a breakdown of cosmological consistency.
  They reflect the partial imprint of a pre-geometric relational phase on the earliest projectable spacetime
  bservables, and are therefore expected to appear primarily at the largest scales.
