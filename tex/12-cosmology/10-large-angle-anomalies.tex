\subsection{Large-Angle Temperature Anomalies}
  \label{subsec:large-angle-anomalies}

  Large-angle anomalies observed in the cosmic microwave background, such as the suppression of power at low multipoles
  and the presence of unexpected large-scale alignments, remain only partially explained within the standard $\Lambda$
  CDM framework~\cite{Planck2018}.

  \begin{figure}[htbp]
    \centering
    \includegraphics[width=0.85\textwidth]{12-cosmology/cmb_lowell_lcdm_cosmo}
    \caption{Low-$\ell$ CMB TT power spectrum comparison.
    The Cosmochrony ansatz (green line) shows a natural suppression of power at large angular scales ($\ell < 10$),
      providing a closer fit to the Planck 2018 data points compared to the standard $\Lambda$CDM best-fit
      (dashed orange line).}
    \label{fig:cmb-low-l-anomalies}
  \end{figure}

  Within the Cosmochrony framework, these features are naturally interpreted as residual relational correlations
  inherited from the pre-geometric regime of the $\chi$ substrate.
  Before the emergence of a stable spacetime description, relational structure was not organized
  according to spatial separation or causal horizons.
  As a result, long-range relational correlations could persist through the projection transition without requiring
  superluminal processes or inflationary amplification.

  When geometric projection becomes admissible, most of these correlations are washed out by subsequent relaxation
  and structure formation.
  However, weak remnants may survive at the largest angular scales, where ``cosmic variance'' is dominant and effective
  geometric descriptions are least constraining.
  Such residual correlations are expected to preferentially reflect the same underlying
  spectral constraints discussed in the context of CMB polarization.

  In this perspective, large-angle CMB anomalies do not signal a breakdown of cosmological consistency.
  They reflect the partial imprint of a pre-geometric relational phase on the earliest projectable spacetime
  observables, and are therefore expected to appear primarily at the largest scales.

  \subsubsection*{Structural Admissibility and Low-$\ell$ Suppression}
  \label{subsec:low-l-admissibility}

  In the Cosmochrony framework, the primordial power spectrum is not assumed to be a
  pure scale-invariant law from the outset.
  Instead, it reflects the set of configurations that are \emph{projectively
admissible} given the relaxation state of the pre-geometric substrate $\chi$.

  At early cosmic times, the relaxation of $\chi$ is strongly constrained by global
  saturation effects, severely limiting the admissibility of complex or highly
  oscillatory configurations.
  As a consequence, the effective primordial spectrum is expected to be modulated by
  a structural constraint reflecting this restricted configuration space.

  We formalize this effect by introducing an \emph{Admissibility Filter}
  $\mathcal{A}(k,t)$ acting on the primordial spectrum:
  \begin{equation}
    P_{\mathrm{obs}}(k,t) = \mathcal{A}^2(k,t)\, P_0(k),
  \end{equation}
  where $P_0(k)$ denotes the unconstrained primordial spectrum.

  A natural phenomenological form for this filter, capturing a structural infrared
  cutoff, is
  \begin{equation}
    \mathcal{A}(k,t) = \exp\!\left[-\left(\frac{k_c(t)}{k}\right)^p\right],
  \end{equation}
  where $k_c(t)$ defines a \emph{coherence scale} associated with the maximal size of
  projectively admissible configurations at time $t$, and $p$ controls the sharpness
  of the transition.

  For modes with $k \ll k_c(t)$, corresponding to the largest angular scales
  (low-$\ell$), power suppression arises not from inflationary dilution or statistical
  variance, but from the structural impossibility of supporting such global modes
  within the available relational complexity of the $\chi$ substrate.

  \subsubsection*{Testable Predictions}

    To distinguish structural admissibility from statistical explanations, the
    Cosmochrony framework leads to several falsifiable predictions:

    \paragraph{A. Correlated Suppression Across TT, TE, and EE Spectra.}
      If low-$\ell$ suppression originates from a projective admissibility constraint,
      it must affect all cosmological observables probing the same primordial modes.
      In particular, the coherence scale $k_c$ (or equivalently $\ell_c \sim 10$--$20$)
      should be consistent across the temperature (TT) and polarization (EE, TE) spectra.
      Future high-precision polarization measurements (e.g.\ \textit{LiteBIRD}) provide
      a decisive test of this prediction.

    \paragraph{B. Absence of Primordial Non-Gaussianities.}
      Because the early Universe in Cosmochrony is characterized by ontological
      simplicity and a severely restricted configuration space, no primordial mechanism
      exists to generate significant non-Gaussian correlations.
      The framework therefore predicts an exceptionally small primordial non-Gaussianity
      parameter $f_{\mathrm{NL}}$.
      A statistically significant detection of primordial non-Gaussianities would falsify
      this minimal relaxation scenario, while increasingly stringent upper bounds would
      support it.

    \paragraph{C. Scale-Dependent Spectral Tilt Near the Cutoff.}
      The admissibility filter implies that the effective spectral index $n_s$ need not be
      strictly constant across all scales.
      A mild running of the spectral index, localized near the transition scale $k_c$, is
      expected as the Universe transitions from a globally constrained regime to one
      allowing richer configuration space.
      Precise measurements of the running parameter $\alpha_s$ correlated with the
      low-$\ell$ region would provide further support for this interpretation.

  \subsubsection*{Ontological Interpretation: The Arrow of Complexity}

    Within this perspective, the suppression of low-$\ell$ modes is not an anomaly but a
    fossil signature of the early Universe's ontological state.
    The second law of thermodynamics is reinterpreted accordingly: entropy growth does
    not correspond to increasing disorder, but to the irreversible enlargement of the
    space of admissible configurations.

    The Universe begins in a state of \emph{ontological poverty}, where only a small
    class of simple, highly coherent configurations can be projected.
    As relaxation proceeds, this admissible space expands, enabling the emergence of
    hierarchical, localized, and complex structures.
    In this sense, cosmological time is simultaneously the arrow of entropy and the
    arrow of increasing descriptive richness.
