\subsection{Dark Matter as Residual Relaxation Effects}
  \label{subsec:dark-matter-phenomenology}

  Cosmochrony addresses dark matter phenomenology not through the addition of hypothetical particles (WIMPs or Axions),
  but as a structural consequence of the substrate's relaxation dynamics.

  \paragraph{Galactic Rotation and Effective Stiffness.}
    The flattening of galactic rotation curves is interpreted as a spatial variation of the effective gravitational
    constant $G_{\mathrm{eff}}$.
    Near galactic centers, the high density of matter localizes the relaxation flow.
    At large radii, the stiffness $K_0$ of the $\chi$ field undergoes a transition, leading to a logarithmic potential
    similar to MOND, but derived from the substrate's elasticity.

  \paragraph{The Bullet Cluster as Relaxation Lag.}
    The observed displacement between baryonic mass and gravitational lensing in systems like the Bullet Cluster is
    reinterpreted as a \textbf{relaxation lag}.
    In high-velocity collisions, the projective geometry $\Pi$ associated with localized solitons (``dark peaks'')
    persists longer than the dissipative gas configurations, manifesting as a ``geometric memory'' in the $\chi$ field.

  \paragraph{Comparison with WIMPs and Axions.}
    While WIMPs require new fundamental fields, Cosmochrony suggests that ``dark'' effects arise from
    \textbf{non-projected spectral modes}—configurations of $\chi$ that possess inertial mass (resistance to relaxation)
    but lack the specific symmetry required for electromagnetic transmittance.
