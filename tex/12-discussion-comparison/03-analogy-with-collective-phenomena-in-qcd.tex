\subsection{Analogy with collective phenomena in QCD}
    \label{subsec:analogy-with-collective-phenomena-in-qcd}

  A useful analogy may be drawn with quantum chromodynamics at low energies, where the fundamental
  degrees of freedom (quarks and gluons) do not correspond directly to observable particles~\cite{Shifman2007QCDVacuum}.
  Instead, hadronic properties and effective masses emerge from a strongly interacting, collective vacuum
  structure often described in terms of a quark--gluon sea.
  In a similar spirit, the present framework does not attribute gravitational phenomena to a fundamental interaction
  mediated by elementary fields, but to collective effects arising from excitations and modulations of the underlying
  $\chi$ field.

  As in QCD, the relevant physical description depends on the scale and regime considered: while the
  microscopic dynamics may be simple in principle, the emergent large-scale behavior is governed by
  nonlinear and collective effects that are more naturally captured by effective, phenomenological
  descriptions.
