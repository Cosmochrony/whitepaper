\subsection{Relation to the Higgs Mechanism}
  \label{subsec:relation-to-the-higgs-mechanism}

  In the Standard Model, particle masses arise through spontaneous symmetry breaking of the electroweak sector, mediated
  by the Higgs field.
  In Cosmochrony, mass is not introduced as a fundamental parameter but emerges as a measure of resistance of localized
  $\chi$ configurations to the global relaxation flow.

  These two descriptions are not in contradiction.
  Rather, the Higgs field may be understood as an effective low-energy manifestation of the interaction between
  solitonic excitations and the surrounding $\chi$ background.
  In this view, the Higgs condensate encodes how localized field configurations acquire inertial properties within an
  already structured geometric substrate.

  Cosmochrony does not deny the empirical success of the Higgs mechanism, nor does it seek to modify its phenomenology
  at accessible energies.
  Instead, it suggests that the Higgs field is not fundamental, but emergent, much like the spacetime metric or quantum
  wavefunctions.
  The observed Higgs boson would then correspond to a collective excitation of the $\chi$ field associated with mass
  stabilization.
  In this framework, the Higgs vacuum expectation value (VEV) would be indirectly determined by
  the local properties of the $\chi$ background, effectively coupling the micro-physics of particle
  masses to the macro-dynamics of cosmic relaxation.

  A detailed derivation of the Higgs sector as an effective theory emerging from $\chi$ dynamics lies beyond the scope
  of the present work and is left for future investigation.
