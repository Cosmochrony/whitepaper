\subsection{Relation to General Relativity}
  \label{subsec:relation-to-general-relativity}

  General Relativity (GR) describes gravitation as the curvature of spacetime induced by energy--momentum.
  In Cosmochrony, no \emph{a priori} metric dynamics is postulated.
  Instead, an effective spacetime geometry emerges from spatial variations in the local relaxation rate of $\chi$.

  Matter configurations, modeled as stable or metastable topological excitations of $\chi$, locally slow the relaxation of
  the field.
  This induces differential proper-time rates between neighboring regions, which can be reinterpreted as an effective
  metric deformation.
  In the weak-field limit, this mechanism reproduces Newtonian gravity, while in the strong-field regime it yields an
  effective Schwarzschild-like geometry.

  From this perspective, gravitation is not a fundamental interaction but an emergent manifestation of temporal
  inhomogeneity in the evolution of $\chi$.
  This interpretation preserves the empirical successes of GR while offering a geometric origin for gravitational time
  dilation and curvature.
