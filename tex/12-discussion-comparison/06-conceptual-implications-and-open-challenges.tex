\subsection{Conceptual Implications and Open Challenges}
  \label{subsec:conceptual-implications-and-open-challenges}

  Cosmochrony offers a unifying geometric narrative in which time, distance, energy, gravitation, and
  quantization originate from a single evolving field.
  This conceptual economy is a central strength of the framework, but it also requires a careful
  reassessment of the status of several foundational notions traditionally treated as independent.

  In particular, Cosmochrony suggests that time, energy, and irreversibility are not separate physical
  primitives.
  The monotonic relaxation of the $\chi$ field provides the fundamental temporal ordering, while energy
  quantifies the residual capacity of $\chi$ configurations to relax.
  Irreversibility, in turn, reflects the progressive exhaustion of this relaxation capacity.
  From this perspective, temporal flow and energetic processes are two complementary descriptions of
  the same underlying geometric dynamics, rather than independent axioms of nature.

  While this reinterpretation resolves several conceptual tensions---such as the origin of the arrow of
  time or the status of energy conservation---it also raises important open questions.
  Among these are:
  \begin{itemize}
    \item the precise mapping between $\chi$ dynamics and observed CMB anisotropies,
    \item the treatment of non-equilibrium quantum measurements and decoherence,
    \item the emergence of gauge symmetries and interaction hierarchies,
    \item and the robustness of solitonic particle configurations under extreme conditions.
  \end{itemize}

  Addressing these challenges will require a combination of analytical, numerical, and experimental
  approaches, including:
  \begin{enumerate}
    \item large-scale numerical simulations of $\chi$ dynamics to quantify structure formation and
    cosmological signatures,
    \item exploration of discretized or network-based realizations of $\chi$ at microscopic scales,
    \item and experimental tests of predicted $\chi$-dependent effects in quantum coherence,
    gravitation, and radiation processes.
  \end{enumerate}

  Progress along these directions may elevate Cosmochrony from a unifying conceptual framework to a
  quantitatively predictive theory, while preserving its minimal ontological foundation.

\subsection{Spectral Origin of Mass and the Secondary Role of $V(\chi)$}
  \label{subsec:spectral_mass}

  A key conjecture of the Cosmochrony framework is that particle masses are not fundamental parameters encoded in the nonlinear potential $V(\chi)$, but instead emerge as spectral properties of a relaxation operator defined on the underlying discrete substrate.

  In this perspective, the role of $V(\chi)$ is secondary and effective: it serves as a convenient coarse-grained description of localization and stability, but does not fundamentally determine the mass spectrum.

  \paragraph{Mass spectrum as eigenmodes of a relaxation operator.}
    Localized particle-like excitations are identified with normal modes of a discrete Laplace--Beltrami operator acting on the graph $G(V,E)$,
    \begin{equation}
      \Delta_G \psi_n = -\lambda_n \psi_n ,
    \end{equation}
    where $\psi_n$ are eigenmodes of the relaxation dynamics.
    The associated particle masses are conjectured to scale as
    \begin{equation}
      m_n c^2 \propto \sqrt{\lambda_n}.
    \end{equation}

    This mechanism is directly analogous to the emergence of discrete acoustic frequencies in bounded elastic systems, where the spectrum is entirely fixed by geometry and boundary conditions rather than by adjustable material parameters.
    Within Cosmochrony, mass hierarchies are thus interpreted as geometric properties of the underlying network topology and connectivity.

    A decisive test of this conjecture consists in computing the low-lying spectrum of $\Delta_G$ on large but finite graphs with physically motivated connectivity rules.
    If even approximate agreement with observed mass ratios were obtained, this would strongly suggest that $V(\chi)$ is not a fundamental ingredient of the theory.

  \paragraph{Residual role of the potential $V(\chi)$.}
    Within this spectral picture, the nonlinear potential $V(\chi)$ may be understood as an effective description of localization mechanisms arising after coarse-graining.
    Its form is constrained by the requirement that it admit stable solitonic solutions corresponding to the low-lying eigenmodes of the relaxation operator, but it does not independently fix their masses.

  \paragraph{Supporting perspectives.}
    Discrete symmetry constraints and information-theoretic considerations may further restrict admissible network structures or provide complementary interpretations of the emergent dynamics.
    However, these directions are secondary to the central spectral hypothesis and are not required for its internal consistency.

    Taken together, these considerations suggest that the primary explanatory burden for mass generation in Cosmochrony lies in the spectral properties of the underlying discrete relaxation dynamics, with $V(\chi)$ playing a derived and non-fundamental role.
