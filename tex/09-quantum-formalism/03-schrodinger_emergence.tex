\subsection{Emergence of the Schr\"odinger Equation as an Effective Description}
  \label{subsec:schrodinger-emergence}

  Within the Cosmochrony framework, quantum dynamics is not postulated as a
  fundamental law.
  The Schr\"odinger equation arises as an effective, long-wavelength and
  non-relativistic description organizing admissible projected descriptions in
  regimes where localization, approximate factorization, and temporal
  projectability are simultaneously valid.

  Rather than emerging from physical fluctuations of an underlying field, the
  Schr\"odinger equation appears as a universal consistency condition governing
  the time evolution of localized projected descriptions whose internal structure
  remains approximately stationary under projection.

  \subsubsection*{Formal Non-Relativistic Limit}
    \label{subsec:kg-to-schrodinger}

    In regimes where an effective relativistic description applies, admissible
    projected descriptions of localized configurations admit a second-order
    hyperbolic evolution equation whose formal structure coincides with the
    Klein--Gordon equation.
    This equation should be understood as an effective geometric encoding of
    stability and admissibility constraints, not as a fundamental field equation
    defined on spacetime.

    In the non-relativistic regime, the effective description separates naturally
    into a rapidly varying phase associated with rest-energy and a slowly varying
    envelope describing spatial localization.
    Formally, this separation may be written as
    \begin{equation}
      \Psi(x,t) = \psi(x,t)\, e^{-i \omega_0 t},
      \qquad
      \omega_0 = \frac{mc^2}{\hbar},
    \end{equation}
    where $\Psi$ denotes an effective relativistic descriptive field and $\psi$ its
    non-relativistic envelope.

    Imposing the condition that the envelope varies slowly compared to the
    rest-energy scale leads, to leading order, to the Schr\"odinger equation,
    \begin{equation}
      i\hbar\,\partial_t \psi
      = -\frac{\hbar^2}{2m}\nabla^2 \psi + V(x)\psi,
      \label{eq:schrodinger-effective}
    \end{equation}
    where $V(x)$ encodes weak external constraints on admissible projected
    descriptions, such as background gravitational or electromagnetic influences.

    \paragraph{Interpretation.}
      The wavefunction $\psi$ does not represent a physical excitation or fluctuation
      of an underlying substrate.
      It is a derived mathematical object encoding the admissible temporal evolution
      of localized projected descriptions once a non-relativistic spacetime
      interpretation becomes applicable.

      In this sense, the Schr\"odinger equation is not a fundamental dynamical law.
      It is the effective evolution equation governing admissible projected
      descriptions in regimes where linearity, localization, and approximate
      factorization provide accurate descriptive approximations.

  \subsubsection*{Operators and Algebra from \texorpdfstring{$\chi$}{χ} Fluctuations}
    \label{subsec:operators-from-chi}

    In the Cosmochrony framework, quantum operators are not introduced as primitive
    postulates acting on an abstract Hilbert space.
    They arise instead as \emph{effective generators} associated with admissible
    variations of projected configurations within a projectable geometric regime.

    As discussed in Sections~\ref{subsec:spectral-scaling-projection-ontology} and~\ref{subsec:h-origin},
    the projection from $\chi$ to effective observables is generically non-injective.
    As a consequence, effective spacetime descriptions generally correspond to
    equivalence classes of underlying relational configurations.
    This structural degeneracy implies that infinitesimal variations of projected
    observables cannot, in general, be represented as commuting operations.

    \paragraph{Emergence of position and momentum operators.}
      In a regime where projected configurations admit a smooth spacetime interpretation,
      effective position $x$ labels equivalence classes of $\chi$ configurations sharing
      identical relational correlations.
      Translations in $x$ therefore correspond to reparametrizations of the projection
      rather than to displacements within a fundamental background space.

      The generator of such translations is naturally represented by an operator
      $\hat{p}$ acting on effective wavefunctions $\psi(x)$,
    \begin{equation}
      \hat{p} = - i \hbar_{\mathrm{eff}} \, \partial_x ,
      \end{equation}
      where $\hbar_{\mathrm{eff}}$ is the projected expression of the fundamental
      reprojection scale $\hbar_\chi$.

      Conversely, the position operator $\hat{x}$ acts multiplicatively on $\psi(x)$
      and encodes the coarse-grained labeling of projected relational configurations.
      Neither $\hat{x}$ nor $\hat{p}$ corresponds to an observable defined at the level
      of the $\chi$ substrate; both are emergent descriptors valid only within a
      projectable geometric regime.

    \paragraph{Origin of non-commutativity.}
      The canonical commutation relation
    \begin{equation}
      [\hat{x}, \hat{p}] = i \hbar_{\mathrm{eff}}
      \end{equation}
      does not reflect a fundamental indeterminacy of nature.
      It expresses the \emph{non-commutativity of geometric measurements} induced by
      projection from the pre-geometric substrate.

      Operationally, performing a localization followed by a translation corresponds
      to a different class of admissible projections than performing the same operations
      in reverse order.
      This mismatch originates from the fact that localization and translation probe
      distinct relational aspects of projected descriptions that cannot be jointly
      resolved within a single effective spacetime representation.

      The uncertainty principle therefore emerges as a geometric consistency condition
      on admissible effective descriptions, rather than as an ontological randomness
      or a fundamental limit imposed at the level of $\chi$ itself.

    \paragraph{Algebra of observables as a projection artifact.}
      More generally, the algebra of quantum observables reflects the structure of the
      space of admissible projections from $\chi$ to effective variables.
      Non-commuting operators encode the incompatibility of simultaneously sharp
      geometric descriptors within a projected spacetime representation.

      In this sense, Hilbert space and operator algebra arise as a \emph{representation
      theory of degeneracy in admissible projected descriptions}, providing a compact and predictive language
      for describing stable regimes of projected $\chi$ dynamics.

    \paragraph{Scope and limitations.}
      The present derivation accounts for the operator structure and non-commutative
      algebra of non-relativistic quantum mechanics.
      Its extension to quantum field theory requires additional structure, namely the
      treatment of fields as infinite collections of coupled projected modes.

      Within Cosmochrony, quantum fields correspond to continuous families of
      collective excitations of the $\chi$ substrate rather than to fundamental operators
      defined at each spacetime point.
      While the emergence of canonical commutation relations for field amplitudes
      is expected in appropriate limits, a full reconstruction of relativistic quantum
      field theory lies beyond the scope of the present work and is left for future study.
