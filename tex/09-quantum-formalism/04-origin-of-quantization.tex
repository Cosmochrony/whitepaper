\subsection{Origin of Quantization}
  \label{subsec:origin-of-quantization}

  In standard quantum theory, quantization is introduced axiomatically through canonical
  commutation relations or path-integral prescriptions.
  Within the Cosmochrony framework, quantization is not fundamental and does not arise
  from an underlying microscopic dynamics.
  It emerges as a structural consequence of stability, consistency, and admissibility
  constraints imposed on projected descriptions.

  Only a restricted class of localized projected configurations admits long-lived,
  internally consistent descriptions.
  Configurations that fail to satisfy these constraints rapidly lose projectability
  and cannot be maintained as persistent physical descriptions.
  As a result, admissible configurations form discrete equivalence classes rather than
  a continuous spectrum.

  Energy quantization reflects this discreteness.
  Energy does not label an intrinsic property of a physical excitation, but characterizes
  the degree of structural persistence of a projected configuration within the relaxation
  ordering.
  Only specific values correspond to stable descriptive regimes, leading to an effective
  discretization of energy exchanges.

  The relation
  \begin{equation}
    E = h \nu
  \end{equation}
  does not express a fundamental oscillatory dynamics.
  It encodes a proportionality between the energetic cost of maintaining a persistent
  projected configuration and the characteristic ordering rate at which its internal
  structure must be consistently re-identified.
  The frequency $\nu$ should therefore be understood as a descriptive rate associated
  with relational re-identification, not as oscillation with respect to a fundamental
  time parameter.

  Within this perspective, Planck’s constant does not represent a fundamental quantum
  of action.
  It emerges as a universal conversion factor characterizing the minimal structural
  scale at which projected descriptions remain stable and coherent across ordering.
  Its apparent universality reflects the universality of the projectability constraints
  themselves, rather than the postulation of an underlying quantized substrate.
