\subsection{Measurement and the Born Rule}
  \label{subsec:measurement-and-the-born-rule}

  Within the Cosmochrony framework, measurement does not involve a fundamental
  wavefunction collapse, nor does it rely on stochastic fluctuations of an underlying
  physical substrate.
  The $\chi$ field evolves continuously and deterministically according to its intrinsic
  relational structure.

  What is conventionally described as a measurement corresponds to an irreversible
  loss of projectability: a localized projected description becomes dynamically coupled
  to a macroscopic environment, preventing the continued joint maintenance of
  incompatible relational alternatives.
  This process is described at the effective level by decoherence, as discussed in
  Section~\ref{subsec:measurement-and-decoherence}.

  Measurement outcomes correspond to effective reprojections onto mutually exclusive
  descriptive regimes.
  These outcomes are not selected by hidden fluctuations or random microscopic events,
  but by the structural compatibility between the pre-measurement description and the
  macroscopic constraints imposed by the measurement apparatus.

  The Born rule does not encode a fundamental probability law.
  It emerges as the unique stable measure on the space of admissible projected
  descriptions that remains invariant under loss of phase coherence and coarse-graining.
  The squared amplitude $|\psi|^2$ quantifies the relative measure of descriptive
  compatibility between a pre-measurement state and the set of macroscopically
  distinguishable outcomes.

  In this sense, $|\psi|^2$ does not represent subjective uncertainty or intrinsic
  randomness.
  It characterizes the structural weight of admissible reprojections consistent with
  both the prior relational configuration and the constraints defining the measurement
  context.
  The Born rule therefore reflects a geometric and consistency-based property of
  projected descriptions, rather than a fundamental stochastic law of nature.
