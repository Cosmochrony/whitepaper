\subsection{Emergence of Hilbert Space Structure}
  \label{subsec:emergence-of-hilbert-space-structure}

  The Hilbert space formalism of quantum mechanics provides a linear structure
  supporting superposition, interference, and unitary evolution.
  Within the Cosmochrony framework, this structure does not reflect a fundamental
  property of an underlying physical substrate.
  It emerges as an effective mathematical organization of admissible projected
  descriptions in regimes where relational constraints between projected descriptions are weak
  and approximately factorizable.

  In such regimes, distinct admissible projected descriptions can be combined
  without introducing strong mutual constraints, giving rise to an approximate
  linear structure.
  Superposition therefore reflects the formal coexistence of multiple compatible
  \emph{descriptive alternatives}, rather than the simultaneous physical
  realization of multiple states.

  \paragraph{Inner Product and Compatibility.}
    The inner-product structure of Hilbert space encodes the degree of mutual
    compatibility between projected descriptions.
    Orthogonality corresponds to mutually exclusive descriptive regimes, while
    non-orthogonal states represent partially compatible projections whose
    distinctions cannot be jointly resolved within a single effective description.

  \paragraph{Role of Complex Phase.}
    The complex phase of the wavefunction does not correspond to an intrinsic
    oscillatory structure of a physical field.
    It encodes relational consistency conditions between alternative projected
    descriptions, ensuring coherent interference patterns within effective
    spacetime representations.

  \paragraph{Validity of Unitary Evolution.}
    Unitary evolution arises as a consistency-preserving transformation within the
    space of admissible projected descriptions, valid so long as projectability and
    approximate factorizability remain intact.
    When these conditions fail---such as during measurement or strong environmental
    coupling---the Hilbert space description ceases to be adequate, and non-unitary
    effective behavior emerges.

    In this perspective, Hilbert space is not a fundamental arena of physical
    reality.
    It is the natural mathematical structure organizing the space of admissible
    descriptions in regimes where linearity and coherence provide accurate effective
    approximations.
