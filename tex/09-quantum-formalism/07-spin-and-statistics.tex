\subsection{Spin and Statistics}
  \label{subsec:spin-and-statistics}

  Within the Cosmochrony framework, spin does not arise as an intrinsic kinematic
  degree of freedom of a particle, nor as a representation of spacetime symmetries.
  It emerges as a topological property of admissible projected configurations
  associated with localized physical descriptions.

  Certain classes of admissible configurations exhibit a non-trivial internal
  topology such that a $2\pi$ effective rotation does not return the configuration
  to an equivalent descriptive state.
  Only after a $4\pi$ rotation is full equivalence restored.
  Projected descriptions with this property correspond to fermionic behavior, while
  configurations that are $2\pi$-periodic correspond to bosonic behavior.

  The spin--statistics connection follows naturally from this topological distinction.
  Configurations with non-trivial covering structure cannot be symmetrically exchanged
  without violating relational consistency, leading to antisymmetric statistics.
  By contrast, configurations with trivial topology admit symmetric exchange.

  In this perspective, spin and statistics are not independent postulates of quantum
  theory.
  They reflect the same underlying topological constraints on the space of admissible
  projected descriptions, providing a unified geometric origin for both phenomena.
  An explicit illustrative construction is discussed in
  Section~\ref{subsec:4pi_soliton}.
