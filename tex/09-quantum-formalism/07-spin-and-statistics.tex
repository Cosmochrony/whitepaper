\subsection{Spin and Statistics}
  \label{subsec:spin-statistics}

  Within the Cosmochrony framework, spin does not arise as an intrinsic kinematic
  degree of freedom of a particle, nor as a fundamental representation of spacetime symmetries.
  Instead, it emerges as a topological property of admissible projected configurations associated with localized
  physical descriptions.

  Certain classes of admissible configurations possess a non-trivial internal covering
  structure such that a $2\pi$ effective rotation does not return the configuration to an
  equivalent descriptive state, while a $4\pi$ rotation does.
  Projected descriptions exhibiting this property correspond to fermionic behavior, whereas configurations
  that are $2\pi$-periodic correspond to bosonic behavior.

  The connection between spin and statistics follows directly from this topological distinction.
  Configurations with non-trivial covering structure cannot be symmetrically
  exchanged without violating relational consistency, leading to antisymmetric exchange behavior.
  By contrast, configurations with trivial topology admit symmetric exchange.

  Spin and statistics are therefore not independent postulates of quantum theory.
  They reflect the same underlying topological constraints on the space of admissible projected descriptions.
  This unified origin accounts simultaneously for half-integer spin, fermionic statistics, and the Pauli exclusion
  principle, without introducing additional quantum axioms.

  A concrete topological construction illustrating these properties is presented in
  Appendix~\ref{subsec:4pi_soliton}.
