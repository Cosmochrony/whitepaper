\subsection{Orbital Geometry as Probabilistic Visibility}
  \label{sec:orbital_geometry_visibility}

  Atomic orbitals do not represent spatially extended material distributions.
  Within the Cosmochrony framework, they correspond to probabilistic visibility
  patterns associated with admissible projected descriptions of localized bound
  configurations.

  Orbital shapes encode structural and symmetry constraints imposed on admissible
  projected descriptions, such as nodal surfaces and angular dependence.
  These features reflect conditions of relational consistency and stability rather
  than the presence of a spatially distributed object.

  The apparent spatial extent of an orbital does not indicate the physical size or
  motion of an underlying entity.
  It reflects the range of effective spatial locations over which a projected
  description remains admissible under repeated reprojection and measurement.
  Regions of high probability correspond to domains where consistent reprojection is
  most robust, while nodal regions correspond to incompatible descriptive regimes.

  Orbital visualizations therefore do not depict occupied regions of space.
  They represent statistical maps of descriptive accessibility within an effective
  geometric representation.
  In this sense, atomic orbitals encode how bound configurations can be consistently
  described in spacetime, rather than revealing the spatial structure of an underlying
  physical object.
