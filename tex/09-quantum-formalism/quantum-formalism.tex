\section{Relation to Quantum Formalism}
  \label{sec:relation-to-quantum-formalism}

  This section does not assign fundamental ontological status to the quantum
  wavefunction nor to Hilbert space structures.
  Instead, it shows how the formal apparatus of quantum mechanics arises as an
  effective description of localized, weakly interacting excitations of the
  $\chi$ field introduced in the preceding sections.

  Quantum mechanics is therefore not replaced, but reinterpreted as a consistent
  coarse-grained framework valid in regimes where $\chi$ admits a stable geometric
  and linearized description.

  \subsection{Status of the Wavefunction}
    \label{subsec:status-of-the-wavefunction}

    In standard quantum mechanics, the wavefunction $\psi$ is a complex-valued object
    defined on configuration space, whose ontological status remains debated.
    Operationally, $|\psi|^2$ encodes measurement probabilities via the Born rule,
    while $\psi$ itself does not correspond directly to a physical field in spacetime.

    In Cosmochrony, the $\chi$ field is not identified with the quantum wavefunction.
    Instead, $\chi$ constitutes a real scalar substratum, from which effective
    quantum wavefunctions emerge as statistical descriptors of coherent excitations.
    The wavefunction is therefore interpreted as a derived object encoding the
    collective behavior of $\chi$-mediated structures rather than as a fundamental
    entity.

    As an illustrative example, the hydrogen atom wavefunctions
    $\psi_{nlm}(r,\theta,\phi)$ correspond to stable solitonic configurations of
    $\chi$ characterized by discrete internal structure.
    The probability density $|\psi|^2$ reflects the spatial visibility of these
    configurations in effective geometric descriptions, while energy quantization
    arises from topological and resonance constraints imposed on stable $\chi$
    excitations.

  \subsection{Emergence of Hilbert Space Structure}
    \label{subsec:emergence-of-hilbert-space-structure}

    The Hilbert space formalism of quantum mechanics provides a linear structure
    supporting superposition, interference, and unitary evolution.
    Within Cosmochrony, this structure emerges as an effective description of weakly
    interacting, small-amplitude fluctuations of the $\chi$ field around a slowly
    varying background.

    Approximate linear superposition reflects the near-independence of such
    fluctuations in regimes where nonlinear constraints are negligible.
    The complex phase of the wavefunction encodes relative internal oscillatory
    structure within $\chi$ rather than representing an intrinsic complex field.

  \subsection{Emergence of the Schr\"odinger Equation from $\chi$ Fluctuations}
    \label{sec:schrodinger_emergence}

    In Cosmochrony, quantum dynamics is not postulated but emerges as a long-wavelength,
    non-relativistic description of coherent $\chi$-field fluctuations around stable
    solitonic configurations.
    In this subsection, the Schr\"odinger equation is recovered as the standard
    non-relativistic limit of such fluctuations.

    \subsubsection{Non-relativistic limit: Klein--Gordon $\rightarrow$ Schr\"odinger}
      \label{sec:KGtoSch}

      Consider a localized excitation around a quasi-stationary soliton background,
      \begin{equation}
        \chi(x,t) = \chi_{\mathrm{sol}}(x) + \delta\chi(x,t),
      \end{equation}
      where $\delta\chi$ denotes a small fluctuation.
      To leading order, $\delta\chi$ obeys an effective Klein--Gordon equation,
      \begin{equation}
        \left(\frac{1}{c^{2}}\partial_{t}^{2}
          - \nabla^{2}
          + \frac{m^{2}c^{2}}{\hbar^{2}}\right)\delta\chi = 0,
        \label{eq:KG_eff}
      \end{equation}
      with $m$ determined by the soliton’s rest-energy.

      In the non-relativistic regime, the field oscillates rapidly at frequency
      $\omega_{0} = mc^{2}/\hbar$, while its envelope varies slowly.
      Using the standard ansatz
      \begin{equation}
        \delta\chi(x,t) = \psi(x,t)e^{-i\omega_{0}t},
        \qquad
        |\partial_t\psi| \ll \omega_{0}|\psi|,
        \label{eq:NR_ansatz}
      \end{equation}
      and neglecting higher-order relativistic corrections yields the Schr\"odinger
      equation,
      \begin{equation}
        i\hbar\,\partial_t\psi
        = -\frac{\hbar^{2}}{2m}\nabla^{2}\psi + V(x)\psi,
        \label{eq:Sch_V}
      \end{equation}
      where $V(x)$ encodes weak interactions with the surrounding $\chi$ background.

      \paragraph{Interpretation.}
        The wavefunction $\psi$ does not represent a fundamental quantum object but an
        effective envelope describing coherent $\chi$ fluctuations.
        The Schr\"odinger equation thus appears as a universal non-relativistic limit of
        localized $\chi$ excitations rather than as a fundamental axiom.

  \subsection{Origin of Quantization}
    \label{subsec:origin-of-quantization}

    Quantization in standard quantum theory is postulated through canonical
    commutation relations or path-integral prescriptions.
    In Cosmochrony, discrete energy exchanges arise from topological and stability
    constraints on localized $\chi$ excitations.

    Only specific internal configurations, resonance conditions, or winding numbers
    are dynamically stable.
    As a result, energy levels are effectively quantized.
    The relation $E = h\nu$ emerges as a proportionality between oscillation frequency
    and relaxation potential stored in localized $\chi$ configurations, with Planck’s
    constant appearing as an effective coupling scale.

  \subsection{Measurement and the Born Rule}
    \label{subsec:measurement-and-the-born-rule}

    Measurement does not involve a fundamental wavefunction collapse.
    Instead, it corresponds to an irreversible interaction between a coherent
    excitation and stochastic fluctuations of the surrounding $\chi$ field.

    Detection events occur when such interactions stabilize a localized configuration.
    The Born rule arises statistically from the distribution of $\chi$ fluctuations
    compatible with the detector’s constraints.
    Thus, $|\psi|^{2}$ represents the density of favorable configurations rather than a
    primitive probability postulate.

    This statistical interpretation is consistent with the decoherence mechanism
    discussed in Section~\ref{subsec:measurement-and-decoherence}.
    Decoherence suppresses interference between relational branches without altering
    the underlying $\chi$ configuration, while measurement outcomes correspond to
    effective reprojections selected by fluctuations.
    The squared amplitude therefore emerges as the unique stable measure preserved
    under loss of phase coherence, reflecting the structural density of admissible
    reprojections rather than subjective uncertainty or fundamental randomness.

  \subsection{Entanglement and Nonlocal Correlations}
    \label{subsec:entanglement-and-nonlocal-correlations}

    Quantum entanglement corresponds to persistent relational structure within a
    single extended $\chi$ configuration.
    Separated particles remain correlated because they are manifestations of the same
    underlying excitation.

    These correlations do not rely on superluminal signaling.
    They reflect the holistic but dynamically local structure of $\chi$, consistent
    with relativistic causality.

  \subsection{Spin and Statistics}
    \label{subsec:spin-and-statistics}

    Spin emerges from topological organization of $\chi$ excitations.
    Configurations requiring a $4\pi$ internal rotation to return to equivalence
    correspond to fermions, while $2\pi$-periodic configurations correspond to bosons.

    The spin--statistics connection follows naturally from the topological stability of
    these structures rather than from imposed quantum postulates.
    An explicit example is discussed in Section~\ref{subsec:4pi_soliton}.

  \subsection{Orbital Geometry as Probabilistic Visibility of Underlying $\chi$ Fluctuations}
    \label{sec:orbital_geometry_visibility}

    Atomic orbitals are not spatial material distributions but probabilistic
    visibility patterns of underlying $\chi$ fluctuations.
    Their shapes encode stable structural features of $\chi$, such as nodal surfaces
    and symmetry constraints, while their apparent extent reflects stochastic
    variations.

    Orbital visualizations therefore represent statistical manifestations of
    stationary $\chi$ modes rather than occupied spatial regions.

  \subsection{Scope and Limitations}
    \label{subsec:scope-and-limitations}

    Cosmochrony does not seek to replace quantum mechanics.
    All standard computational tools remain valid within their domain of
    applicability.

    Its contribution is interpretative and unificatory: it provides a coherent
    geometric and dynamical origin for quantum phenomena without modifying
    experimentally verified predictions.
    A complete mapping between $\chi$ dynamics and operator-based quantum theory is
    left for future work.
