\section{Relation to Quantum Formalism}
  \label{sec:relation-to-quantum-formalism}

  This section does not assign fundamental ontological status to the quantum
  wavefunction, Hilbert space, or operator structures.
  Instead, it shows how the formal apparatus of quantum mechanics arises as an
  effective and internally consistent description organizing admissible projected
  descriptions in regimes where localization, linearity, and approximate
  factorization hold.

  Within the Cosmochrony framework, quantum mechanics is not derived from an
  underlying microscopic dynamics of the $\chi$ substrate.
  Rather, it emerges as a universal coarse-grained formalism governing the
  projectability and temporal consistency of localized physical descriptions once a
  stable geometric interpretation becomes applicable.

  Quantum mechanics is therefore not replaced, but reinterpreted as an effective
  framework whose validity is restricted to regimes where the relational structure
  of $\chi$ admits a stable, approximately linear, spacetime-based description.
  The following subsections clarify the origin and interpretational status of the
  wavefunction, Hilbert space structure, quantization, measurement, entanglement,
  and related quantum concepts within this perspective.

  \subsection{Status of the Wavefunction}
  \label{subsec:status-of-the-wavefunction}

  In standard quantum mechanics, the wavefunction $\psi$ is a complex-valued object
  defined on configuration space, whose ontological status remains debated.
  Operationally, $|\psi|^2$ encodes measurement probabilities via the Born rule, while
  $\psi$ itself does not correspond to a physical field propagating in spacetime.

  Within the Cosmochrony framework, the quantum wavefunction is not identified with any
  fundamental physical entity.
  It arises at the level of effective descriptions as a statistical encoding of the
  set of admissible local reprojections compatible with a given non-factorizable
  projected configuration.

  The wavefunction therefore does not represent an underlying structure or hidden
  dynamics.
  It summarizes, in compact mathematical form, the relative accessibility of different
  local outcomes once a global descriptive coherence has been established.
  Its complex phase encodes relational constraints between alternative local
  descriptions, while its modulus determines the statistical weight of accessible
  reprojections.

  As an illustrative example, the hydrogen atom wavefunctions
  $\psi_{nlm}(r,\theta,\phi)$ do not correspond to localized structures or solitonic
  configurations at a fundamental level.
  They represent stationary admissible projected descriptions characterized by
  specific symmetry and stability properties.
  The probability density $|\psi|^2$ reflects the relative frequency with which
  localized reprojections occur in effective geometric descriptions, while energy
  quantization arises from the discrete admissibility conditions imposed on stationary
  descriptive regimes.

  In this perspective, the wavefunction is neither a physical field nor a direct image
  of an underlying substrate.
  It is a derived statistical object, encoding the structure of admissible projected
  descriptions and the probabilities of their local realization within spacetime.

  \subsection{Emergence of Hilbert Space Structure}
  \label{subsec:emergence-of-hilbert-space-structure}

  The Hilbert space formalism of quantum mechanics provides a linear structure
  supporting superposition, interference, and unitary evolution.
  Within the Cosmochrony framework, this structure does not reflect a fundamental
  property of an underlying physical substrate.
  It emerges as an effective mathematical organization of admissible projected
  descriptions in regimes where relational constraints are weak and approximately
  factorizable.

  In such regimes, distinct admissible projected descriptions can be combined without
  significant mutual constraint, giving rise to an approximate linear structure.
  Superposition reflects the coexistence of multiple compatible descriptive
  alternatives, rather than the simultaneous physical realization of multiple states.

  The inner-product structure of Hilbert space encodes the degree of mutual
  compatibility between projected descriptions.
  Orthogonality corresponds to mutually exclusive descriptive regimes, while
  non-orthogonal states represent partially compatible projections whose distinctions
  cannot be jointly resolved.

  The complex phase of the wavefunction does not correspond to an intrinsic oscillatory
  structure of a physical field.
  It encodes relational consistency conditions between alternative projected
  descriptions, ensuring coherent interference patterns within effective spacetime
  representations.

  Unitary evolution arises as a consistency-preserving transformation within the space
  of admissible projected descriptions, valid so long as projectability and
  factorizability remain approximately intact.
  When these conditions fail, such as during measurement or strong environmental
  coupling, the Hilbert space description ceases to be adequate, and non-unitary
  effective behavior emerges.

  In this perspective, Hilbert space is not a fundamental arena of physical reality.
  It is the natural mathematical structure organizing the space of admissible
  descriptions in regimes where linearity and coherence provide accurate effective
  approximations.

  \subsection{Emergence of the Schr\"odinger Equation as an Effective Description}
  \label{sec:schrodinger_emergence}

  Within the Cosmochrony framework, quantum dynamics is not postulated as a fundamental
  law.
  The Schr\"odinger equation arises as an effective, long-wavelength and
  non-relativistic description organizing admissible projected descriptions in regimes
  where localization, approximate factorization, and temporal projectability are
  simultaneously valid.

  Rather than emerging from physical fluctuations of an underlying field, the
  Schr\"odinger equation appears as the universal consistency condition governing the
  time evolution of localized projected descriptions whose internal structure remains
  approximately stationary.

  \subsubsection*{Formal Non-Relativistic Limit}
    \label{sec:KGtoSch}

    In regimes where an effective relativistic description applies, admissible projected
    descriptions of localized configurations admit a second-order hyperbolic evolution
    equation whose formal structure coincides with the Klein--Gordon equation.
    This equation should be understood as an effective geometric encoding of stability
    and admissibility conditions, not as a fundamental field equation.

    In the non-relativistic regime, the effective description separates naturally into a
    rapidly varying phase associated with rest-energy and a slowly varying envelope
    describing spatial localization.
    Formally, this separation may be written as
    \begin{equation}
      \Psi(x,t) = \psi(x,t)\, e^{-i \omega_0 t},
      \qquad \omega_0 = \frac{mc^2}{\hbar},
    \end{equation}
    where $\Psi$ denotes an effective relativistic descriptive field and $\psi$ its
    non-relativistic envelope.

    Imposing the condition that the envelope varies slowly compared to the rest-energy
    scale leads, to leading order, to the Schr\"odinger equation,
    \begin{equation}
      i\hbar\,\partial_t \psi
      = -\frac{\hbar^2}{2m}\nabla^2 \psi + V(x)\psi,
      \label{eq:Sch_V}
    \end{equation}
    where $V(x)$ encodes weak external constraints on admissible projected descriptions,
    such as background gravitational or electromagnetic effects.

    \paragraph{Interpretation.}
      The wavefunction $\psi$ does not represent a physical excitation or fluctuation of an
      underlying substrate.
      It is a derived mathematical object encoding the admissible time evolution of
      localized projected descriptions once a non-relativistic spacetime interpretation
      becomes applicable.

      In this sense, the Schr\"odinger equation is not a fundamental dynamical law.
      It is the effective evolution equation governing admissible projected descriptions in
      regimes where linearity, localization, and approximate factorization provide accurate
      descriptive approximations.

  \subsection{Origin of Quantization}
  \label{subsec:origin-of-quantization}

  In standard quantum theory, quantization is introduced axiomatically through canonical
  commutation relations or path-integral prescriptions.
  Within the Cosmochrony framework, quantization is not fundamental and does not arise
  from an underlying microscopic dynamics.
  It emerges as a structural consequence of stability, consistency, and admissibility
  constraints imposed on projected descriptions.

  Only a restricted class of localized projected configurations admits long-lived,
  internally consistent descriptions.
  Configurations that fail to satisfy these constraints rapidly lose projectability
  and cannot be maintained as persistent physical descriptions.
  As a result, admissible configurations form discrete equivalence classes rather than
  a continuous spectrum.

  Energy quantization reflects this discreteness.
  Energy does not label an intrinsic property of a physical excitation, but characterizes
  the degree of structural persistence of a projected configuration within the relaxation
  ordering.
  Only specific values correspond to stable descriptive regimes, leading to an effective
  discretization of energy exchanges.

  The relation
  \begin{equation}
    E = h \nu
  \end{equation}
  does not express a fundamental oscillatory dynamics.
  It encodes a proportionality between the energetic cost of maintaining a persistent
  projected configuration and the characteristic ordering rate at which its internal
  structure must be consistently re-identified.
  The frequency $\nu$ should therefore be understood as a descriptive rate associated
  with relational re-identification, not as oscillation with respect to a fundamental
  time parameter.

  Within this perspective, Planck’s constant does not represent a fundamental quantum
  of action.
  It emerges as a universal conversion factor characterizing the minimal structural
  scale at which projected descriptions remain stable and coherent across ordering.
  Its apparent universality reflects the universality of the projectability constraints
  themselves, rather than the postulation of an underlying quantized substrate.

  \subsection{Measurement and the Born Rule}
  \label{subsec:measurement-and-the-born-rule}

  Within the Cosmochrony framework, measurement does not involve a fundamental
  wavefunction collapse, nor does it rely on stochastic fluctuations of an underlying
  physical substrate.
  The $\chi$ field evolves continuously and deterministically according to its intrinsic
  relational structure.

  What is conventionally described as a measurement corresponds to an irreversible
  loss of projectability: a localized projected description becomes dynamically coupled
  to a macroscopic environment, preventing the continued joint maintenance of
  incompatible relational alternatives.
  This process is described at the effective level by decoherence, as discussed in
  Section~\ref{subsec:measurement-and-decoherence}.

  Measurement outcomes correspond to effective reprojections onto mutually exclusive
  descriptive regimes.
  These outcomes are not selected by hidden fluctuations or random microscopic events,
  but by the structural compatibility between the pre-measurement description and the
  macroscopic constraints imposed by the measurement apparatus.

  The Born rule does not encode a fundamental probability law.
  It emerges as the unique stable measure on the space of admissible projected
  descriptions that remains invariant under loss of phase coherence and coarse-graining.
  The squared amplitude $|\psi|^2$ quantifies the relative measure of descriptive
  compatibility between a pre-measurement state and the set of macroscopically
  distinguishable outcomes.

  In this sense, $|\psi|^2$ does not represent subjective uncertainty or intrinsic
  randomness.
  It characterizes the structural weight of admissible reprojections consistent with
  both the prior relational configuration and the constraints defining the measurement
  context.
  The Born rule therefore reflects a geometric and consistency-based property of
  projected descriptions, rather than a fundamental stochastic law of nature.

  \subsection{Entanglement and Nonlocal Correlations}
  \label{subsec:entanglement-and-nonlocal-correlations}

  Within the Cosmochrony framework, quantum entanglement does not correspond to a
  physical linkage or interaction between spatially separated entities.
  It reflects the persistence of a shared relational structure within a single,
  non-factorizable configuration of the $\chi$ substrate.

  Entangled systems are described by projected configurations that cannot be
  decomposed into independent subsystems without loss of relational consistency.
  Although effective spacetime descriptions assign distinct locations to the
  corresponding subsystems, these locations represent different projections of a
  single underlying relational configuration.

  Nonlocal correlations therefore do not arise from superluminal influences or hidden
  signal exchange.
  They follow from the fact that measurement operations act on a globally defined
  relational structure whose admissible projections must remain mutually consistent.
  Once a projection is selected in one region, the set of admissible projections
  elsewhere is correspondingly constrained, without any dynamical transmission.

  In this sense, quantum nonlocality in Cosmochrony is ontological rather than
  dynamical.
  The underlying relational configuration is globally defined, while its evolution
  and reprojection remain locally governed by the relaxation and projectability
  constraints of $\chi$.
  As a result, entanglement correlations are fully compatible with relativistic
  causality and do not require the introduction of nonlocal forces or preferred
  reference frames.

  \subsection{Spin and Statistics}
  \label{subsec:spin-statistics}

  Within the Cosmochrony framework, spin does not arise as an intrinsic kinematic
  degree of freedom of a particle, nor as a fundamental representation of spacetime symmetries.
  Instead, it emerges as a topological property of admissible projected configurations associated with localized
  physical descriptions.

  Certain classes of admissible configurations possess a non-trivial internal covering
  structure such that a $2\pi$ effective rotation does not return the configuration to an
  equivalent descriptive state, while a $4\pi$ rotation does.
  Projected descriptions exhibiting this property correspond to fermionic behavior, whereas configurations
  that are $2\pi$-periodic correspond to bosonic behavior.

  The connection between spin and statistics follows directly from this topological distinction.
  Configurations with non-trivial covering structure cannot be symmetrically
  exchanged without violating relational consistency, leading to antisymmetric exchange behavior.
  By contrast, configurations with trivial topology admit symmetric exchange.

  Spin and statistics are therefore not independent postulates of quantum theory.
  They reflect the same underlying topological constraints on the space of admissible projected descriptions.
  This unified origin accounts simultaneously for half-integer spin, fermionic statistics, and the Pauli exclusion
  principle, without introducing additional quantum axioms.

  A concrete topological construction illustrating these properties is presented in
  Appendix~\ref{subsec:4pi_soliton}.

  \subsection{Orbital Geometry as Probabilistic Visibility}
  \label{sec:orbital_geometry_visibility}

  Atomic orbitals do not represent spatially extended material distributions.
  Within the Cosmochrony framework, they correspond to probabilistic visibility
  patterns associated with admissible projected descriptions of localized bound
  configurations.

  Orbital shapes encode structural and symmetry constraints imposed on admissible
  projected descriptions, such as nodal surfaces and angular dependence.
  These features reflect conditions of relational consistency and stability rather
  than the presence of a spatially distributed object.

  The apparent spatial extent of an orbital does not indicate the physical size or
  motion of an underlying entity.
  It reflects the range of effective spatial locations over which a projected
  description remains admissible under repeated reprojection and measurement.
  Regions of high probability correspond to domains where consistent reprojection is
  most robust, while nodal regions correspond to incompatible descriptive regimes.

  Orbital visualizations therefore do not depict occupied regions of space.
  They represent statistical maps of descriptive accessibility within an effective
  geometric representation.
  In this sense, atomic orbitals encode how bound configurations can be consistently
  described in spacetime, rather than revealing the spatial structure of an underlying
  physical object.

  \subsection{Scope and Limitations}
  \label{subsec:scope-and-limitations}

  Cosmochrony does not aim to replace quantum mechanics as a predictive or
  computational framework.
  All standard quantum-mechanical formalisms, including operator methods, path
  integrals, and perturbative techniques, remain valid and unchanged within their
  established domains of applicability.

  The contribution of Cosmochrony is interpretative and unificatory.
  It provides a coherent pre-geometric and relational origin for quantum phenomena,
  clarifying the ontological status of the wavefunction, quantization, measurement,
  and nonlocal correlations, without altering any experimentally verified
  predictions.

  In this framework, quantum mechanics is understood as an effective theory
  governing admissible projected descriptions in regimes where linearity,
  localization, and factorization hold approximately.
  Cosmochrony does not introduce new degrees of freedom, hidden variables, or
  modifications of quantum dynamics.

  A complete formal correspondence between the relational $\chi$ substrate and the
  operator-based structures of quantum theory, including a systematic derivation of
  Hilbert space, observables, and evolution operators, lies beyond the scope of the
  present work and is left for future investigation.

  Accordingly, the present framework should be regarded as a foundational
  reconstruction rather than a competing physical theory, intended to clarify the
  conceptual origin and domain of validity of quantum-mechanical descriptions.

