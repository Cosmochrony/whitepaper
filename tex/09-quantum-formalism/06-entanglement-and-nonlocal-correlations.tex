\subsection{Entanglement and Nonlocal Correlations}
  \label{subsec:entanglement-and-nonlocal-correlations}

  Within the Cosmochrony framework, quantum entanglement does not correspond to a
  physical linkage or interaction between spatially separated entities.
  It reflects the persistence of a shared relational structure within a single,
  non-factorizable configuration of the $\chi$ substrate.

  Entangled systems are described by projected configurations that cannot be
  decomposed into independent subsystems without loss of relational consistency.
  Although effective spacetime descriptions assign distinct locations to the
  corresponding subsystems, these locations represent different projections of a
  single underlying relational configuration.

  Nonlocal correlations therefore do not arise from superluminal influences or hidden
  signal exchange.
  They follow from the fact that measurement operations act on a globally defined
  relational structure whose admissible projections must remain mutually consistent.
  Once a projection is selected in one region, the set of admissible projections
  elsewhere is correspondingly constrained, without any dynamical transmission.

  In this sense, quantum nonlocality in Cosmochrony is ontological rather than
  dynamical.
  The underlying relational configuration is globally defined, while its evolution
  and reprojection remain locally governed by the relaxation and projectability
  constraints of $\chi$.
  As a result, entanglement correlations are fully compatible with relativistic
  causality and do not require the introduction of nonlocal forces or preferred
  reference frames.
