\subsection{Physical Interpretation}
  \label{subsec:physical-interpretation}

  In Cosmochrony, spacetime is not postulated as a fundamental background structure.
  Instead, it arises as an effective macroscopic description once configurations of
  the relational substrate $\chi$ admit a sufficiently stable and projectable regime.
  Temporal and spatial notions are therefore understood as emergent features of a
  single underlying irreversible ordering process, rather than as independent
  primitives.

  In such regimes, the infra-physical projection from $\chi$ to an effective
  description $\chi_{\mathrm{eff}}$ yields a factorisable structure, allowing an
  approximate decomposition into subsystems and the definition of operational
  observables.
  This factorisation underlies the emergence of classical locality, compatibility of
  measurements, and standard relativistic descriptions in stable domains, as
  illustrated in Fig.~\ref{fig:classical-factorisable}.

  \begin{figure}[t]
    \centering
    \begin{tikzpicture}[
      font=\small,
      node distance=10mm,
      box/.style={draw, rounded corners, align=center, inner sep=6pt},
      arrow/.style={-Latex, thick},
      note/.style={align=left, font=\footnotesize},
      dashedbox/.style={draw, dashed, rounded corners, inner sep=6pt}
    ]

    \node[box] (chi) {$\chi$\\\footnotesize infra-physical substrate};

    \node[box, below=of chi] (chieff) {$\chi_{\mathrm{eff}}$\\\footnotesize factorisable regime};

    \node[dashedbox, below=of chieff, minimum width=6.6cm] (decomp) {
      \begin{tabular}{c}
        \footnotesize $\chi_{\mathrm{eff}} \simeq \chi_{\mathrm{eff}}^{(A)} \otimes \chi_{\mathrm{eff}}^{(B)}$\\
        \footnotesize (independent subsystems)
      \end{tabular}
    };

    \node[box, below left=10mm and 12mm of decomp] (obsA) {Local observables\\in subsystem $A$};
    \node[box, below right=10mm and 12mm of decomp] (obsB) {Local observables\\in subsystem $B$};

    \draw[arrow] (chi) -- node[right=2mm, note] {infra-physical\\projection $\pi$} (chieff);
    \draw[arrow] (chieff) -- (decomp);

    \draw[arrow] (decomp) -- node[left=2mm, note] {operational\\projection $\mathcal{O}_A$} (obsA);
    \draw[arrow] (decomp) -- node[right=2mm, note] {operational\\projection $\mathcal{O}_B$} (obsB);

    \node[note, below=7mm of decomp, align=center] (compat)
    {\footnotesize Compatible operational readings: joint assignment of local observables is well-defined.};

    \end{tikzpicture}
    \caption{Classical (factorisable) regime.
    After the infra-physical projection $\pi$, the effective description
      $\chi_{\mathrm{eff}}$ admits an approximate decomposition into independent subsystems.
      Operational projections $\mathcal{O}_A$ and $\mathcal{O}_B$ then yield compatible local
      observables, recovering standard classical and relativistic descriptions.
      This figure illustrates the hierarchy between infra-physical projection and
      operational observation, and contrasts with the non-factorisable regimes discussed
      in later sections.}
    \label{fig:classical-factorisable}
  \end{figure}

  Because the projection from $\chi$ to $\chi_{\mathrm{eff}}$ is generically
  non-injective, the effective observables obtained in this way summarize relational
  structure without exhausting the underlying degrees of freedom.
  Distinct $\chi$ configurations may therefore correspond to identical effective
  descriptions, while a single underlying configuration may admit multiple correlated
  operational realizations.
  This structural asymmetry is central to the emergence of both classical and quantum
  phenomenology, and is developed in detail in subsequent sections.

  Within this interpretation, temporal ordering, relational separation, and large-scale
  behavior are not treated as independent postulates, but as complementary effective
  summaries of the same underlying relaxation dynamics, appearing at different levels
  of coarse-graining.
  Their precise operational and dynamical roles are introduced progressively in the
  following sections.

  The physical content of the theory thus resides entirely in the dynamics of the
  fundamental $\chi$ substrate, while spacetime notions function as emergent,
  context-dependent descriptive tools rather than as fundamental ingredients.
