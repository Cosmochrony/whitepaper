\subsection{Physical Interpretation}
  \label{subsec:physical-interpretation}

  In Cosmochrony, spacetime is not assumed as a pre-existing background structure.
  Instead, it appears as an effective macroscopic description arising from the
  continuous and monotonic ordering intrinsic to the relational substrate $\chi$.
  What are conventionally described as temporal and spatial features are understood
  here as distinct, but related, descriptive manifestations of this single underlying
  process, once a geometric regime becomes applicable.

  In regimes where projected $\chi$ configurations exhibit sufficiently stable and
  smooth correlation patterns, variations of the effective scalar descriptor
  $\chi_{\mathrm{eff}}$ give rise to a set of operational observables.
  In particular, an increase in $\chi_{\mathrm{eff}}$ along a given physical process
  is associated with:
  \begin{itemize}
    \item the accumulation of operational proper time along that process,
    \item the progressive decorrelation between effective configurations, summarized
    as an emergent spatial separation,
    \item the large-scale expansion behavior observed when the ordering of projected
    $\chi$ configurations is considered at the cosmological level.
  \end{itemize}

  Within this effective description, temporal duration and spatial separation are not
  independent primitives.
  They represent complementary aspects of the same underlying ordering structure,
  captured at different levels of coarse-graining.
  Heuristically, effective distance may be viewed as the persistent imprint of
  relational differentiation that has already occurred, while effective time
  corresponds to the ongoing local ordering of projected $\chi$ configurations.
  These expressions are intended as interpretative guides rather than literal
  definitions, emphasizing their common dynamical origin.

  This unified interpretation is not introduced ad hoc.
  It follows directly from identifying temporal ordering and relational separation as
  distinct effective summaries of the same underlying $\chi$ structure, once a
  macroscopic spacetime description becomes appropriate.
  The physical content of the theory therefore resides entirely in the dynamics of the
  fundamental $\chi$ substrate, while spacetime notions serve as emergent descriptive
  tools valid only within restricted regimes.
