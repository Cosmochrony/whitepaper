\subsection{Physical Interpretation}
  \label{subsec:physical-interpretation}

  In Cosmochrony, spacetime is not assumed as a pre-existing background structure.
  Instead, it appears as an effective macroscopic description arising from the
  continuous and monotonic relaxation dynamics of the scalar quantity $\chi$.
  What are conventionally described as temporal and spatial features are understood
  here as distinct, but related, descriptive manifestations of this single underlying
  process.

  In regimes where $\chi$ exhibits sufficiently stable and smooth correlation
  patterns, variations of the field give rise to a set of effective observables.
  In particular, an increase in $\chi$ is associated with:
  \begin{itemize}
    \item the accumulation of operational proper time along physical processes,
    \item the progressive decorrelation between configurations, summarized as an
    effective spatial separation,
    \item the large-scale expansion behavior observed when the relaxation of $\chi$
    is considered at the cosmological level.
  \end{itemize}

  Within this effective description, temporal duration and spatial separation are not
  independent primitives.
  They represent complementary aspects of the same relaxation dynamics, captured at
  different levels of coarse-graining.
  Heuristically, effective distance may be viewed as the persistent imprint of
  relaxation that has already occurred, while effective time corresponds to the
  ongoing local relaxation of $\chi$.
  These expressions are intended as interpretative guides rather than literal
  definitions, emphasizing their common dynamical origin.

  This unified interpretation is not introduced ad hoc.
  It follows directly from identifying temporal ordering and relational separation as
  distinct summaries of the same scalar relaxation process, once a macroscopic
  description becomes appropriate.
  The physical content of the theory therefore resides entirely in the dynamics of
  $\chi$, while spacetime notions serve as emergent descriptors valid in restricted
  regimes.
