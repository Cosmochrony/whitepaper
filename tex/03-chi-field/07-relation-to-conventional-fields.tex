\subsection{Relation to Conventional Fields}
  \label{subsec:relation-to-conventional-fields}

  Effective descriptions derived from projected $\chi$ configurations may formally
  resemble scalar or tensor fields used in cosmology and particle physics.
  This resemblance, however, reflects the emergence of a spacetime-based descriptive
  language, not the presence of an additional fundamental field.
  The $\chi$ substrate itself remains a pre-geometric relational structure, independent
  of spacetime and field-theoretic notions.

  Within Cosmochrony, energy and quantization are not fundamental attributes of $\chi$.
  They arise only at the effective level, as consequences of the non-injective
  projection from $\chi$ to admissible observables.
  Only certain stable, localized, and spectrally isolated configurations admit a
  particle-like interpretation and can be consistently described using conventional
  quantum field-theoretic tools in regimes where a spacetime description is valid.

  Matter, radiation, and interactions are therefore not associated with independent
  fundamental fields coupled to $\chi$.
  They correspond instead to effective degrees of freedom arising from structural
  constraints, spectral organization, and long-lived relational patterns of projected
  $\chi$ configurations.
  Standard Model fields are recovered as accurate effective descriptions within the
  appropriate coarse-grained regimes.

  From this perspective, Cosmochrony does not extend the Standard Model by introducing
  new fundamental fields.
  Rather, it provides an ontological explanation for the emergence, applicability, and
  structural properties of effective field descriptions themselves.
