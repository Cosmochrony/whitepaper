\subsection{Initial Conditions and Global Structure}
  \label{subsec:initial-conditions-and-global-structure}

  The Cosmochrony framework does not postulate initial conditions in the conventional
  temporal sense.
  Instead, it assumes that the relational substrate $\chi$ admits a minimal admissible
  ordering state, denoted $\chi_0$, corresponding to configurations of maximal effective
  relaxation density.
  This reference state characterizes the earliest physically meaningful configurations
  within projected descriptions, without presupposing a fundamental temporal origin or
  a distinguished initial instant.

  In effective geometric regimes, the characteristic scale associated with projected
  descriptions near $\chi_0$ coincides numerically with the Planck scale.
  This correspondence reflects the breakdown of coarse-grained spacetime descriptions
  below this regime, rather than the presence of a fundamental cutoff, discreteness, or
  microscopic spacetime structure.

  From this perspective, cosmic history is interpreted as the progressive ordering of
  projected $\chi$ configurations away from this minimal admissible state.
  No spacetime singularity is required in the fundamental description.
  Apparent singular behavior arises only when classical notions of time and distance are
  extrapolated beyond the regime in which projected $\chi$ configurations admit a
  stable geometric interpretation.

  The global structure of admissible projected descriptions is thus constrained by the
  ordering properties of the underlying $\chi$ substrate, rather than by arbitrarily
  specified initial data.
  In the following section, we derive a minimal effective dynamical equation governing
  the ordering of projected $\chi$ configurations and explore its immediate physical
  consequences.
