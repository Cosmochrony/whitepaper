\subsection{Monotonicity and Arrow of Time}
  \label{subsec:monotonicity-and-arrow-of-time}

  A fundamental postulate of the theory is that $\chi$ evolves monotonically\cite{Prigogine1997,Penrose1989Weyl}:
  \begin{equation}
    \frac{\partial \chi}{\partial t} \ge 0 .
  \end{equation}

  This monotonicity is not derived from statistical considerations, nor imposed as a thermodynamic boundary condition.
  Rather, it reflects a structural property of the $\chi$ field related to the irreversible character of its
  relaxation dynamics.
  Within Cosmochrony, energy is not treated as a fundamental conserved substance but as a measure of the residual
  capacity of a given $\chi$ configuration to relax.
  Only relaxation processes can dissipate this capacity, while a decrease of $\chi$ would correspond to a spontaneous
  reintroduction of contraction or tension into the field, for which no physical mechanism exists in the framework.

  Irreversibility therefore follows naturally: any decrease of $\chi$ would correspond to a contraction of both temporal
  and spatial structure and to an effective creation of relaxation potential, which is dynamically forbidden.
  The arrow of time is thus identified with the irreversible expenditure of the relaxation capacity of the $\chi$ field,
  rather than with a statistical entropy principle imposed at the outset.
