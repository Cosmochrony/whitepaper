\subsection{Monotonicity and Arrow of Time}
  \label{subsec:monotonicity-and-arrow-of-time}

  A central structural assumption of Cosmochrony is that the scalar quantity $\chi$
  evolves through a monotonic relaxation process:
  \begin{equation}
    \mathcal{D}_{\lambda} \chi \ge 0 .
  \end{equation}
  This condition is not introduced as a statistical statement, nor as a boundary
  condition imposed on an otherwise time-symmetric dynamics.
  Rather, it expresses an intrinsic property of the relaxation process governing
  $\chi$ itself.

  Within this framework, energy is not treated as a fundamental conserved substance,
  but as a measure of the remaining capacity of a given $\chi$ configuration to relax.
  As relaxation proceeds, this capacity is irreversibly expended.
  A hypothetical decrease of $\chi$ would correspond to a spontaneous restoration of
  relaxation capacity, effectively reintroducing contraction or tension into the field.
  No dynamical mechanism within Cosmochrony permits such a process.

  Irreversibility therefore follows directly from the structure of the $\chi$ dynamics.
  Because $\chi$ cannot decrease, the ordering of configurations induced by relaxation
  is intrinsically directed.
  What is conventionally described as the arrow of time is identified here with this
  directional ordering: the irreversible progression from configurations with greater
  relaxation capacity toward configurations in which that capacity has been exhausted.

  Importantly, this arrow is not derived from coarse-graining, probabilistic entropy,
  or special initial conditions.
  It emerges as a direct consequence of the monotonic relaxation of $\chi$, prior to
  any statistical or thermodynamic description.
  Temporal orientation is thus a manifestation of the fundamental dynamics, rather than
  an emergent asymmetry imposed at the macroscopic level.
