\subsection{Monotonicity and Arrow of Time}
  \label{subsec:monotonicity-and-arrow-of-time}

  A central structural assumption of Cosmochrony is that the relational substrate $\chi$
  admits an intrinsic, globally ordered relaxation structure.
  This ordering is reflected, within effective descriptions, by the monotonic behavior
  of the projected scalar descriptor $\chi_{\mathrm{eff}}$ along admissible physical
  processes:
  \begin{equation}
    \mathcal{D}_{\lambda} \chi_{\mathrm{eff}} \ge 0 .
  \end{equation}
  Here $\lambda$ denotes an ordering parameter associated with the relaxation of
  projected $\chi$ configurations, not a fundamental time coordinate.
  The inequality expresses a structural constraint on admissible projected
  representations, rather than the evolution equation of a fundamental scalar quantity.

  This monotonicity is not introduced as a statistical statement, nor as a boundary
  condition imposed on an otherwise time-symmetric dynamics.
  It reflects an intrinsic asymmetry in the ordering structure of $\chi$
  configurations, which constrains the class of effective descriptions compatible
  with the framework.
  In particular, the non-injective nature of the projection from $\chi$ to
  $\chi_{\mathrm{eff}}$ does not permit reversals of this ordering, but only a loss of
  information about underlying configurations.

  Within this perspective, energy is not treated as a fundamental conserved substance,
  but as an effective measure of the remaining capacity of projected $\chi$
  configurations to undergo further relaxation.
  As effective relaxation proceeds, this capacity is irreversibly expended in the
  projected description.
  A hypothetical decrease of $\chi_{\mathrm{eff}}$ along an admissible ordering path
  would correspond to a spontaneous restoration of relaxation capacity, effectively
  reintroducing contraction or tension into the projected representation.
  No mechanism within Cosmochrony allows such a reversal, as it would contradict the
  underlying ordering structure of $\chi$.

  Irreversibility therefore follows directly from the structure of the relaxation
  ordering admitted by $\chi$.
  Because admissible projected descriptions cannot exhibit decreasing
  $\chi_{\mathrm{eff}}$, the ordering of configurations induced by relaxation is
  intrinsically directed.
  What is conventionally described as the arrow of time is identified here with this
  directional ordering: the irreversible progression from configurations with greater
  effective relaxation capacity toward configurations in which that capacity has been
  locally or globally exhausted.

  Importantly, this arrow is not derived from coarse-graining, probabilistic entropy,
  or special initial conditions.
  It arises prior to any statistical or thermodynamic description, as a direct
  consequence of the structural ordering constraints imposed by $\chi$ on its
  projected representations.
  Temporal orientation is thus a manifestation of the fundamental ordering structure
  of the theory, rather than an emergent asymmetry introduced only at the macroscopic
  level.

  This intrinsic ordering underlies the thermodynamic arrow of time discussed in Section~\ref{subsec:entropy-arrow}.
