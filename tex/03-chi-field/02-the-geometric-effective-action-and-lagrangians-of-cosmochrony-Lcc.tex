\subsection{The Geometric Effective Description of $\chi$ Dynamics}
  \label{subsec:geometric-action}

  \subsubsection*{Effective Observables from $\chi$ Correlations}

    In Cosmochrony, quantities conventionally described in geometric terms—such as time
    intervals, spatial separation, and causal ordering—are not taken as primitive.
    They arise as effective descriptive summaries of relational patterns within the
    $\chi$ substrate, accessed only through projected, coarse-grained representations.
    The projection from $\chi$ to effective observables is generically non-injective,
    allowing distinct underlying configurations to correspond to identical effective
    descriptions.
    No background spacetime, coordinate system, or discrete substrate is assumed at any
    stage of the fundamental description.

    The configurations $\sigma$ represent internal relational states of the $\chi$
    substrate and are defined without reference to external spacetime coordinates or
    background geometry.
    They label patterns of internal organization rather than positions in a pre-existing
    space.

    Correlations between configurations $\sigma$ encode the emergent geometric and causal
    structure at the effective level, once a stable spacetime description becomes
    applicable.
    The measure $d\mu(\sigma)$ denotes an invariant integration over configuration space,
    defined intrinsically from the correlation structure associated with $\chi$.
    It ensures that physical observables are independent of the particular parametrization
    chosen to label configurations, and carries no interpretation as a volume element in an
    underlying spacetime.

    \paragraph{Effective scalar descriptor.}
      In regimes where projected $\chi$ configurations admit a stable geometric
      interpretation, it is convenient to introduce an \emph{effective scalar descriptor},
      denoted $\chi_{\mathrm{eff}}$.
      This quantity is a coarse-grained, projected representation of relational and
      spectral features of the $\chi$ substrate, defined only within the emergent spacetime
      description.
      It does not correspond to the fundamental $\chi$ substrate itself, which remains
      non-indexable and devoid of intrinsic values.
      Although $\chi_{\mathrm{eff}}$ is represented as a scalar function in effective
      descriptions, it should not be interpreted as a fundamental field propagating in
      spacetime.

      Operational time intervals are defined from the accumulated ordering of projected
      $\chi$ configurations along paths in configuration space.
      This ordering is quantified using the effective descriptor $\chi_{\mathrm{eff}}$,
      not the fundamental $\chi$ substrate:
    \begin{equation}
      \tau_{AB} \;\propto\;
      \int_{\gamma_{AB}} \mathcal{D}_{\lambda} \chi_{\mathrm{eff}} \, d\lambda ,
      \end{equation}
      where $\mathcal{D}_{\lambda} \chi_{\mathrm{eff}}$ is an effective relaxation functional
      characterizing the ordering of projected configurations along the path
      $\gamma_{AB}$.
      The parameter $\lambda$ is an ordering parameter, not a fundamental time coordinate.

      Geometric observables are constructed from correlations between effective descriptors
      $\chi_{\mathrm{eff}}$ associated with projected configurations.
      These correlations encode relational information about causal connectivity and
      separation once a geometric regime is established.

      Operational spatial separation is quantified by the decay of correlations between
      effective descriptors:
    \begin{equation}
      d(x,y) \;\propto\;
      - \log\!\left(
                \frac{\langle \chi_{\mathrm{eff}}(x)\,\chi_{\mathrm{eff}}(y)\rangle}
                {\langle \chi_{\mathrm{eff}}^{\,2}\rangle}
      \right),
      \end{equation}
      where $x$ and $y$ label effective spacetime events in the emergent description.
      The correlation function involves $\chi_{\mathrm{eff}}$ only and does not imply any
      localization or value assignment at the level of the fundamental $\chi$ substrate.
      Here, $\langle\chi_{\text{eff}}(x)\,\chi_{\text{eff}}(y)\rangle$ denotes an
      \textbf{effective correlation functional} encoding relational proximity between projected configurations $x$
      and $y$, not a statistical average over pre-existing degrees of freedom.

      These constructions are purely relational and make no reference to a pre-existing
      metric or discrete structure.
      They are applicable whenever projected $\chi$ configurations exhibit stable
      correlation patterns.

      Throughout this section, all indexed, correlated, or integrated quantities refer
      exclusively to the effective descriptor $\chi_{\mathrm{eff}}$, not to the
      fundamental $\chi$ substrate.

  \subsubsection*{Effective Metric as a Descriptive Tool}

    In regimes where projected $\chi$ configurations exhibit smooth and stable correlation
    patterns, the relational observables defined above may be compactly summarized by an
    operational tensor $g_{\mu\nu}[\chi_{\mathrm{eff}}]$.
    The notation emphasizes that the metric summarizes correlations of effective
    descriptors, not properties of the fundamental $\chi$ substrate.
    It is not introduced as a fundamental geometric structure, nor as an independent
    degree of freedom, but as a post-hoc parametrization of effective relational regularities.

    The metric provides a convenient summary of how variations in $\chi_{\mathrm{eff}}$
    modulate causal connectivity and relational intervals, and is meaningful only insofar
    as such a coarse-grained description remains valid.
    For example:
    \begin{itemize}
      \item The conformal (lightcone) structure is constrained by the maximal effective
      propagation speed $c$ associated with projected $\chi$ relaxation.
      \item Proper time between effective events is proportional to the accumulated
      $\chi_{\mathrm{eff}}$ ordering along connecting paths.
      \item Spatial distance reflects the decay rate of $\chi_{\mathrm{eff}}$ correlations.
    \end{itemize}

    No discrete-to-continuum limit is invoked.
    The theory is continuous at all scales; apparent granularity (including Planck-scale
    phenomena) arises from non-linear and threshold effects in $\chi$ dynamics and
    projection, not from an underlying discretization.

    No background $\eta_{\mu\nu}$ is assumed.
    Minkowski space appears only as an effective approximation in suitable limits (e.g.,
    weak-gradient regimes), without ontological status at the fundamental level.

  \subsubsection*{Consistency with General Relativity}

    The effective metric $g_{\mu\nu}[\chi_{\mathrm{eff}}]$, constructed as a descriptive
    summary of projected correlations, reproduces the phenomenology of general relativity
    in appropriate regimes:
    \begin{itemize}
      \item \textbf{Weak-field limit:} when $\chi_{\mathrm{eff}}$ gradients are small, the
      effective metric approaches a form compatible with Einstein-like dynamics for a
      fluid-like stress-energy description associated with $\chi$ excitations.
      \item \textbf{Strong-field regimes:} near localized $\chi$ excitations (e.g.,
      solitonic configurations), the metric encodes time dilation and spatial curvature
      as emergent consequences of inhibited $\chi_{\mathrm{eff}}$ relaxation.
      \item \textbf{Cosmological expansion:} homogeneous relaxation of $\chi$ yields an
      effective Hubble-like expansion law for the emergent scale factor.
    \end{itemize}

    Crucially, this is not a bootstrap procedure.
    The metric is never iteratively reconstructed or dynamically postulated; it remains a
    derived descriptor summarizing geometric regularities of projected $\chi$
    configurations.
    All predictive content resides in the underlying $\chi$ dynamics.

  \subsubsection*{Ontological Status of the Metric}

    To avoid confusion, we emphasize:
    \begin{itemize}
      \item $\chi$ is the only fundamental entity.
      Spacetime, metric structure, and matter are emergent descriptions of projected
      $\chi$ configurations.
      \item No ``double ontology'' is assumed: there is no underlying discrete graph or
      lattice.
      Geometric language is introduced only as an effective descriptive tool.
      \item The metric $g_{\mu\nu}[\chi_{\mathrm{eff}}]$ is an effective construct,
      analogous to how temperature emerges in thermodynamics.
      It is useful for coarse-grained description but plays no role in the fundamental
      dynamics.
    \end{itemize}

    \paragraph{Operational origin of the effective metric.}
      The metric tensor $g_{\mu\nu}$ is a derived descriptor summarizing relational distance
      induced by $\chi$ correlations.
      Its explicit construction from operational distances is given in Appendix~E, where
      geometric quantities are shown to arise only in projectable regimes admitting a
      smooth continuum approximation.

    \paragraph{Operational interpretation of the line element.}
      In Cosmochrony, the line element $ds^2$ is not a primitive geometric quantity.
      It is a shorthand for a local quadratic approximation of the operational relational
      distance.
      Concretely, if $d(i,j)$ denotes the operational distance between configurations in
      the relational network of $\chi$ (Appendix~E), then in projectable regimes admitting
      a low-dimensional embedding one may write locally
      \[
        d(i,j)^2 \;\approx\; g_{\mu\nu}(x)\,\Delta x^\mu \Delta x^\nu .
      \]
      The continuum line element $ds^2$ should therefore be understood as a derived
      descriptive construct, not as a fundamental structure.

  \subsubsection*{Summary: A Fully Continuous Framework}

    \begin{itemize}
      \item \textbf{Fundamental level:} only the continuous $\chi$ relational substrate
      exists, characterized by intrinsic ordering and relaxation dynamics.
      \item \textbf{Effective level:} geometric observables (time, distance, metric) emerge
      from correlations between effective descriptors $\chi_{\mathrm{eff}}$ associated
      with projected $\chi$ configurations.
      \item \textbf{No bootstrap:} the metric is never postulated or iteratively constructed;
      it is a derived summary of projected relational structure.
      \item \textbf{No discretization:} apparent ``Planck-scale'' effects arise from
      non-linear and threshold phenomena in $\chi$ dynamics and projection, not from an
      underlying discrete substrate.
    \end{itemize}
