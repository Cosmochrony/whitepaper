\subsection{The Geometric Effective Description of $\chi$ Dynamics}
  \label{subsec:geometric-action}

  \subsubsection{Effective Observables from $\chi$ Correlations}
    In Cosmochrony, quantities conventionally described in geometric terms---such as time
    intervals, spatial separation, and causal ordering---are not taken as primitive.
    They arise as descriptive summaries of relational patterns within the $\chi$ field,
    constructed directly from its internal variation structure.
    No background spacetime, coordinate system, or discrete substrate is assumed at any
    stage of the fundamental description.

  The configurations $\sigma$ represent internal states of the $\chi$ field and are
    defined without reference to external spacetime coordinates or background geometry.
    They label relational degrees of freedom of the field, specifying patterns of
    internal organization rather than positions in a pre-existing space.

    Correlations between configurations $\sigma$ encode the emergent geometric and
    causal structure at the effective level, once a stable spacetime description becomes
    applicable.
    The measure $d\mu(\sigma)$ denotes an invariant integration over this space of
    configurations, defined intrinsically from the correlation structure of $\chi$.
    It ensures that physical observables are independent of the particular
    parametrization chosen to label configurations, and carries no interpretation as a
    volume element in an underlying spacetime.

    Operational time intervals are defined by the accumulated relaxation of $\chi$ along a path
    in configuration space:
    \begin{equation}
      \tau_{AB} \propto \int_{\gamma_{AB}} \mathcal{D}_{\lambda} \chi \, d\lambda,
    \end{equation}
    where $\gamma_{AB}$ is a path connecting configurations $A$ and $B$, and
    $\mathcal{D}_{\lambda}\chi$ is the relaxation rate of $\chi$.

    Operational spatial separation is quantified by the decay of $\chi$ correlations between two
    configurations:
    \begin{equation}
      d(x,y) \propto -\log\left(\frac{\langle \chi(x)\chi(y)\rangle}{\langle \chi^2\rangle}\right),
    \end{equation}
    where $\langle \chi(x)\chi(y)\rangle$ measures the correlation between configurations labeled
    by $x$ and $y$.
    These definitions are purely relational and make no reference to a pre-existing metric or
    discrete structure.
    They are applicable whenever $\chi$ configurations exhibit stable patterns.

  \subsubsection{Effective Metric as a Descriptive Tool}
    In regimes where the $\chi$ field exhibits smooth and stable correlation patterns,
    the relational observables defined above may be compactly summarized by an operational
    tensor $g_{\mu\nu}[\chi]$.
    This object is not introduced as a fundamental geometric structure, nor as an
    independent degree of freedom.
    Rather, it provides a convenient parametrization of how variations in $\chi$
    modulate causal connectivity and effective relational intervals, and is meaningful
    only insofar as such a coarse-grained description remains valid.
    This metric is not a fundamental object, but a compact representation of how $\chi$
    correlations modulate causal connectivity and proper intervals.

    The metric is introduced as a descriptive summary of $\chi$ correlations, not postulated as
    an independent degree of freedom.
    For example:
    \begin{itemize}
      \item The conformal (lightcone) structure is constrained by the maximal relaxation speed
      $c$ of $\chi$.
      \item Proper time between effective events is proportional to the accumulated $\chi$
      relaxation along paths connecting the corresponding configurations.
      \item Spatial distance reflects the decay rate of $\chi$ correlations.
    \end{itemize}

    No discrete-to-continuum limit is invoked.
    The theory is continuous at all scales; apparent granularity (e.g., Planck-scale phenomena)
    is attributed to non-linear $\chi$ dynamics rather than to an underlying discretization.

    No background $\eta_{\mu\nu}$ is assumed.
    Minkowski space appears only as a convenient approximation in suitable limits (e.g.,
    weak-gradient regimes), without ontological status in the fundamental description.

  \subsubsection{Consistency with General Relativity}
    The effective metric $g_{\mu\nu}[\chi]$ constructed as a summary of $\chi$ correlations
    reproduces the phenomenology of general relativity in the following sense:
    \begin{itemize}
      \item \textbf{Weak-field limit:} when $\chi$ gradients are small, the effective metric
      approaches a form compatible with Einstein-like dynamics for a fluid-like
      stress-energy description associated with $\chi$ excitations.
      \item \textbf{Strong-field regimes:} near localized $\chi$ excitations (e.g., solitons), the
      metric encodes time dilation and spatial curvature as emergent effects of slowed $\chi$
      relaxation, without requiring a fundamental gravitational field.
      \item \textbf{Cosmological expansion:} homogeneous relaxation of $\chi$ yields an effective
      Hubble-like expansion law for the emergent scale factor.
    \end{itemize}

    Crucially, this is not a bootstrap procedure.
    The metric is not iteratively reconstructed from $\chi$; it is a post-hoc descriptive tool
    summarizing geometric regularities of $\chi$ configurations.
    The theory's predictive content resides in the dynamics of $\chi$, not in the metric itself.

  \subsubsection{Ontological Status of the Metric}
    To avoid confusion, we emphasize:
    \begin{itemize}
      \item $\chi$ is the only fundamental field.
      Spacetime, metric structure, and matter are emergent descriptions of $\chi$
      configurations.
      \item No ``double ontology'' is assumed: there is no underlying discrete graph or lattice.
      Geometric language is introduced only as an effective tool.
      \item The metric $g_{\mu\nu}[\chi]$ is an effective construct, analogous to how temperature
      emerges in thermodynamics.
      It is useful for coarse-grained description but plays no role in the fundamental
      dynamics.
    \end{itemize}

  \subsubsection{Summary: A Fully Continuous Framework}
    \begin{itemize}
      \item \textbf{Fundamental level:} only the continuous $\chi$ field exists, evolving through
      a monotonic relaxation dynamics.
      \item \textbf{Effective level:} geometric observables (time, distance, metric) emerge from
      $\chi$ correlations in suitable regimes.
      \item \textbf{No bootstrap:} the metric is never iteratively constructed or assumed; it is a
      derived description summarizing the relational structure of $\chi$ configurations.
      \item \textbf{No discretization:} apparent ``Planck-scale'' effects arise from non-linear
      $\chi$ dynamics, not from an underlying discrete substrate.
    \end{itemize}
