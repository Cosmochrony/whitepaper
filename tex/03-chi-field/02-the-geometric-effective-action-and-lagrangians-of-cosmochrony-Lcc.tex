\subsection{The Geometric Effective Action and Lagrangians of Cosmochrony $\left(\mathcal{L}_{\mathrm{CC}}\right)$}
  \label{sec:geometric-action}

  \subsubsection*{Interpretational caution.}
    The action principle presented below employs conventional field-theoretic notation, including a metric tensor $g_{\mu \nu}$ and a four-dimensional integration measure. This should not be interpreted as assuming pre-existing spacetime structure.

    The formalism serves two purposes:
    \begin{enumerate}
      \item To provide a compact representation of $\chi$ dynamics in regimes where an effective spacetime description is valid.
      \item To establish the bridge between the fundamental relational network and the effective manifold description used in standard physics.
    \end{enumerate}

    The fundamental content of the theory is the field $\chi$ and its relaxation dynamics on a discrete graph. The metric $g_{\mu \nu}$ appearing in the action is a statistical emergent structure representing the connectivity and correlation density of the $\chi$ field, not an independent ontological input.

  \subsubsection*{Discrete Network Foundation}
    The dynamics of $\chi$ are fundamentally defined on a \textbf{discrete network}, where each node $i$ represents a local value $\chi_i$, and each link between nodes $i$ and $j$ is characterized by a \textbf{connectivity strength} $K_{ij}$. This matrix $K_{ij}$ encodes the correlation between neighboring $\chi$-values and serves as the microscopic foundation for the emergent geometry. In regimes where $\chi$ admits a quasi-stable geometric interpretation, $K_{ij}$ can be mapped to an effective metric $g_{\mu\nu}$ through the relation:
    \begin{equation}
      g_{\mu \nu} dx^\mu dx^\nu \approx \sum_{(u,v) \in \text{path}} \frac{1}{K_{uv}}
      \label{eq:metric-emergent}
    \end{equation}

  \subsubsection*{Explicit Form of the Connectivity Matrix $K_{ij}$}
    \label{subsec:Kij-definition}
    The connectivity strength $K_{ij}$ between nodes $i$ and $j$ encodes the \textbf{local correlation} between their respective $\chi$-field values. To ensure consistency with the emergent geometry and the dynamical constraints of the $\chi$ field, we adopt the following \textbf{constitutive relation}:
    \begin{equation}
      K_{ij} = K_0 \cdot f\left(\frac{|\chi_i - \chi_j|^2}{\chi_c^2}\right)
      \label{eq:Kij-def}
    \end{equation}
    where:
    \begin{itemize}
      \item $K_0$ is a \textbf{fundamental coupling scale} (with dimensions of $[\text{length}]^{-2}$), representing the maximal connectivity strength in regions where $\chi$ is uniform.
      \item $\chi_c$ is a \textbf{characteristic scale} of the $\chi$ field, naturally associated with the Planck length or the cosmological relaxation scale.
      \item $f(x)$ is a \textbf{dimensionless, monotonically decreasing function} satisfying $f(0) = 1$ and $f(x) \to 0$ as $x \to \infty$.
    \end{itemize}

    A physically motivated choice for $f(x)$ is:
    \begin{equation}
      f(x) = \frac{1}{1 + x}
    \end{equation}
    This ansatz ensures that:
    \begin{enumerate}
      \item \textbf{Symmetry}: $K_{ij} = K_{ji}$, as required for a consistent relational structure.
      \item \textbf{Locality}: $K_{ij}$ depends only on the \textbf{local difference} $|\chi_i - \chi_j|$, reflecting the pre-geometric nature of the $\chi$ field.
      \item \textbf{Boundedness}: $0 < K_{ij} \leq K_0$, preventing unphysical divergences in the emergent metric.
      \item \textbf{Gradient sensitivity}: $K_{ij}$ decreases as $|\chi_i - \chi_j|$ increases, encoding the \textbf{resistance to relaxation} induced by localized excitations (e.g., particles or curvature).
    \end{enumerate}

    This form of $K_{ij}$ provides a \textbf{microscopic foundation} for the emergent metric $g_{\mu\nu}$, as
    detailed in Section~\ref{subsec:emergent-metric}.
    The coupling scale $K_0$ and the characteristic scale $\chi_c$ are expected to be related to fundamental
    constants (e.g., the Planck length $\ell_P$ or the Hubble scale $H_0^{-1}$), but their precise values are left
    as phenomenological parameters to be constrained by observations (see Section~\ref{subsec:normalization-of-the-chi-field}).

  \subsubsection*{Emergent Geometry from $K_{ij}$}
    \label{subsec:emergent-metric}
    The connectivity matrix $K_{ij}$ defines an \textbf{operational distance} between nodes $i$ and $j$ via the minimal path sum:
    \begin{equation}
      d(i, j)^2 = \ell_0^2 \sum_{(u,v) \in \text{path}} \frac{1}{K_{uv}}
    \end{equation}
    where $\ell_0$ is a microscopic length scale (e.g., related to the Planck length).
    In the continuum limit, this discrete sum converges to the line element of an \textbf{emergent metric} $g_{\mu\nu}$:
    \begin{equation}
      ds^2 = g_{\mu\nu} dx^\mu dx^\nu \approx \ell_0^2 \left[ \delta_{\mu\nu} + \mathcal{O}\left(\frac{|\nabla \chi|^2}{\chi_c^2}\right) \right] dx^\mu dx^\nu
    \end{equation}
    Here, the corrections $\mathcal{O}(|\nabla \chi|^2 / \chi_c^2)$ encode the \textbf{curvature induced by localized excitations} (e.g., particles or black holes).
    This construction ensures that:
    \begin{itemize}
      \item \textbf{Flat spacetime} emerges when $\chi$ is uniform ($\nabla \chi = 0$), as $K_{ij} \approx K_0$ and $g_{\mu\nu} \approx \eta_{\mu\nu}$.
      \item \textbf{Curved spacetime} arises in regions where $\nabla \chi \neq 0$, as $K_{ij}$ varies spatially, inducing a non-trivial $g_{\mu\nu}$.
    \end{itemize}

    This mechanism provides a \textbf{geometric interpretation of gravity} as a modulation of the $\chi$-field's connectivity, without invoking a fundamental metric or curvature tensor.
    Further details on the continuum limit and the derivation of the effective field equations are provided in Appendix~\ref{sec:collective-coupling}.

  \subsubsection*{Effective action formulation.}
    In regimes where $\chi$ admits a quasi-stable geometric interpretation, the dynamics may be encoded in an effective action:
    \begin{equation}
      S_{\mathrm{CC}} = \int \mathcal{L}_{\mathrm{CC}} \sqrt{-g} \, d^4 x
    \end{equation}
    where the Lagrangian density decomposes as:
    \begin{equation}
      \mathcal{L}_{\mathrm{CC}} = \mathcal{L}_{\text{Gravity/Time}} + \mathcal{L}_{\chi / \text{Soliton}} + \mathcal{L}_{\text{Forces/Matter}}
    \end{equation}
    The symbol $\sqrt{-g}$ represents the invariant volume element. In regimes where no spacetime interpretation yet exists (e.g., at the nodes of the fundamental graph), this should be understood as an abstract integration measure $d\mu$ on the configuration space of $\chi$.

  \subsubsection*{Status of $g_{\mu\nu}$ in this formulation.}
    The metric $g_{\mu\nu}$ is an effective description of the coupling strengths $K_{ij}$ between $\chi$ nodes. It is defined by the requirement that the distance $ds^2$ in the continuum matches the operational distance derived from the network's connectivity:
    \begin{equation}
      g_{\mu\nu} dx^\mu dx^\nu \approx \sum_{(u,v) \in \text{path}} \frac{1}{K_{uv}}
    \end{equation}
    Consequently, $g_{\mu\nu}$ is a phenomenological summary of the underlying relational dynamics, capturing the local rate of $\chi$-relaxation and its spatial correlations.
