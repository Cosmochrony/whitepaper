\subsection{Local Relaxation Speed}
  \label{subsec:local-relaxation-speed}

  A fundamental structural constraint of the Cosmochrony framework is that the effective
  local ordering rate associated with projected $\chi$ configurations is bounded.
  In effective geometric descriptions, this constraint takes the form
  \begin{equation}
    \left| \mathcal{D}_{\mathrm{loc}} \chi_{\mathrm{eff}} \right| \le c ,
  \end{equation}
  where $\mathcal{D}_{\mathrm{loc}} \chi_{\mathrm{eff}}$ denotes an effective local
  relaxation functional characterizing the maximal admissible ordering of projected
  $\chi$ configurations.
  The constant $c$ is the effective causal bound observed in spacetime.

  This inequality expresses a constraint on admissible projected descriptions, not a
  propagation speed defined at the level of the $\chi$ substrate.
  It limits the maximal rate at which effective causal connectivity and local geometric
  relations can be established within descriptions compatible with the intrinsic
  ordering structure of $\chi$.
  The quantity $c$ therefore characterizes the causal structure of the projected regime
  rather than the dynamics of the pre-geometric substrate itself.

  Local particle propagation, signal transmission, and field interactions are all
  constrained by this bound in effective spacetime descriptions.
  By contrast, the underlying $\chi$ substrate is not subject to spacetime notions of
  velocity or signal propagation.

  Apparent superluminal recession velocities at cosmological scales arise from cumulative
  and global effects of projected $\chi$ ordering and do not violate this local causal
  constraint.
  Local causal relations remain bounded by $c$ in all admissible effective descriptions.
