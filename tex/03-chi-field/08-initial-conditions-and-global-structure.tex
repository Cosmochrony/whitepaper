\subsection{Initial Conditions and Global Structure}
  \label{subsec:initial-conditions-and-global-structure}

  The Cosmochrony framework does not postulate initial conditions in the conventional
  temporal sense.
  Instead, it assumes that the relational substrate $\chi$ admits a minimal admissible
  ordering state, denoted $\chi_0$, which defines a structural boundary of admissible
  projected descriptions.
  This state does not correspond to a distinguished moment in time, but to the earliest
  configurations for which an effective ordering interpretation becomes meaningful.

  In effective geometric regimes, the characteristic scale associated with projected
  descriptions in the vicinity of $\chi_0$ coincides numerically with the Planck scale.
  This correspondence reflects the breakdown of projectability and of coarse-grained
  spacetime descriptions below this regime, rather than the presence of a fundamental
  cutoff, microscopic discreteness, or underlying spacetime lattice.

  From this perspective, cosmic history is interpreted as the progressive and
  irreversible ordering of projected $\chi$ configurations away from this minimal
  admissible boundary.
  No spacetime singularity is required at the fundamental level.
  Apparent singular behavior arises only when classical notions of time, distance, or
  curvature are extrapolated beyond the regime in which projected $\chi$ configurations
  admit a stable geometric and causal interpretation.

  \paragraph{Ontological poverty and the growth of admissible structure.}
    The minimal admissible state $\chi_0$ corresponds to a regime of \emph{ontological
poverty}, as defined in Section~\ref{sec:definition-and-fundamental-properties-of-the-chi-field}.
    Only a severely restricted class of simple and highly coherent configurations can be
    projected in this regime.
    As relaxation proceeds, the space of admissible configurations expands, enabling the
    emergence of increasingly rich, localized, and hierarchical effective structures.
    This growth reflects an expansion of descriptive and relational capacity, rather than
    the unfolding of pre-encoded complexity.

    The global structure of admissible projected descriptions is therefore constrained by
    the ordering properties of the underlying $\chi$ substrate, rather than by arbitrarily
    specified initial data or boundary conditions.
    In the following section, we derive a minimal effective dynamical equation governing
    the ordering of projected $\chi$ configurations and explore its immediate physical
    consequences.
