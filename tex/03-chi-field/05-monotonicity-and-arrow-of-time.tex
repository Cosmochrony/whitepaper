\subsection{Monotonicity and Arrow of Time}
  \label{subsec:monotonicity-and-arrow-of-time}

  A central structural postulate of Cosmochrony is that the relational substrate $\chi$
  admits an intrinsic, globally ordered relaxation structure.
  In effective descriptions, this ordering manifests as the monotonic behavior of the
  projected scalar descriptor $\chi_{\mathrm{eff}}$ along admissible ordering paths:
  \begin{equation}
    \mathcal{D}_{\lambda} \chi_{\mathrm{eff}} \ge 0 .
  \end{equation}
  Here $\lambda$ denotes an ordering parameter associated with the relaxation of
  projected $\chi$ configurations, rather than a fundamental time coordinate.
  The inequality expresses a structural constraint on admissible projected
  representations, not an evolution equation for a fundamental scalar quantity.

  This monotonicity reflects an intrinsic asymmetry in the ordering structure of $\chi$
  configurations.
  Because the projection from $\chi$ to $\chi_{\mathrm{eff}}$ is generically
  non-injective, admissible effective descriptions may lose information about
  underlying configurations, but cannot exhibit reversals of the ordering induced by
  relaxation.

  Within this framework, energy is interpreted as an effective measure of the
  remaining capacity of projected $\chi$ configurations to undergo further
  relaxation.
  As relaxation proceeds, this capacity is irreversibly expended in the projected
  description.
  Admissible ordering paths therefore exclude any effective decrease of
  $\chi_{\mathrm{eff}}$, which would correspond to a restoration of relaxation capacity
  incompatible with the underlying ordering structure.

  Irreversibility follows directly from this structural constraint.
  The arrow of time is identified with the directional ordering induced by relaxation:
  the progression from configurations with greater effective relaxation capacity
  toward configurations in which that capacity has been locally or globally exhausted.

  This temporal orientation is not derived from coarse-graining, entropy production,
  or special initial conditions.
  It arises prior to any statistical or thermodynamic description, as a direct
  consequence of the ordering constraints imposed by $\chi$ on its admissible
  projections.
  This perspective contrasts with approaches in which the arrow of time is attributed
  to entropy-based assumptions, coarse-graining, or special initial conditions~\cite{Prigogine1997,Rovelli1991}.
  The relation to thermodynamic irreversibility is discussed further in
  Section~\ref{subsec:entropy-arrow}.

  \paragraph{Projectability and Kinematic Saturation.}
    \label{par:projectability-kinematics}

    In the Cosmochrony framework, kinematic notions such as velocity do not refer to motion
    within a pre-existing spacetime background, but to the increasing informational load
    carried by the effective projection from the relational substrate~$\chi$ to its
    projectable description~$\chi_{\mathrm{eff}}$.
    A change of velocity corresponds to a modification of the relational coherence
    constraints that must be simultaneously maintained by the projection, including
    relative ordering, synchronization, and admissible causal relations.

    As velocity increases, the amount of relational information required to sustain a
    coherent effective description grows accordingly.
    This increasing informational demand progressively saturates the projectability
    capacity of the projection~$\Pi$.
    The bound $c_\chi$ should therefore not be interpreted as a maximal speed within
    spacetime, but as a structural limit beyond which no stable and globally consistent
    projection can be maintained.

    Approaching this saturation regime, part of the relational content of~$\chi$ becomes
    inaccessible to the effective description and must be traced out.
    This loss of projectability manifests as familiar relativistic effects, such as time
    dilation, length contraction, horizon formation, and observer-dependent effective
    descriptions.
    In this sense, relativistic kinematics emerges as a consequence of finite projection
    capacity, rather than as a fundamental geometric deformation of spacetime itself.

  \paragraph{Planck Scale and Relativistic Bounds as Projection Limits.}
    \label{par:planck-c-projective-limits}

    For readers accustomed to the standard formulation of relativistic and quantum
    physics, it is useful to emphasize that the Cosmochrony framework naturally places
    the speed of light~$c$ and Planck's constant~$h$ on a similar conceptual footing.
    Both constants are interpreted here as manifestations of a finite resolution of the
    projection from the relational substrate~$\chi$ to effective observables.

    The bound~$c$ limits the maximal admissible rate at which relational ordering and
    causal influence can be consistently projected, corresponding to a saturation of
    the effective relational flux.
    By contrast, Planck's constant~$h$ sets a lower bound on the granularity with which
    this flux can be resolved, limiting the minimal distinguishable action or relational
    variation within the projected description.

    In this view, relativistic and quantum constraints do not originate from independent
    postulates, but emerge as complementary facets of a single structural limitation:
    the finite capacity and resolution of the projection.
    The constants~$c$ and~$h$ thus characterize, respectively, the maximal propagation
    speed and the minimal resolvable scale of the same underlying relational dynamics.
