\subsection{Relational Projection and Spectral Admissibility}
  \label{sec:relational_projection}

  In Cosmochrony, admissible physical descriptions are not defined by spacetime
  locality or geometric structure, but by the spectral properties of the relational
  substrate $\chi$.

  \paragraph{Relational operator.}
    Let $\mathcal{H}_\chi$ denote the configuration space of admissible $\chi$ states.
    We introduce a relational operator
    \begin{equation}
      L_\chi : \mathcal{H}_\chi \rightarrow \mathcal{H}_\chi,
    \end{equation}
    defined purely in terms of $\chi$ correlations.
    In discrete implementations (Appendix~D), $L_\chi$ reduces to a graph Laplacian
    associated with relational adjacency; in the continuum limit, it defines an
    effective self-adjoint operator encoding relational connectivity.

  \paragraph{Spectral decomposition.}
    The operator $L_\chi$ admits a spectral decomposition
    \begin{equation}
      L_\chi \psi_n = \lambda_n \psi_n,
    \end{equation}
    where $\{\lambda_n\}$ are non-negative eigenvalues and $\{\psi_n\}$ the associated
    eigenmodes.
    No geometric or spacetime interpretation is assumed at this stage.

  \paragraph{Admissibility as spectral filtering.}
    Projectable configurations are selected by a spectral filter acting on $L_\chi$.
    We define the infra-physical admissibility projection as
    \begin{equation}
      \Pi_{\lambda_*} \;\equiv\; f\!\left(\frac{L_\chi}{\lambda_*}\right),
    \end{equation}
    where $f(x)$ is a fixed, smooth cutoff function satisfying
    $f(x)\to 0$ for $x\ll 1$ and $f(x)\to 1$ for $x\gg 1$.
    The scale $\lambda_*$ sets the characteristic spectral threshold separating
    admissible from non-admissible relational modes.

    This construction defines admissibility without reference to spacetime,
    coordinates, or integration measures, relying solely on spectral properties
    of the relational operator.
