\subsection{The Big Bang as a Maximal Constraint Regime of the $\chi$ Field}
  \label{subsec:big-bang-maximal-constraint}

  Within the Cosmochrony framework, the Big Bang is not interpreted as a spacetime
  singularity, nor as a physical event occurring at a definite moment.
  It corresponds instead to a limiting regime in which the relational structure of
  the $\chi$ substrate is maximally constrained.

  In this regime, the density of structural and topological constraints within $\chi$
  is such that no stable geometric projection is admissible.
  Concepts such as spatial distance, temporal duration, curvature, or causal ordering
  are therefore undefined.
  The apparent singular behavior encountered in standard cosmological models reflects
  the breakdown of effective spacetime descriptions when extrapolated beyond their
  domain of validity.

  Cosmic evolution is interpreted as the progressive relaxation of these maximal
  constraints.
  As the relational structure of $\chi$ becomes less constrained, increasingly stable
  projected descriptions become admissible, allowing effective notions of space,
  time, and geometry to emerge.
  The Big Bang thus marks not the beginning of spacetime, but the boundary beyond
  which spacetime ceases to be a meaningful descriptive framework.

  Within this perspective, the arrow of time does not originate from special initial
  conditions or entropy assumptions.
  It arises intrinsically from the monotonic relaxation ordering of $\chi$ away from
  the maximally constrained regime.
  Temporal ordering is therefore a consequence of the structural evolution of the
  substrate itself, rather than a feature imposed at the level of effective
  cosmological descriptions.
