\subsection{Dark Matter as Residual Relaxation Effects}
  \label{subsec:dark-matter-phenomenology}

  Cosmochrony addresses dark matter phenomenology not through the addition of
  hypothetical particles (WIMPs or Axions), but as a structural consequence of
  the substrate's relaxation dynamics.

  \paragraph{Galactic Rotation and Effective Stiffness.}
    The flattening of galactic rotation curves is interpreted as a spatial variation
    of the effective gravitational constant $G_{\mathrm{eff}}$.
    Near galactic centers, the high density of matter localizes the relaxation flow.
    At large radii, the stiffness $K_0$ of the $\chi$ field undergoes a transition,
    leading to a logarithmic potential similar to MOND, but derived from
    the substrate's elasticity.

  \paragraph{The Bullet Cluster as Relaxation Lag.}
    The observed displacement between baryonic mass and gravitational lensing
    in systems like the Bullet Cluster is reinterpreted as a \textbf{relaxation lag}.
    In high-velocity collisions, the projective geometry $\Pi$ associated with
    localized solitons (dark peaks) persists longer than the dissipative gas
    configurations, manifesting as a ``geometric memory'' in the $\chi$ field.

  \paragraph{Comparison with WIMPs and Axions.}
    While WIMPs require new fundamental fields, Cosmochrony suggests that
    ``dark'' effects arise from \textbf{non-projected spectral modes}—configurations
    of $\chi$ that possess inertial mass (resistance to relaxation) but lack
    the specific symmetry required for electromagnetic transmittance.

\subsection{Phenomenology of Galactic Dynamics and Lensing}
\label{subsec:dm-pheno}

Cosmochrony provides a non-particulate explanation for dark matter phenomena by
coupling the effective gravitational coupling to the substrate's relaxation
density $\Phi_\chi$.

\paragraph{Effective Force Law.}
  The departure from the inverse-square law at galactic scales is described by
  a modified stiffness $K(r) = K_0 \cdot \mathcal{F}(\Phi_\chi)$. In low-density
  regimes, the relaxation flux $\Phi_\chi$ drops below a critical threshold
  $\mathcal{K}_c$, inducing a transition to a regime where the potential
  gradient becomes logarithmic, naturally recovering flat rotation curves
  without the need for dark halos.

\paragraph{Gravitational Lensing as Spectral Refraction.}
  The displacement observed in the Bullet Cluster is interpreted as a \textbf{phase
lag} in the substrate's response.
  While baryonic gas dissipates energy through collisions, the geometric deformations of $\chi$ (solitons) maintain
  their momentum.
  Gravitational lensing occurs due to the \textbf{refractive index gradient} of the substrate, which persists along the
  trajectory of the solitons, independent of the slowed-down gas.

\paragraph{Predictive Distinction from WIMPs.}
  Unlike WIMP models, which predict localized particle scattering, Cosmochrony
  predicts a \textbf{non-local correlation} between the mass discrepancy and the
  global spectral age of the system.
  A specific signature of this framework is the absence of small-scale dark matter cusps, as the substrate's elasticity
  imposes a minimum smoothing scale (the ``spectral graininess'' $h_\chi$).

\subsection{Dark Matter: Spectral Refraction and Substrate Memory}
\label{subsec:dm-refraction}

The dark matter phenomenology is reinterpreted as a direct consequence of the
non-linear elastic response of the \(\chi\) substrate.

\paragraph{Variable Threshold $\mathcal{K}_c$.}
  The observed flat rotation curves arise when the relaxation flux $\Phi_\chi$
  drops below the saturation threshold $\mathcal{K}_c$.
  Unlike the universal constant $a_0$ in MOND, $\mathcal{K}_c$ is a local property of the substrate's spectral density.
  This explains why the ``dark matter'' fraction appears to
  vary between galaxies of different spectral ages or environments, as the
  substrate's stiffness is a dynamical state, not a fixed law.

\paragraph{Gravitational Lensing as Metric Refraction.}
  The displacement in the Bullet Cluster provides evidence for the \textbf{phase
lag} of the projection $\Pi$.
  Light deflection is treated as a refraction process within the spectral gradient of the \(\chi\) field.
  In high-energy collisions, the dissipative baryonic component (gas) decouples
  from the primary solitons (mass peaks).
  The lensing signal tracks the \textbf{residual geometric deformation} of the substrate, effectively
  measuring the ``wake'' left by the passing mass-solitons in the \(\chi\) medium.

\paragraph{Comparison and Predictions.}
  Cosmochrony predicts that dark matter ``halos'' should exhibit \textbf{spectral
echoes}—faint gravitational signatures in regions where matter was
  previously present but has since moved, a phenomenon fundamentally
  incompatible with particulate WIMP models but inherent to a substrate
  with finite relaxation time.
