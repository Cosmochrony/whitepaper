\subsection{The Hubble Tension}
  \label{subsec:hubble-tension}

  The discrepancy between early-universe and late-universe determinations of the
  Hubble constant is now well established observationally~\cite{Riess2019,Riess2022,Planck2020,HubbleTensionReview}.
  Standard cosmological models interpret this tension as a potential indication of
  new physics beyond the $\Lambda$CDM framework.

  Within the Cosmochrony framework, this discrepancy admits a natural qualitative
  interpretation without introducing new fundamental components or modifying the
  underlying relaxation dynamics.
  The key point is that different observational probes access different regimes of
  effective projectability of the relational $\chi$ substrate.

  Early-universe measurements, such as those inferred from the cosmic microwave
  background, probe a regime close to the transition from maximal constraint to
  geometric projectability.
  In this regime, relational constraints remain significant, and the effective
  mapping between $\chi$ relaxation and spacetime observables differs from that
  characterizing the late universe.

  Late-time measurements, based on local distance ladders and astrophysical standard
  candles, probe a regime in which $\chi$ has undergone substantial further
  relaxation.
  In this more weakly constrained regime, effective spacetime descriptions are more
  fully developed, leading to a different inferred relation between relaxation
  ordering and observational distance–redshift relations.

  The resulting difference in inferred values of $H_0$ does not reflect a change in
  a fundamental expansion rate.
  It arises from the use of a single spacetime-based parametrization to describe
  observations sampling distinct stages of relational relaxation.
  In this sense, the Hubble tension reflects a limitation of homogeneous effective
  descriptions when applied across regimes of differing projectability.

  The present discussion is intended as a qualitative explanation rather than a
  precision prediction.
  A more detailed analysis, outlining how different observables map onto the
  relaxation history of $\chi$, is provided in
  Appendix~\ref{app:hubble_tension}.
