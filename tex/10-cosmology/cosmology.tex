\section{Cosmological Implications}
  \label{sec:cosmology}

  \subsection{The Big Bang as a Maximal Constraint Regime of the $\chi$ Field}
    \label{subsec:big-bang-maximal-constraint}

    In the Cosmochrony framework, the Big Bang is not interpreted as a spacetime
    singularity, but as an initial regime in which the $\chi$ field was maximally
    constrained.
    Rather than diverging curvature or density, this regime is characterized by
    extreme structural variations and topological constraints within $\chi$.

    At this stage, no stable geometric interpretation exists.
    Concepts such as distance, duration, or curvature are not yet meaningful.
    The subsequent cosmological evolution corresponds to the progressive relaxation
    of these constraints, through which effective spacetime notions gradually emerge.

    The arrow of time originates directly from this relaxation process.
    Temporal ordering is not imposed by boundary conditions, but arises intrinsically
    from the monotonic evolution of $\chi$ away from its maximally constrained state.

  \subsection{Cosmological Cycles of Constraint and Reprojection}
    \label{subsec:cosmic-reprojection-cycle}

    The maximally constrained initial regime identified with the Big Bang should not
    be interpreted as a unique or irreproducible event.
    Rather, it represents a limiting configuration of the $\chi$ field in which
    structural constraints temporarily dominate over relaxation.

    As the universe evolves, the global relaxation of $\chi$ gives rise to emergent
    spacetime, matter excitations, and large-scale structure.
    However, the same structural bound may be locally reapproached at later epochs,
    most notably in regions of extreme gravitational confinement identified as black
    holes.

    In such regions, the local saturation of the structural bound $c_{\chi}$ leads to
    a breakdown of spatiotemporal descriptions, analogous to the pre-geometric regime
    of the early universe.
    Information encoded in emergent relational degrees of freedom ceases to remain
    expressible within spacetime and is reduced to a purely structural form within the
    $\chi$ substrate.

    Due to intrinsic fluctuations of $\chi$, structurally encoded information remains
    reprojectable.
    Reprojection occurs in discrete units governed by $\hbar_{\chi}$ and manifests in
    emergent spacetime as transient or stable excitations commonly associated with the
    quantum vacuum.
    In this sense, the present-day vacuum reflects the ongoing reprojection of
    structural information originating both from the primordial constrained phase and
    from later localized saturation regimes.

    Cosmological evolution in Cosmochrony therefore involves not a single transition
    from an initial state, but a continuous interplay between relaxation, local
    reconfinement, deprojection, and reprojection across scales.

  \subsection{Cosmic Expansion Without Inflation}
    \label{subsec:expansion-without-inflation}

    Standard cosmology invokes an inflationary phase to account for large-scale
    homogeneity, isotropy, and spatial flatness.
    In Cosmochrony, these features arise naturally from the pre-geometric nature of
    the initial $\chi$ configuration.

    Because the early $\chi$ field was globally connected prior to geometric
    differentiation, no horizon problem arises.
    Homogeneity reflects the continuity of the underlying field rather than the
    outcome of rapid spacetime expansion.

    This framework does not presently provide a detailed quantitative substitute for
    inflation at the level of perturbation spectra.
    However, it offers a conceptually economical explanation for large-scale cosmic
    regularities without introducing additional dynamical fields or phases.

  \subsection{Cosmic Expansion as $\chi$ Relaxation}
    \label{subsec:expansion-as-relaxation}

    Cosmic expansion in Cosmochrony does not correspond to the motion of matter
    through space, but to the progressive relaxation of the $\chi$ field itself.
    As $\chi$ increases monotonically, effective spatial separations emerge and
    grow.

    In this interpretation, expansion is not driven by an external energy component
    but reflects the internal geometric unfolding of the field.
    Matter excitations act as local constraints on this relaxation, leading to
    inhomogeneities that later manifest as large-scale structure.

    This perspective reinterprets cosmological expansion as a purely geometric and
    dynamical process intrinsic to the $\chi$ field.

  \subsection{Emergent Hubble Law}
    \label{subsec:emergent-hubble-law}

    In homogeneous regimes, the relaxation of $\chi$ is uniform, yielding a linear
    relation
    \begin{equation}
      \chi(t) = \chi_0 + c t .
    \end{equation}

    Identifying effective spatial scales with accumulated $\chi$ increments leads
    naturally to a Hubble-like law.
    The Hubble parameter emerges as
    \begin{equation}
      H(t) = \frac{\dot{\chi}}{\chi},
    \end{equation}
    providing a natural cosmological scale without introducing a fundamental scale
    factor.

    The present-day Hubble constant $H_0$ is therefore interpreted as a measure of
    the current global relaxation rate of the $\chi$ field.

  \subsection{Cosmic Acceleration Without Dark Energy}
    \label{subsec:acceleration-without-dark-energy}

    Observed late-time cosmic acceleration does not require a cosmological constant
    or dark energy component in this framework.
    Instead, it arises as an apparent effect produced by the cumulative relaxation
    history of $\chi$.

    As localized structures form, they increasingly constrain local relaxation,
    modifying the global effective expansion rate.
    This leads to an apparent acceleration when interpreted within standard
    spacetime-based cosmological models.

  \subsection{Cosmic Microwave Background}
    \label{subsec:cmb}

    The cosmic microwave background reflects the imprint of early $\chi$ relaxation
    dynamics rather than fluctuations generated during an inflationary phase.
    Large-scale correlations arise from the continuity of the $\chi$ field prior to
    geometric differentiation.

    Acoustic features in the temperature power spectrum can be interpreted as
    arising from oscillatory coupling between $\chi$ and matter excitations during
    the transition to a stable geometric regime.

    No superluminal expansion within spacetime is required to account for observed
    large-angle correlations.

  \subsection{The Hubble Tension}
    \label{subsec:hubble-tension}

    The discrepancy between early-universe and late-universe determinations of the
    Hubble constant finds a natural qualitative explanation within Cosmochrony.
    Different observational methods probe $\chi$ at different stages of its
    relaxation.

    Early-universe measurements are sensitive to a more constrained configuration
    of $\chi$, while late-time measurements reflect a more relaxed state.
    This naturally leads to distinct effective values of $H_0$.

    These estimates should be understood as order-of-magnitude consistency rather
    than precision predictions.
    A more detailed analysis is presented in Appendix~\ref{app:hubble_tension}.

  \subsection{Entropy and the Arrow of Time}
    \label{subsec:entropy-arrow}

    In Cosmochrony, the arrow of time is fundamental and precedes thermodynamic
    considerations.
    Entropy increase emerges as a secondary, statistical description of the
    irreversible relaxation of $\chi$.

    This reverses the standard explanatory order:
    time asymmetry does not arise from entropy growth; rather, entropy growth reflects
    the underlying temporal directionality imposed by $\chi$ relaxation.

    It is important to note that entropy and the arrow of time are defined only within
    the emergent spacetime regime.
    Processes that involve deprojection of information into the $\chi$ substrate, such
    as those associated with extreme gravitational confinement, do not correspond to
    entropy decrease or temporal reversal, but rather to a change in the level of
    physical description.

  \subsection{Large-Angle Temperature Anomalies}
    \label{subsec:large-angle-anomalies}

    Large-angle anomalies observed in the CMB, such as low-$\ell$ power suppression,
    may reflect residual correlations inherited from the pre-geometric phase of the
    $\chi$ field.

    These features are not predicted as precise signatures but arise naturally as
    qualitative consequences of incomplete relaxation at the largest scales.
    Cosmic variance limits the statistical significance of such effects.

  \subsection{Summary}
    \label{subsec:summary9}

    Cosmological phenomena in Cosmochrony emerge from the global relaxation dynamics
    of the $\chi$ field.
    Expansion, acceleration, large-scale homogeneity, and the arrow of time arise
    without invoking inflation, dark energy, or initial spacetime singularities.

    The framework reproduces the phenomenological predictions of standard cosmology
    at large scales, while offering an alternative geometric interpretation rooted
    in a single pre-geometric scalar dynamics.
