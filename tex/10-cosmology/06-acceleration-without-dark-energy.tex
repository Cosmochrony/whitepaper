\subsection{Cosmic Acceleration Without Dark Energy}
  \label{subsec:acceleration-without-dark-energy}

  Within the Cosmochrony framework, the observed late-time cosmic acceleration does not
  require the introduction of a cosmological constant or a dark energy component.
  No additional energy density or repulsive interaction is postulated at the
  fundamental level.

  The apparent acceleration arises as an effective consequence of the cumulative
  relaxation history of the relational $\chi$ substrate.
  As cosmic evolution proceeds, the formation of localized and long-lived structures
  (such as galaxies and clusters) increasingly constrains the local relaxation of
  $\chi$.
  These constraints introduce growing spatial inhomogeneities in the relaxation
  process.

  When interpreted within standard spacetime-based cosmological models, which assume
  a homogeneous and isotropic expansion driven by a global scale factor, these
  inhomogeneities manifest as an apparent acceleration of cosmic expansion.
  The effect reflects a mismatch between the underlying relational relaxation dynamics
  and the assumptions built into effective geometric descriptions.

  In this sense, cosmic acceleration is not a dynamical phenomenon requiring a new
  source of energy.
  It is an emergent, interpretative effect arising from the progressively uneven
  relaxation of $\chi$ across cosmic scales.
  As structure formation proceeds, the effective expansion inferred from observations
  naturally departs from the predictions of homogeneous models, without invoking dark
  energy.

  This interpretation aligns with approaches that attribute late-time acceleration to
  backreaction effects, while providing a unified and pre-geometric origin rooted in
  the relaxation dynamics of the $\chi$ substrate.
