\subsection{Cosmological Cycles of Constraint and Reprojection}
  \label{subsec:cosmic-reprojection-cycle}

  The maximally constrained regime identified with the Big Bang should not be
  interpreted as a unique or irreproducible event.
  It corresponds instead to a limiting configuration of the $\chi$ substrate in which
  structural constraints dominate to the extent that no stable spacetime projection
  is admissible.

  As global relaxation proceeds, increasingly stable projected descriptions become
  possible, giving rise to emergent spacetime, matter configurations, and large-scale
  cosmological structure.
  However, this maximal constraint regime is not confined to the early universe.
  It may be locally reapproached whenever structural constraints on $\chi$ saturate,
  most notably in regions of extreme gravitational confinement identified as black
  holes.

  In such regions, effective spacetime descriptions progressively lose validity.
  Relational information encoded in emergent degrees of freedom ceases to remain
  expressible within spacetime and undergoes deprojection into the purely relational
  $\chi$ substrate.
  This process does not destroy information but renders it inaccessible to spacetime
  descriptions.

  Crucially, deprojection does not imply irreversibility at the level of the substrate.
  As structural constraints relax, the same relational content may again admit
  admissible projected descriptions.
  Reprojection should therefore be understood not as a discrete physical process,
  but as the restoration of descriptive projectability once relational consistency
  conditions are satisfied.

  From this perspective, phenomena commonly associated with the quantum vacuum reflect
  the persistent presence of reprojectable relational structures within $\chi$, rather
  than the activity of fluctuating fields.
  The vacuum is not empty, but represents a regime of minimal yet non-vanishing
  projectability.

  Cosmological evolution in Cosmochrony thus involves a continuous interplay between
  global relaxation, local reconfinement, deprojection, and reprojection across scales.
  The universe is not characterized by a single origin or terminal state, but by
  recurrent transitions between regimes of descriptive admissibility and breakdown.
