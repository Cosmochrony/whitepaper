\subsection{Particles as Stable Wave Configurations}
  \label{subsec:particles-as-stable-wave-configurations}

  Within the cosmochrony framework, particles are not fundamental point-like objects but stable, localized
  excitations of the $\chi$ field\cite{Rajaraman1982}.
  They correspond to persistent wave configurations that locally constrain the relaxation of $\chi$.

  These configurations may be interpreted as soliton-like structures: they maintain their identity during
  propagation and interaction, while remaining fully embedded in the underlying $\chi$ dynamics.

\subsection{Topological Stability}
  \label{subsec:topological-stability}

  The stability of particle-like excitations is attributed to topological constraints rather than to conserved
  charges postulated a priori\cite{rajaraman1982solitons}.
  Nontrivial phase winding, torsion, or knot-like structures in $\chi$
  prevent continuous deformation into the vacuum state.

  Such topological protection naturally explains the discreteness of particle species and their robustness under
  perturbations.

  For instance, an electron corresponds to a localized knot in $\chi$
  with a specific winding number, where the knot's energy (proportional to its curvature) determines the
  particle's mass, and its topological charge (e.g., $4\pi$-periodicity) determines its spin-1/2 nature.

  The stability of solitonic excitations arises from a balance between the nonlinear self-interaction term $V(\chi)$
  (which tends to localize the field) and the gradient energy $|\nabla \chi|^2$
  (which tends to disperse it). Topological invariants, such as the winding number
  $n = \frac{1}{2\pi} \oint \nabla \arg(\chi) \cdot d\mathbf{l}$
  , further protect these configurations from decay, ensuring their persistence as particle-like objects.

  More topological configurations are discussed in ~\ref{subsec:topological_solitons}

\subsection{Mass as Resistance to $\chi$ Relaxation}
  \label{subsec:mass-as-resistance-to-chi-relaxation}

  In this framework, mass is not an intrinsic attribute but an emergent measure of how strongly a localized
  excitation resists the local increase of $\chi$.

  Regions containing particle excitations exhibit increased spatial gradients:
  \begin{equation}
    |\nabla \chi| > 0 ,
  \end{equation}
  which, according to Eq.~\eqref{eq:chi_dynamics}, reduces the local relaxation rate $\partial_t \chi$.

  This reduction manifests macroscopically as time dilation and gravitational mass.

  Mapping of the energy of solitons with the masses of observed particles is detailled in ~\ref{subsec:soliton_energy_mass}.

\subsection{Energy--Frequency Relation}
  \label{subsec:energy-frequency-solitons}

  The energy associated with a particle excitation is linked to the internal oscillation frequency of its wave
  configuration.
  Higher-frequency structures correspond to tighter localization and stronger gradients in $\chi$.

  This provides a geometric interpretation of the relation
  \begin{equation}
    E \propto \nu ,
  \end{equation}
  with Planck's constant emerging as an effective proportionality factor determined by the properties of the $\chi$
  field rather than assumed as a fundamental constant.

  A more explicit geometric derivation of this relation, in the context of radiation and photon-like
  excitations of the $\chi$ field, will be presented in Section~\ref{subsec:energy-frequency-radiation}.

\subsection{Fermions and Bosons}\label{subsec:fermions-and-bosons}

  Particle statistics arise from the topology of the underlying excitation.
  Configurations requiring a $4\pi$
  phase rotation to return to their original state correspond to fermions, while integer-winding configurations
  correspond to bosons.

  This topological distinction naturally reproduces the spin-statistics connection without invoking additional
  quantum postulates.

  Consider a localized soliton solution $\chi(\mathbf{x}) = \chi_0 \tanh(r/\xi)$, where $r$
  is the radial coordinate and $\xi$ sets the soliton size. For fermionic excitations, the phase of $\chi$
  must wind by $4\pi$ to return to its original value, reflecting a M\"
  obius-like twist in the field configuration. This topological constraint enforces the spin-statistics theorem:
  only configurations with half-integer winding numbers (fermions) can exhibit such $4\pi$
  -periodicity, while integer windings (bosons) correspond to $2\pi$-periodic solutions.

\subsection{Antiparticles}\label{subsec:antiparticles}

  Antiparticles are interpreted as phase-inverted counterparts of particle excitations.
  Their annihilation corresponds to topological unwinding, releasing stored curvature energy back into propagating
  $\chi$ waves.

  This process conserves total $\chi$-structure while converting localized constraints into delocalized radiation.

\subsection{Particle Creation and Destruction}\label{subsec:particle-creation-and-destruction}

  Particles are created when propagating $\chi$
  waves self-interfere or interact with existing excitations strongly enough to form stable localized
  configurations.
  Conversely, particle destruction corresponds to the loss of topological stability through interaction or
  decoherence.

  This view removes the need for particle ontology as a primitive concept and replaces it with a purely
  dynamical description.

\subsection{Summary}\label{subsec:summary7}

  Particles emerge as stable, localized excitations of the $\chi$ field that resist its relaxation.
  Their mass, energy, spin, and statistics follow from geometric and topological properties of $\chi$
  , providing a unified description compatible with both relativistic and quantum phenomena.
