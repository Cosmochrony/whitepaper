\subsection{Gravitational Shadows and the Spectral Wake}
  \label{sec:gravitational-shadows}

  The mass-generating torsion $\Omega_w$ is not strictly localized within the projected fiber;
  it induces a long-range spectral deformation in the surrounding substrate $\chi$.
  This ``spectral wake'' or \emph{gravitational shadow} represents a zone of reduced relaxation frequency.

  This mechanism provides an ontological basis for "Dark Matter" effects without the need for additional particles.
  The shadow manifests as a spectral persistence of the substrate:
  \begin{itemize}
    \item \textbf{Elastic Remanence:}
    The curvature (shadow) persists in the substrate even if the baryonic node is displaced, explaining the spatial
    offsets observed in galactic collisions such as the Bullet Cluster.
    \item \textbf{Non-Local Susceptibility:} The effective gravitational acceleration $g$
    emerges from the global relaxation flow.
    When the local gradient falls below the threshold $a_0 \approx c H_0$, the substrate's response becomes non-linear,
    recovering the MOND-like regime as a purely elastic phase transition of $\chi$.
  \end{itemize}
