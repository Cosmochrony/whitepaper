\subsection{Strong-Field Constitutive Coupling Near a Schwarzschild Black Hole}
  \label{app:keff_schwarzschild}

  \paragraph{Purpose and status.}
    Section~\ref{subsec:microscopic-origin-coupling} introduced an effective constitutive
    relation for the relaxation conductivity,
    \begin{equation}
      K_{\mathrm{eff}} = K_0 \exp\!\left(-\frac{(\Delta\chi)^2}{\chi_c^2}\right),
      \label{eq:keff_constitutive}
    \end{equation}
    as a coarse-grained way of encoding how strong internal structure of $\chi$ reduces
    the effectiveness of relaxation.
    In weak-field regimes this leads to a Poisson-like description and to the recovery of
    Schwarzschild phenomenology at leading order (Section~\ref{subsec:recovery_schwarzschild}).
    The goal of this appendix is to make explicit a consistent \emph{strong-field}
    profile $K_{\mathrm{eff}}(r)$ in the spherically symmetric case, suitable for describing
    the approach to an effective horizon (Section~\ref{subsec:strong_gravity_black_holes}).

  \paragraph{Operational time-dilation factor.}
    In the emergent geometric regime, gravitational time dilation is encoded by a local
    slowdown of the relaxation rate of $\chi$ relative to its asymptotic value far from the
    source. We define the dimensionless lapse-like factor
    \begin{equation}
      N(r) \;\equiv\; \frac{D_{\mathrm{loc}}\chi(r)}{D_0\chi},
      \qquad 0 < N(r) \le 1,
      \label{eq:lapse_def}
    \end{equation}
    where $D_0\chi$ denotes the asymptotic relaxation rate in a homogeneous background.
    In the weak-field limit, one may write $N \simeq 1+\Phi/c^2$ for an effective Newtonian
    potential $\Phi$ (Section~\ref{subsec:microscopic-origin-coupling}).

  \paragraph{Matching to Schwarzschild form.}
    Section~\ref{subsec:recovery_schwarzschild} argues that, in regimes where a stable
    geometric description applies, the external field of an isolated compact source may be
    summarized by a Schwarzschild-like line element. We adopt the standard form
    \begin{equation}
      ds^2 = -f(r)c^2 dt^2 + f(r)^{-1} dr^2 + r^2 d\Omega^2,
      \qquad f(r)=1-\frac{r_s}{r},
      \label{eq:schwarzschild_standard}
    \end{equation}
    with $r_s = 2GM/c^2$ defined operationally by the asymptotic weak-field matching.
    Consistency with the interpretation of $N(r)$ as the local time-dilation factor implies
    \begin{equation}
      N(r)^2 = f(r) = 1-\frac{r_s}{r}.
      \label{eq:lapse_schwarzschild}
    \end{equation}
    Thus $N(r)\to 0$ as $r\to r_s^+$, capturing the asymptotic freeze-out of local relaxation
    identified with an effective horizon.

  \paragraph{From lapse to strong-field conductivity.}
    To connect $N(r)$ to $K_{\mathrm{eff}}(r)$ we need a strong-field identification between the
    relaxation slowdown and the reduction of conductivity.
    A minimal and self-consistent choice is to assume that the \emph{fractional} slowdown
    is directly controlled by the \emph{fractional} conductivity,
    \begin{equation}
      \frac{K_{\mathrm{eff}}(r)}{K_0} \;\equiv\; N(r)^2,
      \label{eq:keff_lapse_ansatz}
    \end{equation}
    which (i) reproduces the weak-field expansion to leading order,
    (ii) ensures $K_{\mathrm{eff}}\to 0$ at the horizon (no relaxation transport across an
    asymptotically frozen region), and (iii) remains bounded and monotone.

    Combining \eqref{eq:keff_lapse_ansatz} with \eqref{eq:lapse_schwarzschild} yields an explicit
    strong-field profile:
    \begin{equation}
      K_{\mathrm{eff}}(r)
      \;=\;
      K_0\!\left(1-\frac{r_s}{r}\right),
      \qquad r>r_s.
      \label{eq:keff_schwarzschild_profile}
    \end{equation}
    This expression should be read as an \emph{effective constitutive law} in the emergent
    geometric regime; it does not assert that $K_{\mathrm{eff}}$ is fundamental.

  \paragraph{Implied structural variation $\Delta\chi(r)$.}
    Using the constitutive relation \eqref{eq:keff_constitutive} together with
    \eqref{eq:keff_lapse_ansatz}, one obtains the corresponding strong-field variation measure:
    \begin{equation}
      \frac{(\Delta\chi(r))^2}{\chi_c^2}
      \;=\;
      -\ln\!\left(\frac{K_{\mathrm{eff}}(r)}{K_0}\right)
      \;=\;
      -\ln\!\left(1-\frac{r_s}{r}\right),
      \label{eq:deltachi_schwarzschild}
    \end{equation}
    so that $\Delta\chi(r)$ diverges logarithmically as $r\to r_s^+$,
    \begin{equation}
      \Delta\chi(r) \sim \chi_c \sqrt{-\ln\!\left(1-\frac{r_s}{r}\right)}.
      \label{eq:deltachi_horizon_asymptotic}
    \end{equation}
    This divergence should not be interpreted as a spacetime singularity.
    It reflects that the coarse-grained structural measure $\Delta\chi$ ceases to remain small and that the
    effective geometric parametrization is pushed to its limit of validity near the horizon.

  \paragraph{Interpretation: horizons as vanishing relaxation conductivity.}
    Equations \eqref{eq:keff_schwarzschild_profile}--\eqref{eq:deltachi_schwarzschild} provide a
    compact strong-field completion of the weak-field Poisson description:
    a Schwarzschild horizon corresponds to a \emph{conductivity zero} of the relaxation flow,
    $K_{\mathrm{eff}}\to 0$, rather than to a fundamental geometric singularity.
    In this sense, black holes are regions where the collective structural constraints in $\chi$
    asymptotically inhibit relaxation, producing the temporal and gravitational shadows discussed
    in Section~\ref{subsec:strong-gravity-and-black-holes}.

\subsubsection{Scope and Open Questions}

  The effective description provided by $S[\chi,\rho]$ raises several open issues:
  \begin{itemize}
    \item the microscopic origin of the coupling constant $\alpha$,
    \item the role of $S[\chi,\rho]$ in quantum regimes where $\rho$ is replaced by
    excitation amplitudes,
    \item possible observational signatures arising from nonlinear relaxation
    effects.
  \end{itemize}

  These questions are deferred to future work.
  The present formulation is intended as a bridge between the fundamental
  relaxation dynamics of $\chi$ and the effective gravitational and cosmological
  phenomena observed at macroscopic scales.

\subsubsection{Conclusion}

  The effective source term $S[\chi,\rho]$ provides a consistent and economical way
  to encode the influence of localized excitations on the relaxation of the $\chi$
  field.
  While not fundamental, it allows Cosmochrony to recover known gravitational
  phenomena and to articulate clear predictions in regimes where deviations from
  standard theories may arise.
