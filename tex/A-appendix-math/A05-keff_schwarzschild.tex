\subsection{Strong-Field Constitutive Coupling Near a Schwarzschild Black Hole}
  \label{app:keff_schwarzschild}

  \paragraph{Purpose and epistemic status.}
    In Section~\ref{subsec:microscopic-origin-of-the-coupling-tensor-and-the-poisson-equation},
    an effective constitutive relation was introduced to encode how strong internal
    structure of the $\chi$ field reduces the efficiency of relaxation:
    \begin{equation}
      K_{\mathrm{eff}}
      =
      K_0 \exp\!\left(-\frac{(\Delta\chi)^2}{\chi_c^2}\right).
      \label{eq:keff_constitutive}
    \end{equation}
    This relation is not fundamental.
    It provides a coarse-grained parametrization of how collective structural
    constraints within $\chi$ suppress relaxation transport in regimes where an
    emergent geometric description becomes applicable.

    In weak-field situations, this description leads to a Poisson-like equation and
    to the recovery of Schwarzschild phenomenology at leading order
    (Section~\ref{subsec:recovery-of-the-schwarzschild-metric}).
    The purpose of the present appendix is to construct a \emph{consistent strong-field
completion} of this picture in the spherically symmetric case, suitable for
    describing the approach to an effective horizon without introducing geometric
    singularities.

  \paragraph{Operational time-dilation factor.}
    In the emergent spacetime regime, gravitational time dilation is operationally
    encoded as a local slowdown of the relaxation rate of $\chi$ relative to its
    asymptotic homogeneous value.
    We define the dimensionless lapse-like factor
    \begin{equation}
      N(r)
      \;\equiv\;
      \frac{\mathcal{D}_{\mathrm{loc}}\chi(r)}{\mathcal{D}_0\chi},
      \qquad
      0 < N(r) \le 1 ,
      \label{eq:lapse_def}
    \end{equation}
    where $\mathcal{D}_0\chi$ denotes the relaxation rate far from localized
    excitations.
    In the weak-field limit, this quantity reduces to
    \(N \simeq 1 + \Phi/c^2\),
    with $\Phi$ an effective Newtonian potential.

  \paragraph{Matching to Schwarzschild phenomenology.}
    In regimes where a stable geometric description applies, the exterior field of an
    isolated compact excitation may be summarized by a Schwarzschild-like metric.
    We therefore adopt, purely as an \emph{effective descriptor},
    \begin{equation}
      ds^2
      =
      -f(r)c^2 dt^2
      + f(r)^{-1} dr^2
      + r^2 d\Omega^2,
      \qquad
      f(r) = 1 - \frac{r_s}{r},
      \label{eq:schwarzschild_standard}
    \end{equation}
    where $r_s = 2GM/c^2$ is defined operationally by asymptotic weak-field matching.

    Consistency with the interpretation of $N(r)$ as the local time-dilation factor
    then implies
    \begin{equation}
      N(r)^2 = f(r) = 1 - \frac{r_s}{r}.
      \label{eq:lapse_schwarzschild}
    \end{equation}
    The limit $N(r)\to 0$ as $r\to r_s^+$ corresponds to an asymptotic freeze-out of
    local relaxation and defines an \emph{effective horizon}.

  \paragraph{From relaxation slowdown to conductivity.}
    To relate the relaxation slowdown to the constitutive conductivity, we adopt a
    minimal and self-consistent identification:
    \begin{equation}
      \frac{K_{\mathrm{eff}}(r)}{K_0}
      \;\equiv\;
      N(r)^2 .
      \label{eq:keff_lapse_ansatz}
    \end{equation}
    This choice is not postulated as fundamental.
    It is selected because it:
    (i) reproduces the weak-field expansion,
    (ii) ensures $K_{\mathrm{eff}}\to 0$ at the horizon, preventing relaxation transport
    across an asymptotically frozen region,
    and (iii) remains monotone and bounded.

    Combining \eqref{eq:keff_lapse_ansatz} with
    \eqref{eq:lapse_schwarzschild} yields the explicit strong-field profile
    \begin{equation}
      K_{\mathrm{eff}}(r)
      =
      K_0\!\left(1 - \frac{r_s}{r}\right),
      \qquad
      r > r_s .
      \label{eq:keff_schwarzschild_profile}
    \end{equation}
    This expression should be read as an \emph{effective constitutive law} valid only
    within the emergent geometric regime.

  \paragraph{Implied structural variation of $\chi$.}
    Inverting the constitutive relation
    \eqref{eq:keff_constitutive} using
    \eqref{eq:keff_schwarzschild_profile} gives
    \begin{equation}
      \frac{(\Delta\chi(r))^2}{\chi_c^2}
      =
      -\ln\!\left(1 - \frac{r_s}{r}\right).
      \label{eq:deltachi_schwarzschild}
    \end{equation}
    As $r \to r_s^+$, the structural variation measure diverges logarithmically:
    \begin{equation}
      \Delta\chi(r)
      \sim
      \chi_c
      \sqrt{-\ln\!\left(1 - \frac{r_s}{r}\right)} .
      \label{eq:deltachi_horizon_asymptotic}
    \end{equation}

    This divergence must not be interpreted as a physical singularity.
    It indicates that the coarse-grained structural measure $\Delta\chi$ ceases to
    remain small and that the effective geometric parametrization is pushed beyond
    its domain of validity near the horizon.

  \paragraph{Interpretation: horizons as conductivity zeros.}
    Equations
    \eqref{eq:keff_schwarzschild_profile}--\eqref{eq:deltachi_schwarzschild}
    provide a coherent strong-field completion of the weak-field Poisson description.
    In Cosmochrony, a Schwarzschild horizon corresponds to a \emph{vanishing relaxation
conductivity},
    \[
      K_{\mathrm{eff}} \to 0,
    \]
    rather than to a fundamental spacetime singularity.
    Black holes are therefore interpreted as regions where collective structural
    constraints within $\chi$ asymptotically inhibit relaxation, producing the
    effective causal and temporal horizons discussed in the main text.

  \paragraph{Scope and open issues.}
    The constitutive description developed here raises several open questions,
    including:
    \begin{itemize}
      \item the microscopic origin of the effective conductivity scale $K_0$,
      \item the extension of the constitutive relation to quantum regimes where
      $\rho$ is replaced by excitation amplitudes,
      \item and potential observational signatures of nonlinear relaxation effects
      near compact objects.
    \end{itemize}
    These issues are left for future work.

  \paragraph{Conclusion.}
    The strong-field constitutive profile constructed in this appendix provides a
    consistent and non-singular description of black-hole-like regimes within
    Cosmochrony.
    It preserves agreement with Schwarzschild phenomenology while reinterpreting
    horizons as limits of relaxation transport rather than as fundamental geometric
    pathologies.
