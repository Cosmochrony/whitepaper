\subsection{Effective Evolution Equation}
  \label{subsec:effective-evolution-equation}

  Once a stable geometric description has emerged from the underlying relaxation
  dynamics of the $\chi$ field, it becomes possible to express its large-scale
  behavior using effective differential operators familiar from relativistic field
  theory.
  At this phenomenological level only, the evolution of $\chi$ may be summarized by
  an effective equation of the form
  \begin{equation}
    \Box_{\mathrm{eff}} \chi = S[\chi,\rho],
  \end{equation}
  where $\Box_{\mathrm{eff}}$ denotes the d'Alembert operator associated with the
  \emph{emergent} metric, and $\rho$ represents the density of localized excitations
  (matter).

  This equation is not fundamental.
  Both the operator $\Box_{\mathrm{eff}}$ and the source term $S$ arise as
  coarse-grained descriptors of the underlying relaxation process and should be
  understood as bookkeeping devices rather than primary dynamical laws.

  \paragraph{Physical interpretation of the source term.}
    The term $S[\chi,\rho]$ does not represent an external force acting on $\chi$.
    Instead, it encodes the effective resistance of localized excitations to the
    global relaxation of the field.
    Regions containing matter correspond to structured configurations of $\chi$ that
    locally reduce the relaxation rate, giving rise to spatial gradients and
    differential proper-time flow.

    In weak-field regimes, where matter-induced gradients are small, this effective
    resistance may be approximated as linear in the excitation density:
    \begin{equation}
      S[\chi,\rho] \simeq -\alpha \rho ,
    \end{equation}
    with $\alpha$ an effective coupling constant.
    Matching with the Newtonian limit identifies $\alpha \sim G/c^2$, where $G$
    emerges as the macroscopic coupling between matter density and relaxation
    slowdown.

    Beyond this regime, nonlinear corrections to $S[\chi,\rho]$ are expected to
    reflect saturation effects in the relaxation dynamics.
    Such corrections prevent unphysical halting of the field evolution and encode
    departures from classical gravitational behavior without modifying the
    fundamental ontological structure of Cosmochrony.
