\subsection{Effective Evolution Equation}
  \label{subsec:effective-evolution-equation}

  \paragraph{Scope and ontological status.}
    This subsection does \emph{not} introduce a fundamental dynamical law for the
    pre-geometric $\chi$ substrate.
    At the level of $\chi$ itself, no spacetime, metric structure, differential operator,
    or evolution equation exists.
    The relations presented below belong \emph{exclusively} to the effective,
    post-projective description, and provide a compact rewriting of admissible
    $\chi$-orderings once a stable geometric regime has emerged.
    All differential operators, source terms, and variational structures introduced here
    should therefore be understood as descriptive tools, not as generators of the
    microscopic dynamics of $\chi$.

    Once a stable geometric description has emerged from the underlying relaxation
    structure of the $\chi$ substrate, it becomes possible to summarize large-scale
    regularities of admissible projected configurations using differential operators
    familiar from relativistic field theory.
    This step introduces no new fundamental content: it merely provides a convenient
    phenomenological language for regimes in which spacetime notions are
    operationally meaningful.

    At this effective level only, the ordering of projected $\chi$ configurations may be
    written in the schematic form
    \begin{equation}
      \Box_{\mathrm{eff}} \chi = S[\chi,\rho],
    \end{equation}
    where $\Box_{\mathrm{eff}}$ denotes the d'Alembert operator associated with the
    \emph{emergent} metric, and $\rho$ represents the effective density of localized,
    relaxation-resistant configurations.
    Neither the operator nor the source term is fundamental: both arise through
    coarse-graining and projection of the underlying relational relaxation structure.

    This equation must therefore be read as an effective encoding of admissible
    $\chi$-orderings once a spacetime description becomes applicable.
    It does not govern the pre-geometric evolution of the $\chi$ substrate, but
    summarizes how large-scale variations of projected descriptions respond to the
    presence of structured configurations that inhibit relaxation.

  \paragraph{Physical meaning of the source term.}
    The term $S[\chi,\rho]$ does not represent an external force acting on $\chi$.
    Instead, it compactly encodes the effective resistance imposed by localized
    configurations on the global relaxation ordering.
    Matter corresponds to persistent, structured patterns of $\chi$ whose internal
    organization locally reduces the admissible relaxation rate, inducing gradients
    in the effective description.

    Within an emergent geometric language, these gradients are reinterpreted as
    gravitational time dilation and spacetime curvature.
    The source term therefore summarizes, in a single effective expression, several
    related manifestations of inhibited relaxation: inertial resistance, local slowdown
    of ordering, and the appearance of effective gravitational potentials.

  \paragraph{Weak-field approximation.}
    In regimes where matter-induced gradients remain small and relaxation stays close
    to homogeneous, the source term may be approximated as linear in the excitation
    density,
    \begin{equation}
      S[\chi,\rho] \simeq -\alpha \rho ,
    \end{equation}
    where $\alpha$ is an emergent effective coupling constant.
    Matching this expression to the Newtonian limit identifies
    $\alpha \sim G/c^2$, with the gravitational constant $G$ appearing as a macroscopic
    parameter characterizing the sensitivity of effective relaxation ordering to localized
    structural constraints.

    In this approximation, the effective evolution equation reproduces the Poisson
    equation for the gravitational potential and yields Schwarzschild-like solutions
    in spherically symmetric configurations.
    These results reflect the weak-structure limit of constrained relaxation ordering,
    not the presence of an independent gravitational interaction.

  \paragraph{Beyond the weak-field regime.}
    In regions of high excitation density or strong confinement—such as near compact
    objects or during early cosmological phases—nonlinear corrections to
    $S[\chi,\rho]$ are expected.
    These corrections encode saturation effects imposed by the universal kinematic
    constraint $\partial_t \chi \leq c$, inherited from the invariant structural bound
    defined at the level of the $\chi$ substrate.

    Such nonlinearities prevent unphysical halting or divergence of the effective
    description and signal the breakdown of the geometric approximation rather than
    a failure of the underlying relaxation structure of $\chi$.
    They mark the limits of validity of the effective spacetime language, while leaving
    the pre-geometric ontological simplicity of Cosmochrony intact.
