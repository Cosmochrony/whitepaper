\subsection{Effective Evolution Equation}
  \label{subsec:effective-evolution-equation}

  Once a stable geometric description has emerged from the underlying relaxation
  dynamics of the $\chi$ field, it becomes possible to summarize its large-scale
  behavior using differential operators familiar from relativistic field theory.
  This step does not introduce new fundamental dynamics, but provides a convenient
  phenomenological language for describing regimes in which spacetime notions are
  operationally meaningful.

  At this effective level only, the evolution of $\chi$ may be written in the form
  \begin{equation}
    \Box_{\mathrm{eff}} \chi = S[\chi,\rho],
  \end{equation}
  where $\Box_{\mathrm{eff}}$ denotes the d'Alembert operator associated with the
  \emph{emergent} metric, and $\rho$ represents the density of localized excitations
  (matter).
  Neither the operator nor the source term is fundamental: both arise through
  coarse-graining of the underlying relational relaxation dynamics.

  This equation should therefore be understood as an effective rewriting of the
  $\chi$ dynamics once a spacetime description has become applicable.
  It does not govern the microscopic evolution of the field, but summarizes how
  large-scale variations of $\chi$ respond to the presence of structured,
  relaxation-resisting configurations.

  \paragraph{Physical meaning of the source term.}
    The term $S[\chi,\rho]$ does not represent an external force acting on $\chi$.
    Instead, it encodes the effective resistance imposed by localized excitations on
    the global relaxation flow of the field.
    Matter corresponds to persistent, structured configurations of $\chi$ that
    locally reduce the admissible relaxation rate, inducing spatial gradients and
    differential effective time flow.

    Within an emergent geometric description, these gradients are naturally
    reinterpreted as gravitational time dilation and spacetime curvature.
    The source term thus compactly summarizes several related effects:
    the inertial resistance of matter, the slowing of local relaxation, and the
    emergence of effective gravitational potentials.

  \paragraph{Weak-field approximation.}
    In regimes where matter-induced gradients are small and relaxation remains close
    to homogeneous, the source term may be approximated as linear in the excitation
    density:
    \begin{equation}
      S[\chi,\rho] \simeq -\alpha \rho ,
    \end{equation}
    where $\alpha$ is an effective coupling constant.
    Matching this expression with the Newtonian limit identifies
    $\alpha \sim G/c^2$, with the gravitational constant $G$ emerging as a macroscopic
    parameter characterizing the sensitivity of the relaxation rate to localized
    excitation density.

    In this approximation, the effective evolution equation reproduces the Poisson
    equation for the gravitational potential and yields Schwarzschild-like solutions
    in spherically symmetric configurations.

  \paragraph{Beyond the weak-field regime.}
    In regions of high excitation density or strong confinement, such as near compact
    objects or during early cosmological phases, nonlinear corrections to
    $S[\chi,\rho]$ are expected.
    These corrections reflect saturation effects imposed by the minimal kinematic
    constraint $\partial_t \chi \leq c$ and prevent unphysical halting or divergence
    of the field evolution.

    Such nonlinearities encode departures from classical gravitational behavior while
    preserving the underlying ontological simplicity of Cosmochrony.
    They signal the breakdown of the effective geometric description rather than a
    failure of the fundamental relaxation dynamics of $\chi$.
