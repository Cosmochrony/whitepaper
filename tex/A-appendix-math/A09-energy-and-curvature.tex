\subsection{Energy and Curvature}
  \label{subsec:energy-and-curvature}

  In the Cosmochrony framework, energy is not introduced as a fundamental conserved
  quantity.
  Instead, it emerges as an effective and relational measure of how strongly a given
  configuration of the $\chi$ field resists the global relaxation process.
  Energy is therefore not a primitive substance, but a diagnostic of constrained
  relaxation within an otherwise monotonically evolving substrate.

  Once an effective geometric description becomes applicable, this resistance may be
  summarized by quantities that resemble familiar energy densities.
  At this phenomenological level, it is convenient to introduce the functional
  \begin{equation}
    \mathcal{E}_\chi^{\mathrm{eff}}
    \;=\;
    \frac{1}{2}
    \left[
      (\partial_t \chi)^2 + (\nabla \chi)^2
    \right],
  \end{equation}
  which provides a coarse-grained measure of temporal and spatial deformation of the
  $\chi$ field.
  This expression should be understood strictly as a bookkeeping device defined
  within the emergent geometric regime, and not as a fundamental Hamiltonian density
  governing the underlying dynamics.

  Regions in which $\mathcal{E}_\chi^{\mathrm{eff}}$ is large correspond to
  configurations where $\chi$ exhibits strong internal gradients or reduced local
  relaxation rates.
  Such configurations store a significant amount of relaxation potential and are
  interpreted as localized resistances to the global evolution of the field.
  In the effective spacetime description, these regions are naturally identified
  with particle-like excitations carrying inertial and gravitational properties.

  Within this context, the notion of ``curvature'' associated with $\chi$ should be
  interpreted with care.
  It does not refer to spacetime curvature as a fundamental geometric object, but to
  the internal deformation of the $\chi$ configuration itself.
  Effective spacetime curvature arises only secondarily, as a macroscopic descriptor
  of how such deformations modulate relaxation and correlation propagation across
  extended regions.

  Stable solitonic configurations arise when nonlinear self-interaction effects of
  the $\chi$ field balance the dispersive tendency associated with spatial gradients.
  This balance allows localized resistance to relaxation to persist over extended
  durations, providing a dynamical and geometric origin for long-lived particle
  excitations without invoking fundamental energy conservation laws or independent
  geometric degrees of freedom.
