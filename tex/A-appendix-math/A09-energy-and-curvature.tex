\subsection{Energy and Curvature}
  \label{subsec:energy-and-curvature}

  In Cosmochrony, energy is not introduced as a fundamental conserved quantity.
  Instead, it emerges as an effective measure of the resistance of $\chi$
  configurations to the global relaxation process.
  Once a geometric description becomes applicable, this resistance may be
  summarized by an effective energy density associated with spatial and temporal
  variations of the field.

  At this phenomenological level, one may define a diagnostic functional of the
  form
  \begin{equation}
    \mathcal{E}_\chi^{\mathrm{eff}}
    =
    \frac{1}{2}
    \left[
      (\partial_t \chi)^2 + (\nabla \chi)^2
    \right],
  \end{equation}
  which should be understood as a bookkeeping device rather than a fundamental
  Hamiltonian density.

  Regions where $\mathcal{E}_\chi^{\mathrm{eff}}$ is large correspond to
  configurations with strong internal gradients, in which the relaxation of
  $\chi$ is locally constrained.
  Such regions are interpreted as localized concentrations of relaxation
  potential and are identified with particle-like excitations.

  In this effective description, what may be loosely referred to as
  ``curvature'' of the $\chi$ field does not denote spacetime curvature in a
  fundamental sense, but rather the degree of internal deformation of the field
  configuration.
  Stable solitonic structures arise when nonlinear self-interaction terms balance
  the dispersive tendency of gradients, allowing localized resistance to
  relaxation to persist over extended times.
