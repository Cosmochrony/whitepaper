\subsection{Relational Consistency of the Effective Lagrangian}
  \label{sec:born-lagrangian_derivation}

  The effective Lagrangian for the \(\chi\) field is \textbf{not postulated arbitrarily} but constructed as a \textbf{canonical representation} of the relational dynamics introduced in Section~\ref{sec:dynamical-equation-for-the-chi-field}. This appendix demonstrates how its Born--Infeld-like form emerges from fundamental principles, clarifies its systematic selection, and addresses the continuum limit with full mathematical rigor.

  \subsubsection*{Step 1: Relational Constraint and Bounded Variations}
    \label{subsec:A12-relational-constraint_v171}

    At the fundamental level, the dynamics of \(\chi\) are governed by a \textbf{discrete relational constraint}:
    \begin{equation}
      \mathcal{C}_i[\chi] \equiv \sum_j K_{ij} (\chi_i - \chi_j)^2 \le \chi_c^2,
      \label{eq:A12-relational-constraint_v171}
    \end{equation}
    where \(K_{ij} = K_{ji}\) is a symmetric connectivity matrix and \(\chi_c\) is the correlation scale. This constraint enforces bounded relative variations without assuming pre-existing spacetime, acting as a \textbf{structural causality condition}.

  \subsubsection*{Step 2: Variational Formulation with Global Order}
    \label{subsec:A12-variational-structure_v171}

    The dynamics are described by a constrained action with KKT conditions:
    \begin{equation}
      S[\{\chi_i\}, \{\mu_i\}] = \int d\lambda \left[ \sum_i \frac{1}{2} \left(\frac{d\chi_i}{d\lambda}\right)^2 - U[\{\chi_i\}] - \sum_i \mu_i(\lambda) \left(\mathcal{C}_i[\chi] - \chi_c^2 \right) \right],
      \label{eq:A12-action_v171}
    \end{equation}
    where:
    \begin{itemize}
      \item The kinetic term \(\frac{1}{2}(d\chi_i/d\lambda)^2\) is the \textbf{leading-order expansion} of any smooth functional governing ordered relaxation, with \(U[\{\chi_i\}]\) encoding additional constraints.
      \item The \textbf{global order} is ensured by the functional \(\Xi[\chi(\lambda)] \equiv \sum_i \chi_i(\lambda)\), with \(\frac{d\Xi}{d\lambda} \ge 0\).
      \item KKT conditions guarantee \(\mu_i(\lambda) \ge 0\) and \(\mu_i(\lambda)(\mathcal{C}_i[\chi] - \chi_c^2) = 0\).
    \end{itemize}

  \subsubsection*{Step 3: Continuum Limit and Canonical Form}
    \label{subsec:A12-continuum-limit_v171}

    In \textbf{projectable regimes}, the discrete constraint maps to a continuum bound:
    \begin{equation}
      |\nabla \chi|^2 \le c^2,
      \label{eq:A12-continuum-bound_v171}
    \end{equation}
    where \(\nabla\) is an emergent operator. For a lattice of spacing \(a\), we define:
    \begin{equation}
      (\nabla \chi)^2 \approx \frac{1}{a^2} \sum_{\langle i,j \rangle} (\chi_i - \chi_j)^2,
    \end{equation}
    yielding \(|\nabla \chi|^2 \le c^2\) with \(c^2 \equiv a^2 \chi_c^2 / K_0\). The continuum limit \(a \to 0\) is well-defined if \(K_0 \sim a^{-2}\).

    \paragraph*{Canonical Selection.}
      We seek a functional \(L_{\text{eff}} = f(|\nabla \chi|^2/c^2)\) satisfying:
    \begin{itemize}
      \item Free theory normalization: \(f(0) = -c^2\),
      \item Saturation: \(f(1) = 0\),
      \item Monotonicity: \(f'(x) > 0\) for \(x \in [0,1]\),
      \item Regularity: \(f''(x)\) finite.
      \end{itemize}
      The \textbf{canonical representation} is:
    \begin{equation}
      f(x) = -c^2 \sqrt{1 - x},
      \label{eq:A12-born-infeld-derivation_v171}
      \end{equation}
      yielding the Born--Infeld-like Lagrangian:
    \begin{equation}
      \mathcal{L}_{\text{eff}} = -c^2 \sqrt{1 - \frac{|\nabla \chi|^2}{c^2}} + \partial_t \chi.
      \label{eq:A12-effective-lagrangian_v171}
      \end{equation}

    \paragraph*{Selection Criteria.}
      The Born--Infeld form corresponds to the \textbf{minimal non-polynomial functional} satisfying boundedness, smooth saturation, and finite characteristic speeds. Other choices are admissible but introduce either degeneracies, non-saturating behavior, or additional scales.

  \subsubsection*{Step 4: Role of the Potential \(U[\{\chi_i\}]\)}
    \label{subsec:A12-potential-role_v171}

    The potential \(U[\{\chi_i\}]\) encodes additional relational constraints (e.g., topological terms).
    In the continuum limit:
    \begin{equation}
      U[\{\chi_i\}] \to \int d^3x \, V(\chi),
    \end{equation}
    where \(V(\chi)\) is an effective potential.
    The connection to the main text's \(V(\chi)\) is established via
    coarse-graining, ensuring consistency with solitonic solutions (Section~\ref{subsec:stability-analysis}).

  \subsubsection*{Step 5: Connection to Emergent Geometry}
    \label{subsec:A12-emergent-geometry_v171}

    A coarse-graining procedure \emph{admits} an auxiliary effective Lagrangian representation of the form:
    \begin{equation}
      \mathcal{L}_{\text{eff}} = -c^2 \sqrt{1 - \frac{|\nabla \chi|^2}{c^2}} + \partial_t \chi,
    \end{equation}
    where the linear term \(\partial_t \chi\) does not affect the equations of motion but fixes the orientation of the
    effective evolution parameter.

    The effective metric is defined via the Hessian of \(\mathcal{L}_{\text{eff}}\):
    \begin{equation}
      g_{\mu\nu}^{\text{eff}} \propto \frac{\partial^2 \mathcal{L}_{\text{eff}}}{\partial (\partial_\mu \chi) \partial (\partial_\nu \chi)},
      \label{eq:A12-emergent-metric_v171}
    \end{equation}
    up to conformal rescalings.
    This construction is valid in projectable regimes where \(K_{ij}\) approximates a
    continuum Laplacian (Section~\ref{subsec:numerical-validation-of-the-chi-rightarrow-chi_eff-transition}).

  \subsubsection*{Summary of Key Improvements}
    \begin{itemize}
      \item \textbf{Systematic selection} of the Born--Infeld form from first-principle constraints (boundedness, monotonicity, regularity).
      \item \textbf{Explicit continuum limit} with spectral Laplacian connection.
      \item \textbf{Clarified role of \(U[\{\chi_i\}]\)} and its continuum counterpart.
      \item \textbf{No circularity}: \(\mathcal{L}_{\text{eff}}\) is consistent with (not derived from) relational dynamics.
    \end{itemize}

  \subsubsection*{Scope and Limitations}
    The Born--Infeld-like Lagrangian is a \textbf{canonical representation} valid in projectable regimes.
    Outside these regimes:
    \begin{itemize}
      \item No spacetime description exists,
      \item Alternative functionals may be required,
      \item The discrete dynamics (Eq.~\eqref{eq:A12-relational-constraint_v171}) remain fundamental.
    \end{itemize}

  \subsubsection*{Continuum Limit and Emergence of the Laplace--Beltrami Operator}
    \label{subsec:A12-continuum-limit}

    The discrete relational constraint
    \begin{equation}
      \mathcal{C}_i[\chi] = \sum_j K_{ij}(\chi_i - \chi_j)^2
    \end{equation}
    defines a weighted graph Laplacian acting on the configuration space of $\chi$.
    To establish its continuum limit, we introduce a local volume element $V_i$
    associated with node $i$, and a spectral distance $d_{ij}$ between nodes.

    We assume that the coupling coefficients admit the scaling
    \begin{equation}
      K_{ij} = \frac{1}{V_i}\, w\!\left(\frac{d_{ij}}{\varepsilon}\right),
    \end{equation}
    where $w$ is a symmetric, rapidly decaying kernel and $\varepsilon$ characterizes
    the microscopic relational scale.

    In the dense limit, where the number of nodes $N\to\infty$,
    the typical separation $d_{ij}\to 0$, and the node distribution becomes asymptotically uniform
    with respect to the emergent measure $\sqrt{|g|}\,d^n x$,
    defined by the relational density of the $\chi$ substrate.
    The discrete sum satisfies
    \begin{equation}
      \lim_{\varepsilon\to 0}
      \frac{1}{V_i} \sum_j w\!\left(\frac{d_{ij}}{\varepsilon}\right)
      (\chi_i - \chi_j)^2
      =
      \int_{\mathcal{M}} g^{ab}\,\partial_a \chi\,\partial_b \chi\,
      \sqrt{|g|}\, d^n x .
    \end{equation}

    This convergence follows from standard results on graph Laplacians,
    which guarantee that the weighted Laplacian of a dense graph converges
    to the Laplace--Beltrami operator on an emergent Riemannian manifold
    $\mathcal{M}$ defined by the relational density of nodes.

    Crucially, no background geometry is assumed \emph{a priori}:
    the metric $g_{ab}$ arises as the continuum encoding of the microscopic
    connectivity structure of the $\chi$ substrate.
    The Dirichlet energy functional is therefore not postulated,
    but emerges uniquely as the thermodynamic limit of the discrete
    relational constraint.

  \subsubsection*{Necessity of the Born--Infeld Structure from Causal Saturation}
    \label{subsec:A12-born-infeld-necessity}

    The effective Lagrangian governing the projected dynamics of $\chi$
    cannot be chosen arbitrarily.
    In particular, a purely quadratic functional
    \begin{equation}
      \mathcal{L}_{\mathrm{quad}} = \frac{1}{2}\,\partial_\mu \chi\,\partial^\mu \chi
    \end{equation}
    is incompatible with the existence of a fundamental upper bound
    on the relaxation speed of the $\chi$ substrate.

    Quadratic actions permit unbounded gradients and therefore allow
    arbitrarily large relaxation fluxes, corresponding to instantaneous
    propagation of constraints.
    Such behavior contradicts the existence of a maximal relaxation speed
    $c_\chi$, required for the causal consistency of the projection
    onto effective spacetime.

    Imposing the condition that the relaxation flux saturates at $c_\chi$
    uniquely constrains the functional form of the action.
    The Lagrangian density must interpolate smoothly between
    the quadratic regime at low gradients and a strictly bounded regime
    at high gradients.

    The minimal functional satisfying these requirements is of Born--Infeld type:
    \begin{equation}
      \mathcal{L}_{\mathrm{BI}}
      =
      b^2\!\left(
             1 - \sqrt{1 - \frac{1}{b^2}\,\partial_\mu \chi\,\partial^\mu \chi}
      \right),
    \end{equation}
    where the parameter $b$ is directly related to the maximal relaxation
    speed $c_\chi$.

    This structure ensures:
    \begin{itemize}
      \item recovery of the quadratic theory in the low-gradient limit,
      \item a strict upper bound on $|\partial_\mu \chi|$,
      \item causal saturation of relaxation fluxes at $c_\chi$.
    \end{itemize}

    Alternative polynomial expansions (e.g.\ $\chi^4$ or higher-order terms)
    fail to enforce such a bound without introducing additional
    ad hoc mass scales.
    They therefore violate high-energy spectral invariance and permit
    superluminal relaxation modes in the $\chi$ substrate.

    Moreover, the Born--Infeld structure is intrinsically self-regularizing:
    configurations that would produce divergent gradients in a quadratic
    theory are smoothly regulated by the square-root saturation mechanism.
    The energy density and relaxation flux remain finite for all admissible
    field configurations, without the need for external cutoffs or
    renormalization prescriptions.

    This self-regularization property is absent from polynomial theories,
    which generically develop gradient singularities and therefore require
    additional ultraviolet completion.

    The Born--Infeld structure is thus not a convenient representation,
    but the \emph{unique} effective functional compatible with
    bounded relaxation, causal projection, and parameter-free saturation
    of the $\chi$ dynamics.

    \paragraph{Relation Between Spectral Saturation ($b$) and the Speed of Light ($c$)}

      It is essential to distinguish the saturation constant $b$ from the
      phenomenological parameter $c$ (the speed of light).
      Within the Cosmochrony framework, $b$ represents the upper bound on the
      relaxation speed of the $\chi$ substrate itself—a pre-geometric causal
      constraint governing the internal dynamics of the fundamental relational
      structure.

      The speed of light $c$, by contrast, emerges as an effective quantity:
      it corresponds to the group velocity of projected perturbative modes
      propagating on the emergent effective metric $g_{\mu\nu}^{\mathrm{eff}}$.
      As such, $c$ is not a fundamental constant of the substrate, but a
      derived parameter characterizing the kinematics of the projected
      description.

      Mathematically, $c$ depends on $b$ through a filtering determined by the
      microscopic coupling density $K_0$, encoding how relaxation dynamics in
      $\chi$ are transcribed into effective spacetime propagation.
      This hierarchical structure ensures that physical information can never
      propagate faster than the underlying relaxation processes of the
      substrate, thereby enforcing the inequality
    \begin{equation}
      c \leq b .
      \end{equation}

      Any hypothetical regime in which $c$ would exceed $b$ would imply that
      projected perturbations propagate faster than the substrate can
      relax, leading to an immediate loss of coherence of solitonic
      configurations and, consequently, to the destabilization of matter
      itself.
