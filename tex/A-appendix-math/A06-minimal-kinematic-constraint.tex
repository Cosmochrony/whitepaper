\subsection{Minimal Kinematic Constraint}
  \label{subsec:minimal-kinematic-constraint}

  A central assumption of Cosmochrony is the existence of a maximal local relaxation rate
  for the $\chi$ field:
  \begin{equation}
    0 \leq \partial_t \chi \leq c ,
  \end{equation}
  where $c$ is identified with the invariant speed that appears in relativistic
  kinematics.

  This bound is not introduced as a dynamical equation or a cosmological driving
  mechanism, but as a minimal kinematic constraint on the unfolding of the $\chi$
  field.
  It ensures that the local progression of physical time remains finite and that
  no influence associated with $\chi$-relaxation can propagate arbitrarily fast.

  Rather than postulating a cosmological constant or an initial expansion impulse,
  Cosmochrony attributes the large-scale expansion of the universe to the cumulative
  effect of this local bound applied to a globally relaxing field.
  At the same time, the constraint guarantees causal consistency and allows the
  emergence of effective spacetime descriptions compatible with special relativity,
  without assuming relativistic structure at the fundamental level.
