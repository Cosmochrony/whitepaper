\subsection{Minimal Kinematic Constraint}
  \label{subsec:minimal-kinematic-constraint}

  A foundational assumption of Cosmochrony is the existence of a universal upper bound
  on the local relaxation rate of the $\chi$ field:
  \begin{equation}
    0 \;\leq\; \partial_t \chi \;\leq\; c ,
  \end{equation}
  where $c$ denotes the maximal admissible rate of relaxation.
  This constant is identified, at the level of effective descriptions, with the
  invariant speed that characterizes relativistic kinematics.

  This bound is not introduced as a dynamical equation, a force law, or a cosmological
  driving term.
  It constitutes a purely kinematic constraint on admissible $\chi$ configurations,
  specifying the maximal rate at which the relational structure of the field may
  unfold.
  In particular, it does not prescribe how $\chi$ evolves, but only restricts which
  evolutions are physically admissible.

  The presence of a maximal relaxation rate serves two closely related purposes.
  First, it ensures that the local progression of effective physical time remains
  finite and well-defined, preventing pathological behavior such as instantaneous
  global reconfiguration.
  Second, it enforces causal consistency by guaranteeing that no influence associated
  with $\chi$ relaxation can propagate arbitrarily fast across the relational
  structure.

  Importantly, this constraint precedes any notion of spacetime geometry.
  It is imposed directly on the pre-geometric dynamics of $\chi$ and does not rely on
  light cones, metrics, or Lorentz symmetry as fundamental ingredients.
  Rather, the familiar relativistic causal structure emerges \emph{a posteriori} as
  an effective description of systems whose dynamics saturate, but do not exceed,
  this universal bound.

  At cosmological scales, the same constraint acquires a global interpretation.
  When applied to a nearly homogeneous configuration of $\chi$, the bound
  $\partial_t \chi \leq c$ implies a monotonic and approximately uniform increase of
  $\chi$, which underlies the effective expansion of space discussed in
  Section~\ref{subsec:expansion-as-relaxation}.
  In this sense, large-scale cosmic expansion is not driven by an external impulse or
  a vacuum energy term, but reflects the cumulative consequence of a local kinematic
  limitation applied consistently across the field.

  The minimal kinematic constraint therefore plays a unifying role in Cosmochrony.
  It anchors causal consistency, bounds temporal unfolding, and provides the
  structural basis from which relativistic spacetime behavior emerges, all without
  introducing additional dynamical postulates or background geometric structures.
