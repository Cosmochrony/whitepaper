\subsection{Minimal Kinematic Constraint}
  \label{subsec:minimal-kinematic-constraint}

  A central postulate of Cosmochrony is the existence of a \emph{universal kinematic bound}
  on admissible configurations of the $\chi$ substrate.
  This bound is not derived from an action, a metric structure, or a variational principle,
  but is imposed \emph{ab initio} at the pre-geometric level as a constraint on admissible
  relaxation patterns.

  In effective descriptions admitting a scalar representation $\chi_{\mathrm{eff}}$,
  this postulate may be written in the compact form
  \begin{equation}
    (\partial_t \chi)^2 + |\nabla \chi|^2 = c^2 ,
    \label{eq:minimal-kinematic-constraint}
  \end{equation}
  where the symbols $\partial_t$ and $\nabla$ denote \emph{effective} temporal and relational
  variations introduced only once a projectable geometric regime is established.

  This constraint underlies the Hamiltonian derivation in Section~\ref{subsec:hamiltonian-derivation}.

  \paragraph{Pre-geometric status.}
    Equation~\eqref{eq:minimal-kinematic-constraint} does \emph{not} presuppose the existence
    of a spacetime metric, light cones, or a Lorentzian structure.
    It is not the norm of a four-gradient taken with respect to an underlying Minkowski metric,
    nor does it encode any invariant interval.
    Rather, it expresses a purely kinematic saturation condition: admissible projected
    descriptions must lie on the boundary of a fixed relaxation budget.

    At the level of the fundamental $\chi$ substrate, no notions of time, space, distance,
    or orthogonality are defined.
    The constant $c$ appearing in~\eqref{eq:minimal-kinematic-constraint} is therefore not
    a velocity in spacetime, but the effective manifestation of the invariant structural bound
    $c_\chi$ introduced in Section~\ref{subsec:role-of-cchi}.
    Only after projection does this bound acquire an operational interpretation as a maximal
    local ordering rate.

  \paragraph{Interpretation as a kinematic postulate.}
    The constraint~\eqref{eq:minimal-kinematic-constraint} should be read as a statement of
    \emph{admissibility}, not of dynamics.
    It does not determine how $\chi$ evolves, propagates, or responds to sources.
    Instead, it restricts the class of effective descriptions that can consistently represent
    underlying $\chi$ configurations.
    In particular, it forbids arbitrarily rapid local ordering or instantaneous global
    reconfiguration, thereby enforcing causal consistency without postulating causality as
    a primitive notion.

  \paragraph{Emergent relativistic structure.}
    Only in regimes where projected configurations admit a locally injective and smooth
    representation does~\eqref{eq:minimal-kinematic-constraint} take on a familiar relativistic
    form.
    In such cases, the bound $c$ coincides numerically with the invariant speed appearing
    in special relativity, and the constraint can be rewritten in a form suggestive of a
    light-cone structure.
    This resemblance is emergent and descriptive, not ontological: Lorentz symmetry
    arises \emph{a posteriori} as a property of saturated relaxation, rather than as a
    fundamental symmetry imposed on spacetime.

  \paragraph{Cosmological and gravitational implications.}
    In homogeneous regimes, the constraint enforces uniform saturation of relaxation,
    leading directly to linear growth of $\chi_{\mathrm{eff}}$ and, consequently, to effective
    cosmic expansion.
    In inhomogeneous regimes, partial saturation manifests as local slowdown of relaxation,
    which underlies gravitational time dilation and curvature effects discussed in later
    sections.
    In all cases, the same minimal kinematic postulate governs admissibility, without the
    introduction of additional geometric or dynamical assumptions.

    The minimal kinematic constraint therefore constitutes one of the foundational pillars
    of Cosmochrony.
    It precedes spacetime, metric structure, and relativistic symmetry, and provides the
    structural origin from which effective causality and relativistic kinematics emerge.

  \paragraph{Limits of the continuum description.}
    The formulation of the minimal kinematic constraint in terms of continuous derivatives
    implicitly assumes a regime in which projected $\chi$ configurations admit a smooth
    effective description.
    At scales comparable to the fundamental relational spacing---for instance near the
    Planck scale---this continuum approximation is expected to break down.

    In such regimes, the gradient operator $\nabla \chi$ should be replaced by finite
    difference expressions defined on the underlying relational graph, and the constraint
    \eqref{eq:minimal-kinematic-constraint} must be rederived from purely discrete
    considerations.
    While the existence of a universal saturation bound is expected to persist, its precise
    mathematical expression may differ from the continuum form used here.

    A fully discrete formulation of the kinematic constraint and its implications for
    early-universe dynamics is deferred to future work.

  \begin{tcolorbox}[colback=white, colframe=green!75!black, title=Example: Homogeneous Relaxation]
      Consider a spatially homogeneous configuration where $|\nabla \chi| = 0$.
      The minimal kinematic constraint reduces to
      \[
        (\partial_t \chi)^2 \leq c^2 .
      \]
      In regimes where the bound is locally saturated, this implies
      \[
        \partial_t \chi = c ,
      \]
      leading to a linear effective evolution
      \[
        \chi(t) = \chi_0 + ct .
      \]
      When the effective scale factor satisfies $a(t) \propto \chi(t)$, this directly yields
      the Hubble law discussed in Section~\ref{subsec:homogeneous-cosmological-limit}.
    \end{tcolorbox}
