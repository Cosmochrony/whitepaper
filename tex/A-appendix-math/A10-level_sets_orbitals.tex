\subsection{Level Sets, Projections, and Apparent Orbital Geometry}
  \label{app:level_sets_orbitals}

  This appendix establishes a general geometric property of continuous scalar fields
  that is directly relevant to the interpretation of atomic orbitals and similar
  structures as threshold-visible manifestations of an underlying continuum.
  The results presented here are purely mathematical and do not rely on any specific
  physical interpretation.

  Level sets of $\chi$ are introduced solely as visualization tools.
  They do not correspond to fundamental spatial structures, nor to independently
  localized entities.
  Instead, they provide a convenient way of identifying regions of comparable
  field value within effective geometric descriptions.

  \subsubsection*{Level Sets of Continuous Scalar Fields}

    Let $\phi : \mathbb{R}^3 \rightarrow \mathbb{R}$ be a continuous scalar field.
    For a given constant $c \in \mathbb{R}$, the associated level set (or isosurface)
    is defined as
    \begin{equation}
      \mathcal{L}_c = \{ \mathbf{x} \in \mathbb{R}^3 \mid \phi(\mathbf{x}) = c \}.
    \end{equation}

    If $\phi$ is smooth, $\mathcal{L}_c$ is generically a two-dimensional surface,
    possibly composed of several disconnected components.
    Such level sets are routinely used in the visualization of scalar fields by
    retaining only regions exceeding a prescribed threshold.

    Crucially, the existence of multiple disconnected components of $\mathcal{L}_c$
    does \emph{not} imply that the underlying field $\phi$ itself is discontinuous or
    decomposed into independent objects.

  \subsubsection*{Projection-Induced Apparent Discontinuities}

    Consider the projection of the level-set condition onto a single coordinate,
    for instance $z$.
    Define the projected set
    \begin{equation}
      P_c =
      \{ z \in \mathbb{R} \mid \exists (x,y)\in\mathbb{R}^2 \text{ such that }
      \phi(x,y,z) \ge c \}.
    \end{equation}

    Even when $\phi$ is continuous everywhere, $P_c$ typically consists of a finite
    union of disjoint intervals.
    These intervals correspond to regions where the level set intersects planes of
    constant $z$.

    This fragmentation is a purely geometric consequence of thresholding followed by
    projection.
    It reflects the fact that only portions of the field exceeding the chosen
    threshold are retained.
    No discontinuity of $\phi$ is involved.

  \subsubsection*{Envelope Function and Threshold Visibility}

    Define the envelope function
    \begin{equation}
      f(z) = \max_{x,y} \phi(x,y,z).
    \end{equation}

    The projected set can then be written equivalently as
    \begin{equation}
      P_c = \{ z \in \mathbb{R} \mid f(z) \ge c \}.
    \end{equation}

    The envelope function $f(z)$ is continuous whenever $\phi$ is continuous.
    However, the condition $f(z)\ge c$ generically selects disconnected regions of
    the domain.
    The appearance and disappearance of these regions as $c$ varies reflect changes
    in \emph{visibility}, not in the underlying field structure.

    Thus, threshold-based visualizations reveal sections of a continuous field rather
    than discrete or independently localized objects.

  \subsubsection*{Non-Uniqueness of Inverse Reconstruction}

    Given a projected set $P_c$ or a collection of disconnected level-set components,
    the inverse problem of reconstructing $\phi$ is underdetermined.
    Infinitely many continuous scalar fields may share identical threshold projections.

    Recovering $\phi$ uniquely requires additional assumptions, such as symmetry,
    smoothness, minimal curvature, or governing differential equations.
    The present analysis therefore establishes a structural constraint, not a
    reconstruction algorithm.

  \subsubsection*{Summary}

    Thresholded and projected visualizations of continuous scalar fields generically
    produce apparently disjoint structures.
    These structures arise from geometric selection effects and do not correspond to
    independent physical entities.

    This result is entirely model-independent.
    However, it provides a natural mathematical framework for understanding
    orbital-like patterns, nodal structures, and probabilistic visibility regions as
    emergent manifestations of an underlying continuous field.

    The apparent discreteness of such patterns reflects projection and detection
    criteria rather than fundamental discontinuity of the substrate.
