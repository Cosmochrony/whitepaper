\subsection{Emergent Electrodynamics from $\chi$ Dynamics}
  \label{app:emergent-electrodynamics}

  The dynamics of $\chi$ perturbations are governed by the
      \textbf{effective wave equation} (Section~\ref{subsec:variational-formulation}):
  \[
    \boxed{
      \nabla \cdot \left( \frac{\nabla \chi}{\sqrt{1 - |\nabla \chi|^2 / c^2}} \right) - \frac{1}{c^2}
      \frac{\partial^2 \chi}{\partial t^2} = \frac{4 \pi G_{\text{eff}}}{c^2} \rho_{\chi}.
    }
  \]
  In the \textbf{weak-field limit} ($|\nabla \chi| \ll c$), this reduces to:
  \[
    \nabla^2 \chi - \frac{1}{c^2} \frac{\partial^2 \chi}{\partial t^2} = 4 \pi G_{\text{eff}} \rho_{\chi}.
  \]

  \paragraph{Scalar and Vector Potentials}
    Decompose $\chi$ into \textbf{scalar} and \textbf{vector} components:
    \[
      \chi(\mathbf{x}, t) = \chi_0 + \phi(\mathbf{x}, t) + \nabla \cdot \mathbf{A}(\mathbf{x}, t),
    \]
    \textbf
    {should be understood as an effective Helmholtz-like projection induced by the topology of solitonic excitations,
      rather than as a fundamental field split.} Specifically:
    \begin{itemize}
      \item The scalar component $\phi$ encodes \textbf{longitudinal modes} associated with charge density
      $\rho_{\text{em}} \propto \Delta_G^{(0)} \chi$.
      \item The vector component $\mathbf{A}$ emerges from \textbf{transverse gradients} of $\chi$ (i.e.,
      $\nabla \times \nabla \chi \neq 0$), which are inherently divergence-free due to the solitonic topology (Section~
      \ref{subsec:vortices}).
    \end{itemize}

    The vector potential $\mathbf{A}$ is not an independent degree of freedom but arises from the
        \textbf{transverse sector} of $\chi$ gradients\footnote{
  The transverse nature of $\mathbf{A}$ follows from the solitonic topology: charged solitons (
  Section~\ref{subsec:vortices}) are characterized by closed loop integrals $\oint \nabla \chi \cdot d\mathbf{l} \neq 0$
  , which imply $\nabla \times \mathbf{A} \neq 0$ while enforcing $\nabla \cdot \mathbf{A} = 0$
  via Stokes' theorem. This is analogous to the Helmholtz decomposition in fluid dynamics,
  where the vector field is derived from the scalar field's gradients.
}.

  \paragraph{Field Definitions}
    Define the \textbf{electric} and \textbf{magnetic fields} as:
    \[
      \mathbf{E} = -\nabla \phi - \frac{1}{c} \frac{\partial \mathbf{A}}{\partial t}, \quad \mathbf{B} = \nabla \times
      \mathbf{A}.
    \]
    These fields satisfy the \textbf{Maxwell equations}:
    \[
      \boxed{
        \begin{aligned}
          \nabla \cdot \mathbf{E} &= 4 \pi G_{\text{eff}} \rho_{\text{em}}, \\
          \nabla \times \mathbf{E} + \frac{1}{c} \frac{\partial \mathbf{B}}{\partial t} &= 0, \\
          \nabla \cdot \mathbf{B} &= 0, \\
          \nabla \times \mathbf{B} - \frac{1}{c} \frac{\partial \mathbf{E}}{\partial t} &=
          \frac{4 \pi G_{\text{eff}}}{c} \mathbf{J}_{\text{em}}.
        \end{aligned}
      }
    \]

  \paragraph{Gauge Invariance}
    The decomposition $\chi = \chi_0 + \phi + \nabla \cdot \mathbf{A}$ is \textbf{not unique}. A
        \textbf{gauge transformation}:
    \[
      \phi \to \phi - \frac{1}{c} \frac{\partial \Lambda}{\partial t}, \quad \mathbf{A} \to \mathbf{A} + \nabla \Lambda,
    \]
    leaves the fields $\mathbf{E}$ and $\mathbf{B}$ invariant. This \textbf{U(1) gauge symmetry} is
        \textbf{derived from the relational nature of $\chi$}, where only \textbf{gradient differences}
        are observable (Section~\ref{subsec:relational-foundation}).
