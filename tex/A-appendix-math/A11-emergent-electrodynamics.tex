\subsection{Emergent Electrodynamics from $\chi$ Dynamics}
  \label{app:emergent-electrodynamics}

  In regimes where the $\chi$ field admits a smooth geometric and weak-gradient
  description, small perturbations around a slowly varying background obey an
  effective wave equation derived from the variational formulation
  (Section~\ref{subsec:variational-formulation}):
  \begin{equation}
    \nabla \cdot
    \left(
      \frac{\nabla \chi}{\sqrt{1 - |\nabla \chi|^2 / c^2}}
    \right)
    - \frac{1}{c^2} \frac{\partial^2 \chi}{\partial t^2}
    =
    \frac{4 \pi G_{\mathrm{eff}}}{c^2} \rho_\chi .
  \end{equation}

  In the weak-field limit, $|\nabla \chi| \ll c$, this equation linearizes to
  \begin{equation}
    \nabla^2 \chi
    - \frac{1}{c^2} \frac{\partial^2 \chi}{\partial t^2}
    =
    4 \pi G_{\mathrm{eff}} \rho_\chi ,
  \end{equation}
  which admits propagating solutions interpreted as radiative disturbances of the
  $\chi$ field.

  \subsubsection*{Emergent Scalar and Vector Potentials}

    Electromagnetic-like degrees of freedom arise from the \emph{geometric structure}
    of $\chi$ perturbations rather than from independent fundamental fields.
    In regions containing charged solitonic excitations, the spatial gradients of
    $\chi$ acquire both longitudinal and transverse components.

    Accordingly, the spatial gradient of $\chi$ may be decomposed (at the effective
    level) into longitudinal and transverse parts:
    \begin{equation}
      \nabla \chi = - \nabla \phi + \mathbf{A}_{\mathrm{T}},
    \end{equation}
    where $\phi$ is an effective scalar potential and $\mathbf{A}_{\mathrm{T}}$ is a
    divergence-free vector field,
    \begin{equation}
      \nabla \cdot \mathbf{A}_{\mathrm{T}} = 0 .
    \end{equation}

    This decomposition should be understood as an effective Helmholtz projection
    induced by the topology of localized $\chi$ excitations, rather than as a
    fundamental split of degrees of freedom.
    The scalar component encodes longitudinal relaxation gradients associated with
    effective charge density, while the transverse component arises from solitonic
    configurations with non-trivial circulation.

  \subsubsection*{Topological Origin of the Vector Potential}

    The transverse component $\mathbf{A}_{\mathrm{T}}$ originates from solitonic
    configurations of $\chi$ characterized by non-vanishing loop integrals
    \begin{equation}
      \oint \nabla \chi \cdot d\mathbf{l} \neq 0 ,
    \end{equation}
    as discussed in Section~\ref{subsec:vortices}.
    Such configurations imply the existence of an effective vector potential whose
    curl is non-zero, while its divergence vanishes identically.

    The vector potential is therefore not an independent dynamical field.
    It is a derived quantity encoding the transverse sector of $\chi$ gradients,
    analogous to vorticity-induced vector fields in fluid dynamics.

  \subsubsection*{Emergent Electromagnetic Fields}

    Within this effective description, the electric and magnetic fields are defined as
    \begin{equation}
      \mathbf{E}
      =
      - \nabla \phi
      - \frac{1}{c} \frac{\partial \mathbf{A}_{\mathrm{T}}}{\partial t},
      \qquad
      \mathbf{B}
      =
      \nabla \times \mathbf{A}_{\mathrm{T}} .
    \end{equation}

    These fields satisfy the Maxwell equations:
    \begin{align}
      \nabla \cdot \mathbf{E}
      &= 4 \pi G_{\mathrm{eff}} \rho_{\mathrm{em}}, \\
      \nabla \times \mathbf{E}
      + \frac{1}{c} \frac{\partial \mathbf{B}}{\partial t}
      &= 0, \\
      \nabla \cdot \mathbf{B}
      &= 0, \\
      \nabla \times \mathbf{B}
      - \frac{1}{c} \frac{\partial \mathbf{E}}{\partial t}
      &=
      \frac{4 \pi G_{\mathrm{eff}}}{c} \mathbf{J}_{\mathrm{em}},
    \end{align}
    where $\rho_{\mathrm{em}}$ and $\mathbf{J}_{\mathrm{em}}$ denote the effective charge
    and current densities associated with solitonic $\chi$ excitations.

  \subsubsection*{Gauge Invariance}

    The decomposition of $\nabla \chi$ into scalar and transverse components is not
    unique.
    A transformation of the form
    \begin{equation}
      \phi \rightarrow \phi - \frac{1}{c} \frac{\partial \Lambda}{\partial t},
      \qquad
      \mathbf{A}_{\mathrm{T}} \rightarrow \mathbf{A}_{\mathrm{T}} + \nabla \Lambda ,
    \end{equation}
    leaves the observable fields $\mathbf{E}$ and $\mathbf{B}$ invariant.

    This emergent $U(1)$ gauge symmetry reflects the relational nature of $\chi$:
    only differences of gradients have operational meaning, while the absolute
    potential is unobservable
    (Section~\ref{subsec:relational-foundation-pointer}).

  \subsubsection*{Interpretational Status}

    The Maxwell-like structure derived here is not fundamental.
    It arises as a universal effective description of transverse $\chi$ perturbations
    in regimes where solitonic topology and weak gradients coexist.
    Electromagnetism therefore appears as a geometric manifestation of the relational
    and topological structure of the $\chi$ field, rather than as an independent
    interaction mediated by elementary gauge fields.
