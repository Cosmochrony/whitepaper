\subsection{Emergent Electrodynamics from \texorpdfstring{$\chi$}{χ} Dynamics}
  \label{app:emergent-electrodynamics}

  \paragraph{Scope and ontological status.}
    This subsection does \emph{not} introduce an independent electromagnetic field
    or a new fundamental interaction.
    All structures discussed below arise exclusively within the effective,
    post-projective description of admissible $\chi$ configurations once a smooth
    geometric regime has emerged.
    No additional degrees of freedom beyond $\chi$ are postulated.

    In regimes where the $\chi$ field admits a smooth geometric and weak-gradient
    description, small perturbations around a slowly varying background obey an
    effective wave equation derived from the variational formulation
    (Section~\ref{subsec:variational-formulation}):
    \begin{equation}
      \nabla \cdot
      \left(
        \frac{\nabla \chi}{\sqrt{1 - |\nabla \chi|^2 / c^2}}
      \right)
      - \frac{1}{c^2} \frac{\partial^2 \chi}{\partial t^2}
      =
      \frac{4 \pi G_{\mathrm{eff}}}{c^2} \rho_\chi .
    \end{equation}

    In the weak-field limit, $|\nabla \chi| \ll c$, this equation linearizes to
    \begin{equation}
      \nabla^2 \chi
      - \frac{1}{c^2} \frac{\partial^2 \chi}{\partial t^2}
      =
      4 \pi G_{\mathrm{eff}} \rho_\chi ,
    \end{equation}
    which admits propagating solutions.
    These solutions should be interpreted as admissible collective modes of the
    projected description, not as fundamental propagating excitations of the $\chi$
    substrate.

\subsubsection*{Emergent Scalar and Vector Potentials}

  Electromagnetic-like degrees of freedom arise from the \emph{geometric and
topological structure} of $\chi$ perturbations rather than from independent
  gauge fields.
  In regions containing charged solitonic excitations, the spatial gradients of
  $\chi$ acquire both longitudinal and transverse components.

  At the effective level, the spatial gradient of $\chi$ may therefore be
  decomposed as
  \begin{equation}
    \nabla \chi = - \nabla \phi + \mathbf{A}_{\mathrm{T}},
  \end{equation}
  where $\phi$ is an effective scalar potential and $\mathbf{A}_{\mathrm{T}}$ is a
  divergence-free vector field,
  \begin{equation}
    \nabla \cdot \mathbf{A}_{\mathrm{T}} = 0 .
  \end{equation}

  This decomposition is not a fundamental split of degrees of freedom.
  It corresponds to an effective Helmholtz projection induced by the topology of
  localized $\chi$ excitations and the admissibility constraints of the projected
  description.
  The scalar component encodes longitudinal relaxation gradients associated with
  effective charge density, while the transverse component arises from solitonic
  configurations with non-trivial circulation.

\subsubsection*{Charge as Transverse Torsion of the Relaxation Flux}
  \label{subsec:charge-as-torsion}

  Within the effective regime in which a geometric description of $\chi$ variations
  is admissible, the decomposition
  \begin{equation}
    \nabla \chi = -\nabla \phi + \mathbf{A}_T ,
  \end{equation}
  with $\nabla \cdot \mathbf{A}_T = 0$, admits a natural structural interpretation.
  The transverse component $\mathbf{A}_T$ does not represent an independent gauge
  field, but encodes a \emph{directional or chiral organization} of the admissible
  relaxation flux associated with localized excitations.

  The effective relaxation flux derived from the Born--Infeld--like Lagrangian,
  \begin{equation}
    \mathcal{L}_{\mathrm{eff}}(\chi)
    \;\sim\;
    -c^2 \sqrt{1 - \frac{|\nabla \chi|^2}{c^2}} ,
  \end{equation}
  defines a bounded canonical current
  \begin{equation}
    \mathbf{J}_\chi
    \;\equiv\;
    \frac{\partial \mathcal{L}_{\mathrm{eff}}}{\partial (\nabla \chi)}
    \;\propto\;
    \frac{\nabla \chi}{\sqrt{1 - |\nabla \chi|^2 / c^2}} ,
  \end{equation}
  whose magnitude approaches saturation as $|\nabla \chi| \to c$.
  In this context, $\mathbf{A}_T$ may be understood as a \emph{torsional component}
  of the effective relaxation flux, characterizing a non-trivial orientation of
  $\mathbf{J}_\chi$ in the projected description.

  A localized excitation is said to carry an effective electric charge if the
  associated transverse component $\mathbf{A}_T$ exhibits a topologically non-trivial
  circulation. Formally, an effective charge invariant may be defined as
  \begin{equation}
    q
    \;\equiv\;
    \kappa \oint_{\gamma} \mathbf{A}_T \cdot d\boldsymbol{\ell}
    \;=\;
    \kappa \int_{S} (\nabla \times \mathbf{A}_T)\cdot d\mathbf{S},
  \end{equation}
  where $\gamma$ is a closed loop enclosing the excitation, $S$ is a spanning surface,
  and $\kappa$ is a normalization constant.
  The sign of $q$ reflects the orientation (chirality) of the transverse torsion,
  while its stability follows from the topological character of the circulation.

  In this formulation, electric charge is not introduced as a fundamental attribute
  or coupling, but emerges as a structural property of the admissible relaxation flux.
  Charge conjugation corresponds to a reversal of the transverse orientation of
  $\mathbf{A}_T$, consistent with its interpretation as a relational inversion rather
  than an internal symmetry operation.

  The magnetic field appearing in the effective electrodynamic description is then
  naturally identified with the vorticity of the transverse flux,
  \begin{equation}
    \mathbf{B} \;\equiv\; \nabla \times \mathbf{A}_T ,
  \end{equation}
  while electric effects arise from longitudinal relaxation gradients and temporal
  variations of the torsional structure, in agreement with the standard effective
  kinematics.

  Importantly, no fundamental gauge symmetry is postulated at the level of the
  $\chi$ substrate. Apparent gauge redundancies reflect the non-uniqueness of the
  effective potential description of $\mathbf{A}_T$, while physically meaningful
  quantities are encoded in topological and flux invariants of the relaxation field.

\subsubsection*{Topological Origin of the Vector Potential}

  The transverse component $\mathbf{A}_{\mathrm{T}}$ originates from solitonic
  configurations of $\chi$ characterized by non-vanishing loop integrals
  \begin{equation}
    \oint \nabla \chi \cdot d\mathbf{l} \neq 0 ,
  \end{equation}
  as discussed in Section~\ref{subsec:vortices}.
  Such configurations imply the existence of an effective vector potential whose
  curl is non-zero, while its divergence vanishes identically.

  The vector potential is therefore not an independent dynamical field.
  It is a derived quantity encoding the transverse sector of admissible $\chi$
  gradients, analogous to vorticity-induced vector fields in hydrodynamic systems.

\subsubsection*{Emergent Electromagnetic Fields}

  Within this effective description, the electric and magnetic fields are defined as
  \begin{equation}
    \mathbf{E}
    =
    - \nabla \phi
    - \frac{1}{c} \frac{\partial \mathbf{A}_{\mathrm{T}}}{\partial t},
    \qquad
    \mathbf{B}
    =
    \nabla \times \mathbf{A}_{\mathrm{T}} .
  \end{equation}

  These fields satisfy a closed system of Maxwell-like relations:
  \begin{align}
    \nabla \cdot \mathbf{E}
    &= 4 \pi G_{\mathrm{eff}} \rho_{\mathrm{em}}, \\
    \nabla \times \mathbf{E}
    + \frac{1}{c} \frac{\partial \mathbf{B}}{\partial t}
    &= 0, \\
    \nabla \cdot \mathbf{B}
    &= 0, \\
    \nabla \times \mathbf{B}
    - \frac{1}{c} \frac{\partial \mathbf{E}}{\partial t}
    &=
    \frac{4 \pi G_{\mathrm{eff}}}{c} \mathbf{J}_{\mathrm{em}},
  \end{align}
  where $\rho_{\mathrm{em}}$ and $\mathbf{J}_{\mathrm{em}}$ denote effective charge
  and current densities associated with solitonic $\chi$ configurations.

  These relations do not constitute fundamental field equations.
  They express compatibility conditions obeyed by the transverse and longitudinal
  sectors of admissible $\chi$ perturbations within the emergent geometric regime.

\subsubsection*{Gauge Invariance}

  The decomposition of $\nabla \chi$ into scalar and transverse components is not
  unique.
  A transformation of the form
  \begin{equation}
    \phi \rightarrow \phi - \frac{1}{c} \frac{\partial \Lambda}{\partial t},
    \qquad
    \mathbf{A}_{\mathrm{T}} \rightarrow \mathbf{A}_{\mathrm{T}} + \nabla \Lambda ,
  \end{equation}
  leaves the observable fields $\mathbf{E}$ and $\mathbf{B}$ invariant.

  This emergent $U(1)$ gauge symmetry reflects the relational nature of $\chi$:
  only differences of gradients have operational meaning, while absolute potential
  values are gauge-dependent descriptors without physical significance
  (Section~\ref{subsec:relational-foundation-pointer}).

\subsubsection*{Interpretational Status}

  The Maxwell-like structure derived here is not fundamental.
  It arises as a universal effective description of transverse $\chi$ perturbations
  in regimes where solitonic topology, weak gradients, and a stable geometric
  approximation coexist.

  Electromagnetism therefore appears in Cosmochrony as a geometric and topological
  manifestation of the relational structure of the $\chi$ substrate, rather than as
  an independent interaction mediated by elementary gauge fields.
