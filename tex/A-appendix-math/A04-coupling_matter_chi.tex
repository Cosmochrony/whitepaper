\subsection{Coupling with Matter: Effective Source Term $S[\chi,\rho]$}
  \label{subsec:coupling_matter_chi}

  In regimes where the $\chi$ field admits a smooth and slowly varying geometric
  description, its collective relaxation dynamics can be represented using
  effective differential operators familiar from continuum field theory.
  Within this \emph{emergent} spacetime description, the influence of localized
  excitations (identified with matter through an effective density $\rho$) on the
  relaxation of $\chi$ may be summarized by an effective source term:
  \begin{equation}
    \square_{\mathrm{eff}} \chi = S[\chi,\rho].
    \label{eq:chi_effective_source}
  \end{equation}

  This equation is not fundamental.
  Both the operator $\square_{\mathrm{eff}}$ and the source term $S[\chi,\rho]$
  arise only after coarse-graining the underlying relational relaxation dynamics
  of $\chi$.
  They provide a macroscopic parametrization valid exclusively in regimes where
  spacetime notions are operationally meaningful.

  \subsubsection*{Physical Interpretation of the Source Term}

    The term $S[\chi,\rho]$ must not be interpreted as an external force acting on
    $\chi$, nor as an independent dynamical input.
    Instead, it encodes the \emph{effective resistance} of localized excitations to
    the global relaxation flow of the field.

    Matter corresponds, in Cosmochrony, to structured and long-lived configurations
    of $\chi$ (solitonic or topologically constrained excitations).
    Such configurations locally inhibit relaxation, inducing spatial gradients and
    differential ordering rates when described in an effective geometric language.

    Within this interpretation, the source term $S[\chi,\rho]$ provides a compact
    phenomenological summary of several emergent effects:
    \begin{itemize}
      \item gravitational time dilation as a manifestation of locally slowed
      $\chi$ relaxation,
      \item inertial mass as persistent resistance to the global relaxation flow,
      \item effective spacetime curvature as a coarse-grained representation of
      spatial variations in relaxation efficiency.
    \end{itemize}

    Importantly, $\rho$ does not represent a fundamental energy density.
    It is an effective descriptor of the density of relaxation-resistant
    configurations within an emergent spacetime regime.

  \subsubsection*{Weak-Field Regime and Linear Approximation}

    In weak-field regimes, where matter-induced gradients remain small and
    relaxation is only mildly perturbed, the source term may be approximated as
    linear in the effective excitation density:
    \begin{equation}
      S[\chi,\rho] \;\simeq\; -\,\alpha\,\rho ,
      \label{eq:linear_source}
    \end{equation}
    where $\alpha$ is an effective coupling parameter.

    Matching this description with the Newtonian limit of the emergent geometric
    regime identifies
    \[
      \alpha \sim \frac{G}{c^2},
    \]
    where the gravitational constant $G$ appears as a macroscopic coupling
    quantifying how strongly localized excitations impede the relaxation of $\chi$.

    Within this approximation, Equation~\eqref{eq:chi_effective_source} reproduces
    the Poisson equation for the effective gravitational potential and yields the
    Schwarzschild solution at leading order.
    No independent gravitational interaction is introduced; gravity arises entirely
    from relaxation resistance.

  \subsubsection*{Strong-Field Regimes and Nonlinear Corrections}

    In regimes of high excitation density—such as near compact objects or during
    early cosmological epochs—the linear approximation breaks down.
    Strong structural constraints lead to saturation effects in the relaxation
    dynamics, requiring nonlinear corrections to the effective source term.

    A generic parametrization of these effects takes the form
    \begin{equation}
      S[\chi,\rho]
      =
      -\,\alpha\,\rho\;
      F\!\left(\frac{\rho}{\rho_c},\,\chi\right),
      \label{eq:nonlinear_source}
    \end{equation}
    where $F$ is a bounded function and $\rho_c$ denotes a characteristic density
    scale beyond which relaxation resistance saturates.

    These nonlinearities prevent unphysical halting of the relaxation flow and
    encode departures from classical gravity in strong-field regimes, while leaving
    the underlying ontological structure unchanged.
    Apparent singular behavior in effective geometric descriptions therefore signals
    the breakdown of the hydrodynamic approximation, not a failure of the
    fundamental $\chi$ dynamics.

  \subsubsection*{Status of the Effective Description}

    The source term $S[\chi,\rho]$ does not define an additional physical field or
    interaction.
    It is a bookkeeping device summarizing how localized, structured configurations
    of $\chi$ modify the collective relaxation flow once spacetime notions have
    emerged.

    At the fundamental level, only the relational relaxation dynamics of $\chi$
    exists.
    Matter, sources, and curvature appear jointly and only as emergent,
    regime-dependent descriptions of this single underlying process.
