\subsection{Coupling with Matter: Effective Source Term $S[\chi,\rho]$}
  \label{subsec:coupling_matter_chi}

  In regimes where the $\chi$ field admits a smooth geometric interpretation, its
  dynamics may be expressed using effective differential operators familiar from
  field theory.
  Within this \emph{emergent} description, the influence of localized excitations
  (matter or energy density $\rho$) on the relaxation of $\chi$ can be summarized
  by an effective source term $S[\chi,\rho]$:
  \begin{equation}
    \square_{\text{eff}} \chi = S[\chi,\rho].
  \end{equation}
  This equation should not be interpreted as fundamental.
  Both the operator $\square_{\text{eff}}$ and the source term $S$ arise only after
  a spacetime description has emerged from the underlying relaxation dynamics of
  $\chi$.

  \subsubsection{Physical Meaning of $S[\chi,\rho]$}

    The term $S[\chi,\rho]$ does not represent an external force acting on $\chi$.
    Rather, it encodes the \emph{effective resistance} of localized excitations to
    the global relaxation of the field.
    Regions containing matter correspond to structured configurations of $\chi$
    (solitons) that locally reduce the relaxation rate, inducing spatial gradients
    and differential proper-time flow.

    Within this interpretation, $S[\chi,\rho]$ provides a compact phenomenological
    description of several emergent effects:
    \begin{itemize}
      \item gravitational time dilation as a consequence of slowed $\chi$ relaxation,
      \item inertial mass as persistent resistance to relaxation,
      \item effective spacetime curvature as a coarse-grained description of spatial
      variations in $\chi$.
    \end{itemize}

  \subsubsection{Effective Form and Weak-Field Limit}

    In weak-field regimes, where matter-induced gradients are small, the effective
    source term may be approximated as linear in the excitation density:
    \begin{equation}
      S[\chi,\rho] \simeq -\alpha \rho ,
    \end{equation}
    with $\alpha$ an effective coupling constant.
    Matching with the Newtonian limit identifies $\alpha \sim G/c^2$, where $G$
    emerges as the macroscopic coupling between matter density and relaxation
    slowdown.

    This linear approximation is sufficient to recover the Poisson equation for the
    effective gravitational potential and the Schwarzschild solution at leading
    order.

  \subsubsection{Strong-Field Regimes and Nonlinear Corrections}

    In regimes of high excitation density, such as near compact objects or in the
    early universe, nonlinear corrections to $S[\chi,\rho]$ are expected.
    These corrections reflect saturation effects in the relaxation dynamics and
    prevent unphysical halting of the field evolution:
    \begin{equation}
      S[\chi,\rho] = -\alpha \rho \, F\!\left(\frac{\rho}{\rho_c}, \chi\right),
    \end{equation}
    where $F$ is a bounded function and $\rho_c$ is a characteristic density scale.
    Such nonlinearities encode departures from classical gravity without modifying
    the underlying ontological structure.
