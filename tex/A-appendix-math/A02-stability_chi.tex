\subsection{Stability Analysis of the $\chi$-Field Dynamics}
  \label{subsec:stability-analysis}

  The stability of the $\chi$-field dynamics, governed by
  \begin{equation}
    \partial_t \chi = c \sqrt{1 - \frac{|\nabla \chi|^2}{c^2}},
  \end{equation}
  is a central requirement for Cosmochrony to define a physically consistent framework.
  Since $\chi$ is interpreted as a fundamental geometric substrate, its evolution must
  remain well-behaved under perturbations, without runaway growth or singular behavior.

  Below we analyze the response of the system to small deviations around homogeneous
  relaxation states.

  \subsubsection{Perturbative Structure and Marginal Linear Stability}

    Consider a spatially homogeneous background solution
    \[
      \chi_0(t) = c t + \chi_{0,0},
    \]
    which satisfies $\nabla \chi_0 = 0$ and $\partial_t \chi_0 = c$.
    We introduce a small perturbation
    \[
      \chi(x,t) = \chi_0(t) + \delta \chi(x,t),
      \qquad |\nabla \delta \chi| \ll c .
    \]

    Substituting into the evolution equation and expanding the square root yields
    \begin{equation}
      \partial_t \delta \chi
      =
      - \frac{1}{2c} |\nabla \delta \chi|^2
      + \mathcal{O}\!\left(|\nabla \delta \chi|^4\right).
    \end{equation}

    Importantly, no term linear in $\delta \chi$ appears.
    The homogeneous solution is therefore \emph{marginally stable at linear order}:
    infinitesimal perturbations neither grow nor propagate dynamically at first order.
    This absence of linear instabilities reflects the purely relaxational
    character of the $\chi$ dynamics.

  \subsubsection{Nonlinear Stability and Dissipative Behavior}

    Although linear perturbations are marginal, the leading nonlinear correction is
    strictly negative.
    This implies that any spatial inhomogeneity in $\chi$ reduces the local relaxation
    rate and is therefore dynamically damped.

    To make this explicit, consider the functional
    \begin{equation}
      E[\delta \chi] = \frac{1}{2} \int |\nabla \delta \chi|^2 \, d^3x ,
    \end{equation}
    which measures the total geometric tension stored in the perturbation.
    Using the evolution equation, one finds that $E[\delta \chi]$ is a
    non-increasing function of time.
    Spatial gradients are therefore progressively smoothed, and perturbations remain
    bounded for all times.

    This establishes \emph{nonlinear stability} of the relaxation dynamics:
    the system is dissipative and contractive in configuration space, with no mechanism
    for amplification of perturbations.

  \subsubsection{Special Configurations}

    For simple perturbative profiles, the qualitative behavior is transparent:
    \begin{itemize}
      \item \textbf{Planar perturbations:} Spatial oscillations do not propagate as waves,
      but are progressively flattened due to the reduction of the local relaxation rate.
      \item \textbf{Spherically symmetric perturbations:} Radial inhomogeneities decay
      monotonically, corresponding to a diffusive relaxation of geometric tension.
    \end{itemize}

    In all cases, the dynamics suppresses sharp gradients and prevents the formation of
    singular structures.

  \subsubsection{Conclusion}

    The $\chi$-field dynamics are marginally stable at linear order and strictly stable
    once nonlinear effects are taken into account.
    This guarantees that the irreversible relaxation of $\chi$ defines a robust and
    physically consistent substrate for the emergence of spacetime geometry, gravitation,
    and quantum phenomena within the Cosmochrony framework.
