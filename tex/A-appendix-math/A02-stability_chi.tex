\subsection{Stability Analysis of the $\chi$-Field Dynamics}
  \label{subsec:stability-analysis}

  The stability of the $\chi$-field dynamics is a central requirement for Cosmochrony
  to define a physically consistent framework.
  Since $\chi$ is interpreted as a fundamental pre-geometric substrate, its evolution
  must remain well-behaved under perturbations, without runaway growth or singular
  behavior.

  In regimes where a smooth geometric description is applicable, the effective
  relaxation dynamics of $\chi$ may be written as
  \begin{equation}
    \partial_t \chi
    =
    c \sqrt{1 - \frac{|\nabla \chi|^2}{c^2}},
  \end{equation}
  where $\partial_t$ denotes an effective ordering parameter associated with the
  relaxation process, not a fundamental time variable.
  This representationity is introduced solely for analytical convenience in the
  hydrodynamic regime.

  Below we analyze the response of this dynamics to small deviations around homogeneous
  relaxation states.

  \subsubsection{Perturbative Structure and Marginal Linear Stability}

    Consider a spatially homogeneous background solution
    \[
      \chi_0(t) = c\,t + \chi_{0,0},
    \]
    satisfying $\nabla \chi_0 = 0$ and $\partial_t \chi_0 = c$.
    We introduce a small perturbation
    \[
      \chi(x,t) = \chi_0(t) + \delta \chi(x,t),
      \qquad
      |\nabla \delta \chi| \ll c .
    \]

    Substituting into the evolution equation and expanding the square root yields
    \begin{equation}
      \partial_t \delta \chi
      =
      - \frac{1}{2c}\,|\nabla \delta \chi|^2
      + \mathcal{O}\!\left(|\nabla \delta \chi|^4\right).
    \end{equation}

    Importantly, no term linear in $\delta \chi$ appears.
    The homogeneous relaxation solution is therefore \emph{marginally stable at linear
order}: infinitesimal perturbations neither grow nor propagate dynamically at first
    order.
    This reflects the purely relaxational character of the $\chi$ dynamics and the
    absence of fundamental propagating modes at the linearized level.

  \subsubsection{Nonlinear Stability and Dissipative Behavior}

    Although linear perturbations are marginal, the leading nonlinear correction is
    strictly negative.
    Any spatial inhomogeneity in $\chi$ therefore reduces the local relaxation rate and
    is dynamically suppressed.

    To make this explicit, consider the functional
    \begin{equation}
      E[\delta \chi]
      =
      \frac{1}{2}
      \int |\nabla \delta \chi|^2 \, d^3x ,
    \end{equation}
    which measures the geometric tension associated with spatial variations of the
    perturbation.
    Using the evolution equation, one finds that $E[\delta \chi]$ is a non-increasing
    function of the ordering parameter.
    This follows from the fact that the relaxation flow is negative-definite in the
    presence of spatial gradients, acting systematically to reduce
    $|\nabla \delta \chi|^2$.

    Spatial gradients are therefore progressively smoothed, and perturbations remain
    bounded for all values of the ordering parameter.
    The dynamics is dissipative and contractive in configuration space, with no
    mechanism for amplification of perturbations.

    This establishes \emph{nonlinear stability} of the $\chi$ relaxation dynamics.

  \subsubsection{Special Configurations}

    For simple classes of perturbations, the qualitative behavior is transparent:
    \begin{itemize}
      \item \textbf{Planar perturbations:}
      Spatial oscillations do not propagate as waves, but are progressively flattened
      as the local relaxation rate decreases in regions of nonzero gradient.

      \item \textbf{Spherically symmetric perturbations:}
      Radial inhomogeneities decay monotonically, corresponding to a
      diffusion-like relaxation of geometric tension.
    \end{itemize}

    In all cases, the dynamics suppresses sharp gradients and prevents the formation
    of singular structures within the effective description.

  \subsubsection{Conclusion}

    The $\chi$-field dynamics are marginally stable at linear order and strictly stable
    once nonlinear effects are taken into account.
    This guarantees that the irreversible relaxation of $\chi$ defines a robust and
    physically consistent substrate for the emergence of spacetime geometry,
    gravitation, and quantum phenomena within the Cosmochrony framework.

    Notably, this stability property is inseparable from the monotonic character of the
    relaxation process: the same mechanism that defines the arrow of time also
    precludes dynamical instabilities.
