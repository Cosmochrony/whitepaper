\subsection{Relational Foundation and Emergent Geometry}
  \label{subsec:relational-foundation-pointer}

  Throughout the main text, the $\chi$ field has been described using a continuous
  representation.
  This choice is not meant to attribute fundamental significance to continuity or to
  spacetime fields, but reflects a pragmatic strategy aimed at maximizing contact
  with established geometric, field-theoretic, and cosmological formalisms.

  Crucially, the emergence of geometric notions in Cosmochrony does \emph{not} depend
  on the assumption of an underlying continuous manifold.
  Continuity is introduced only as an effective approximation, valid in regimes where
  the relational structure of $\chi$ varies smoothly and admits coarse-grained
  descriptions.

  At a more fundamental level, Cosmochrony can be formulated in purely relational
  terms.
  In such a formulation, neither spacetime points, nor distances, nor a metric are
  assumed \emph{a priori}.
  Temporal ordering arises from the monotonic relaxation ordering of $\chi$,
  while spatial relations and effective geometry are reconstructed operationally
  from patterns of correlation, resistance, and connectivity within the field.

  A concrete realization of this relational perspective is developed in
  Appendix~\ref{app:relational_formulation}.
  There, geometric quantities are shown to emerge as effective summaries of
  relational properties of $\chi$, such as the ease with which relaxation-induced
  variations propagate between configurations.
  The metric appears only as a derived object encoding these relational properties,
  not as a fundamental dynamical entity.

  The role of the present subsection is therefore purely clarificatory.
  It emphasizes that the continuous description employed in the main text is a
  representational convenience rather than an ontological commitment.
  All core claims of Cosmochrony—including the emergence of time, geometry, and
  gravitation—remain valid independently of this choice and rest ultimately on the
  relational dynamics of the $\chi$ field itself.
