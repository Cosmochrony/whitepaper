\subsection{Analytical Solutions of the $\chi$-Field Dynamics}
  \label{subsec:analytical-solutions}

  To illustrate the behavior of the $\chi$ field, we derive explicit analytical solutions
  of the dynamical equation
  \begin{equation}
    \partial_t \chi
    =
    c \sqrt{1 - \frac{|\nabla \chi|^2}{c^2}},
  \end{equation}
  in a set of simple but physically meaningful configurations.
  These solutions are not intended to exhaust the dynamics, but to clarify its
  geometric and causal structure.

  \subsubsection{Homogeneous Solution}

    In a spatially homogeneous configuration, $\nabla \chi = 0$ and the evolution equation
    reduces to
    \begin{equation}
      \partial_t \chi = c .
    \end{equation}
    Integration yields
    \begin{equation}
      \chi(t) = \chi_0 + c t ,
    \end{equation}
    where $\chi_0$ is the initial value.
    This solution defines the homogeneous cosmological background of Cosmochrony, in which
    the global relaxation of $\chi$ provides a natural origin for cosmic expansion and the
    Hubble law.

  \subsubsection{Spherically Symmetric Gradient-Saturated Profiles}

    Consider a spherically symmetric configuration $\chi = \chi(r,t)$.
    The evolution equation becomes
    \begin{equation}
      \partial_t \chi
      =
      c \sqrt{1 - \frac{(\partial_r \chi)^2}{c^2}} .
    \end{equation}

    Configurations satisfying $|\partial_r \chi| = c$ correspond to a complete local
    suppression of relaxation, $\partial_t \chi = 0$.
    Such profiles take the form
    \begin{equation}
      \chi(r) = \chi_0 \pm c r ,
    \end{equation}
    and represent limiting configurations in which the local unfolding of time is halted.
    Although these profiles cannot be realized globally, they play an important conceptual
    role as idealized models of horizons and maximally constrained regions.

  \subsubsection{Linear Front Solutions}

    A simple class of exact solutions is given by linear fronts of the form
    \begin{equation}
      \chi(x,t) = \chi_0 + c t \pm x ,
    \end{equation}
    for which $|\nabla \chi| = 1 < c$ (in suitable units) and the evolution equation is
    satisfied identically.
    These solutions describe propagating relaxation fronts separating regions of different
    $\chi$ values.
    They do not correspond to waves in the usual sense, but to kinematic boundaries imposed
    by the maximal relaxation speed.

  \subsubsection{Absence of Linear Wave Solutions}

    It is important to emphasize that the $\chi$-field dynamics does not admit linear wave
    solutions.
    Small perturbations do not propagate as oscillatory modes, but are smoothed through the
    nonlinear relaxation mechanism discussed in
    Section~\ref{subsec:stability-analysis}.
    Apparent wave-like phenomena (gravitational or electromagnetic radiation) arise only as
    effective descriptions associated with structured matter excitations and will be
    discussed in later sections.

  \subsubsection{Conclusion}

    These analytical solutions illustrate the fundamentally relaxational character of the
    $\chi$ dynamics.
    Homogeneous growth underpins cosmological expansion, while gradient-saturated profiles
    and relaxation fronts clarify the emergence of horizons and causal structure.
    Together, they confirm the internal consistency of Cosmochrony and prepare the ground
    for effective wave phenomena introduced at the macroscopic level.
