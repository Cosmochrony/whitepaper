\subsection{Analytical Solutions of the $\chi$-Field Dynamics}
  \label{subsec:analytical-solutions}

  To illustrate the qualitative behavior of the $\chi$ field, we derive a set of
  explicit analytical solutions of the effective relaxation equation
  \begin{equation}
    \partial_t \chi
    =
    c \sqrt{1 - \frac{|\nabla \chi|^2}{c^2}},
    \label{eq:chi_effective_evolution}
  \end{equation}
  valid in regimes where a smooth geometric description is applicable.
  Here, $\partial_t$ denotes an effective ordering parameter associated with the
  relaxation process, not a fundamental time derivative.

  These solutions are not intended to exhaust the full dynamics of $\chi$, but to
  clarify its causal structure, limiting configurations, and relaxational character.

  \subsubsection*{Homogeneous Relaxation Solution}

    In a spatially homogeneous configuration, spatial variations vanish,
    \[
      \nabla \chi = 0 ,
    \]
    and the evolution equation reduces to
    \begin{equation}
      \partial_t \chi = c .
    \end{equation}

    Integration yields
    \begin{equation}
      \chi(t) = \chi_0 + c\,t ,
    \end{equation}
    where $\chi_0$ is a constant labeling the initial relaxation state.
    This solution defines the homogeneous background of Cosmochrony, corresponding
    to uniform global relaxation.

    When interpreted within an effective spacetime description, this homogeneous
    relaxation underlies the emergence of cosmological expansion and leads
    naturally to a Hubble-like relation, as discussed in
    Section~\ref{subsec:homogeneous-cosmological-limit}.

  \subsubsection*{Spherically Symmetric Gradient-Saturated Profiles}

    Consider a spherically symmetric configuration $\chi = \chi(r,t)$.
    The effective evolution equation becomes
    \begin{equation}
      \partial_t \chi
      =
      c \sqrt{1 - \frac{(\partial_r \chi)^2}{c^2}} .
    \end{equation}

    Configurations satisfying
    \[
      |\partial_r \chi| = c
    \]
    correspond to complete local saturation of the relaxation bound.
    In this case,
    \[
      \partial_t \chi = 0 ,
    \]
    indicating a local freezing of the effective temporal ordering.

    Such profiles take the form
    \begin{equation}
      \chi(r) = \chi_0 \pm c\,r ,
    \end{equation}
    and represent limiting configurations in which relaxation is entirely inhibited
    by maximal structural gradients.

    Although these configurations cannot be realized globally in a regular manner,
    they play an important conceptual role as idealized models of horizons and
    maximally constrained regions.
    In effective geometric descriptions, they correspond to boundaries beyond which
    spacetime notions cease to be well-defined.

  \subsubsection*{Linear Relaxation Fronts}

    A simple class of exact solutions is given by linear fronts of the form
    \begin{equation}
      \chi(x,t) = \chi_0 + c\,t \pm v\,x ,
    \end{equation}
    with $|v| < c$.
    For such configurations,
    \[
      |\nabla \chi| = |v| < c ,
    \]
    and the evolution equation~\eqref{eq:chi_effective_evolution} is satisfied
    identically.

    These solutions describe propagating relaxation fronts separating regions of
    different $\chi$ values.
    They do not correspond to propagating waves or oscillatory modes, but to
    kinematic boundaries determined by the maximal admissible relaxation rate.

    The velocity $v$ characterizes the spatial steepness of the front rather than a
    signal propagation speed, which remains bounded by $c$.

  \subsubsection*{Absence of Linear Wave Solutions}

    A crucial structural feature of the $\chi$ dynamics is the absence of linear
    wave solutions.
    Small perturbations around homogeneous relaxation states do not propagate as
    oscillatory modes.

    As shown in Section~\ref{subsec:stability-analysis}, infinitesimal perturbations
    are marginal at linear order and are damped once nonlinear effects are taken
    into account.
    The dynamics is therefore purely relaxational.

    Apparent wave-like phenomena, such as gravitational or electromagnetic radiation,
    arise only at the effective level, through collective excitations associated with
    structured matter configurations.
    These emergent phenomena will be discussed in later sections and should not be
    confused with fundamental propagating modes of the $\chi$ field.

  \subsubsection*{Conclusion}

    These analytical solutions illustrate the central features of the $\chi$
    dynamics:
    homogeneous relaxation underpins cosmological expansion, gradient saturation
    defines causal and horizon-like limits, and relaxation fronts clarify the
    kinematic structure imposed by the universal bound $c$.

    Together, they confirm the internal consistency and stability of the
    Cosmochrony framework and prepare the ground for the emergence of effective
    geometric and radiative phenomena at macroscopic scales.
