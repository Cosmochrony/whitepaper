\subsection{Effective Lagrangian Description as a Hydrodynamic Limit}
  \label{subsec:hydrodynamic-limit}

  \noindent\emph{The purpose of this subsection is not to introduce any additional
fundamental structure into Cosmochrony, but to provide an effective hydrodynamic
tool for connecting the relational $\chi$ framework to standard geometric
formulations in regimes where a spacetime description becomes operationally
meaningful.}

  \subsubsection*{From Relational Dynamics to an Effective Continuum Description}

    At the fundamental level, Cosmochrony is defined without reference to any
    pre-existing spacetime manifold or metric structure.
    The dynamics of the $\chi$ field are relational and are specified directly in
    terms of local relaxation rules and coupling relations between configurations
    (Section~\ref{subsec:geometric-action}).

    In regimes where $\chi$ varies smoothly over large scales, it becomes convenient
    to introduce a continuum approximation in order to compare the theory with
    standard geometric and field-theoretic formulations.
    This approximation does not alter the underlying ontology but provides a
    coarse-grained description suitable for analytical calculations and contact with
    general relativity.

  \subsubsection*{Hydrodynamic Limit and Emergent Geometry}

    In this hydrodynamic regime, the discrete relational couplings encoded in the
    connectivity matrix $K_{ij}$ can be summarized by effective continuum quantities.
    Operationally, distances are defined through the resistance encountered by the
    propagation of $\chi$ relaxation across the network.
    In the continuum limit, this leads schematically to an effective line element of
    the form
    \[
      g_{\mu\nu}\,dx^\mu dx^\nu \;\sim\; \sum_{(u,v)\in\text{path}} \frac{1}{K_{uv}},
    \]
    which should be understood as a diagnostic illustration rather than a defining
    relation.
    This expression does not define a unique metric tensor, but captures how effective
    distance emerges as cumulative resistance to $\chi$ relaxation.

    The effective metric $g_{\mu\nu}$ therefore encodes the coarse-grained density of
    correlations in the $\chi$ field and serves as a macroscopic summary of its
    relational dynamics.

  \subsubsection*{Effective Lagrangian Representation}

    To reproduce the continuum evolution equations obtained from the discrete
    relaxation dynamics (Equation~\ref{eq:discrete-dynamics}), one may introduce an
    effective Lagrangian density $\mathcal{L}_{\mathrm{CC}}$.
    This Lagrangian is constructed to match the hydrodynamic behavior of the $\chi$
    field in the smooth regime, while remaining fully subordinate to the underlying
    relational description.

    In this representation, terms resembling those of standard geometric theories
    naturally appear.
    In particular, a curvature-like contribution emerges as the leading-order
    descriptor of spatial variations in the relaxation rate:
    \[
      \mathcal{L}_{\text{eff}}
      = \frac{1}{16\pi G_{\text{eff}}}\,F(\chi)\,R
      - \Lambda_{\text{flow}}^{4}\,\chi
      + \cdots
    \]
    where $R$ is the Ricci scalar associated with the effective metric.
    Its appearance reflects the fact that, at leading order in a derivative expansion,
    $R$ provides the most general local scalar invariant encoding slow spatial
    variations of the relaxation structure.
    The function $F(\chi)$ is not an independent coupling, but parametrizes how the
    underlying $\chi$ relaxation dynamics is encoded in the effective geometric
    description.

    Crucially, this Lagrangian does \emph{not} define the fundamental dynamics of the
    theory.
    It is an auxiliary representation that reproduces the macroscopic behavior of
    $\chi$ once a geometric interpretation becomes applicable.

  \subsubsection*{Status and Limitations}

    The hydrodynamic Lagrangian formulation presented here should be understood as an
    auxiliary representation, not as an alternative foundation of Cosmochrony.
    All physical content remains encoded in the relational relaxation dynamics of the
    $\chi$ field.

    The emergence of Einstein-like field equations in this limit reflects the
    universality of geometric descriptions for slowly varying collective phenomena,
    rather than the presence of a fundamental spacetime structure.
    Accordingly, singularities or breakdowns of the effective metric signal only the
    limits of the hydrodynamic approximation, not a failure of the underlying $\chi$
    dynamics.
