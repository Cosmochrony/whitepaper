\subsection{Topological Stability of Relational \texorpdfstring{$\chi$}{χ} Configurations}
  \label{app:relational_topological_stability}

  In the fully relational formulation of Cosmochrony, particle-like excitations are
  identified with nontrivial, internally organized configurations of the
  \(\chi\) field in its abstract configuration space.
  They are not defined as objects localized in a pre-existing spacetime manifold,
  but as relational patterns whose stability is guaranteed by intrinsic topological
  constraints.

  \paragraph{Relational notion of topology.}
    Unlike conventional field theories, where topological invariants are defined with
    respect to spatial embeddings or boundary conditions on a manifold, the invariants
    relevant in Cosmochrony are purely relational.
    They characterize inequivalent classes of \(\chi\) configurations that cannot be
    continuously transformed into one another without violating the internal
    relaxation constraints or inducing a discontinuous reorganization of relational
    structure.

    Topology, in this sense, is a property of configuration space rather than of
    physical space.
    It encodes global consistency conditions on how relational links within a
    configuration may be rearranged while preserving admissibility.

  \paragraph{Origin of stability.}
    The stability of a relational configuration arises from the existence of
    topological obstructions to global relaxation.
    Certain configurations cannot relax continuously into the homogeneous vacuum
    state without passing through forbidden regions of configuration space.
    As a result, they persist as long-lived or stable excitations.

    This stability does not rely on conserved charges imposed by symmetry principles.
    It is a structural property of the relational organization of \(\chi\) itself,
    independent of any geometric or gauge-theoretic framework.

  \paragraph{Geometric metaphors and their limits.}
    For heuristic purposes, relational topological structures may be illustrated using
    geometric metaphors such as knots, twists, vortices, or defects.
    These images provide intuition when configurations admit an effective geometric
    projection.

    However, such metaphors should not be taken literally.
    At the relational level, there are no spatial loops, cores, or embedding spaces.
    All stability properties are encoded in the global pattern of relational
    constraints rather than in spatial winding.

  \paragraph{Example: \texorpdfstring{$4\pi$}{4π}-periodic configurations.}
    A paradigmatic example is provided by relational configurations exhibiting an
    intrinsic \(4\pi\)-periodic internal structure.
    Such configurations cannot be continuously unwound into a trivial state and
    therefore belong to a distinct topological sector of configuration space.

    When projected onto an effective geometric description, these configurations
    exhibit spinorial transformation properties and fermion-like behavior.
    The appearance of spin-\(\tfrac{1}{2}\) is thus traced back to a topological feature
    of the relational configuration, rather than to a fundamental spinor field.

  \paragraph{Particle identity and mass.}
    Distinct particle species correspond, in this picture, to inequivalent topological
    sectors of relational \(\chi\) configurations.
    Particle identity is therefore associated with topological class membership rather
    than with localization or internal labels.

    The energetic cost of deforming a stable configuration is determined by the
    internal resistance of the \(\chi\) field to relaxation.
    This resistance provides a unified origin for particle mass, stability, and
    spectral separation, without invoking externally imposed charges, gauge groups, or
    symmetry-breaking mechanisms.

  \paragraph{Scope and interpretation.}
    The relational-topological picture developed here is not intended as a complete or
    unique classification of all admissible configurations.
    It serves as an explicit realization of the ontological principles underlying
    Cosmochrony, demonstrating how particle-like properties may emerge from the
    internal organization of \(\chi\).

    Crucially, this formulation remains fully compatible with the effective geometric
    and dynamical descriptions employed in the main text.
    Topological stability is defined prior to and independently of any geometric
    projection, ensuring consistency across all descriptive levels of the framework.
