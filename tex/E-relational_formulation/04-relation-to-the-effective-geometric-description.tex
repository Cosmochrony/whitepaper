\subsection{Relation to the Effective Geometric Description}
  \label{subsec:relation-to-the-effective-geometric-description}

  The effective geometric structures introduced in the main text—such as metric
  fields, spatial gradients, connection-like objects, and Poisson-type
  equations—do not represent fundamental degrees of freedom in Cosmochrony.
  They arise as coarse-grained summaries of relational configurations of the
  \(\chi\) field once a projectable regime becomes applicable.

  \paragraph{From relational structure to geometric representation.}
    At the relational level, configurations of \(\chi\) are specified entirely by
    internal structural relations and bounded relaxation constraints.
    No notion of distance, angle, or curvature is defined.
    However, when relational variations become sufficiently smooth and hierarchically
    organized, it becomes possible to represent these configurations using effective
    geometric descriptors.

    This representation associates relational gradients with spatial gradients of a
    projected field \(\chi_{\mathrm{eff}}\), and collective relaxation constraints with
    geometric quantities such as curvature or gravitational potential.
    The resulting geometric language provides a compact and operationally useful
    summary of the relational organization, but it is neither unique nor exact.

  \paragraph{Status of the effective metric.}
    The effective metric introduced in the main text is not postulated as a fundamental
    object.
    It is defined implicitly through the propagation properties of perturbations and
    the operational comparison of relaxation rates.
    In this sense, the metric encodes how relational distinctions are mapped onto
    effective notions of spatial separation and temporal ordering.

    Because this mapping is many-to-one, distinct relational configurations may
    correspond to the same effective metric.
    Conversely, changes in the relational structure may occur without any
    corresponding change in the effective geometric description.
    The metric therefore captures only a restricted subset of the information
    contained in the relational configuration.

  \paragraph{Emergence of field equations.}
    Poisson-type and wave-like equations appearing in the effective description arise
    from linearizing the relational relaxation dynamics around quasi-homogeneous
    configurations.
    They express how small deviations from uniform relaxation propagate and combine at
    the macroscopic level.

    These equations should not be interpreted as fundamental dynamical laws.
    They are regime-dependent approximations whose validity is limited to weak-field,
    slow-variation conditions.
    Outside these regimes, the effective geometric description ceases to provide a
    faithful account of the underlying relational dynamics.

  \paragraph{Consistency across descriptive levels.}
    No contradiction exists between the relational and geometric formulations.
    They apply to different descriptive levels of the same underlying theory.
    The relational formulation specifies the fundamental ontology and dynamics,
    whereas the geometric description provides an efficient and empirically successful
    approximation in appropriate regimes.

    Importantly, the direction of conceptual dependence is unambiguous:
    the geometric description depends on the relational one, but not conversely.
    All geometric notions are secondary constructs whose meaning and applicability are
    derived from the relational organization of \(\chi\).

  \paragraph{Conceptual role.}
    This subsection clarifies that the effective geometric language employed throughout
    the main text is a representational tool rather than an ontological commitment.
    Its role is to connect the relational foundations of Cosmochrony with familiar
    macroscopic descriptions of spacetime and gravity, while preserving the
    non-geometric nature of the fundamental theory.

    The relational formulation therefore underwrites the validity of the effective
    geometric description without being reducible to it, ensuring conceptual coherence
    across all levels of the framework.
