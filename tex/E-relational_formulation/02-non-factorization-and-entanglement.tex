\subsection{Non-Factorization and Entanglement}
  \label{subsec:non-factorization-and-entanglement}

  Within the relational formulation of Cosmochrony, configurations of the
  \(\chi\) field do not generically decompose into independent subsystems.
  The notion of factorization—central to both classical separability and standard
  quantum tensor-product structures—is therefore not fundamental, but emerges only
  in restricted regimes of relational organization.

  A composite relational configuration of \(\chi\) is said to be
  \emph{non-factorizable} when no decomposition exists that preserves the internal
  relaxation structure while isolating disjoint subsets of relations.
  In such cases, what appear as multiple subsystems at the effective geometric level
  correspond, at the relational level, to a single indivisible configuration.

  \paragraph{Relational origin of entanglement.}
    Quantum entanglement arises naturally within this framework as a manifestation of
    persistent non-factorization.
    When an initially unified relational configuration admits an effective projection
    onto spatially separated degrees of freedom, parts of the configuration may become
    geometrically distant while remaining relationally inseparable.

    Entanglement is therefore not understood as the result of superluminal influences
    or nonlocal signal exchange.
    It reflects the fact that the underlying relational structure cannot be expressed
    as a product of independent configurations, even though an effective spacetime
    description assigns distinct locations to its components.

  \paragraph{Effective separability and its limits.}
    In regimes where relational couplings are weak or hierarchically organized,
    approximate factorization becomes possible.
    Such regimes admit an effective description in which subsystems behave
    independently to a good approximation, and classical notions of locality and
    separability apply.

    However, this separability is always conditional and approximate.
    When relational constraints are strong, no refinement of the effective geometric
    description restores full independence.
    Residual correlations persist regardless of spatial separation, reflecting the
    holistic nature of the underlying \(\chi\) configuration.

  \paragraph{Measurement and relational projection.}
    In an effective quantum description, measurements correspond to projections that
    select particular relational features of a configuration while suppressing others.
    For non-factorizable configurations, such projections necessarily act on the
    configuration as a whole.
    As a result, measurement outcomes associated with one effective subsystem
    constrain the set of admissible outcomes for other subsystems, even when these are
    geometrically distant.

    This constraint does not arise from a dynamical update propagating through space,
    but from the incompatibility of certain relational patterns with the selected
    projection.
    In this sense, quantum correlations reflect constraints on relational consistency
    rather than causal influence.

  \paragraph{Relation to nonlocality and causality.}
    The non-factorization underlying entanglement should not be conflated with
    dynamical nonlocality.
    All dynamical evolution of \(\chi\) remains governed by bounded relaxation
    constraints that respect the invariant speed \(c\) once an effective causal
    structure emerges.

    Entanglement correlations therefore do not enable superluminal signaling.
    They express the global structure of relational configurations, which is already
    fully specified prior to projection onto spacetime.

  \paragraph{Conceptual role.}
    This relational interpretation reframes entanglement as an ontological property of
    the configuration space of \(\chi\), rather than as a paradoxical feature of
    measurement or wavefunction collapse.
    It provides a unified explanation for quantum correlations that is consistent with
    relativistic causality and does not require additional postulates beyond the
    relational dynamics of the fundamental field.

    The present subsection establishes the conceptual basis for the discussion of
    spin--statistics relations and topological stability developed in subsequent
    sections of this appendix.
