\subsection{Topological Origin of Fermionic and Bosonic Statistics}
  \label{app:relational_spin_statistics}

  Within the fully relational formulation of Cosmochrony, the distinction between
  fermionic and bosonic behavior does not arise from imposed quantum statistics or
  from a fundamental spinorial ontology.
  Instead, it originates from the internal topological structure of localized
  configurations of the \(\chi\) field in its configuration space.

  \paragraph{Internal rotations and configuration space topology.}
    At the relational level, the notion of rotation is not defined with respect to
    physical space.
    It refers instead to closed paths in the configuration space of admissible
    \(\chi\) configurations.
    Two configurations are considered equivalent if they are related by a continuous
    deformation that preserves all relational relaxation constraints.

    Certain classes of configurations exhibit nontrivial topology in this
    configuration space.
    For these configurations, a closed path corresponding to a \(2\pi\) internal
    reorientation does not return the system to an equivalent configuration.
    Only a full \(4\pi\) cycle restores relational equivalence.
    This topological obstruction defines a non-orientable structure in configuration
    space.

  \paragraph{Emergence of fermionic behavior.}
    Configurations with intrinsic \(4\pi\)-periodicity belong to topological sectors
    that are double-valued under \(2\pi\) reorientation.
    When such configurations admit an effective geometric projection, this internal
    property manifests as fermion-like behavior:
    \begin{itemize}
      \item a sign change under \(2\pi\) rotations,
      \item restoration of equivalence only after a \(4\pi\) cycle,
      \item and transformation properties characteristic of spin-\(\tfrac{1}{2}\)
      degrees of freedom.
    \end{itemize}

    These features arise without introducing fundamental spinors.
    They reflect the topology of the underlying relational configuration rather than a
    representation of the Lorentz group imposed at the outset.

  \paragraph{Emergence of bosonic behavior.}
    Other classes of relational configurations are topologically orientable.
    For these configurations, a \(2\pi\) internal reorientation is sufficient to return
    to an equivalent state.
    When projected onto an effective geometric description, such configurations
    exhibit boson-like behavior, including integer-spin transformation properties and
    the absence of sign inversion under \(2\pi\) rotations.

    The distinction between fermionic and bosonic excitations is therefore encoded in
    the topology of configuration space rather than in any dynamical or statistical
    assumption.

  \paragraph{Geometric metaphors and their limits.}
    Geometric metaphors—such as Möbius twists, non-orientable loops, or knotted
    structures—may be used heuristically to visualize these internal topological
    features.
    However, such images are meaningful only after projection onto an effective
    geometric description.
    At the relational level, no spatial embedding exists, and these metaphors serve
    solely as intuitive aids.

  \paragraph{Relation to the spin--statistics connection.}
    The relational-topological distinction between \(4\pi\)- and \(2\pi\)-periodic
    configurations provides a natural qualitative explanation of the
    spin--statistics connection.
    Fermionic and bosonic behavior emerge as consequences of internal topological
    constraints rather than as independent quantum postulates.

    While this construction does not constitute a formal proof of the
    spin--statistics theorem, it demonstrates that the observed dichotomy between
    fermions and bosons can arise consistently from the internal organization of
    \(\chi\), prior to and independently of any effective geometric or quantum
    description.

  \paragraph{Conceptual role.}
    This subsection completes the relational account of particle properties in
    Cosmochrony by showing how spin and statistics arise from topology alone.
    It reinforces the view that quantum transformation properties are emergent features
    of relational structure, not fundamental ingredients of the theory.
