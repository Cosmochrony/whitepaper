\subsection{Relational Configurations of \texorpdfstring{$\chi$}{χ}}
  \label{subsec:relational-configurations-of-chi}

  At the most fundamental level, the \(\chi\) field is not defined as a function on
  spacetime, nor as a field assigned to points of a manifold.
  It is instead specified as a complete relational configuration, characterized
  entirely by internal structural relations.
  No coordinates, distances, durations, or background geometric notions are assumed
  or required.

  A relational configuration of \(\chi\) is defined by the pattern of mutual
  constraints governing its relaxation structure.
  Two configurations are distinct if and only if they differ in their internal
  relational organization.
  Conversely, configurations that are related by a global reparameterization or
  relaxation-preserving transformation are considered physically equivalent.

  Within this formulation, there is no primitive notion of localization.
  Concepts such as position, separation, or spatial extension have no meaning at
  the relational level.
  What later appears as spatial organization arises only when a configuration
  admits a projectable regime in which geometric descriptors become effective.

  \paragraph{Configuration space and relational equivalence.}
    The space of all admissible \(\chi\) configurations may be viewed as an abstract
    configuration space equipped with equivalence classes defined by relational
    symmetries.
    Physical states correspond to equivalence classes of configurations rather than
    to individual realizations.

    This perspective replaces geometric invariance with relational invariance:
    transformations that preserve the internal relaxation structure of \(\chi\) leave
    the physical content unchanged, even if they would correspond to nontrivial
    coordinate transformations in an emergent geometric description.

  \paragraph{Absence of factorization.}
    Because \(\chi\) is fundamentally relational, its configurations do not generally
    factorize into independent subsystems.
    What appears as a composite system in an effective spacetime description may
    correspond to a single, indivisible relational configuration at the fundamental
    level.

    This absence of factorization is not a dynamical interaction but a structural
    property of the configuration space itself.
    It underlies the emergence of nonlocal correlations and provides the foundation
    for the discussion of entanglement in
    Section~\ref{subsec:non-factorization-and-entanglement}.

  \paragraph{From relational structure to effective description.}
    Only under specific conditions—such as approximate homogeneity, bounded gradients,
    and stable relaxation regimes—does a relational configuration admit an effective
    projection onto a geometric description.
    In that regime, relational distinctions are mapped onto spatial relations,
    durations, and causal ordering.

    Importantly, this mapping is many-to-one and inherently approximate.
    Different relational configurations may correspond to the same effective geometric
    state, and no inverse mapping is defined.
    As a result, the relational formulation provides the ontological foundation of
    Cosmochrony, while effective geometric descriptions serve as emergent,
    context-dependent representations.

  \paragraph{Conceptual role.}
    This relational perspective establishes that the fundamental content of
    Cosmochrony resides entirely in the internal organization of the \(\chi\)
    configuration.
    All subsequent notions—particles, fields, spacetime geometry, and quantum
    correlations—are secondary constructs arising from specific regimes of relational
    organization.

    The purpose of this subsection is therefore not to introduce additional structure,
    but to make explicit the non-geometric and non-local ontology on which the rest of
    the framework is built.
