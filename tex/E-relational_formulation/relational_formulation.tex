\section{Relational Formulation of $\chi$ Dynamics}
  \label{app:relational_formulation}

  This appendix develops a fully relational and explicitly non-geometric
  formulation of particle-like excitations in the $\chi$ framework.
  It is intended to clarify and extend the interpretation of topological stability
  introduced in Sec.~\ref{subsec:topological-stability}, but is not required for
  the effective dynamical description presented in the main text.

  The constructions discussed here illustrate how notions such as particle
  identity, spin, charge, and statistics may arise from internal topological
  features of $\chi$ configurations.
  They should be understood as a concrete realization of the relational ontology
  underlying Cosmochrony, rather than as a closed or unique classification of
  physical particles.

  \subsection{Relational Configurations of $\chi$}

    At the most fundamental level, the $\chi$ field is described as a complete
    relational configuration, without reference to spacetime points, coordinates,
    or an underlying manifold.
    Physical distinctions arise solely from internal relational differences between
    configurations, rather than from localization within a pre-existing geometric
    structure.

  \subsection{Non-Factorization and Entanglement}

    Within this relational description, composite configurations of $\chi$ need not
    factorize into independent subsystems.
    Such non-factorizable configurations naturally account for quantum entanglement,
    which appears as the persistence of shared relational structure even when
    effective spatial separation emerges.

  \subsection{Locality, Causality, and the Role of the Bound $c$}

    Although relational correlations may extend across large effective distances,
    any modification of $\chi$ configurations is constrained by the universal bound
    $c$.
    This ensures causal consistency without invoking signal propagation or
    superluminal influence.

  \subsection{Relation to the Effective Geometric Description}

    The effective metric, spatial gradients, and Poisson-type equations introduced
    in the main text arise as coarse-grained summaries of relational $\chi$
    configurations.
    They provide a convenient macroscopic language, but do not constitute
    fundamental degrees of freedom.

  \subsubsection{Planck Scale and \(\chi\)-Field Parameters}
    \label{sec:planck_scale_chi}

    The relationship between the \(\chi\)-field parameters and the Planck scale \(L_P = \sqrt{\hbar G / c^3}\) is **not fundamental**, but arises in regimes where both quantum and gravitational descriptions become applicable. Here, we clarify this connection to avoid misinterpretation:

    \begin{enumerate}
      \item **Emergent Gravity from \(\chi\)**:
      In the weak-field limit, the effective gravitational coupling \(G_{\text{eff}}\) is related to the \(\chi\)-field parameters by (Section 5.3):
      \[
        G_{\text{eff}} \sim \frac{c^4}{K_0 \chi_c^2}.
      \]
      This is **not a derivation of \(G\)**, but an identification of how gravitational phenomena emerge from \(\chi\) dynamics.

      \item **Planck Scale as a Derived Quantity**:
      Combining the above with \(\hbar_{\chi} = c^3 / (K_0 \chi_c)\), we find:
      \[
        \frac{\hbar_{\chi}}{L_P^2 c} = \frac{c^3 / (K_0 \chi_c)}{(\hbar G / c^3) c} = \frac{c^5}{\hbar G K_0 \chi_c} \sim \mathcal{O}(1).
      \]
      This **dimensionless ratio** suggests that:
      \begin{itemize}
        \item The \(\chi\)-field parameters \(K_0\) and \(\chi_c\) are **compatible with Planck-scale physics**,
        \item But **do not require** the Planck scale for their definition.
      \end{itemize}

      \item **Conceptual Distinction**:
      Unlike in quantum gravity approaches, where \(L_P\) is fundamental, here:
      \begin{itemize}
        \item \(L_P\) is an **emergent scale** in regimes where both quantum and gravitational descriptions apply,
        \item The fundamental scales are \(K_0\), \(\chi_c\), and \(c\), which **precede** the emergence of spacetime and quantum mechanics.
      \end{itemize}
    \end{enumerate}

    \noindent
    \textbf{Key Message}:
    The connection to \(L_P\) is a **consistency check**, not a foundational element of Cosmochrony. It demonstrates that the framework is **compatible with** (but not dependent on) Planck-scale physics.


  \subsection{Topological Stability of Relational $\chi$ Configurations}
    \label{app:relational_topological_stability}

    In a fully relational formulation of Cosmochrony, particle-like excitations are
    identified with nontrivial, localized configurations of the $\chi$ field within
    its internal configuration space, rather than as objects embedded in a
    pre-existing spacetime manifold.
    Their stability arises from intrinsic topological constraints that obstruct any
    continuous relaxation into the homogeneous vacuum configuration.

    Unlike conventional field theories, where topological invariants are defined
    with respect to spatial geometry, the invariants relevant here are purely
    internal to the relational structure of $\chi$.
    They characterize inequivalent classes of configurations that cannot be
    continuously transformed into one another without discontinuity or singular
    reorganization of the field.

    Such relational topological structures may be heuristically described using
    geometric metaphors—such as knots, twists, or vortices—but these should be
    understood as representations in an emergent descriptive language, not as
    fundamental spatial entities.
    At the relational level, stability is encoded in global consistency conditions
    on the internal correlations of $\chi$.

    A simple illustrative example is provided by configurations exhibiting
    $4\pi$-periodic internal phase structure.
    These configurations cannot be continuously unwound into a trivial state and
    naturally give rise to fermion-like behavior in effective descriptions.
    More generally, distinct particle families may correspond to inequivalent
    topological sectors of the relational $\chi$ configuration space.

    The energetic cost associated with deforming such configurations is determined
    by the internal stress of the $\chi$ field, quantified by its resistance to
    relaxation.
    This provides a unified origin for particle mass, stability, and identity,
    without invoking externally imposed charges or symmetries.

    Importantly, this relational-topological picture does not require a unique or
    exhaustive classification of all possible configurations.
    It should be understood as a concrete realization of the ontological framework
    underlying Cosmochrony, illustrating how particle properties may emerge from the
    internal organization of $\chi$, while remaining compatible with the effective
    geometric and dynamical descriptions developed in the main text.

  \subsection{Topological Origin of Fermionic and Bosonic Statistics}
    \label{app:relational_spin_statistics}

    In a fully relational formulation, the distinction between fermionic and bosonic
    excitations arises from the internal topological structure of localized $\chi$
    configurations, rather than from imposed quantum statistics.
    The relevant notion of rotation is not defined in physical space, but within the
    internal configuration space of $\chi$.

    Certain classes of configurations require a $4\pi$ internal phase rotation to
    return to an equivalent state.
    Such configurations are topologically twisted and cannot be continuously
    reoriented through a $2\pi$ transformation.
    In effective descriptions, these structures exhibit fermion-like behavior,
    including the characteristic sign change under $2\pi$ rotation and the need for
    a full $4\pi$ cycle to restore equivalence.

    Other configurations are $2\pi$-periodic and correspond to orientable internal
    structures.
    These give rise to boson-like excitations in effective geometric descriptions.
    The distinction between the two classes is therefore topological and relational,
    not dynamical or statistical in origin.

    Geometric metaphors such as Möbius twists, knots, or non-orientable loops may be
    used to visualize these internal structures, but they should be understood as
    illustrative representations valid only once an effective spatial description
    has emerged.
    At the fundamental level, the distinction is encoded in the global consistency
    conditions of the $\chi$ configuration.

    This relational-topological perspective provides a natural qualitative
    explanation of the spin--statistics connection.
    While it does not constitute a proof in the axiomatic sense, it shows how the
    observed fermionic and bosonic behavior may arise from the internal organization
    of $\chi$ without introducing independent quantum postulates or fundamental
    spin degrees of freedom.

  \subsection{Vacuum Energy versus Relaxation Capacity of the $\chi$ Field}
    \label{subsec:vacuum-energy-relaxation-capacity}

    In conventional quantum field theory, the notion of \emph{vacuum energy} refers to
    a non-vanishing energy density associated with zero-point fluctuations of quantum
    fields.
    This quantity is typically treated as an extensive, locally defined property of
    spacetime, contributing directly to the stress--energy tensor and therefore to
    gravitation.
    Its naive application leads to the well-known cosmological constant problem.

    In Cosmochrony, no such fundamental vacuum energy density is postulated.
    The $\chi$ field does not possess an intrinsic, additive energy content in the
    absence of interactions or constraints.
    Instead, what is commonly interpreted as vacuum energy is reinterpreted as a
    \emph{relational relaxation capacity} of the $\chi$ field.

    This relaxation capacity characterizes the potential of $\chi$ to undergo further
    structural reconfiguration.
    It is not a local scalar density, but a contextual and non-extensive property that
    only becomes physically meaningful when relational constraints are imposed.
    In the absence of boundaries, matter excitations, or topological obstructions,
    this capacity has no observable manifestation.

    The Casimir effect provides a paradigmatic illustration of this distinction.
    In the Cosmochrony framework, the presence of conducting boundaries constrains the
    admissible modes of $\chi$ relaxation between plates.
    The resulting force does not arise from an absolute vacuum energy stored in the
    intervening region, but from a differential in relaxation capacity between
    constrained and unconstrained configurations of the field.

    Because relaxation capacity is inherently relational and does not correspond to a
    uniform energy density permeating spacetime, it does not gravitate in the manner
    predicted by standard vacuum energy arguments.
    This resolves, at the conceptual level, the tension between observable vacuum
    phenomena and the absence of an enormous cosmological constant.

    In this sense, Cosmochrony does not deny the physical reality of vacuum-related
    effects.
    Rather, it reclassifies them as manifestations of the dynamical structure of the
    $\chi$ field under constraint, eliminating the need for a fundamental vacuum
    energy while preserving all empirically verified phenomena.

  \subsection{Conceptual Positioning with Respect to Existing Frameworks}
    \label{app:conceptual-positioning}

    For clarity, Table~\ref{tab:conceptual-comparison} provides a high-level overview
    of how the relational $\chi$ framework is positioned with respect to several
    established theoretical approaches.
    The comparison is intended to highlight differences in ontological organization
    and scope, rather than empirical adequacy or predictive performance.

    \begin{table}[htbp]
      \centering
      \renewcommand{\arraystretch}{1.25}
      \begin{tabular}{|p{3.4cm}|p{3.6cm}|p{3.6cm}|p{3.8cm}|}
        \hline
        \textbf{Conceptual Aspect} &
        \textbf{Quantum Formalism (QM / QFT)} &
        \textbf{Geometric Gravity (GR / related)} &
        \textbf{Cosmochrony} \\
        \hline
        Primary ontology &
        Quantum states and fields &
        Spacetime geometry &
        Relational scalar substrate $\chi$ \\
        \hline
        Status of spacetime &
        Fixed or effective background &
        Fundamental dynamical entity &
        Emergent effective description \\
        \hline
        Nature of time &
        External parameter or operator &
        Coordinate-dependent &
        Intrinsic ordering via relaxation \\
        \hline
        Gravitation &
        Not fundamental &
        Metric curvature &
        Collective slowdown of $\chi$ relaxation \\
        \hline
        Quantum behavior &
        Postulated formalism &
        Externally imposed or emergent &
        Emergent from $\chi$ excitations \\
        \hline
        Vacuum structure &
        Zero-point fluctuations &
        Geometric ground state &
        Contextual relaxation capacity \\
        \hline
        Particle ontology &
        Fundamental entities &
        Geometric excitations &
        Topological $\chi$ configurations \\
        \hline
        Cosmic expansion &
        Not addressed &
        Requires matter/energy content &
        Geometric unfolding of $\chi$ \\
        \hline
        Inflation / initial conditions &
        Not addressed &
        External mechanism &
        Not required (pre-geometric continuity) \\
        \hline
        Empirical status &
        Highly successful &
        Highly successful &
        Exploratory \\
        \hline
      \end{tabular}
      \caption{High-level conceptual positioning of the relational $\chi$ framework
      with respect to established quantum and geometric approaches.
      The comparison emphasizes ontological structure rather than empirical validity.}
      \label{tab:conceptual-comparison}
    \end{table}
