\clearpage
\section{Relational Formulation of \texorpdfstring{$\chi$}{χ} Dynamics}
  \label{app:relational_formulation}

  This appendix develops a fully relational and explicitly non-geometric formulation
  of the dynamics of the \(\chi\) field.
  Its purpose is to make explicit the ontological foundations underlying the
  Cosmochrony framework, independently of any effective spacetime or metric-based
  description.

  The constructions presented here are not required for the operational,
  projected dynamics discussed in the main text.
  Rather, they serve to clarify how particle-like properties, topological stability,
  and quantum correlations may arise from the intrinsic relational structure of
  \(\chi\), prior to the emergence of geometry.
  In this sense, the appendix complements—but does not extend—the effective
  dynamical framework developed elsewhere.

  \paragraph{Status and scope.}
    The relational formulation does not assume:
    \begin{itemize}
      \item a background spacetime or metric,
      \item spatial localization or distance,
      \item a tensorial or spinorial fundamental ontology,
      \item or an underlying Hilbert space structure.
    \end{itemize}

    Instead, it treats \(\chi\) as a single relational substrate whose configurations
    are defined entirely by internal structural relations and bounded relaxation
    constraints.
    Concepts such as position, duration, causal order, and particle identity emerge
    only at the level of effective projection and are therefore absent from the
    foundational description presented here.

  \paragraph{Relational origin of particle properties.}
    Within this framework, particle-like excitations are identified with internally
    stable relational configurations of \(\chi\).
    Their apparent properties—such as mass, charge, spin, and statistics—are not
    assigned a priori.
    They emerge from topological and organizational features of relational
    configurations once a projectable regime becomes applicable.

    The constructions developed in this appendix illustrate how:
    \begin{itemize}
      \item particle identity corresponds to relational equivalence classes,
      \item charge reflects oriented asymmetries in relaxation constraints,
      \item spin arises from nontrivial transformation properties of configuration
      space,
      \item fermionic and bosonic behavior follow from topological obstructions to
      continuous factorization.
    \end{itemize}

    These mechanisms are presented as existence proofs rather than as a unique or
    exhaustive classification of physical particles.

  \paragraph{Relation to quantum phenomena.}
    Several core features of quantum physics—most notably non-factorization,
    entanglement, and spin--statistics correlations—acquire a natural interpretation
    within the relational formulation.
    In Cosmochrony, these phenomena are not imposed through quantization rules.
    They reflect the holistic structure of \(\chi\) configurations that cannot be
    decomposed into independent subsystems once relationally coupled.

    This perspective aligns with, but is distinct from, other non-local or
    pre-geometric approaches.
    It emphasizes structural coherence rather than signal propagation or information
    exchange.

  \paragraph{Relation to effective geometric descriptions.}
    The relational formulation provides the ontological underpinning for the effective
    geometric descriptions introduced elsewhere in the paper.
    Once coarse-graining and projection become valid, relational configurations admit
    approximate geometric representations in terms of fields on spacetime.
    The correspondence between the relational and geometric levels is many-to-one and
    regime-dependent.

    Importantly, no contradiction arises between the relational and geometric
    descriptions.
    They apply to different descriptive levels of the same underlying dynamics.

  \paragraph{Purpose of the appendix.}
    This appendix serves three complementary roles:
    \begin{itemize}
      \item to demonstrate that Cosmochrony admits a fully non-geometric formulation,
      \item to clarify the ontological meaning of particle-like excitations and quantum
      correlations,
      \item to prevent misinterpretations that would reintroduce spacetime or quantum
      postulates at the fundamental level.
    \end{itemize}

    Readers interested primarily in phenomenology or effective dynamics may skip this
    appendix without loss of continuity.
    Readers concerned with the conceptual foundations and internal coherence of the
    framework may find the relational formulation essential.

    \subsection{Relational Configurations of \texorpdfstring{$\chi$}{χ}}
  \label{subsec:relational-configurations-of-chi}

  At the most fundamental level, the \(\chi\) field is not defined as a function on
  spacetime, nor as a field assigned to points of a manifold.
  It is instead specified as a complete relational configuration, characterized
  entirely by internal structural relations.
  No coordinates, distances, durations, or background geometric notions are assumed
  or required.

  A relational configuration of \(\chi\) is defined by the pattern of mutual
  constraints governing its relaxation structure.
  Two configurations are distinct if and only if they differ in their internal
  relational organization.
  Conversely, configurations that are related by a global reparameterization or
  relaxation-preserving transformation are considered physically equivalent.

  Within this formulation, there is no primitive notion of localization.
  Concepts such as position, separation, or spatial extension have no meaning at
  the relational level.
  What later appears as spatial organization arises only when a configuration
  admits a projectable regime in which geometric descriptors become effective.

  \paragraph{Configuration space and relational equivalence.}
    The space of all admissible \(\chi\) configurations may be viewed as an abstract
    configuration space equipped with equivalence classes defined by relational
    symmetries.
    Physical states correspond to equivalence classes of configurations rather than
    to individual realizations.

    This perspective replaces geometric invariance with relational invariance:
    transformations that preserve the internal relaxation structure of \(\chi\) leave
    the physical content unchanged, even if they would correspond to nontrivial
    coordinate transformations in an emergent geometric description.

  \paragraph{Absence of factorization.}
    Because \(\chi\) is fundamentally relational, its configurations do not generally
    factorize into independent subsystems.
    What appears as a composite system in an effective spacetime description may
    correspond to a single, indivisible relational configuration at the fundamental
    level.

    This absence of factorization is not a dynamical interaction but a structural
    property of the configuration space itself.
    It underlies the emergence of nonlocal correlations and provides the foundation
    for the discussion of entanglement in
    Section~\ref{subsec:non-factorization-and-entanglement}.

  \paragraph{From relational structure to effective description.}
    Only under specific conditions—such as approximate homogeneity, bounded gradients,
    and stable relaxation regimes—does a relational configuration admit an effective
    projection onto a geometric description.
    In that regime, relational distinctions are mapped onto spatial relations,
    durations, and causal ordering.

    Importantly, this mapping is many-to-one and inherently approximate.
    Different relational configurations may correspond to the same effective geometric
    state, and no inverse mapping is defined.
    As a result, the relational formulation provides the ontological foundation of
    Cosmochrony, while effective geometric descriptions serve as emergent,
    context-dependent representations.

  \paragraph{Conceptual role.}
    This relational perspective establishes that the fundamental content of
    Cosmochrony resides entirely in the internal organization of the \(\chi\)
    configuration.
    All subsequent notions—particles, fields, spacetime geometry, and quantum
    correlations—are secondary constructs arising from specific regimes of relational
    organization.

    The purpose of this subsection is therefore not to introduce additional structure,
    but to make explicit the non-geometric and non-local ontology on which the rest of
    the framework is built.

    \input{E-relational_formulation/E02-non-factorization-and-entanglement}
    \subsection{Locality, Causality, and the Role of the Bound \texorpdfstring{$c$}{c}}
  \label{subsec:locality-causality-and-the-role-of-the-bound-c}

  Within the relational formulation of Cosmochrony, correlations between
  configurations of the \(\chi\) field may extend across arbitrarily large effective
  distances once a geometric description becomes applicable.
  However, the existence of such correlations does not imply unrestricted dynamical
  influence.
  All modifications of relational configurations are constrained by a universal
  kinematic bound, denoted by \(c\).

  \paragraph{Role of the bound \texorpdfstring{$c$}{c}.}
    The constant \(c\) does not represent a signal velocity in a pre-existing
    spacetime.
    At the relational level, it defines a fundamental upper bound on the rate at which
    the internal structure of a \(\chi\) configuration may change.
    This bound is encoded directly in the relaxation constraints governing the
    evolution of \(\chi\).

    As a consequence, while relational configurations may be globally defined and
    non-factorizable, their evolution remains locally constrained once an effective
    notion of causality emerges.
    No relational reconfiguration can induce arbitrarily rapid changes in projected
    descriptions.

  \paragraph{Emergence of locality.}
    Locality is not assumed at the fundamental level.
    It arises only when relational configurations admit a projectable regime in which
    spatial organization becomes meaningful.
    In such regimes, the bound \(c\) manifests operationally as a maximum propagation
    speed for disturbances in the projected field \(\chi_{\mathrm{eff}}\).

    This emergent locality ensures that effective interactions respect relativistic
    causal ordering.
    Dynamical influences propagate continuously and are limited by the same invariant
    speed that governs relativistic kinematics.

  \paragraph{Compatibility with entanglement.}
    The coexistence of global relational correlations and bounded dynamical evolution
    resolves the apparent tension between quantum entanglement and relativistic
    causality.
    Entangled configurations correspond to non-factorizable relational structures that
    are globally specified.
    Their correlated measurement outcomes do not result from superluminal influences,
    but from the consistency constraints imposed by a shared relational configuration.

    Because no dynamical update propagates between subsystems during measurement,
    entanglement does not violate the causal bound \(c\).
    All observable dynamical effects remain subluminal in the emergent geometric
    description.

  \paragraph{Causality as a projected concept.}
    Causality, in Cosmochrony, is not a primitive feature of the relational ontology.
    It is a property of the effective geometric description that emerges once temporal
    ordering and spatial separation become meaningful.
    The bound \(c\) ensures that this emergent causal structure is well-defined and
    internally consistent.

    Thus, Cosmochrony distinguishes sharply between:
    \begin{itemize}
      \item \emph{relational correlations}, which may be global and nonlocal in a
      geometric sense, and
      \item \emph{dynamical causation}, which is always constrained by the bound \(c\)
      once projected.
    \end{itemize}

  \paragraph{Conceptual role.}
    This distinction clarifies how Cosmochrony accommodates both the holistic features
    of quantum phenomena and the strict causal structure of relativistic physics
    without contradiction.
    The bound \(c\) acts as the bridge between the relational and geometric levels,
    ensuring that emergent spacetime dynamics remain causal even though the underlying
    ontology is non-geometric and nonlocal.

    The present subsection therefore establishes the foundation for reconciling
    entanglement, relativistic causality, and the emergence of spacetime within a
    single coherent framework.

    \input{E-relational_formulation/04-relational-distance}
    \input{E-relational_formulation/05-chi-eff-derivation}
    \input{E-relational_formulation/06-relation-to-the-effective-geometric-description}
    \input{E-relational_formulation/07-emergent-coordinates}
    \input{E-relational_formulation/08-relational_topological_stability}
    \input{E-relational_formulation/09-relational_spin_statistics}
    \input{E-relational_formulation/10-vacuum-energy-relaxation-capacity}
    \subsection{Conceptual Positioning with Respect to Existing Frameworks}
  \label{app:conceptual-positioning}

  This subsection situates the relational \(\chi\) framework with respect to several
  established theoretical approaches.
  The purpose of this comparison is strictly conceptual.
  It aims to clarify differences in ontological commitments, explanatory strategy,
  and scope, rather than to assess empirical adequacy or predictive performance.

  Cosmochrony is not presented as a direct competitor to quantum mechanics,
  quantum field theory, or general relativity.
  Instead, it is positioned as a foundational framework intended to underlie and
  contextualize these effective theories by identifying a deeper, pre-geometric
  level of description.

  \paragraph{Scope of the comparison.}
    The comparison emphasizes:
    \begin{itemize}
      \item what is taken as fundamental in each framework,
      \item how spacetime and geometry are treated,
      \item the status of time, particles, and vacuum structure,
      \item and the role of initial conditions and large-scale coherence.
    \end{itemize}

    No claim is made that Cosmochrony currently matches the quantitative success of
    established theories.
    Its empirical status remains exploratory, and its primary contribution at this
    stage is conceptual unification and reinterpretation.

    \begin{table}[htbp]
      \centering
      \renewcommand{\arraystretch}{1.25}
      \begin{tabular}{|p{3.6cm}|p{3.8cm}|p{3.8cm}|p{4.0cm}|}
        \hline
        \textbf{Conceptual Aspect} &
        \textbf{Quantum Formalism (QM / QFT)} &
        \textbf{Geometric Gravity (GR and extensions)} &
        \textbf{Cosmochrony} \\
        \hline
        Primary ontology &
        Quantum states and operator-valued fields &
        Spacetime geometry and metric structure &
        Relational scalar substrate \(\chi\) \\
        \hline
        Status of spacetime &
        Fixed background or effective stage &
        Fundamental dynamical entity &
        Emergent, projective description \\
        \hline
        Nature of time &
        External parameter or operator &
        Coordinate-dependent geometric quantity &
        Intrinsic ordering via relaxation of \(\chi\) \\
        \hline
        Gravitation &
        Not fundamental; introduced externally &
        Manifestation of metric curvature &
        Collective slowdown of \(\chi\) relaxation \\
        \hline
        Quantum behavior &
        Postulated formal structure &
        Added or emergent from quantization &
        Emergent from relational \(\chi\) configurations \\
        \hline
        Vacuum structure &
        Zero-point energy of quantum fields &
        Geometric ground state &
        Contextual relaxation capacity \\
        \hline
        Particle ontology &
        Fundamental entities or excitations &
        Geometric or field excitations &
        Topologically stable relational configurations \\
        \hline
        Cosmic expansion &
        Not addressed intrinsically &
        Requires matter/energy content &
        Emergent geometric unfolding of \(\chi\) \\
        \hline
        Inflation / initial conditions &
        Outside scope &
        Requires external mechanisms &
        Not required (pre-geometric continuity) \\
        \hline
        Empirical status &
        Highly successful &
        Highly successful &
        Exploratory and foundational \\
        \hline
      \end{tabular}
      \caption{High-level conceptual positioning of the relational \(\chi\) framework
      with respect to quantum and geometric approaches.
      The comparison highlights ontological structure and explanatory strategy rather
      than empirical validation.}
      \label{tab:conceptual-comparison}
    \end{table}

  \paragraph{Interpretive caution.}
    The similarities highlighted in this table—such as the emergence of geometry,
    the recovery of relativistic causality, or the appearance of quantum correlations—
    should not be interpreted as equivalence.
    Cosmochrony deliberately refrains from adopting the formal postulates of either
    quantum mechanics or general relativity at the fundamental level.

    Conversely, differences in ontology do not imply incompatibility.
    The effective regimes of Cosmochrony are constructed precisely so that standard
    quantum and geometric descriptions are recovered where they are empirically
    validated.

  \paragraph{Conceptual contribution.}
    The distinctive contribution of Cosmochrony lies in its attempt to:
    \begin{itemize}
      \item unify quantum, gravitational, and cosmological phenomena within a single
      relational substrate,
      \item eliminate the need for independent postulates for spacetime, quantum
      statistics, and vacuum energy,
      \item and reinterpret long-standing conceptual tensions as artifacts of applying
      effective descriptions beyond their domain of validity.
    \end{itemize}

    In this sense, Cosmochrony should be viewed as a foundational and exploratory
    framework.
    Its role is to provide a coherent ontological backdrop against which established
    theories may be understood as complementary, regime-dependent descriptions rather
    than as mutually incompatible fundamentals.

  \paragraph{Framework-level comparison.}
    While Table~\ref{tab:conceptual-comparison} contrasts broad ontological commitments,
    the following comparison focuses on cosmological and gravitational frameworks
    commonly discussed in the literature.

    \begin{table}[htbp]
      \centering
      \caption{Cosmological and gravitational comparison between Cosmochrony,
        the standard $\Lambda$CDM model, and Loop Quantum Gravity (LQG).}
      \label{tab:comparison_cosmochrony}
      \renewcommand{\arraystretch}{1.2}
      \begin{tabular}{|p{4.2cm}|p{3.5cm}|p{3.5cm}|p{3.5cm}|}
        \hline
        \textbf{Aspect} &
        \textbf{Cosmochrony} &
        \textbf{$\Lambda$CDM} &
        \textbf{LQG} \\
        \hline

        Fundamental ontology &
        Single pre-geometric relational substrate $\chi$ &
        Spacetime metric $g_{\mu\nu}$ + matter fields + $\Lambda$ &
        Quantum geometry (spin networks, holonomies) \\
        \hline

        Status of spacetime &
        Emergent, projected, non-fundamental &
        Fundamental background (classical) &
        Quantized but kinematically assumed \\
        \hline

        Degrees of freedom (fundamental) &
        One relational entity (no local field DOF) &
        Metric DOF + multiple matter fields &
        Discrete graph-based DOF \\
        \hline

        Nature of time &
        Emergent ordering from irreversible relaxation &
        External parameter &
        Problematic / relational \\
        \hline

        Quantum gravity &
        Emergent from projection and spectral constraints &
        Absent &
        Fundamental (background independent) \\
        \hline

        Dark energy &
        Not required (cosmic expansion from $\chi$ relaxation) &
        Explicit $\Lambda$ term &
        No consensus (emergent or absent) \\
        \hline

        Inflation &
        Not required &
        Required (inflaton field) &
        Alternative scenarios \\
        \hline

        Origin of mass &
        Spectral inhibition of relaxation &
        Higgs mechanism &
        Not intrinsic \\
        \hline

        Discreteness &
        Projective and spectral (non-injective projection) &
        None &
        Fundamental (Planck-scale geometry) \\
        \hline

        Testable predictions &
        $H(z)$ evolution, CMB low-$\ell$ anomalies, redshift drift &
        $H_0$, $w$, structure growth &
        Planck-scale signatures, area spectra \\
        \hline

        Primary conceptual aim &
        Ontological unification of time, geometry, and matter &
        Phenomenological concordance model &
        Quantization of spacetime geometry \\
        \hline
      \end{tabular}
    \end{table}

