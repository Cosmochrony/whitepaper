\subsection{Conceptual Positioning with Respect to Existing Frameworks}
  \label{app:conceptual-positioning}

  This subsection situates the relational \(\chi\) framework with respect to several
  established theoretical approaches.
  The purpose of this comparison is strictly conceptual.
  It aims to clarify differences in ontological commitments, explanatory strategy,
  and scope, rather than to assess empirical adequacy or predictive performance.

  Cosmochrony is not presented as a direct competitor to quantum mechanics,
  quantum field theory, or general relativity.
  Instead, it is positioned as a foundational framework intended to underlie and
  contextualize these effective theories by identifying a deeper, pre-geometric
  level of description.

  \paragraph{Scope of the comparison.}
    The comparison emphasizes:
    \begin{itemize}
      \item what is taken as fundamental in each framework,
      \item how spacetime and geometry are treated,
      \item the status of time, particles, and vacuum structure,
      \item and the role of initial conditions and large-scale coherence.
    \end{itemize}

    No claim is made that Cosmochrony currently matches the quantitative success of
    established theories.
    Its empirical status remains exploratory, and its primary contribution at this
    stage is conceptual unification and reinterpretation.

    \begin{table}[htbp]
      \centering
      \renewcommand{\arraystretch}{1.25}
      \begin{tabular}{|p{3.6cm}|p{3.8cm}|p{3.8cm}|p{4.0cm}|}
        \hline
        \textbf{Conceptual Aspect} &
        \textbf{Quantum Formalism (QM / QFT)} &
        \textbf{Geometric Gravity (GR and extensions)} &
        \textbf{Cosmochrony} \\
        \hline
        Primary ontology &
        Quantum states and operator-valued fields &
        Spacetime geometry and metric structure &
        Relational scalar substrate \(\chi\) \\
        \hline
        Status of spacetime &
        Fixed background or effective stage &
        Fundamental dynamical entity &
        Emergent, projective description \\
        \hline
        Nature of time &
        External parameter or operator &
        Coordinate-dependent geometric quantity &
        Intrinsic ordering via relaxation of \(\chi\) \\
        \hline
        Gravitation &
        Not fundamental; introduced externally &
        Manifestation of metric curvature &
        Collective slowdown of \(\chi\) relaxation \\
        \hline
        Quantum behavior &
        Postulated formal structure &
        Added or emergent from quantization &
        Emergent from relational \(\chi\) configurations \\
        \hline
        Vacuum structure &
        Zero-point energy of quantum fields &
        Geometric ground state &
        Contextual relaxation capacity \\
        \hline
        Particle ontology &
        Fundamental entities or excitations &
        Geometric or field excitations &
        Topologically stable relational configurations \\
        \hline
        Cosmic expansion &
        Not addressed intrinsically &
        Requires matter/energy content &
        Emergent geometric unfolding of \(\chi\) \\
        \hline
        Inflation / initial conditions &
        Outside scope &
        Requires external mechanisms &
        Not required (pre-geometric continuity) \\
        \hline
        Empirical status &
        Highly successful &
        Highly successful &
        Exploratory and foundational \\
        \hline
      \end{tabular}
      \caption{High-level conceptual positioning of the relational \(\chi\) framework
      with respect to quantum and geometric approaches.
      The comparison highlights ontological structure and explanatory strategy rather
      than empirical validation.}
      \label{tab:conceptual-comparison}
    \end{table}

  \paragraph{Interpretive caution.}
    The similarities highlighted in this table—such as the emergence of geometry,
    the recovery of relativistic causality, or the appearance of quantum correlations—
    should not be interpreted as equivalence.
    Cosmochrony deliberately refrains from adopting the formal postulates of either
    quantum mechanics or general relativity at the fundamental level.

    Conversely, differences in ontology do not imply incompatibility.
    The effective regimes of Cosmochrony are constructed precisely so that standard
    quantum and geometric descriptions are recovered where they are empirically
    validated.

  \paragraph{Conceptual contribution.}
    The distinctive contribution of Cosmochrony lies in its attempt to:
    \begin{itemize}
      \item unify quantum, gravitational, and cosmological phenomena within a single
      relational substrate,
      \item eliminate the need for independent postulates for spacetime, quantum
      statistics, and vacuum energy,
      \item and reinterpret long-standing conceptual tensions as artifacts of applying
      effective descriptions beyond their domain of validity.
    \end{itemize}

    In this sense, Cosmochrony should be viewed as a foundational and exploratory
    framework.
    Its role is to provide a coherent ontological backdrop against which established
    theories may be understood as complementary, regime-dependent descriptions rather
    than as mutually incompatible fundamentals.
