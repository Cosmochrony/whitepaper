\subsection{Vacuum Energy versus Relaxation Capacity of the \texorpdfstring{$\chi$}{χ} Field}
  \label{subsec:vacuum-energy-relaxation-capacity}

  In conventional quantum field theory, the notion of \emph{vacuum energy} refers to a
  non-vanishing energy density associated with zero-point fluctuations of quantum
  fields.
  This quantity is treated as a locally defined, extensive property of spacetime and
  is assumed to contribute directly to the stress--energy tensor.
  When extrapolated to cosmological scales, this interpretation leads to the well-known
  cosmological constant problem.

  In Cosmochrony, no fundamental vacuum energy density is postulated.
  The \(\chi\) field does not carry an intrinsic additive energy in the absence of
  constraints or excitations.
  Instead, phenomena commonly attributed to vacuum energy are reinterpreted in terms
  of the \emph{relaxation capacity} of the \(\chi\) field.

  \paragraph{Relaxation capacity as a relational notion.}
    Relaxation capacity characterizes the ability of a relational configuration of
    \(\chi\) to undergo further structural reorganization under the bounded relaxation
    constraints.
    It is not a local scalar density and cannot be meaningfully assigned to spacetime
    points.
    Rather, it is a contextual and non-extensive property of an entire relational
    configuration.

    In the absence of matter excitations, boundaries, or topological obstructions, this
    capacity has no observable manifestation.
    A perfectly unconstrained configuration corresponds to a state in which relaxation
    capacity is uniform and physically inert.

  \paragraph{Constraints and observable vacuum effects.}
    Observable vacuum phenomena arise only when relational constraints restrict the
    space of admissible \(\chi\) configurations.
    Boundaries, material structures, or imposed conditions modify the allowed patterns
    of relaxation and thereby redistribute relaxation capacity.

    The Casimir effect provides a paradigmatic illustration.
    Within the Cosmochrony framework, the presence of conducting plates constrains the
    admissible relational configurations of \(\chi\) between them.
    The resulting force does not originate from an absolute vacuum energy stored in the
    intervening region.
    It arises from a differential in relaxation capacity between constrained and
    unconstrained configurations.

    This interpretation preserves the empirically observed magnitude and sign of the
    effect while eliminating the need to attribute a large, homogeneous energy density
    to empty space.

  \paragraph{Gravitational implications.}
    Because relaxation capacity is inherently relational and non-extensive, it does not
    enter gravitational dynamics as a uniform source term.
    Only changes in the relaxation structure induced by localized excitations or
    topological constraints contribute to effective gravitational behavior.

    As a result, there is no reason for relaxation capacity to gravitate in the manner
    predicted by standard vacuum energy arguments.
    This provides a natural conceptual resolution of the cosmological constant problem:
    the enormous vacuum energy inferred from zero-point counting is not a physically
    meaningful quantity in the Cosmochrony ontology.

  \paragraph{Conceptual reinterpretation of the vacuum.}
    Cosmochrony does not deny the physical reality of vacuum-related phenomena.
    Rather, it reclassifies them as manifestations of constrained relational dynamics of
    the \(\chi\) field.
    What appears as vacuum energy in effective descriptions corresponds, at the
    fundamental level, to differences in relaxation capacity between relational
    configurations.

    In this view, the vacuum is not an energetic substance filling spacetime, but a
    relational state whose physical relevance emerges only through constraints.
    This reinterpretation preserves all empirically verified vacuum effects while
    eliminating the need for a fundamental vacuum energy density or a finely tuned
    cosmological constant.
