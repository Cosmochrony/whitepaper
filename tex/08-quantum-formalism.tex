This section does not assign fundamental ontological status to the quantum wavefunction
or to Hilbert space structures.
Instead, it shows how the formal apparatus of quantum mechanics can be understood
as an effective framework emerging from the dynamics of localized and weakly interacting
$\chi$-field excitations described in the preceding sections.

\subsection{Status of the Wavefunction}\label{subsec:status-of-the-wavefunction}

  In standard quantum mechanics, the wavefunction $\psi$
  is a complex-valued object defined on configuration space, whose ontological status remains debated.
  Operationally, $|\psi|^2$ encodes measurement probabilities via the Born rule, while $\psi$
  itself does not correspond directly to a physical field in spacetime.

  In cosmochrony, the $\chi$ field is not identified with the quantum wavefunction.
  Instead, $\chi$
  constitutes a real geometric substratum defined on spacetime, from which effective quantum wavefunctions
  emerge as coarse-grained descriptions of localized excitations.
  The quantum wavefunction is thus interpreted as a derived object encoding the statistical behavior of $\chi$
  -mediated structures rather than as a fundamental entity.

  As an example, the hydrogen atom's wavefunction $\psi_{nlm}(r, \theta, \phi)$
  corresponds to a stable solitonic configuration of $\chi$ with radial nodes $n$, angular momentum $l$,
  and magnetic quantum number $m$. The probability density $|\psi|^2$ reflects the spatial distribution of
  $\chi$-curvature, while energy quantization $E_n \propto -1/n^2$
  arises from the discrete topological winding numbers permitted by the boundary conditions at $r \to 0$ and
  $r \to \infty$.

\subsection{Emergence of Hilbert Space Structure}\label{subsec:emergence-of-hilbert-space-structure}

  The Hilbert space formalism of quantum mechanics provides a linear structure supporting superposition,
  interference, and unitary evolution.
  Within cosmochrony, this structure arises as an effective description of weakly interacting excitations of the
  $\chi$ field.

  Linear superposition reflects the approximate independence of small-amplitude perturbations propagating on a
  slowly varying $\chi$ background.
  The complex phase of the wavefunction encodes relative geometric shifts within the underlying $\chi$
  oscillations rather than representing an intrinsic complex field.

\subsection{Emergence of the Schrödinger Equation from $\chi$ Fluctuations}
  \label{sec:schrodinger_emergence}

  In Cosmochrony, quantum behavior is not postulated but emerges as an effective,
  long-wavelength description of small fluctuations of the fundamental $\chi$ field
  around stable solitonic configurations. In this section, we provide an explicit and
  standard derivation of the Schr\"odinger equation as the non-relativistic limit of such
  fluctuations, making all approximations transparent.

  \subsubsection{Non-relativistic limit: Klein--Gordon $\rightarrow$ Schr\"odinger}
    \label{sec:KGtoSch}

    Consider a localized, massive excitation of the Cosmochrony field around a quasi-stationary soliton
    background,
    \begin{equation}
      \chi(x,t)=\chi_{\mathrm{sol}}(x)+\delta\chi(x,t),
    \end{equation}
    and assume that, to leading order in small fluctuations, $\delta\chi$ obeys an effective
    Klein--Gordon equation with a mass scale $m$ set by the soliton's rest-energy:
    \begin{equation}
      \left(\frac{1}{c^{2}}\partial_{t}^{2}-\nabla^{2}+\frac{m^{2}c^{2}}{\hbar^{2}}\right)\delta\chi=0.
      \label{eq:KG_eff}
    \end{equation}

    In the non-relativistic regime, the field oscillates rapidly at the rest-energy frequency
    $\omega_{0}=mc^{2}/\hbar$, while its envelope varies slowly. We therefore use the standard ansatz
    \begin{equation}
      \delta\chi(x,t)=\psi(x,t)\,e^{-i\omega_{0}t},
      \qquad
      \left|\partial_t\psi\right|\ll\omega_{0}|\psi|.
      \label{eq:NR_ansatz}
    \end{equation}
    Compute derivatives:
    \begin{align}
      \partial_t\delta\chi &= e^{-i\omega_0 t}\left(\partial_t\psi-i\omega_0\psi\right),\\
      \partial_t^2\delta\chi &= e^{-i\omega_0 t}
      \left(\partial_t^2\psi-2i\omega_0\partial_t\psi-\omega_0^2\psi\right),
    \end{align}
    and substitute into Eq.~\eqref{eq:KG_eff}. Using $\omega_0=mc^2/\hbar$ cancels the large rest-mass
    terms ($-\omega_0^2/c^2$ against $+m^2c^2/\hbar^2$), yielding
    \begin{equation}
      \frac{1}{c^2}\partial_t^2\psi-\frac{2i\omega_0}{c^2}\partial_t\psi-\nabla^2\psi=0.
      \label{eq:KG_env}
    \end{equation}
    Under the slow-envelope condition, $\left|\partial_t^2\psi\right|\ll\omega_0\left|\partial_t\psi\right|$
    (neglecting terms of order $v^4/c^4$), Eq.~\eqref{eq:KG_env} reduces to
    \begin{equation}
      -\frac{2i\omega_0}{c^2}\partial_t\psi-\nabla^2\psi=0
      \quad\Longrightarrow\quad
      i\hbar\,\partial_t\psi=-\frac{\hbar^2}{2m}\nabla^2\psi.
      \label{eq:Sch_free}
    \end{equation}

    A weak interaction with the surrounding $\chi$ background (or other solitons) can be encoded at the
    envelope level by an effective potential $V(x)$, giving the standard Schr\"odinger form
    \begin{equation}
      i\hbar\,\partial_t\psi=\left[-\frac{\hbar^2}{2m}\nabla^2+V(x)\right]\psi.
      \label{eq:Sch_V}
    \end{equation}

    \paragraph{Cosmochrony-specific content.}
      Equations~\eqref{eq:Sch_free}--\eqref{eq:Sch_V} establish the rigorous non-relativistic limit of a
      relativistic scalar excitation. In Cosmochrony, the program is to derive (i) the effective mass
      $m$ from soliton energetics and (ii) the form of $V(x)$ from interaction-induced deformations of
      $\chi_{\mathrm{sol}}(x)$.

  \subsubsection{Interpretation}

    In this framework, the complex wavefunction $\psi$ does not represent a fundamental
    quantum object but an effective description of coherent $\chi$-field fluctuations
    around a solitonic particle state. The Schr\"odinger equation thus appears as the
    universal non-relativistic limit of localized $\chi$ excitations, rather than as a
    fundamental postulate.

    The Cosmochrony program is to relate the effective mass and interaction potential
    entering the Schr\"odinger dynamics to soliton energetics and interaction-induced
    deformations of the $\chi$ background. A detailed derivation of these quantities from
    the full $\chi$ action is left for future work.

\subsection{Origin of Quantization}\label{subsec:origin-of-quantization}

  Quantization in standard quantum theory is postulated through canonical commutation relations or path-integral
  prescriptions.

  In cosmochrony, discrete energy exchanges arise from topological constraints on stable excitations of the $\chi$
  field.
  Only certain winding numbers, knot structures, or resonance conditions are dynamically stable, leading to
  effectively quantized energy levels.
  The relation $E = h \nu$
  emerges as a geometric proportionality between oscillation frequency and curvature energy stored in localized
  $\chi$ configurations.

\subsection{Measurement and the Born Rule}\label{subsec:measurement-and-the-born-rule}

  The measurement postulate remains one of the most conceptually opaque elements of quantum mechanics.
  In cosmochrony, measurement corresponds to a local irreversible interaction between a structured excitation and
  stochastic fluctuations of the $\chi$ field.

  Detection events occur when interference between the excitation and ambient $\chi$
  fluctuations produces a stable localized crest.
  The Born rule arises statistically from the distribution of these fluctuations, with $|\psi|^2$
  representing the density of favorable geometric configurations rather than a fundamental probability axiom.

  During a measurement, the detector's macroscopic degrees of freedom impose boundary conditions that select a
  specific topological sector of the $\chi$
  -field. For example, a photon detector absorbs energy by fixing a localized crest in $\chi$
  , effectively ``cutting'' the extended wave configuration and collapsing it to a particle-like excitation. The
  Born rule $P \propto |\psi|^2$ then follows from the statistical distribution of $\chi$
  -fluctuations that satisfy the detector's constraints, without requiring an intrinsic probabilistic postulate.

\subsection{Entanglement and Nonlocal Correlations}\label{subsec:entanglement-and-nonlocal-correlations}

  Quantum entanglement is traditionally described as nonlocal correlation between subsystems whose joint
  wavefunction cannot be factorized.

  In cosmochrony, entanglement corresponds to persistent geometric connectedness within a single extended $\chi$
  configuration.
  Separated particles remain correlated because they are manifestations of the same underlying wave segment.
  Decoherence corresponds to the progressive tearing or dispersion of this shared geometric structure due to
  environmental interactions.

  This interpretation preserves the empirical predictions of quantum mechanics while avoiding superluminal
  signaling, as no information propagates faster than the local relaxation rate of $\chi$.

\subsection{Spin and Statistics}\label{subsec:spin-and-statistics}

  Spin is treated in quantum mechanics as an intrinsic degree of freedom associated with representations of the
  rotation group.
  The necessity of $4\pi$
  rotations for fermions is usually accepted as a mathematical fact without deeper geometric explanation.

  Within cosmochrony, half-integer spin emerges from topological twists of $\chi$ excitations.
  Fermionic states correspond to M\"obius-like configurations requiring $4\pi$
  rotations to return to identity, while bosonic states correspond to untwisted or integer-winding structures.
  The spin-statistics connection follows naturally from the topological stability of these configurations.

  See an illustrated example in ~\ref{subsec:4pi_soliton}.

\subsection{Scope and Limitations}\label{subsec:scope-and-limitations}

  Cosmochrony does not aim to replace the quantum formalism.
  All standard computational tools of quantum mechanics remain valid within their domain of applicability.

  The contribution of cosmochrony is interpretative and unificatory: it proposes a geometric origin for quantum
  behavior, measurement statistics, and nonlocal correlations, without modifying experimentally tested predictions.
  Further work is required to formalize the precise mapping between $\chi$
  dynamics and operator-based quantum theory.
