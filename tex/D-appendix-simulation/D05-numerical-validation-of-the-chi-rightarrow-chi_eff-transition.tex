\subsection{Numerical validation of the \texorpdfstring{$\chi \rightarrow \chi_{\mathrm{eff}}$}{χ→χeff} transition}
  \label{subsec:numerical-validation-of-the-chi-rightarrow-chi_eff-transition}

  This subsection provides a numerical validation of the relational-to-effective transition
  $\chi \rightarrow \chi_{\mathrm{eff}}$ introduced in Appendix~E.
  The goal is not physical realism, but a constructive demonstration that an explicit
  relational relaxation rule on a discrete network admits a coarse-grained description
  whose evolution is consistent with the coarse-grained micro-dynamics in projectable
  regimes.

  \paragraph{Distinction Between Numerical Stability and Projectability.}
    The numerical validation presented in this subsection evaluates two distinct but often
    conflated properties:

    \begin{itemize}
      \item \textbf{Numerical stability}, measured by the normalized residual $\epsilon$,
      ensures that the effective field $\chi_{\text{eff}}$ converges to a quasi-stationary
      solution under iterative relaxation.
      \item This is a \emph{local} and algorithm-dependent property.

      \item \textbf{Projectability} is a \emph{geometric} property of the projection
      $\Pi : \chi \rightarrow \chi_{\text{eff}}$, requiring that relational configurations
      admit a faithful and locally injective effective description.
    \end{itemize}

    A configuration may therefore be numerically stable ($\epsilon \ll 1$) yet
    non-projectable if multiple distinct $\chi$ configurations map to the same
    $\chi_{\text{eff}}$, as occurs in strong-structure or deprojection regimes.
    The numerical diagnostics introduced below are explicitly designed to separate these
    two notions.

  \paragraph{Discrete model and operators.}
    We consider a three-dimensional cubic lattice graph with periodic boundary conditions,
    containing $N^3$ nodes and nearest-neighbor adjacency $\mathcal{N}(i)$.
    Each node $i$ carries a scalar value $\chi_i(t)$.
    All operators are defined purely in terms of neighbor relations (graph locality) and do
    not presuppose any background continuum geometry.

  \paragraph{Explicit update rule and saturation.}
    The discrete relaxation step is defined by the local slope functional
    \begin{equation}
      S_i(\chi) \;\equiv\; \frac{1}{c^2}\sum_{j\in\mathcal{N}(i)} K_{ij}\,(\chi_i-\chi_j)^2,
      \qquad
      K_{ij} \;=\; \frac{K_0}{1+(\chi_i-\chi_j)^2/\chi_c^2},
      \label{eq:D4_Si_Kij}
    \end{equation}
    and the bounded relaxation rate
    \begin{equation}
      R_i \;\equiv\; c\,\sqrt{\max(0,\,1-S_i)}.
      \label{eq:D4_Ri_def}
    \end{equation}
    The explicit update is
    \begin{equation}
      \chi_i(t+\Delta t) \;=\; \chi_i(t) + \Delta t\Big(R_i(t) + \kappa\,(\Delta_G\chi)_i(t)\Big),
      \label{eq:D4_update}
    \end{equation}
    where $(\Delta_G\chi)_i=\sum_{j\in\mathcal{N}(i)}(\chi_j-\chi_i)$ is the graph Laplacian.
    If $S_i>1$, the bounded term saturates to $R_i=0$ (radicand clipping), and the evolution
    remains well-defined; the Laplacian term tends to reduce local slopes and assists the
    formation of a projectable regime.

  \paragraph{Coarse-graining and definition of \texorpdfstring{$\chi_{\mathrm{eff}}$}{χeff}.}
    The effective field $\chi_{\mathrm{eff}}$ is obtained by block coarse-graining at scale
    $\ell_0$ (in lattice units), i.e. by averaging $\chi$ over disjoint cubic blocks,
    yielding a reduced lattice that represents the effective degrees of freedom:
    \begin{equation}
      \chi_{\mathrm{eff}}(t) \;\equiv\; \mathrm{CG}\big(\chi(t)\big).
      \label{eq:D4_chi_eff_def}
    \end{equation}
    No differential structure is introduced at this stage.

  \paragraph{Correct validation target: coarse-grained micro-dynamics.}
    Because the evolution operator is nonlinear and includes saturation, coarse-graining does
    not commute with the dynamics in general:
    \[
      \mathrm{CG}\!\big(\mathcal{R}(\chi)\big)\neq \mathcal{R}\!\big(\mathrm{CG}(\chi)\big).
    \]
    Accordingly, the validation targets the \emph{coarse-grained micro-dynamics}:
    \begin{equation}
      \partial_t \chi_{\mathrm{eff}} \;\approx\;
      \mathrm{CG}\!\Big(
      c\sqrt{\max(0,1-S(\chi))} + \kappa\,\Delta_G \chi
      \Big),
      \label{eq:D4_target_cg_dynamics}
    \end{equation}
    where $S(\chi)$ is defined by Eq.~\eqref{eq:D4_Si_Kij}.
    Operationally, the right-hand side is computed on the micro-lattice and then
    coarse-grained, ensuring that the comparison is performed at a consistent descriptive
    level.

  \paragraph{Residual metric.}
    Let $\chi_{\mathrm{eff}}(t)=\mathrm{CG}(\chi(t))$ and define
    \[
      \partial_t \chi_{\mathrm{eff}}(t) \approx
      \frac{\chi_{\mathrm{eff}}(t+\Delta t)-\chi_{\mathrm{eff}}(t)}{\Delta t}.
    \]
    Define the coarse-grained right-hand side
    \[
      \mathcal{R}_{\mathrm{eff}}(t) \equiv
      \mathrm{CG}\!\Big(
      c\sqrt{\max(0,1-S(\chi(t)))} + \kappa\,\Delta_G \chi(t)
      \Big).
    \]
    We then evaluate the normalized residual
    \begin{equation}
      \varepsilon(t) \equiv
      \frac{\left\|\partial_t \chi_{\mathrm{eff}}(t) - \mathcal{R}_{\mathrm{eff}}(t)\right\|}
      {\left\|\partial_t \chi_{\mathrm{eff}}(t)\right\|},
      \label{eq:D4_epsilon_def}
    \end{equation}
    where $\|\cdot\|$ denotes an $L^2$ norm over the effective lattice.

  \paragraph{Scope of the residual diagnostic.}
    The normalized residual $\epsilon$ provides a quantitative measure of the
    \emph{algorithmic consistency} between the time evolution of the coarse-grained field
    $\chi_{\text{eff}}$ and the coarse-grained micro-dynamics. As such, it is a diagnostic of
    numerical convergence and internal consistency of the relaxation scheme. However,
    $\epsilon$ does not encode geometric information about the projection
    $\Pi : \chi \rightarrow \chi_{\text{eff}}$. In particular, a small residual
    $\epsilon \ll 1$ does not imply that the projection is locally injective or that the
    corresponding effective description is geometrically faithful. Distinguishing
    numerically stable configurations from genuinely projectable ones therefore requires
    additional, independent criteria beyond the residual metric alone.

  \paragraph{Initial conditions.}
    Unless stated otherwise, simulations start from an i.i.d. Gaussian field
    $\chi_i(0)\sim\mathcal{N}(0,\sigma^2)$ with $\sigma=0.2$ (dimensionless units).
    A \emph{smooth} run includes a short pre-smoothing stage consisting of
    $n_{\mathrm{pre}}=10$ iterations of
    \[
      \chi \leftarrow \chi + \alpha\,\Delta_G\chi,
      \qquad \alpha=0.2,
    \]
    whose only role is to suppress high-frequency modes and place the system within a
    projectable regime. A \emph{rough} run corresponds to the same i.i.d. draw without
    pre-smoothing.

  \paragraph{Representative results and temporal diagnostics.}
    For $N=32$, $\ell_0=4$ lattice units, $\Delta t=0.03$ and dimensionless normalization
    $c=1$ (with parameters chosen for numerical stability on modest lattice sizes), we find
    a final normalized residual of order $10^{-2}$ in projectable regimes.
    In a representative smooth run, the final values are
    $\varepsilon_{L^2}\approx 9.3\times 10^{-3}$ and $\varepsilon_{L^\infty}\approx 1.4\times 10^{-2}$;
    in a representative rough run, $\varepsilon_{L^2}\approx 1.45\times 10^{-2}$ and
    $\varepsilon_{L^\infty}\approx 1.63\times 10^{-2}$.

    Importantly, the same order of magnitude is observed when increasing the micro-lattice
    resolution (e.g.\ $N=48$ at fixed $\ell_0=4$), indicating that the small-residual regime
    is not a resolution-dependent artifact but reflects a genuine coarse-grained consistency.

    The temporal evolution $\varepsilon(t)$ is shown in Fig.~\ref{fig:D4-epsilon-time},
    and the distribution of pointwise residuals for the smooth run is shown in
    Fig.~\ref{fig:D4-residual-hist}.
    These results provide explicit numerical evidence that the relational-to-effective
    transition is consistent with the effective description \emph{at the level of
coarse-grained dynamics}.

    \begin{figure}[t]
      \centering
      \includegraphics[width=0.78\linewidth]{D-appendix-simulation/D04_epsilon_vs_time_compare}
      \caption{\textbf{Residual versus time.}
      Normalized residual $\varepsilon(t)$ (Eq.~\eqref{eq:D4_epsilon_def}) for a
      representative smooth run and a rough run, illustrating convergence toward a
      small-error regime of order $10^{-2}$ over the simulated time window.}
      \label{fig:D4-epsilon-time}
    \end{figure}

    \begin{figure}[t]
      \centering
      \includegraphics[width=0.78\linewidth]{D-appendix-simulation/D04_saturation_fraction_compare}
      \caption{\textbf{Saturation fraction versus time.}
      Fraction of lattice sites satisfying $S>1$ for smooth, rough, and nonprojectable runs.
      The nonprojectable configuration rapidly reaches $f_{\mathrm{sat}}\simeq 1$,
        indicating a fully saturated and effectively frozen regime, while smooth and rough
        cases remain partially saturated and dynamically active.}
      \label{fig:D4-satfrac-compare}
    \end{figure}

    \begin{figure}[t]
      \centering
      \includegraphics[width=0.72\linewidth]{D-appendix-simulation/D04_residual_hist_smooth}
      \caption{\textbf{Pointwise residual distribution (smooth run).}
      Histogram of $\partial_t \chi_{\mathrm{eff}} - \mathcal{R}_{\mathrm{eff}}$ over the
      effective lattice at the final time. The distribution is centered around zero and
      remains narrow compared to the typical scale of $\partial_t\chi_{\mathrm{eff}}$,
        consistent with a small normalized residual.}
      \label{fig:D4-residual-hist}
    \end{figure}

  \paragraph{Spatial structure of the effective field and residual.}
    In addition to the quantitative diagnostics, spatial snapshots are shown to
    illustrate the geometric character of the coarse-grained field and the nature
    of the remaining discrepancies.
    Figure~\ref{fig:D4-spatial-slices} displays representative slices of
    $\chi_{\mathrm{eff}}$ and of the corresponding residual field at the final time
    for a smooth run.

    The effective field $\chi_{\mathrm{eff}}$ is observed to be smooth across
    multiple coarse-graining cells, while the residual field exhibits no coherent
    long-wavelength structure.
    This supports the interpretation that the remaining error is dominated by
    local discretization effects rather than by a breakdown of the effective
    description.

    \begin{figure}[t]
      \centering
      \includegraphics[width=0.47\linewidth]{D-appendix-simulation/D04_chi_eff_slice_smooth}
      \hfill
      \includegraphics[width=0.47\linewidth]{D-appendix-simulation/D04_residual_slice_smooth}
      \caption{\textbf{Spatial slices of the effective field and residual (smooth run).}
      \emph{Left:} slice of the coarse-grained field $\chi_{\mathrm{eff}}$ at fixed $z$,
        showing a smooth large-scale structure.
        \emph{Right:} corresponding slice of the residual
        $\partial_t \chi_{\mathrm{eff}} - \mathcal{R}_{\mathrm{eff}}$ at the same time,
        exhibiting no coherent long-wavelength pattern.}
      \label{fig:D4-spatial-slices}
    \end{figure}

  \paragraph{Interpretation and limitations.}
    This toy model demonstrates constructively that the operational coarse-graining
    procedure defining $\chi_{\text{eff}}$ yields an effective description compatible with
    the coarse-grained micro-dynamics in \emph{projectable} regimes, as indicated by a
    small normalized residual $\epsilon = O(10^{-2})$ for smooth and rough configurations
    across multiple lattice resolutions.

    However, numerical stability alone is not a sufficient criterion for projectability.
    In fully saturated configurations, where $S_i > 1$ on (nearly) all lattice sites, the
    bounded relaxation term vanishes ($R_i = 0$) and the effective dynamics becomes
    quasi-static. In this regime, $\partial_t \chi_{\text{eff}} \approx 0$ and the residual
    $\epsilon$ becomes trivially small, even though no faithful geometric interpretation
    exists.

    Such configurations correspond to \emph{stable but non-projectable} regimes, in which
    multiple distinct $\chi$ configurations map to the same effective field
    $\chi_{\text{eff}}$. This loss of local injectivity reflects the emergence of
    \emph{rank-deficient projection fibers}, rather than any physical singularity or
    pathology of the underlying $\chi$ dynamics. These regimes mark the limits of
    applicability of the continuum geometric description and must be identified using
    criteria independent of the residual $\epsilon$.

  \paragraph{Reproducibility.}
    All figures and numerical values reported in this subsection are reproducible using an
    independent Python implementation provided as supplementary material in a separate
    repository.
    The implementation follows Eqs.~\eqref{eq:D4_Si_Kij}--\eqref{eq:D4_epsilon_def}
    exactly, including the pre-smoothing protocol and the residual diagnostics.

  \paragraph{Resolution and coarse-graining dependence.}
    To assess robustness beyond a single configuration, the validation program includes
    parameter sweeps in lattice resolution $N$ and coarse-graining scale $\ell_0$.
    Preliminary resolution sweeps at fixed $\ell_0=4$ (e.g.\ $N=32$ and $N=48$) show that the
    normalized residual remains of order $10^{-2}$ in projectable regimes, while fully
    saturated configurations remain clearly identifiable by an independent saturation
    indicator.

    More extensive sweeps in $(N,\ell_0)$ space are left for future work; however, the
    present results already demonstrate that the observed agreement is not a numerical
    coincidence tied to a single lattice size.

  Taken together, these results transform the $\chi\rightarrow\chi_{\mathrm{eff}}$
    transition from a purely programmatic statement into an explicit and numerically
    demonstrated construction within a controlled toy model.
