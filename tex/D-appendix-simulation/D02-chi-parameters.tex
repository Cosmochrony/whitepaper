\subsection{Estimates of \texorpdfstring{$\chi$}{χ}-Field Parameters}
  \label{subsec:chi-parameters}

  The quantities introduced in this section—effective coupling scales, spectral
  parameters, and characteristic lengths—should be understood as properties of a
  \emph{projected relaxation operator} acting on a finite-dimensional function space.
  They characterize the response of localized \(\chi\) configurations to
  perturbations within a given resolution scale and do not represent fundamental
  degrees of freedom of the theory.

  \textbf{Distinction between Bare and Effective Scales:}
  As established in Section~\ref{subsec:renormalization-hbar}, we distinguish
  between the \textbf{bare} parameters (\(K_{0,\text{bare}}\), \(\chi_{c,\text{bare}}\)),
  which are universal substrate invariants, and the \textbf{effective} parameters
  discussed here (\(K_{0,\text{eff}}\), \(\chi_{c,\text{eff}}\)). These encode
  how localized structures constrain relaxation once a coarse-grained geometric
  description becomes applicable, effectively describing a \textbf{projective renormalization}
  of the substrate's stiffness.

  The relevant effective parameters include:
  \begin{itemize}
    \item the \textbf{effective coupling scale} \(K_{0,\text{eff}}\) entering the
    projected response operator \(K_{ij}\),
    \item the \textbf{characteristic scale} \(\chi_{c,\text{eff}}\) at which
    macroscopic geometric effects emerge,
    \item effective solitonic parameters \((\lambda, \eta)\) controlling stabilization
    mechanisms in reduced descriptions,
    \item the maximal relaxation speed \(c\), which remains an invariant link
    between the bare and effective regimes.
  \end{itemize}

  \subsubsection*
  {Effective Coupling Scale \texorpdfstring{$K_0$}{K0} and Characteristic Scale \texorpdfstring{$\chi_c$}{chic}}

    The scale \(\chi_{c,\text{eff}}\) sets the characteristic magnitude of \(\chi\)
    over which structural variations significantly modulate relaxation and induce
    macroscopic geometric effects.
    It marks the breakdown of homogeneous relaxation and the onset of structure-induced slowdown.

    The emergent gravitational constant \(G\) is driven by the ratio
    \(K_{0,\text{eff}} / \chi_{c,\text{eff}}^2\).
    Equation~\eqref{eq:G-emergent} admits two illustrative normalization regimes, highlighting the scale-dependency
    of these effective ``dressed'' parameters:

    \paragraph{Planck-scale normalization.}
      If \(\chi_{c,\text{eff}}\) is associated with the Planck length
      \(\ell_P \simeq 1.6 \times 10^{-35}\,\mathrm{m}\), one finds
    \begin{equation}
      K_{0,\text{eff}} \sim 10^{93}\,\mathrm{m}^{-2}.
      \end{equation}
      In this regime, the effective relaxation dynamics is extremely stiff, and
      gravitational phenomena are interpreted as structural constraints near the
      limit of applicability of classical spacetime descriptions.

    \paragraph{Cosmological-scale normalization.}
      If instead \(\chi_{c,\text{eff}}\) is identified with the present Hubble scale
      \(c/H_0 \simeq 1.4 \times 10^{26}\,\mathrm{m}\), the inferred coupling scale
      becomes
    \begin{equation}
      K_{0,\text{eff}} \sim 10^{-52}\,\mathrm{m}^{-2}.
      \end{equation}
      This regime corresponds to a much softer collective response dominated by
      large-scale cosmological relaxation.

      \textbf{Conclusion on Scalability:}
      Both normalizations are internally consistent at the level of dimensional analysis.
      Their coexistence suggests that the Cosmochrony dynamics are
      \textbf{spectrally self-similar}: the fundamental physics remains invariant
      under the transformation of scales, provided the ratio of effective stiffness
      to correlation length is preserved.
      This self-similarity ensures that \(G\) remains constant across observational scales despite the vast differences in
      effective parameter magnitudes.
