\subsection{Simulation Algorithms for \texorpdfstring{$\chi$}{χ}-Field Dynamics}
  \label{subsec:simulation-algorithms}

  The numerical simulations presented in this subsection implement finite-dimensional
  approximations of the fundamentally continuous relaxation dynamics of the \(\chi\) field.
  They do not assume an underlying network, lattice, or discretized spacetime structure.
  Instead, they rely on auxiliary basis representations introduced solely for
  numerical stability, convergence control, and diagnostic clarity, in close analogy
  with spectral, finite-element, or wavelet-based methods used in continuum field
  theories.

  Any apparent graph-like structure arising in the implementation reflects the
  choice of numerical basis and sampling strategy.
  It does not correspond to a physical discretization of the \(\chi\) substrate, nor
  to a fundamental causal or spatial connectivity.

  \paragraph{Objectives of the numerical simulations.}
    The simulations pursue four complementary goals:
    \begin{enumerate}
      \item to verify the internal consistency of the bounded relaxation dynamics,
      \item to test the spontaneous formation and long-term stability of localized
      configurations,
      \item to study the response of the \(\chi\) field to perturbations and imposed
      constraints,
      \item to extract structural spectral features associated with stable
      configurations.
    \end{enumerate}

    These goals are exploratory rather than predictive.
    The simulations are designed to probe qualitative mechanisms of the theory in regimes where analytic treatment is
    impractical.

  \paragraph{Numerical representation and computational substrate.}
    For computational purposes, the \(\chi\) field is represented by a finite set of
    degrees of freedom \(\{\chi_i(\lambda)\}\), where the index \(i\) labels elements of
    a chosen numerical basis and \(\lambda\) denotes the monotonic relaxation parameter
    introduced in Section~\ref{subsec:parameter-independent-relaxation}.

    Interactions between these degrees of freedom are encoded through a coupling
    operator \(K_{ij}\), which represents a finite-dimensional projection of the effective relaxation response kernel.
    The indices \(i\) and \(j\) do not label spatial sites or causal nodes.
    They index basis functions in the chosen representation.

    Different numerical bases and sampling strategies—including regular grids,
    irregular samplings, or weighted connectivity graphs—lead to qualitatively similar behavior.
    This robustness indicates that the observed phenomena are intrinsic features of
    the bounded relaxation dynamics rather than artifacts of a particular numerical scheme.

  \paragraph{Relaxation update rule.}
    The numerical evolution follows a bounded relaxation rule inspired by the minimal
    kinematic constraint discussed in
    Section~\ref{subsec:minimal-kinematic-constraint}.
    In the chosen representation, the evolution equation is implemented as
    \begin{equation}
      \label{eq:discrete-dynamics}
      \frac{d\chi_i}{d\lambda}
      =
      c \sqrt{
        1 -
        \frac{1}{c^2}
        \sum_j K_{ij} (\chi_i - \chi_j)^2
      } .
    \end{equation}

    This update rule enforces:
    \begin{itemize}
      \item strict monotonicity of \(\chi\),
      \item a universal upper bound on the local relaxation rate,
      \item suppression of gradient-driven instabilities.
    \end{itemize}

    Time integration is performed using adaptive stepping schemes with explicit stability control.
    Alternative numerical implementations respecting the same kinematic bound produce
    equivalent qualitative behavior, confirming that the results do not depend sensitively on algorithmic details.

  \paragraph{Reference pseudocode for bounded relaxation dynamics.}
    The following pseudocode summarizes a minimal numerical implementation of the bounded
    relaxation rule~\eqref{eq:discrete-dynamics}. It is intentionally representation-agnostic:
    indices label basis coefficients, and the coupling operator \(K_{ij}\) encodes the chosen
    finite-dimensional approximation of the effective relaxation kernel.

    \begin{algorithm}[t]
\caption{Bounded \(\chi\)-relaxation with stability diagnostics}
\label{alg:bounded-chi-relaxation}
\begin{algorithmic}[1]
  \Require Initial coefficients \(\chi^{(0)}_i\), coupling operator \(K_{ij}\), bound \(c\),
  tolerances \(\varepsilon_\chi,\varepsilon_S\), max steps \(N_{\max}\)
  \Ensure Relaxed configuration \(\chi^\star\) and diagnostics \((S_{\max}, \Theta_p, \{\lambda_n\})\)

  \State \(n \gets 0\), \(\chi \gets \chi^{(0)}\)
  \State Initialize diagnostic logs \(\mathcal{L}\)

  \While{\(n < N_{\max}\)}
    \State \textbf{Compute local saturation density}:
    \For{each index \(i\)}
      \State \(S_i \gets \frac{1}{c^2}\sum_j K_{ij}\,(\chi_i - \chi_j)^2\)
      \State \(S_i \gets \min(S_i,\,1)\) \Comment{numerical safety clamp}
    \EndFor
    \State \(S_{\max} \gets \max_i S_i\)

    \If{\(S_{\max} > 1 - \varepsilon_S\)}
      \State \textbf{Flag near-saturation regime} in \(\mathcal{L}\)
      \Comment{may indicate metastability or impending reconfiguration}
    \EndIf

    \State \textbf{Bounded relaxation update}:
    \For{each index \(i\)}
      \State \(v_i \gets c\,\sqrt{1 - S_i}\) \Comment{implements Eq.~\eqref{eq:discrete-dynamics}}
    \EndFor

    \State \textbf{Adaptive step control}:
    \State Choose \(\Delta\lambda\) such that \(\max_i |\Delta\lambda\, v_i| \le \Delta\chi_{\max}\)
    \Comment{e.g., \(\Delta\chi_{\max}\) fixed small fraction of typical \(|\chi|\)}

    \State \textbf{Update} \(\chi_i \gets \chi_i + \Delta\lambda\, v_i\) for all \(i\)

    \State \textbf{Convergence test}:
    \State \(\delta \gets \max_i |\chi_i^{(n+1)} - \chi_i^{(n)}|\)
    \If{\(\delta < \varepsilon_\chi\)}
      \State \textbf{break}
    \EndIf

    \State \textbf{Optional: linearized spectrum around current state}:
    \If{diagnostics scheduled at step \(n\)}
      \State Construct linearized response operator \(L[\chi]\) around current \(\chi\)
      \State Compute low-lying eigenpairs \(\{(\lambda_k,\psi_k)\}_{k=1..m}\)
      \State Evaluate projectability threshold \(\Theta_p\) from spectral gap and Dirichlet energy
      \State Log \((S_{\max}, \{\lambda_k\}, \Theta_p)\) into \(\mathcal{L}\)
    \EndIf

    \State \(n \gets n+1\)
  \EndWhile

  \State \(\chi^\star \gets \chi\)
  \State \Return \(\chi^\star, \mathcal{L}\)
\end{algorithmic}
    \end{algorithm}

    The clamp is a numerical safeguard and does not modify the conceptual bound $S\le 1$.
    Loops are defined in the computational representation only, as a diagnostic of torsional organization; they do not
    imply spatial connectivity.

  \paragraph{Optional diagnostic: chiral--torsional charge invariant.}
    To quantify charge as a chiral--torsional invariant of the relaxation flux,
    we define a discrete flux on relational links \(J_{ij} \sim (\chi_j-\chi_i)\).
    For a chosen set of closed loops \(\gamma\) in the numerical representation,
    a winding-like invariant can be monitored by transporting the local flux orientation
    around each loop and computing the net defect:
    \begin{equation}
      \label{eq:charge_winding_pseudocode}
      Q_\gamma \;\equiv\; \frac{1}{2\pi}\sum_{(i,j)\in\gamma}\Delta \theta_{ij},
    \end{equation}
    where \(\theta_{ij}\) encodes the local oriented direction of the bounded flux and
    \(\Delta\theta_{ij}\) is taken modulo \(2\pi\).
    Stable charged excitations correspond to persistent nonzero \(Q_\gamma\) values,
    while near-saturation events (large \(S_{\max}\)) typically coincide with
    reconfiguration of torsional patterns and a redistribution of \(Q_\gamma\).

  \paragraph{Emergence and persistence of localized configurations.}
    Starting from generic initial conditions, the simulations robustly exhibit the
    spontaneous emergence of localized configurations in which structural variations of
    \(\chi\) remain persistently large.
    These configurations locally resist the global relaxation flow and remain stable over many relaxation intervals.

    Such structures are interpreted as numerical counterparts of the solitonic
    excitations discussed in Section~\ref{sec:particles-as-localized-excitations-of-the-chi-field}.
    They arise dynamically without being imposed by hand and do not require fine-tuned initial conditions.

    Perturbative tests indicate that small disturbances around these configurations
    decay rather than grow, confirming their dynamical stability within the bounded relaxation framework.

  \paragraph{Spectral analysis and response modes.}
    To probe the internal organization of stable configurations, the effective
    relaxation operator is linearized around a stationary background configuration.
    The resulting eigenvalue problem defines a discrete set of response modes
    characterizing how the configuration reacts to small perturbations within the chosen numerical representation.

    A systematic spectral analysis reveals a robust separation between:
    \begin{itemize}
      \item a small number of low-lying modes associated with coherent, collective deformations of the configuration,
      \item a dense set of higher modes that are rapidly damped by the relaxation dynamics.
    \end{itemize}

    This separation is observed across different bases, resolutions, and boundary conditions.
    It provides a structural fingerprint of the degree of internal organization and
    resistance to deformation of each stable excitation.

    At this stage, these response modes are not identified with observed particle masses.
    They are interpreted as intrinsic stability scales of localized configurations.
    Possible connections between spectral hierarchies and physical mass spectra are
    discussed conceptually in Appendix~\ref{subsec:spectral_mass}, without invoking numerical matching.

  \paragraph{Effective entanglement as a constrained observable.}
    Numerical investigations reveal that non-factorizable correlations do not emerge
    as a smooth or monotonic function of compression.
    Instead, they appear intermittently, during specific spectral reorganization events of the relaxation operator.
    This behavior indicates that entanglement is not a generic consequence of moderate
    compression, but a critically activated phenomenon tied to the internal restructuring of admissible modes.

    While spectral diagnostics such as mode crowding or near-degeneracy provide a
    quantitative measure of the non-injectivity of the projection, they do not by
    themselves characterize the ability of a configuration to sustain observable quantum correlations.

    In particular, strong Born--Infeld saturation of the relaxation flux suppresses
    the dynamical mobility required for correlations to remain operationally exploitable.
    Spectral non-factorization is therefore a necessary but not sufficient condition for effective entanglement.

    To account for this competition of constraints, we define an \emph{effective entanglement observable} as
    \begin{equation}
      \label{eq:effective-entanglement}
      E_{\mathrm{eff}}(\mathcal{C})
      \;\equiv\;
      \Delta_{\Pi}(\mathcal{C})\,
      \bigl(1 - \mathcal{C}^{\nu}\bigr),
      \qquad \nu > 0 ,
    \end{equation}
    where $\Delta_{\Pi}$ quantifies the degree of spectral non-injectivity associated
    with the projection, and the factor $(1-\mathcal{C}^{\nu})$ encodes the loss of
    projective mobility as the Born--Infeld saturation bound is approached.

    The functional form of $E_{\mathrm{eff}}$ is not postulated as a universal law,
    nor intended as a quantitative entanglement measure.
    It provides a minimal structural diagnostic encoding the competition between
    two independent constraints:
    (i) the degree of spectral non-injectivity of the projection,
    and (ii) the remaining dynamical mobility of the relaxation flux prior to Born--Infeld saturation.

    In particular, peaks of $E_{\mathrm{eff}}$ need not form a smooth envelope.
    They may occur as isolated maxima associated with discrete spectral rearrangements,
    reflecting the intermittent activation of non-factorizable correlations.

    This distinction is not imposed by construction but emerges generically from the simulations,
    making explicit the link between spectral mode crowding, the effective width of the projection
    fiber, and the stability of topological versus spectrally activated observables.

  \begin{figure}[t]
      \centering
      \includegraphics[width=0.95\linewidth]{D-appendix-simulation/D04_entanglement_intermittence_b}
      \caption{Numerical diagnostics of effective entanglement and chiral bias as a function
      of the compression parameter $\mathcal{C}$.
      \textbf{Top:} spectral non-injectivity proxy $\Delta_{\Pi}$ exhibiting discrete peaks
      associated with spectral reorganization events.
      \textbf{Middle:} effective entanglement observable $E_{\mathrm{eff}}$, showing
      intermittent activation in narrow compression windows and suppression at high
      saturation.
      \textbf{Bottom:} chiral (CP) bias, displaying a monotonic and robust dependence on
      compression.
      This contrast highlights the distinct ontological status of entanglement as a
      critical phenomenon, versus charge as a stable topological invariant.}
      \label{fig:D4-entanglement-intermittence}
    \end{figure}

    This definition does not introduce a new dynamical assumption.
    It expresses the fact that effective quantum correlations arise only in regimes
    where relational non-factorization coexists with sufficient dynamical freedom.
    In the limits of vanishing compression or full saturation, $E_{\mathrm{eff}}$
    vanishes, recovering respectively classical factorization and frozen,
    non-projectable regimes.

    The resulting behavior generically exhibits a maximum at intermediate values of
    the compression parameter $\mathcal{C}$, identifying a critical projection regime
    in which quantum entanglement emerges as a compromise between resolution and
    saturation constraints.

    \textit{The absence of a smooth ‘entanglement bell curve’ is not a numerical artifact but a structural prediction of
    Cosmochrony: entanglement emerges only during discrete spectral reorganization events of the projection fiber.”}

  \paragraph{Interpretation, scope, and limitations.}
    The appearance of discrete spectral hierarchies and long-lived localized
    configurations is a robust and reproducible numerical result.
    Within the present work, their role is structural rather than predictive.

    The simulations do not include quantum fluctuations, fully relativistic covariance,
    or higher-order backreaction effects.
    They are not intended to provide quantitative predictions for particle physics or precision cosmology.

  \paragraph{The Projectability Threshold \texorpdfstring{$\Theta_p$}{Θp}.}
    A critical diagnostic in our simulations is the \textbf{projectability threshold} $\Theta_p$, which defines when a
    relational configuration becomes admissible for a smooth spacetime projection $\Pi$.

    This threshold is monitored through two spectral conditions:
    \begin{itemize}
      \item \textbf{Spectral Gap Stability:}
      The projection is only valid if a clear separation exists in the relaxation spectrum, typically when
      $\frac{\lambda_2 - \lambda_1}{\text{Tr}(L)} > \epsilon_p$.
      Below this, the substrate is in a ``pre-geometric'' state where distance metrics are ill-defined.
      \item \textbf{Topological Coherence:}
      The Dirichlet energy of the mapped configuration must remain below a saturation bound,
      $\mathcal{E}_{proj} < \mathcal{E}_{max}$, ensuring that the emergent manifold is structurally stable and
      non-singular.
    \end{itemize}
    Crossing $\Theta_p$ marks the transition from ontological ``poverty'' (where only global, low-frequency modes are
    supported) to the emergence of complex, localized solitonic structures.

    \medskip
    \noindent
    \textit{Order-of-magnitude interpretation.}
    Although $\epsilon_p$ enters the simulations as a dimensionless diagnostic threshold,
    it is not intended to be an arbitrarily tunable numerical parameter.
    Its role is to encode the existence of a minimal resolvable spectral separation required
    for a configuration to admit a stable spacetime projection.

    From a physical perspective, this threshold plays a role analogous to an effective
    quantum of action: it marks the point below which fluctuations cannot be cleanly
    separated into distinct relational modes.
    For this reason, its natural order of magnitude is expected to be set by the emergence
    scale of $\hbar$ in effective descriptions, rather than by numerical resolution alone.

    Equivalently, the existence of a nonzero projectability gap implies a minimal
    resolvable geometric scale in projected configurations.
    When interpreted in continuum terms, this naturally corresponds to the Planck scale.
    In this sense, the Planck length is not imposed as a fundamental cutoff in the simulations,
    but can be understood as the effective manifestation of a nonzero projection threshold
    $\epsilon_p$ at the interface between pre-geometric and geometric regimes.

  \paragraph{Conclusion.}
    This subsection demonstrates that the bounded relaxation dynamics of the \(\chi\)
    field can be implemented numerically in a stable and controlled manner using
    finite-dimensional representations, without invoking a background geometry or
    additional fundamental degrees of freedom.

    In particular, the simulations reveal that quantum entanglement, understood as
    non-factorizable projective correlation, is a critically intermittent phenomenon.
    It emerges only during specific spectral reorganization events of the $\chi$
    substrate and is suppressed both in under-constrained and strongly saturated regimes.
    This behavior provides numerical support for the interpretation of entanglement
    developed in Section~\ref{subsec:entanglement-critical-compression}.
