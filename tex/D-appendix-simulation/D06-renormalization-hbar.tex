\subsection{Renormalization and the Universality of \texorpdfstring{$\hbar$}{ℏ}}
  \label{subsec:renormalization-hbar}

  To ensure the logical closure of the framework, we distinguish between the \textbf{bare substrate parameters}
  and their \textbf{effective counterparts} emerging through coarse-graining (as detailed in Appendix D):

  \begin{itemize}
    \item \textbf{Bare Parameters ($K_{0,\mathrm{bare}}, \chi_{c,\mathrm{bare}}$):}
    Universal, non-observable invariants of the $\chi$ substrate. They define the fundamental quantum of action:
    \begin{equation}
      \hbar_\chi \equiv \frac{c^3}{K_{0,\mathrm{bare}} \, \chi_{c,\mathrm{bare}}}.
    \end{equation}
    \item \textbf{Effective Parameters ($K_{0,\mathrm{eff}}, \chi_{c,\mathrm{eff}}$):}
    Environment-dependent values that incorporate the local density of relaxation constraints.
  \end{itemize}

  \paragraph{The "Firewall" of Constancy.}
    It is crucial to note that no observable variation of $\hbar$
    arises within a fixed relaxation epoch, as the bare substrate parameters remain invariant.
    The perceived universality of $\hbar$ and the spectral invariant $\alpha_{\mathrm{spec}}$
    stems from their exclusive dependence on the ratio of these bare quantities, which remain invariant under
    projective scaling.
    This construction transforms the effective descriptions of Appendix~\ref{sec:appendix-technical} into a
    rigorous
    theory of \textbf{projective renormalization}.
