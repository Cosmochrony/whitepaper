\subsection{Collective Gravitational Coupling and Operational Geometry}
  \label{subsec:collective-coupling}

  The fundamental field \(\chi\) is continuous and governed by nonlinear,
  non-perturbative relaxation constraints.
  Its dynamics does not admit a closed-form spectral decomposition, nor a simple linearization valid across all regimes.
  As a result, any explicit investigation of stability, collective response, or
  mode structure necessarily relies on auxiliary representations that approximate
  the underlying functional dynamics.

  In this appendix, we introduce such representations strictly as \textbf{effective surrogates}.
  They provide finite-dimensional bridges between the fundamental substrate and
  the effective response observed at macroscopic scales.
  These constructions do not reflect any fundamental discreteness of the substrate, nor do they
  define preferred spatial locations or a background geometry.
  They serve only to render certain collective effects computationally accessible.

  \paragraph{Collective coupling as a dressed response operator.}
    Localized excitations of the \(\chi\) field act as persistent resistances to global relaxation.
    When many such excitations are present, their influence combines collectively, modulating the relaxation flow at
    macroscopic scales.

    At the effective level, this collective influence is summarized by a response
    operator \(K_{ij}\), interpreted as a finite-dimensional representation of the
    linearized relaxation constraints.
    Crucially, \(K_{ij}\) is the \textbf{dressed counterpart} of the bare relational connectivity \(K_{0,\text{bare}}\)
    introduced in Section~\ref{subsec:renormalization-hbar}.
    While the bare coupling determines the universal quantum of action \(\hbar_\chi\), the effective operator
    \(K_{ij}\) encodes the spatial distribution of these constraints, effectively
    "mapping" the emergent geometry through the local spectral density.

  \paragraph{Effective gravitational potential in the weak-structure regime.}
    In regimes where localized resistances are sparse, the modulation of the
    global relaxation flow \(\Phi_\chi\) can be approximated as a perturbation
    of a uniform background.
    The effective potential \(\Phi_{\text{eff}}\) governing the motion of test excitations is derived from the local
    slowdown of the relaxation tempo.
    For a static source of mass \(M\), the operational distance \(r\) is defined by the propagation time of
    \(\chi\)-fluctuations, yielding:
    \begin{equation}
      \nabla^2 \Phi_{\text{eff}} \approx 4\pi G_{\text{eff}} \rho,
      \label{eq:poisson-emergent}
    \end{equation}
    where \(\rho\) is the density of relaxation resistance and \(G_{\text{eff}}\)
    is the emergent gravitational constant. The relation between the
    stiffness \(K_0\) and $G$ is given by:
    \begin{equation}
      G_{\text{eff}} \approx \frac{c^4}{K_{0,\text{eff}} \, \chi_{c,\text{eff}}^2}.
      \label{eq:G-emergent}
    \end{equation}
    This shows that the Newtonian limit is not a postulate, but the leading-order
    description of collective relaxation interference.

  \paragraph{Operational geometry.}
    Because Cosmochrony does not postulate a fundamental spacetime metric, spatial
    geometry is defined operationally.
    Two configurations are considered close if perturbations of \(\chi\) propagate efficiently between them, and distant
    otherwise.
    In the weak-gradient regime, this induces an effective spatial geometry that coincides with Newtonian gravity.
    \textbf{Gravity is thus recovered as a macroscopic manifestation of relaxation resistance.}

  \paragraph{Scope and limitations.}
    The construction presented here is restricted to quasi-static, weak-field regimes.
    Its purpose is to demonstrate that classical gravitational behavior
    can be recovered consistently without introducing a fundamental metric structure.
