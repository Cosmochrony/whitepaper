\subsection{Simulation Algorithms for \texorpdfstring{$\chi$}{χ}-Field Dynamics}
  \label{subsec:simulation-algorithms}

  The numerical simulations presented in this subsection implement finite-dimensional
  approximations of the fundamentally continuous relaxation dynamics of the \(\chi\) field.
  They do not assume an underlying network, lattice, or discretized spacetime structure.
  Instead, they rely on auxiliary basis representations introduced solely for
  numerical stability, convergence control, and diagnostic clarity, in close analogy
  with spectral, finite-element, or wavelet-based methods used in continuum field
  theories.

  Any apparent graph-like structure arising in the implementation reflects the
  choice of numerical basis and sampling strategy.
  It does not correspond to a physical discretization of the \(\chi\) substrate, nor
  to a fundamental causal or spatial connectivity.

  \paragraph{Objectives of the numerical simulations.}
    The simulations pursue four complementary goals:
    \begin{enumerate}
      \item to verify the internal consistency of the bounded relaxation dynamics,
      \item to test the spontaneous formation and long-term stability of localized
      configurations,
      \item to study the response of the \(\chi\) field to perturbations and imposed
      constraints,
      \item to extract structural spectral features associated with stable
      configurations.
    \end{enumerate}

    These goals are exploratory rather than predictive.
    The simulations are designed to probe qualitative mechanisms of the theory in regimes where analytic treatment is
    impractical.

  \paragraph{Numerical representation and computational substrate.}
    For computational purposes, the \(\chi\) field is represented by a finite set of
    degrees of freedom \(\{\chi_i(\lambda)\}\), where the index \(i\) labels elements of
    a chosen numerical basis and \(\lambda\) denotes the monotonic relaxation parameter
    introduced in Section~\ref{subsec:parameter-independent-relaxation}.

    Interactions between these degrees of freedom are encoded through a coupling
    operator \(K_{ij}\), which represents a finite-dimensional projection of the effective relaxation response kernel.
    The indices \(i\) and \(j\) do not label spatial sites or causal nodes.
    They index basis functions in the chosen representation.

    Different numerical bases and sampling strategies—including regular grids,
    irregular samplings, or weighted connectivity graphs—lead to qualitatively similar behavior.
    This robustness indicates that the observed phenomena are intrinsic features of
    the bounded relaxation dynamics rather than artifacts of a particular numerical scheme.

  \paragraph{Relaxation update rule.}
    The numerical evolution follows a bounded relaxation rule inspired by the minimal
    kinematic constraint discussed in
    Section~\ref{subsec:minimal-kinematic-constraint}.
    In the chosen representation, the evolution equation is implemented as
    \begin{equation}
      \label{eq:discrete-dynamics}
      \frac{d\chi_i}{d\lambda}
      =
      c \sqrt{
        1 -
        \frac{1}{c^2}
        \sum_j K_{ij} (\chi_i - \chi_j)^2
      } .
    \end{equation}

    This update rule enforces:
    \begin{itemize}
      \item strict monotonicity of \(\chi\),
      \item a universal upper bound on the local relaxation rate,
      \item suppression of gradient-driven instabilities.
    \end{itemize}

    Time integration is performed using adaptive stepping schemes with explicit stability control.
    Alternative numerical implementations respecting the same kinematic bound produce
    equivalent qualitative behavior, confirming that the results do not depend sensitively on algorithmic details.

  \paragraph{Emergence and persistence of localized configurations.}
    Starting from generic initial conditions, the simulations robustly exhibit the
    spontaneous emergence of localized configurations in which structural variations of
    \(\chi\) remain persistently large.
    These configurations locally resist the global relaxation flow and remain stable over many relaxation intervals.

    Such structures are interpreted as numerical counterparts of the solitonic
    excitations discussed in Section~\ref{sec:particles-as-localized-excitations-of-the-chi-field}.
    They arise dynamically without being imposed by hand and do not require fine-tuned initial conditions.

    Perturbative tests indicate that small disturbances around these configurations
    decay rather than grow, confirming their dynamical stability within the bounded relaxation framework.

  \paragraph{Spectral analysis and response modes.}
    To probe the internal organization of stable configurations, the effective
    relaxation operator is linearized around a stationary background configuration.
    The resulting eigenvalue problem defines a discrete set of response modes
    characterizing how the configuration reacts to small perturbations within the chosen numerical representation.

    A systematic spectral analysis reveals a robust separation between:
    \begin{itemize}
      \item a small number of low-lying modes associated with coherent, collective deformations of the configuration,
      \item a dense set of higher modes that are rapidly damped by the relaxation dynamics.
    \end{itemize}

    This separation is observed across different bases, resolutions, and boundary conditions.
    It provides a structural fingerprint of the degree of internal organization and
    resistance to deformation of each stable excitation.

    At this stage, these response modes are not identified with observed particle masses.
    They are interpreted as intrinsic stability scales of localized configurations.
    Possible connections between spectral hierarchies and physical mass spectra are
    discussed conceptually in Appendix~\ref{subsec:spectral_mass}, without invoking numerical matching.

  \paragraph{Interpretation, scope, and limitations.}
    The appearance of discrete spectral hierarchies and long-lived localized
    configurations is a robust and reproducible numerical result.
    Within the present work, their role is structural rather than predictive.

    The simulations do not include quantum fluctuations, fully relativistic covariance,
    or higher-order backreaction effects.
    They are not intended to provide quantitative predictions for particle physics or precision cosmology.

  \paragraph{The Projectability Threshold $\Theta_p$.}
    A critical diagnostic in our simulations is the \textbf{projectability threshold} $\Theta_p$, which defines when a
    relational configuration becomes admissible for a smooth spacetime projection $\Pi$.

    This threshold is monitored through two spectral conditions:
    \begin{itemize}
      \item \textbf{Spectral Gap Stability:}
      The projection is only valid if a clear separation exists in the relaxation spectrum, typically when
      $\frac{\lambda_2 - \lambda_1}{\text{Tr}(L)} > \epsilon_p$.
      Below this, the substrate is in a ``pre-geometric'' state where distance metrics are ill-defined.
      \item \textbf{Topological Coherence:}
      The Dirichlet energy of the mapped configuration must remain below a saturation bound,
      $\mathcal{E}_{proj} < \mathcal{E}_{max}$, ensuring that the emergent manifold is structurally stable and
      non-singular.
    \end{itemize}
    Crossing $\Theta_p$ marks the transition from ontological ``poverty'' (where only global, low-frequency modes are
    supported) to the emergence of complex, localized solitonic structures.

    \medskip
    \noindent
    \textit{Order-of-magnitude interpretation.}
    Although $\epsilon_p$ enters the simulations as a dimensionless diagnostic threshold,
    it is not intended to be an arbitrarily tunable numerical parameter.
    Its role is to encode the existence of a minimal resolvable spectral separation required
    for a configuration to admit a stable spacetime projection.

    From a physical perspective, this threshold plays a role analogous to an effective
    quantum of action: it marks the point below which fluctuations cannot be cleanly
    separated into distinct relational modes.
    For this reason, its natural order of magnitude is expected to be set by the emergence
    scale of $\hbar$ in effective descriptions, rather than by numerical resolution alone.

    Equivalently, the existence of a nonzero projectability gap implies a minimal
    resolvable geometric scale in projected configurations.
    When interpreted in continuum terms, this naturally corresponds to the Planck scale.
    In this sense, the Planck length is not imposed as a fundamental cutoff in the simulations,
    but can be understood as the effective manifestation of a nonzero projection threshold
    $\epsilon_p$ at the interface between pre-geometric and geometric regimes.

  \paragraph{Conclusion.}
    This subsection demonstrates that the bounded relaxation dynamics of the \(\chi\)
    field can be implemented numerically in a stable and controlled manner using
    finite-dimensional representations, without invoking a background geometry or
    additional fundamental degrees of freedom.

    The spontaneous emergence of localized configurations and the associated hierarchy
    of response modes provide strong numerical support for the conceptual foundations
    of Cosmochrony and establish a solid basis for future, more quantitative computational investigations.
