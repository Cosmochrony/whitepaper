\subsection{Estimates of \texorpdfstring{$\chi$}{χ}-Field Parameters}
  \label{subsec:chi-parameters}

  The quantities introduced in this section—effective coupling scales, spectral
  parameters, and characteristic lengths—should be understood as properties of a
  \emph{projected relaxation operator} acting on a finite-dimensional function
  space.
  They characterize the response of localized \(\chi\) configurations to
  perturbations within a given resolution scale and do not represent fundamental
  degrees of freedom of the theory.

  All parameters discussed below are therefore \emph{effective} and
  \emph{environment-dependent}.
  They encode how localized structures constrain relaxation once a coarse-grained
  geometric description becomes applicable.
  No claim is made that they correspond to universal constants of nature.

  The relevant effective parameters include:
  \begin{itemize}
    \item the \textbf{effective coupling scale} \(K_0\) entering the projected
    response operator \(K_{ij}\),
    \item the \textbf{characteristic scale} \(\chi_c\) at which macroscopic geometric
    effects emerge,
    \item effective solitonic parameters \((\lambda,\eta)\) controlling stabilization
    mechanisms in reduced descriptions,
    \item the maximal relaxation speed \(c\), identified with the invariant speed of
    relativistic kinematics.
  \end{itemize}

  \subsubsection{Effective Coupling Scale \texorpdfstring{$K_0$}{K0} and Characteristic Scale \texorpdfstring{$\chi_c$}{χc}}

    In weak-field regimes admitting an effective geometric description, gravitational
    coupling emerges from the collective stiffness of the \(\chi\) relaxation
    dynamics.
    Dimensional consistency and matching to the Newtonian limit lead to the effective
    relation
    \begin{equation}
      G = \frac{c^4}{16\pi K_0 \chi_c^2},
      \label{eq:G-emergent}
    \end{equation}
    which links the observed gravitational constant \(G\) to two emergent scales.

    The parameter \(K_0\), with dimensions of inverse length squared, characterizes the
    maximal stiffness of the projected relaxation response in a homogeneous background.
    It is a property of the projected operator used for numerical and phenomenological
    analysis, not a microscopic connectivity or lattice spacing.

    Crucially, \(K_0\) is not scale-invariant.
    It is defined only after projection and coarse-graining of the underlying
    \(\chi\)-field dynamics and therefore depends explicitly on the resolution scale
    at which the operator is constructed.
    This dependence may be summarized as
    \begin{equation}
      K_0 \equiv K_0(\ell_{\mathrm{cg}}),
    \end{equation}
    where \(\ell_{\mathrm{cg}}\) denotes the characteristic coarse-graining scale.

    The large variation of \(K_0\) across different regimes—Planckian, mesoscopic, or
    cosmological—is therefore not a sign of fine-tuning.
    It is a generic consequence of scale-dependent projection in an emergent
    framework.

    The scale \(\chi_c\) sets the characteristic magnitude of \(\chi\) over which
    structural variations significantly modulate relaxation and induce macroscopic
    geometric effects.
    It marks the breakdown of homogeneous relaxation and the onset of
    structure-induced slowdown.

    Equation~\eqref{eq:G-emergent} admits two illustrative normalization regimes:

    \paragraph{Planck-scale normalization.}
      If \(\chi_c\) is associated with the Planck length
      \(\ell_P \simeq 1.6 \times 10^{-35}\,\mathrm{m}\),
      one finds
      \begin{equation}
        K_0 \sim 10^{93}\,\mathrm{m}^{-2}.
      \end{equation}
      In this regime, the effective relaxation dynamics is extremely stiff, and
      gravitational phenomena are interpreted as arising from structural constraints
      near the limit of applicability of classical spacetime descriptions.

    \paragraph{Cosmological-scale normalization.}
      If instead \(\chi_c\) is identified with the present Hubble scale
      \(c/H_0 \simeq 1.4 \times 10^{26}\,\mathrm{m}\),
      the inferred coupling scale becomes
      \begin{equation}
        K_0 \sim 10^{-52}\,\mathrm{m}^{-2}.
      \end{equation}
      This regime corresponds to a much softer collective response dominated by
      large-scale cosmological relaxation.
      Both normalizations are internally consistent at the level of dimensional
      analysis; discriminating between them requires additional observational or
      dynamical input.
