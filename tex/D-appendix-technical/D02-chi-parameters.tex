\subsection{Estimates of \texorpdfstring{$\chi$}{χ}-Field Parameters}
  \label{subsec:chi-parameters}

  The quantities introduced in this section—effective coupling scales, spectral
  parameters, and characteristic lengths—should be understood as properties of a
  \emph{projected relaxation operator} acting on a finite-dimensional function
  space.
  They characterize the response of localized \(\chi\) configurations to
  perturbations within a given resolution scale and do not represent fundamental
  degrees of freedom of the theory.

  All parameters discussed below are therefore \emph{effective} and
  \emph{environment-dependent}.
  They encode how localized structures constrain relaxation once a coarse-grained
  geometric description becomes applicable.
  No claim is made that they correspond to universal constants of nature.

  The relevant effective parameters include:
  \begin{itemize}
    \item the \textbf{effective coupling scale} \(K_0\) entering the projected
    response operator \(K_{ij}\),
    \item the \textbf{characteristic scale} \(\chi_c\) at which macroscopic geometric
    effects emerge,
    \item effective solitonic parameters \((\lambda,\eta)\) controlling stabilization
    mechanisms in reduced descriptions,
    \item the maximal relaxation speed \(c\), identified with the invariant speed of
    relativistic kinematics.
  \end{itemize}

  \subsubsection{Effective Coupling Scale \texorpdfstring{$K_0$}{K0} and Characteristic Scale \texorpdfstring{$\chi_c$}{χc}}

    In weak-field regimes admitting an effective geometric description, gravitational
    coupling emerges from the collective stiffness of the \(\chi\) relaxation
    dynamics.
    Dimensional consistency and matching to the Newtonian limit lead to the effective
    relation
    \begin{equation}
      G = \frac{c^4}{16\pi K_0 \chi_c^2},
      \label{eq:G-emergent}
    \end{equation}
    which links the observed gravitational constant \(G\) to two emergent scales.

    The parameter \(K_0\), with dimensions of inverse length squared, characterizes the
    maximal stiffness of the projected relaxation response in a homogeneous background.
    It is a property of the projected operator used for numerical and phenomenological
    analysis, not a microscopic connectivity or lattice spacing.

    Crucially, \(K_0\) is not scale-invariant.
    It is defined only after projection and coarse-graining of the underlying
    \(\chi\)-field dynamics and therefore depends explicitly on the resolution scale
    at which the operator is constructed.
    This dependence may be summarized as
    \begin{equation}
      K_0 \equiv K_0(\ell_{\mathrm{cg}}),
    \end{equation}
    where \(\ell_{\mathrm{cg}}\) denotes the characteristic coarse-graining scale.

    The large variation of \(K_0\) across different regimes—Planckian, mesoscopic, or
    cosmological—is therefore not a sign of fine-tuning.
    It is a generic consequence of scale-dependent projection in an emergent
    framework.

    The scale \(\chi_c\) sets the characteristic magnitude of \(\chi\) over which
    structural variations significantly modulate relaxation and induce macroscopic
    geometric effects.
    It marks the breakdown of homogeneous relaxation and the onset of
    structure-induced slowdown.

    Equation~\eqref{eq:G-emergent} admits two illustrative normalization regimes:

    \paragraph{Planck-scale normalization.}
      If \(\chi_c\) is associated with the Planck length
      \(\ell_P \simeq 1.6 \times 10^{-35}\,\mathrm{m}\),
      one finds
      \begin{equation}
        K_0 \sim 10^{93}\,\mathrm{m}^{-2}.
      \end{equation}
      In this regime, the effective relaxation dynamics is extremely stiff, and
      gravitational phenomena are interpreted as arising from structural constraints
      near the limit of applicability of classical spacetime descriptions.

    \paragraph{Cosmological-scale normalization.}
      If instead \(\chi_c\) is identified with the present Hubble scale
      \(c/H_0 \simeq 1.4 \times 10^{26}\,\mathrm{m}\),
      the inferred coupling scale becomes
      \begin{equation}
        K_0 \sim 10^{-52}\,\mathrm{m}^{-2}.
      \end{equation}
      This regime corresponds to a much softer collective response dominated by
      large-scale cosmological relaxation.
      Both normalizations are internally consistent at the level of dimensional
      analysis; discriminating between them requires additional observational or
      dynamical input.

\subsection{Order-of-Magnitude Consistency Checks}
  \label{sec:consistency_checks}

  Precise numerical values of \(K_0\) and \(\chi_c\) require dedicated numerical
  simulations of the \(\chi\) relaxation dynamics.
  At the present stage, we restrict attention to order-of-magnitude consistency
  checks.
  These are not predictions, but sanity tests ensuring that the framework operates
  in a phenomenologically viable regime.

  \begin{enumerate}
    \item \textbf{Electron mass scale.}
    For an electron-like solitonic excitation with rest energy
    \(m_e c^2 \approx 0.5\,\mathrm{MeV}\), the lowest stability eigenvalue
    \(\lambda_1\) of the projected operator must satisfy
    \begin{equation}
      \lambda_1 \sim \left(\frac{m_e c^2}{\hbar_{\mathrm{eff}}}\right)^2 .
    \end{equation}
    Assuming \(\hbar_{\mathrm{eff}} \approx \hbar\) at microscopic scales yields
    \(\lambda_1 \sim 10^{41}\,\mathrm{s^{-2}}\).
    For a representative numerical resolution \(a \sim 10^{-15}\,\mathrm{m}\),
    this implies
    \begin{equation}
      K_0 \sim 10^{31}\,\mathrm{m}^{-2},
    \end{equation}
    consistent with the stability of localized solitonic configurations but not
    uniquely fixed.

    \item \textbf{Correlation scale \(\chi_c\).}
    The scale \(\chi_c\) sets the transition between effectively symmetric and
    structurally broken relaxation regimes.
    Requiring compatibility with electroweak-scale physics suggests the bound
    \begin{equation}
      \chi_c \lesssim \frac{\hbar c}{v} \sim 10^{-18}\,\mathrm{m},
    \end{equation}
    where \(v \simeq 246\,\mathrm{GeV}\).
    This is not a prediction but a consistency requirement ensuring that particle
    masses emerge at the correct energy scales.

    \item \textbf{Absence of fine-tuning.}
    The parameters \(K_0\) and \(\chi_c\) are constrained, not fine-tuned.
    Viable regimes are defined by:
    \begin{itemize}
      \item soliton stability (\(K_0 a^2 \gg 1\)),
      \item emergence of particle mass scales (\(\chi_c \lesssim 10^{-18}\,\mathrm{m}\)),
      \item absence of ultraviolet instabilities (\(K_0 \lesssim c^2/a^2\)).
    \end{itemize}
    These inequalities define a parameter window rather than a unique solution.
  \end{enumerate}

  \paragraph{Important note.}
    All numerical values quoted above are illustrative.
    Precise determination of effective parameters requires:
    \begin{itemize}
      \item numerical simulations of \(\chi\)-field dynamics
      (Appendix~D.3),
      \item matching to the particle mass spectrum
      (Section~B),
      \item consistency with cosmological observations
      (Appendix~C).
    \end{itemize}
    No claim is made that these parameters are predicted at this stage; they are
    constrained by internal and observational consistency.

\subsubsection{Relaxation Speed and Cosmological Constraints}

  The maximal relaxation speed \(c\) is identified with the invariant speed of
  relativistic kinematics.
  At the cosmological level, homogeneous relaxation implies
  \begin{equation}
    H(t) \simeq \frac{\dot{\chi}}{\chi},
  \end{equation}
  so that at the present epoch
  \begin{equation}
    \chi(t_0) \simeq \frac{c}{H_0}
    \sim 4 \times 10^{26}\,\mathrm{m}.
  \end{equation}

  This identification reproduces the observed age of the universe,
  \(t_0 \sim \chi(t_0)/c \simeq 13.8\,\mathrm{Gyr}\),
  without introducing additional cosmological parameters.

\subsubsection{Observational Constraints}

  Current observations impose indirect constraints on the effective parameter
  space:
  \begin{itemize}
    \item \textbf{CMB anisotropies} constrain large-scale \(\chi\) fluctuations and
    disfavor values of \(\chi_c\) that would excessively amplify low-\(\ell\) modes.
    \item \textbf{The Hubble tension} may be interpreted as probing different
    effective relaxation regimes at low and high redshift.
    \item \textbf{Gravitational-wave observations} constrain variations of the
    effective coupling scale \(K_0\) in strong-field environments to remain
    subdominant.
  \end{itemize}

\subsubsection{Summary and Status}
  \label{subsec:chi-parameters-summary-status}

  Table~\ref{tab:chi-parameters} summarizes indicative consistency ranges for the
  effective parameters discussed above.
  These ranges define admissible windows rather than predictions and are presented
  for orientation only.

  \begin{table}[htbp]
    \centering
    \renewcommand{\arraystretch}{1.2}
    \begin{tabular}{|p{3.3cm}|p{4.3cm}|p{7.0cm}|}
      \hline
      \textbf{Quantity} &
      \textbf{Indicative scale / range} &
      \textbf{Interpretation in Cosmochrony} \\
      \hline
      \(K_0(\ell_{\mathrm{cg}})\) &
      Scale-dependent; examples span
      \(10^{-52}\) to \(10^{93}\,\mathrm{m^{-2}}\) depending on the identification of \(\chi_c\) &
      Effective stiffness of the projected relaxation response operator at coarse-graining scale
      \(\ell_{\mathrm{cg}}\); not fundamental and not expected to be universal across regimes. \\
      \hline
      \(\chi_c\) &
      Regime-dependent characteristic scale; illustrative identifications include
      \(\ell_P\) (Planck) or \(c/H_0\) (cosmological) &
      Characteristic \(\chi\)-scale at which structural variations significantly modulate relaxation
      and induce macroscopic geometric effects; interpretation depends on the projection regime. \\
      \hline
      \(\lambda_1\) (lowest response mode) &
      Order-of-magnitude diagnostic scale (model- and resolution-dependent) &
      Lowest stability/response eigenvalue of the linearized projected operator around a localized
      configuration; a structural stability indicator, not a particle-mass prediction at this stage. \\
      \hline
      \(\hbar_{\mathrm{eff}}\) &
      Treated as approximately \(\hbar\) in conventional microscopic regimes (assumption) &
      Effective quantization scale of the projected description; may encode coarse-graining and
      regime dependence; not fixed by the relational formulation alone. \\
      \hline
      \(a_0(t)\) &
      Emergent scale of order \(cH(t)\) in late-time regimes &
      Phenomenological acceleration scale arising from bounded relaxation and cosmological evolution;
      may lead to MOND-like behavior without interpolation functions. \\
      \hline
    \end{tabular}
    \caption{Indicative consistency windows for effective \(\chi\)-field parameters.
    These values are not predictions; they summarize scale-dependent ranges and
    diagnostic quantities used for internal and phenomenological consistency checks.}
    \label{tab:chi-parameters}
  \end{table}

  A first-principles derivation of these effective quantities from the fundamental
  relational \(\chi\) dynamics remains an open problem and is identified as a
  central objective for future analytical and numerical work.
