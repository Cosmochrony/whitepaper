\subsection{Estimates of $\chi$-Field Parameters}
  \label{subsec:chi-parameters}

  The quantities introduced in this section, including effective coupling matrices and spectral
  modes, should be understood as properties of a projected relaxation operator acting on a finite
  function space.

  They characterize how localized $\chi$ configurations respond to perturbations within a given
  resolution scale, rather than representing fundamental degrees of freedom of the theory.
  The parameters considered here characterize the effective response of the $\chi$
  relaxation dynamics in regimes where a coarse-grained geometric description applies.
  They do not correspond to fundamental constants of the theory, but to emergent or
  environment-dependent scales encoding how localized structures constrain relaxation.

  The relevant parameters include:
  \begin{itemize}
    \item the \textbf{effective coupling scale} $K_0$ entering the projected kernel $K_{ij}$,
    \item the \textbf{characteristic $\chi$ scale} $\chi_c$ at which macroscopic geometric
    effects become significant,
    \item effective solitonic parameters $(\lambda,\eta)$ controlling stabilization
    mechanisms,
    \item the maximal relaxation speed $c$.
  \end{itemize}

  \subsubsection{Effective Coupling Scale $K_0$ and Characteristic Scale $\chi_c$}

    In weak-field regimes admitting an effective geometric description, the gravitational
    coupling emerges from the collective stiffness of the $\chi$ relaxation dynamics.
    Dimensional consistency and matching to the Newtonian limit yield the relation
    \begin{equation}
      G = \frac{c^4}{16 \pi K_0 \chi_c^2},
      \label{eq:G-emergent}
    \end{equation}
    which links the effective gravitational constant $G$ to two emergent scales.

    The parameter $K_0$, with dimensions $[\mathrm{length}]^{-2}$, characterizes the maximal
    strength of the collective relaxation response in a homogeneous background.
    It should be understood as a property of the projected relaxation operator introduced
    for numerical and phenomenological purposes, not as a microscopic connectivity.

    Importantly, this statement does not imply that $K_0$ is scale-invariant.
    Rather, $K_0$ is an \emph{effective} parameter defined only after projection and
    coarse-graining of the underlying $\chi$-field dynamics.
    Its numerical value therefore depends explicitly on the scale at which the
    relaxation operator is defined and on the subset of modes retained in the
    effective description.

    In this sense, $K_0$ should be understood as a scale-dependent spectral stiffness
    associated with the projected relaxation operator,
    \begin{equation}
      K_0 \;\equiv\; K_0(\ell_{\mathrm{cg}}),
    \end{equation}
    where $\ell_{\mathrm{cg}}$ denotes the characteristic coarse-graining scale.
    The large variation of $K_0$ across different regimes, as reported in
    Table~2, reflects the fact that the same underlying $\chi$ dynamics admits
    very different effective descriptions at Planckian, mesoscopic, or cosmological
    scales.
    This behavior is therefore not a sign of fine-tuning, but a generic consequence
    of scale-dependent projection in an emergent framework.

    The scale $\chi_c$ sets the characteristic magnitude of $\chi$ over which structural
    variations significantly modulate relaxation and thereby induce macroscopic geometric
    effects.
    It marks the breakdown scale of the homogeneous relaxation regime, beyond which
    localized configurations noticeably slow the relaxation flow.

    Equation~\eqref{eq:G-emergent} admits two physically distinct normalization regimes:

    \paragraph{Planck-scale normalization.}
      If $\chi_c$ is associated with the Planck length
      $\ell_P \simeq 1.6 \times 10^{-35}\,\mathrm{m}$,
      one finds
      \begin{equation}
        K_0 \sim 10^{93}\,\mathrm{m}^{-2}.
      \end{equation}
      In this regime, the effective relaxation dynamics is extremely stiff, and gravitational
      phenomena are interpreted as arising from structural constraints operating near the
      threshold where classical spacetime descriptions cease to apply.

    \paragraph{Cosmological-scale normalization.}
      If instead $\chi_c$ is identified with the present Hubble scale
      $c/H_0 \simeq 1.4 \times 10^{26}\,\mathrm{m}$,
      the inferred effective coupling scale is
      \begin{equation}
        K_0 \sim 10^{-52}\,\mathrm{m}^{-2}.
      \end{equation}
      This regime corresponds to a much softer collective response, dominated by large-scale
      cosmological relaxation.
      Both normalizations are internally consistent at the level of dimensional analysis;
      discriminating between them requires additional observational input.

\subsection{Order-of-Magnitude Consistency Checks}
  \label{sec:consistency_checks}

  While precise numerical estimates of \(K_0\) and \(\chi_c\) require full lattice simulations (see Appendix D.3), we perform **order-of-magnitude consistency checks** to ensure that the \(\chi\)-field framework operates in a phenomenologically viable regime. These checks are **not predictions**, but sanity tests for the parameter space.

  \begin{enumerate}
    \item **Electron Mass Constraint**:
    For an electron-like soliton with \(m_e \approx 0.5\) MeV and \(\lambda_1 \approx (m_e c^2 / \hbar_{\text{eff}})^2\), we require:
    \[
      \lambda_1 \approx \left( \frac{0.5 \text{ MeV} \cdot c^2}{\hbar_{\text{eff}}} \right)^2.
    \]
    Assuming \(\hbar_{\text{eff}} \approx \hbar\) for microscopic scales, this yields:
    \[
      \lambda_1 \approx (5 \times 10^{20} \, \text{s}^{-1})^2 = 2.5 \times 10^{41} \, \text{s}^{-2}.
    \]
    For a lattice spacing \(a \approx 10^{-15}\) m, this implies:
    \[
      K_0 \approx \frac{\lambda_1}{c^2 / a^2} \approx 2.5 \times 10^{31} \, \text{m}^{-2}.
    \]
    This value is **consistent with the stability of solitons** on the lattice, but not uniquely determined.

    \item **Correlation Length \(\chi_c\)**:
    The scale \(\chi_c\) sets the transition between the **symmetric** (\(\chi < \chi_c\)) and **broken** (\(\chi > \chi_c\)) phases. For the electroweak scale \(v \approx 246\) GeV, we expect:
    \[
      \chi_c \lesssim \frac{\hbar c}{v} \approx 10^{-18} \, \text{m},
    \]
    where the inequality reflects that \(\chi_c\) is a **pre-geometric scale**. This is **not a prediction**, but a consistency bound ensuring that particle masses emerge at the correct energy scales.

    \item **Avoiding Fine-Tuning**:
    The parameters \(K_0\) and \(\chi_c\) are **not fine-tuned**, but constrained by:
    \begin{itemize}
      \item The **stability of solitons** (requires \(K_0 a^2 \gg 1\)),
      \item The **emergence of the electroweak scale** (requires \(\chi_c \lesssim 10^{-18}\) m),
      \item The **absence of UV divergences** (requires \(K_0 \lesssim c^2 / a^2\)).
    \end{itemize}
    These bounds define a **viable parameter space**, not a unique solution.
  \end{enumerate}

  \noindent
  \textbf{Important Note}:
  The above estimates are **illustrative only**. Precise values of \(K_0\) and \(\chi_c\) will be determined by:
  \begin{itemize}
    \item **Lattice simulations** of \(\chi\)-field dynamics (Appendix D.3),
    \item **Matching to the particle mass spectrum** (Section B.8.4),
    \item **Cosmological observations** (e.g., CMB anisotropies, Section 10.7).
  \end{itemize}
  No claim is made that these parameters are **predicted** at this stage; they are **constrained** by consistency with known physics.

  \subsubsection{Relaxation Speed and Cosmological Constraints}

    The maximal relaxation speed $c$ is identified with the invariant speed of relativistic
    kinematics.
    At the cosmological level, the homogeneous relaxation dynamics imply
    \begin{equation}
      H(t) \simeq \frac{\dot{\chi}}{\chi},
    \end{equation}
    so that at the present epoch
    \begin{equation}
      \chi(t_0) \simeq \frac{c}{H_0}
      \sim 4 \times 10^{26}\,\mathrm{m}.
    \end{equation}

    This identification reproduces the observed age of the Universe,
    $t_0 \sim \chi(t_0)/c \simeq 13.8\,\mathrm{Gyr}$,
    without introducing additional cosmological parameters or modifying late-time
    dynamics.

  \subsubsection{Observational Constraints}

    Current observations impose indirect constraints on the allowed effective parameter
    space:
    \begin{itemize}
      \item \textbf{CMB anisotropies} constrain large-scale $\chi$ fluctuations, disfavouring
      values of $\chi_c$ that would excessively amplify low-$\ell$ modes.
      \item \textbf{The Hubble tension} may be interpreted as probing different effective
      $\chi$ relaxation regimes at low and high redshift.
      \item \textbf{Gravitational-wave propagation} constrains variations of the effective
      coupling scale $K_0$ in strong-field environments to remain subdominant.
    \end{itemize}

  \subsubsection{Summary and Status}

    Table~\ref{tab:chi-parameters} summarizes the indicative ranges discussed above.
    These values define consistency windows rather than predictions.

    \begin{table}[h]
      \centering
      \caption{Indicative ranges for effective $\chi$-field parameters}
      \label{tab:chi-parameters}
      \begin{tabular}{|c|c|c|}
        \hline
        Parameter & Planck-scale regime & Cosmological-scale regime \\ \hline
        $K_0$ & $\sim 10^{93}\,\mathrm{m}^{-2}$ & $\sim 10^{-52}\,\mathrm{m}^{-2}$ \\ \hline
        $\chi_c$ & $\sim 10^{-35}\,\mathrm{m}$ & $\sim 10^{26}\,\mathrm{m}$ \\ \hline
        $\lambda$ & $\ll 1$ (effective) & $\ll 1$ (effective) \\ \hline
        $\eta$ & $\mathcal{O}(1)$ & $\gg 1$ \\ \hline
      \end{tabular}
    \end{table}

    A first-principles derivation of these effective parameters from the underlying
    $\chi$ relaxation dynamics remains an open problem and is identified as a central
    target for future analytical and numerical work.
