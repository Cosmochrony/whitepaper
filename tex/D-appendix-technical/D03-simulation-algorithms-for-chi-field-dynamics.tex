\subsection{Simulation Algorithms for $\chi$-Field Dynamics}
  \label{subsec:simulation_algorithms}

  This appendix outlines a concrete numerical program aimed at testing the spectral
  hypothesis for inertial mass in Cosmochrony.
  The goal is not to compute the global spectrum of a homogeneous relaxation network,
  but to extract the stability spectrum of localized solitonic configurations of the
  $\chi$ field, whose lowest eigenmodes encode the inertial masses of particle-like
  excitations.

  \subsubsection{Discrete Relaxation Network and Baseline Operator}

    We consider a discrete relaxation network $G(V,E)$ with fixed geometric couplings
    $K^{(0)}_{ij}$.
    The corresponding baseline relaxation operator is the graph Laplacian
    \begin{equation}
    (\Delta_G^{(0)} \psi)
      _i = \sum_{j \sim i} K^{(0)}_{ij} (\psi_i - \psi_j),
    \end{equation}
    which encodes the geometric connectivity of the underlying substrate.

    Periodic or open boundary conditions may be imposed at large distances to minimize
    finite-size effects.
    No physical interpretation is assigned to the global eigenmodes of $\Delta_G^{(0)}$
    itself, which represent extended lattice modes rather than particle-like excitations.

  \subsubsection{Construction of Localized Solitonic Configurations}

    A localized solitonic configuration $\chi_{\mathrm{sol}}$ is constructed as a stationary
    solution of the $\chi$-field relaxation dynamics.
    Operationally, this configuration may be obtained either by numerical relaxation from
    a topologically non-trivial initial condition (e.g., kink-, vortex-, or Skyrmion-like
    profiles) or by imposing an analytical ansatz adapted to the dimensionality of the
    network.

    The soliton is required to be spatially localized, energetically stable, and stationary
    under the relaxation dynamics, thereby representing a particle-like excitation of the
    $\chi$ field.

  \subsubsection{Linearized Stability Operator Around a Soliton}

    Small perturbations $\delta\chi$ around the solitonic configuration,
    \begin{equation}
      \chi = \chi_{\mathrm{sol}} + \delta\chi,
    \end{equation}
    are governed by a linearized stability operator of the form
    \begin{equation}
      \mathcal{L}_{\mathrm{sol}} = \Delta_G^{(0)} + U_{\mathrm{sol}},
    \end{equation}
    where $U_{\mathrm{sol}}$ is a localized restoring operator determined by the background
    configuration $\chi_{\mathrm{sol}}$.

    The operator $U_{\mathrm{sol}}$ is non-vanishing only in regions where the soliton induces
    strong gradients or topological constraints.
    It may be constructed from local geometric quantities such as gradient saturation,
    discrete curvature, or the second variation of an effective localization functional.

  \subsubsection{Spectral Problem and Mass Identification}

    In the regime where an effective wave description applies, perturbations satisfy a
    Klein--Gordon-type equation,
    \begin{equation}
      \left(\frac{1}{c^2} \partial_t^2 + \mathcal{L}_{\mathrm{sol}}\right) \delta\chi = 0.
    \end{equation}

    Seeking normal-mode solutions $\delta\chi(t) = e^{-i\omega_n t}\psi_n$ leads to the
    spectral problem
    \begin{equation}
      \mathcal{L}_{\mathrm{sol}} \psi_n = \lambda_n \psi_n, \qquad \omega_n^2 = c^2 \lambda_n.
    \end{equation}

    The inertial masses associated with localized modes are identified as
    \begin{equation}
      m_n = \frac{\hbar}{c} \sqrt{\lambda_n}.
    \end{equation}

    A crucial diagnostic is the spatial localization of $\psi_n$, ensuring that the
    corresponding eigenmodes represent soliton-bound states rather than extended lattice
    modes.

  \subsubsection{Numerical Implementation and Diagnostics}

    The operator $\mathcal{L}_{\mathrm{sol}}$ is sparse and can be diagonalized using
    iterative eigensolvers such as ARPACK or LOBPCG.
    Shift--invert techniques are employed to resolve the lowest eigenvalues when large
    spectral gaps are present.

    Localization of eigenmodes is quantified using measures such as the inverse
    participation ratio (IPR).
    Robustness of the extracted spectrum is tested against variations in lattice size,
    boundary conditions, and moderate deformations of the solitonic configuration.

  \subsubsection{Success Criteria and Falsifiability}

    A successful outcome of this numerical program would be the emergence of a small number
    of localized eigenmodes whose mass ratios fall within the observed ranges of elementary
    particles (e.g., electron, pion, proton) up to order-of-magnitude accuracy, without
    fine-tuning of microscopic parameters.

    Conversely, failure to obtain such hierarchical spectra under controlled conditions
    would directly falsify the spectral hypothesis for mass generation in Cosmochrony.

  \subsubsection{Preliminary Spectral Results on Paired Localized Modes}
    \label{subsec:paired_localized_modes}

    As an initial validation of the numerical methodology, we have performed a series of
    one-dimensional toy simulations involving pinned kink--antikink configurations.
    While these simulations are not intended to model physical particles directly, they
    provide a controlled setting to probe the qualitative structure of the stability
    spectrum associated with localized $\chi$ configurations.

    \begin{figure}[t]
      \centering
      \includegraphics[width=\linewidth]{spectral_scan_sep}
      \caption{
        Emergence and pairing of localized stability modes in a one-dimensional pinned
        kink--antikink configuration.
        Degenerate low-lying eigenvalues correlate with strong spatial localization
        (high IPR) and reflect the duplication of local excitations associated with two
        equivalent solitonic centers, rather than particle--antiparticle creation.
      }
      \label{fig:paired_modes}
    \end{figure}

  The appearance of degenerate low-lying eigenvalues correlates with strong spatial
    localization, as quantified by the inverse participation ratio (IPR), and reflects
    the duplication of local excitations associated with two equivalent solitonic centers.

    When two identical localized centers are present, the lowest eigenvalues of
    $\mathcal{L}_{\mathrm{sol}}$ systematically appear as degenerate pairs.
    Inverse participation ratio (IPR) diagnostics confirm that each member of a degenerate
    pair is strongly localized around one of the two centers.
    Introducing a small asymmetry in the pinning strengths immediately lifts the degeneracy
    and increases the localization of each mode, demonstrating that these paired eigenvalues
    correspond to independent local excitations rather than global symmetric or antisymmetric
    combinations.

    As the separation between the kink and antikink is reduced, a clear transition is
    observed.
    Below a critical separation, localized low-lying modes disappear entirely, and the
    spectrum becomes dominated by extended lattice modes.
    Near this threshold, a soft quasi-zero mode emerges, characterized by a very small
    eigenvalue and moderate localization.
    This behavior is consistent with a weakly constrained local displacement (or sliding)
    degree of freedom, rather than with a tunneling-induced level splitting.

    These preliminary results indicate that the pairing of low-lying eigenmodes arises from
    the duplication of local degrees of freedom associated with multiple solitonic centers,
    and not from a dynamical coupling mechanism.
    They provide a concrete illustration of how particle-like excitations in Cosmochrony
    are expected to emerge as localized stability modes tied to individual solitons, with
    degeneracies controlled by symmetry and lifted by small perturbations.
