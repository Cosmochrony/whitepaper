\subsection{Collective Gravitational Coupling and Operational Geometry}
  \label{subsec:collective-coupling}

  The fundamental field \(\chi\) is continuous and governed by nonlinear,
  non-perturbative relaxation constraints.
  Its dynamics does not admit a closed-form spectral decomposition, nor a simple
  linearization valid across all regimes.
  As a result, any explicit investigation of stability, collective response, or
  mode structure necessarily relies on auxiliary representations that approximate
  the underlying functional dynamics.

  In this appendix, we introduce such representations strictly as numerical and
  operational tools.
  They provide finite-dimensional surrogates for aspects of the continuous
  relaxation operator governing \(\chi\).
  These constructions do not reflect any fundamental discreteness of the
  substrate, nor do they define preferred spatial locations or background geometry.
  They serve only to render certain collective effects computationally accessible.

  \paragraph{Collective coupling as an effective response operator.}
    Localized excitations of the \(\chi\) field act as persistent resistances to
    global relaxation.
    When many such excitations are present, their influence combines collectively,
    modulating the relaxation flow at macroscopic scales.

    At the effective level, this collective influence may be summarized by a response
    operator \(K_{ij}\), interpreted as a finite-dimensional representation of the
    linearized response of the relaxation dynamics to structural variations.
    The indices \(i\) and \(j\) label elements of a chosen numerical or functional
    basis used to represent configurations of \(\chi\).
    They do not correspond to fundamental spatial points, lattice sites, or metric
    locations.

    The operator \(K_{ij}\) depends only on relative variations of \(\chi\) between
    configurations and encodes the stiffness of the relaxation dynamics.
    It does not presuppose any background notion of distance, adjacency, or spatial
    embedding.
    Different choices of basis or discretization scheme lead to different numerical
    representations of \(K_{ij}\) without altering its conceptual role.

  \paragraph{Effective gravitational potential in the weak-structure regime.}
    In regimes where variations of the projected field \(\chi_{\mathrm{eff}}\) are
    small and smoothly distributed, the collective relaxation dynamics admit a
    coarse-grained description in which geometric notions become operationally
    meaningful.

    In this weak-structure regime, local variations of the relaxation rate may be
    parametrized by an effective gravitational potential \(\Phi\), defined
    operationally through the relative slowdown of relaxation,
    \begin{equation}
      \frac{\partial_t \chi}{c}
      \simeq
      1 + \frac{\Phi}{c^2},
    \end{equation}
    where \(\Phi\) is dimensionally an energy per unit mass.

    This definition introduces \(\Phi\) not as a fundamental field, but as a compact
    summary of how localized excitations collectively constrain the relaxation flow
    of \(\chi\).
    In the limit of weak gradients and slow variation, this parametrization leads to
    an effective Poisson-like relation,
    \begin{equation}
      \nabla^2 \Phi = 4\pi G\,\rho,
    \end{equation}
    where \(\rho\) denotes the effective density of localized, relaxation-resistant
    configurations.

    The gravitational constant \(G\) appears here as an emergent coupling parameter
    that relates the collective response of the \(\chi\) relaxation dynamics to the
    distribution of such excitations.
    Its numerical value is fixed empirically at the effective level and reflects the
    stiffness scale of the underlying relaxation process.
    No claim is made that \(G\) is derived from first principles within this
    approximation.

  \paragraph{Operational emergence of geometry.}
    Because Cosmochrony does not postulate a fundamental spacetime metric, spatial
    geometry is defined operationally.
    Two configurations are considered close if perturbations of \(\chi\) propagate
    efficiently between them, and distant otherwise.

    In the weak-gradient regime, this operational notion induces an effective spatial
    geometry that coincides with the Newtonian and post-Newtonian descriptions of
    gravity.
    Distances, curvature, and gravitational potentials arise as macroscopic
    descriptors of how localized excitations collectively modulate the propagation
    and attenuation of relaxation perturbations.

    From this perspective, spacetime curvature is not a primitive geometric object.
    It is a derived quantity encoding the large-scale organization of the relaxation
    flow of \(\chi\).

  \paragraph{Scope and limitations.}
    The construction presented in this subsection is restricted to quasi-static,
    weak-field regimes in which a smooth geometric description provides an accurate
    summary of the underlying relaxation dynamics.
    It does not address strong-field configurations, highly nonlinear regimes, or
    quantum fluctuations of the \(\chi\) field.

    Its purpose is to demonstrate that classical gravitational behavior can be
    recovered consistently as an effective manifestation of collective relaxation,
    without introducing a fundamental metric structure or an independent
    gravitational interaction.
