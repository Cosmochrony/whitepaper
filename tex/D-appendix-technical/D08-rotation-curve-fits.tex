\subsection{Galactic Rotation Curves as Tests of Saturation Dynamics}
  \label{subsec:rotation-curve-fits}

  \subsubsection*{Rotation curve data and baryonic decomposition}

    Rotation curve data are taken from the literature for NGC~3198, NGC~2403,
    and NGC~5055, including gas and stellar surface densities when available.
    We rely on the original references and baryonic decompositions used in
    SPARC-like analyses, ensuring transparency and reproducibility.
    Distances, inclinations, and geometric parameters are fixed to their
    observationally inferred values.

  \subsubsection*{Effective saturation model}

    The effective acceleration entering the rotation-curve prediction is modeled as
    \begin{equation}
      g_{\mathrm{eff}}(r) = \sqrt{g_N(r)^2 + a_0(t)\,g_N(r)},
    \end{equation}
    where $g_N(r)$ is the Newtonian baryonic acceleration inferred from the observed
    gas and stellar distributions, and $a_0(t)$ is the emergent cosmological
    relaxation scale discussed in Section~\ref{subsec:phi-eff-galaxies}.
    In practice, the cosmological scale $a_0(t)$ entering the fits is taken as
    $a_0(t_0)=\eta\,cH_0$, with a universal projection efficiency factor
    $\eta=0.15$ fixed once and for all, and not adjusted on a galaxy-by-galaxy basis.
    This interpolation is not postulated as a fundamental law, but serves as a compact
    operational representation of the crossover between unsaturated and
    saturation-dominated $\chi$-relaxation regimes.

  \subsubsection*{Fitting procedure}

    Fits are performed by minimizing
    \begin{equation}
      \chi^2 = \sum_i \frac{\bigl[V_{\mathrm{obs}}(r_i)-V_{\mathrm{model}}(r_i)\bigr]^2}
      {\sigma_i^2},
    \end{equation}
    with the stellar mass-to-light ratio $\Upsilon_\star$ as the sole free parameter
    for each galaxy.
    The acceleration scale $a_0(t_0)$ is fixed by the cosmological relation
    $a_0(t)\sim cH(t)$ and is not adjusted on a galaxy-by-galaxy basis.
    No dark matter halo or additional degree of freedom is introduced.

    The reduced $\chi^2$ values should not be interpreted as strict goodness-of-fit
    estimators, as rotation-curve data points are affected by correlated systematic
    uncertainties (inclination, non-circular motions, disk thickness, asymmetric drift)
    that are not fully captured by the quoted statistical errors.
    The purpose of the fits is to assess the reproduction of global radial trends
    across different morphological classes.

    The effective saturation hypothesis would be falsified if:
    (i) a single $\Upsilon_\star$ fails simultaneously to reproduce inner and outer regions,
    (ii) systematic overshooting occurs in declining rotation curves,
    or (iii) flat rotation curves require galaxy-dependent acceleration scales.
