\subsection{Numerical validation of the $\chi \rightarrow \chi_{\mathrm{eff}}$ transition}
  \label{subsec:numerical-validation-of-the-chi-rightarrow-chi_eff-transition}

  This subsection provides a numerical validation of the relational-to-effective transition
  $\chi \rightarrow \chi_{\mathrm{eff}}$ introduced in Appendix~E.
  The goal is not physical realism, but a constructive demonstration that an explicit
  relational relaxation rule on a discrete network admits a coarse-grained description
  whose evolution is consistent with the coarse-grained micro-dynamics in projectable
  regimes.

  \paragraph{Discrete model and operators.}
    We consider a three-dimensional cubic lattice graph with periodic boundary conditions,
    containing $N^3$ nodes and nearest-neighbor adjacency $\mathcal{N}(i)$.
    Each node $i$ carries a scalar value $\chi_i(t)$.
    All operators are defined purely in terms of neighbor relations (graph locality) and do
    not presuppose any background continuum geometry.

  \paragraph{Explicit update rule and saturation.}
    The discrete relaxation step is defined by the local slope functional
    \begin{equation}
      S_i(\chi) \;\equiv\; \frac{1}{c^2}\sum_{j\in\mathcal{N}(i)} K_{ij}\,(\chi_i-\chi_j)^2,
      \qquad
      K_{ij} \;=\; \frac{K_0}{1+(\chi_i-\chi_j)^2/\chi_c^2},
      \label{eq:D4_Si_Kij}
    \end{equation}
    and the bounded relaxation rate
    \begin{equation}
      R_i \;\equiv\; c\,\sqrt{\max(0,\,1-S_i)}.
      \label{eq:D4_Ri_def}
    \end{equation}
    The explicit update is
    \begin{equation}
      \chi_i(t+\Delta t) \;=\; \chi_i(t) + \Delta t\Big(R_i(t) + \kappa\,(\Delta_G\chi)_i(t)\Big),
      \label{eq:D4_update}
    \end{equation}
    where $(\Delta_G\chi)_i=\sum_{j\in\mathcal{N}(i)}(\chi_j-\chi_i)$ is the graph Laplacian.
    If $S_i>1$, the bounded term saturates to $R_i=0$ (radicand clipping), and the evolution
    remains well-defined; the Laplacian term tends to reduce local slopes and assists the
    formation of a projectable regime.

  \paragraph{Coarse-graining and definition of $\chi_{\mathrm{eff}}$.}
    The effective field $\chi_{\mathrm{eff}}$ is obtained by block coarse-graining at scale
    $\ell_0$ (in lattice units), i.e. by averaging $\chi$ over disjoint cubic blocks,
    yielding a reduced lattice that represents the effective degrees of freedom:
    \begin{equation}
      \chi_{\mathrm{eff}}(t) \;\equiv\; \mathrm{CG}\big(\chi(t)\big).
      \label{eq:D4_chi_eff_def}
    \end{equation}
    No differential structure is introduced at this stage.

  \paragraph{Correct validation target: coarse-grained micro-dynamics.}
    Because the evolution operator is nonlinear and includes saturation, coarse-graining does
    not commute with the dynamics in general:
    \[
      \mathrm{CG}\!\big(\mathcal{R}(\chi)\big)\neq \mathcal{R}\!\big(\mathrm{CG}(\chi)\big).
    \]
    Accordingly, the validation targets the \emph{coarse-grained micro-dynamics}:
    \begin{equation}
      \partial_t \chi_{\mathrm{eff}} \;\approx\;
      \mathrm{CG}\!\Big(
      c\sqrt{\max(0,1-S(\chi))} + \kappa\,\Delta_G \chi
      \Big),
      \label{eq:D4_target_cg_dynamics}
    \end{equation}
    where $S(\chi)$ is defined by Eq.~\eqref{eq:D4_Si_Kij}.
    Operationally, the right-hand side is computed on the micro-lattice and then
    coarse-grained, ensuring that the comparison is performed at a consistent descriptive
    level.

  \paragraph{Residual metric.}
    Let $\chi_{\mathrm{eff}}(t)=\mathrm{CG}(\chi(t))$ and define
    \[
      \partial_t \chi_{\mathrm{eff}}(t) \approx
      \frac{\chi_{\mathrm{eff}}(t+\Delta t)-\chi_{\mathrm{eff}}(t)}{\Delta t}.
    \]
    Define the coarse-grained right-hand side
    \[
      \mathcal{R}_{\mathrm{eff}}(t) \equiv
      \mathrm{CG}\!\Big(
      c\sqrt{\max(0,1-S(\chi(t)))} + \kappa\,\Delta_G \chi(t)
      \Big).
    \]
    We then evaluate the normalized residual
    \begin{equation}
      \varepsilon(t) \equiv
      \frac{\left\|\partial_t \chi_{\mathrm{eff}}(t) - \mathcal{R}_{\mathrm{eff}}(t)\right\|}
      {\left\|\partial_t \chi_{\mathrm{eff}}(t)\right\|},
      \label{eq:D4_epsilon_def}
    \end{equation}
    where $\|\cdot\|$ denotes an $L^2$ norm over the effective lattice.

  \paragraph{Initial conditions.}
    Unless stated otherwise, simulations start from an i.i.d. Gaussian field
    $\chi_i(0)\sim\mathcal{N}(0,\sigma^2)$ with $\sigma=0.2$ (dimensionless units).
    A \emph{smooth} run includes a short pre-smoothing stage consisting of
    $n_{\mathrm{pre}}=10$ iterations of
    \[
      \chi \leftarrow \chi + \alpha\,\Delta_G\chi,
      \qquad \alpha=0.2,
    \]
    whose only role is to suppress high-frequency modes and place the system within a
    projectable regime. A \emph{rough} run corresponds to the same i.i.d. draw without
    pre-smoothing.

  \paragraph{Representative results and temporal diagnostics.}
    For $N=32$, $\ell_0=4$ lattice units, $\Delta t=0.03$ and dimensionless normalization
    $c=1$ (with parameters chosen for numerical stability on modest lattice sizes), we find
    a final residual of order $10^{-2}$.
    In a representative smooth run, the final values are
    $\varepsilon_{L^2}\approx 9.3\times 10^{-3}$ and $\varepsilon_{L^\infty}\approx 1.4\times 10^{-2}$;
    in a representative rough run, $\varepsilon_{L^2}\approx 1.45\times 10^{-2}$ and
    $\varepsilon_{L^\infty}\approx 1.63\times 10^{-2}$.
    The temporal evolution $\varepsilon(t)$ is shown in Fig.~\ref{fig:D4-epsilon-time},
    and the distribution of pointwise residuals for the smooth run is shown in
    Fig.~\ref{fig:D4-residual-hist}.
    These results provide explicit numerical evidence that the relational-to-effective
    transition is consistent with the effective description \emph{at the level of
coarse-grained dynamics}.

    \begin{figure}[t]
      \centering
      \includegraphics[width=0.78\linewidth]{D-appendix-technical/fig_D4_epsilon_vs_time_compare}
      \caption{\textbf{Residual versus time.}
      Normalized residual $\varepsilon(t)$ (Eq.~\eqref{eq:D4_epsilon_def}) for a
      representative smooth run and a rough run, illustrating convergence toward a
      small-error regime of order $10^{-2}$ over the simulated time window.}
      \label{fig:D4-epsilon-time}
    \end{figure}

    \begin{figure}[t]
      \centering
      \includegraphics[width=0.72\linewidth]{D-appendix-technical/fig_D4_residual_hist_smooth}
      \caption{\textbf{Pointwise residual distribution (smooth run).}
      Histogram of $\partial_t \chi_{\mathrm{eff}} - \mathcal{R}_{\mathrm{eff}}$ over the
      effective lattice at the final time. The distribution is centered around zero and
      remains narrow compared to the typical scale of $\partial_t\chi_{\mathrm{eff}}$,
        consistent with a small normalized residual.}
      \label{fig:D4-residual-hist}
    \end{figure}

  \paragraph{Spatial structure of the effective field and residual.}
    In addition to the quantitative diagnostics, spatial snapshots are shown to
    illustrate the geometric character of the coarse-grained field and the nature
    of the remaining discrepancies.
    Figure~\ref{fig:D4-spatial-slices} displays representative slices of
    $\chi_{\mathrm{eff}}$ and of the corresponding residual field at the final time
    for a smooth run.

    The effective field $\chi_{\mathrm{eff}}$ is observed to be smooth across
    multiple coarse-graining cells, while the residual field exhibits no coherent
    long-wavelength structure.
    This supports the interpretation that the remaining error is dominated by
    local discretization effects rather than by a breakdown of the effective
    description.

    \begin{figure}[t]
      \centering
      \includegraphics[width=0.47\linewidth]{D-appendix-technical/fig_D4_chi_eff_slice_smooth}
      \hfill
      \includegraphics[width=0.47\linewidth]{D-appendix-technical/fig_D4_residual_slice_smooth}
      \caption{\textbf{Spatial slices of the effective field and residual (smooth run).}
      \emph{Left:} slice of the coarse-grained field $\chi_{\mathrm{eff}}$ at fixed $z$,
        showing a smooth large-scale structure.
        \emph{Right:} corresponding slice of the residual
        $\partial_t \chi_{\mathrm{eff}} - \mathcal{R}_{\mathrm{eff}}$ at the same time,
        exhibiting no coherent long-wavelength pattern.}
      \label{fig:D4-spatial-slices}
    \end{figure}

  \paragraph{Interpretation and limitations.}
    This toy model demonstrates constructively that the operational coarse-graining
    procedure defining $\chi_{\mathrm{eff}}$ yields an effective description compatible with
    the coarse-grained micro-dynamics in a projectable regime.
    The present runs do not yet exhibit a sharp \emph{failure mode} (non-projectable regime)
    in which $\varepsilon(t)$ remains $\mathcal{O}(1)$; such regimes can be obtained by
    increasing initial roughness/amplitude, reducing smoothing, or forcing persistent
    saturation ($S_i>1$) through parameter choices, and are left to extended numerical
    sweeps.

  \paragraph{Reproducibility.}
    All figures and numerical values reported in this subsection are reproducible using an
    independent Python implementation provided as supplementary material in a separate
    repository. The implementation follows Eqs.~\eqref{eq:D4_Si_Kij}--\eqref{eq:D4_epsilon_def}
    exactly, including the pre-smoothing protocol and the residual diagnostics.

  \paragraph{Resolution and coarse-graining dependence (planned sweeps).}
    To assess robustness rather than a single-case agreement, the validation program is
    designed to include parameter sweeps in (i) lattice resolution $N$, and (ii) coarse-graining
    scale $\ell_0$, while keeping $\ell_0/N$ within a fixed projectable range.
    The convergence criterion is
    \[
      \varepsilon(N,\ell_0) \rightarrow 0
      \quad\text{as}\quad N\rightarrow\infty
      \ \text{with}\ \ell_0/N \ \text{fixed}.
    \]
