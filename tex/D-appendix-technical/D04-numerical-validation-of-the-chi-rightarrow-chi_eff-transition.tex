\subsection{Numerical validation of the $\chi \rightarrow \chi_{\mathrm{eff}}$ transition}
  \label{subsec:numerical-validation-of-the-chi-rightarrow-chi_eff-transition}

  This subsection presents a numerical validation of the relational-to-effective
  transition $\chi \rightarrow \chi_{\mathrm{eff}}$ introduced in Appendix~E.
  The purpose of this validation is not to establish physical realism, but to
  demonstrate explicitly that the coarse-graining procedure defining
  $\chi_{\mathrm{eff}}$ yields, in projectable regimes, an effective field obeying
  the continuum evolution equation assumed in the effective description.

  \paragraph{Discrete model and relaxation dynamics.}
    We consider a three-dimensional cubic lattice graph with periodic boundary
    conditions, containing $N^3$ nodes and nearest-neighbor adjacency.
    Each node $i$ carries a scalar value $\chi_i(t)$.
    The discrete evolution rule implements a local relaxation dynamics consistent
    with the constraints introduced in Appendix~E, combining:
    (i) bounded local propagation at maximal relaxation speed $c$, and
    (ii) a smoothing term ensuring stability in the projectable regime.
    All operators are defined purely in terms of discrete neighbor relations and do
    not presuppose any background geometry.

  \paragraph{Extraction of the effective field.}
    The effective field $\chi_{\mathrm{eff}}$ is obtained by block coarse-graining
    over relational neighborhoods of characteristic scale $\ell_0$, following the
    definition given in Appendix~E.
    Operationally, this corresponds to averaging $\chi$ over disjoint cubic blocks,
    yielding a reduced lattice that represents the effective degrees of freedom.
    No differential or geometric structure is introduced at this stage.

  \paragraph{Continuum equation and verification metric.}
    The effective description predicts that, in projectable regimes,
    $\chi_{\mathrm{eff}}$ satisfies a nonlinear continuum evolution equation of the
    form
    \begin{equation}
      \partial_t \chi_{\mathrm{eff}} \;=\;
      c\,\sqrt{1 - \frac{|\nabla \chi_{\mathrm{eff}}|^2}{c^2}},
      \label{eq:chi-eff-continuum-eq}
    \end{equation}
    where spatial derivatives are understood as effective shorthand for discrete
    operators acting on the coarse-grained field.

    To assess the validity of this equation, we compute a normalized residual
    \begin{equation}
      \varepsilon \;\equiv\;
      \frac{\left\|\partial_t \chi_{\mathrm{eff}}
              - c\sqrt{1 - |\nabla \chi_{\mathrm{eff}}|^2/c^2}\right\|}
      {\left\|\partial_t \chi_{\mathrm{eff}}\right\|},
      \label{eq:pde-residual-definition}
    \end{equation}
    where $\|\cdot\|$ denotes an $L^2$ norm over the effective lattice.

  \paragraph{Results.}
    For representative simulations with $N=32$, a coarse-graining scale
    $\ell_0=4$ lattice units, and smooth initial conditions, the normalized residual
    is found to be of order
    \[
      \varepsilon \sim 4 \times 10^{-4},
    \]
    with pointwise (maximum-norm) deviations remaining below $10^{-3}$.

    \begin{figure}[t]
      \centering
      \includegraphics[width=0.65\linewidth]{D-appendix-technical/fig_D4_residual_hist}
      \caption{\textbf{Distribution of the pointwise residual}
        $\partial_t \chi_{\mathrm{eff}} -
        c\sqrt{1-|\nabla\chi_{\mathrm{eff}}|^2/c^2}$ over the effective lattice.
        The distribution is centered around zero and exhibits no large-scale bias,
        indicating consistency with the continuum equation.}
      \label{fig:D4-residual-hist}
    \end{figure}

    \begin{figure}[t]
      \centering
      \includegraphics[width=0.48\linewidth]{D-appendix-technical/fig_D4_chi_eff_slice}
      \hfill
      \includegraphics[width=0.48\linewidth]{D-appendix-technical/fig_D4_residual_slice}
      \caption{\textbf{Spatial structure of the coarse-grained field and residual.}
      \emph{Left:} slice of $\chi_{\mathrm{eff}}$ at fixed $z$, showing a smooth
      large-scale structure.
      \emph{Right:} corresponding slice of the residual, displaying no coherent
      long-wavelength pattern. Residuals are dominated by small-scale discretization
      effects rather than systematic deviations.}
      \label{fig:D4-slices}
    \end{figure}

  \paragraph{Interpretation and limitations.}
    These results provide explicit numerical evidence that the relational
    coarse-graining procedure defining $\chi_{\mathrm{eff}}$ is consistent with the
    continuum effective equation in projectable regimes.
    They do not constitute a proof of uniqueness, nor do they address regimes where
    projectability fails (e.g., near singular configurations or horizons).
    The model should therefore be understood as a minimal demonstration that the
    $\chi \rightarrow \chi_{\mathrm{eff}}$ transition is not merely formal, but can
    be realized constructively in a discrete setting.

  \paragraph{Reproducibility.}
    All figures and numerical values reported in this subsection are reproducible
    using an independent Python implementation made available as supplementary
    material.
    The code implements the discrete relaxation dynamics, the
    coarse-graining procedure, and the residual diagnostics exactly as described above.
