\section{Discussion and Comparison with Existing Frameworks}
  \label{sec:discussion-and-comparison-with-existing-frameworks}

  The Cosmochrony framework proposes a minimal pre-geometric substrate, described by
  a single scalar quantity $\chi$, whose irreversible relaxation underlies both
  microscopic and cosmological phenomena.
  Spacetime geometry, gravitational dynamics, and quantum behavior arise only as
  effective descriptions of this underlying relaxation process.

  In this section, we discuss how this approach relates to established theoretical
  frameworks, highlight its conceptual implications, and identify open challenges.
  Particular emphasis is placed on clarifying points of contact and distinction with
  general relativity, quantum mechanics, and standard cosmological models, as well as
  on assessing the ontological and methodological economy of the framework.

  The goal is not to claim empirical superiority over existing theories, but to
  clarify the conceptual role of Cosmochrony as a deeper explanatory layer from which
  standard frameworks emerge in appropriate regimes.

  \subsection{Relation to General Relativity}
  \label{subsec:relation-to-general-relativity}

  General Relativity (GR) describes gravitation as the curvature of spacetime induced
  by energy--momentum.
  In Cosmochrony, no \emph{a priori} metric dynamics is postulated at the fundamental
  level.
  Instead, an effective spacetime geometry emerges as a descriptive framework from
  variations in the local relaxation dynamics of the $\chi$ field.

  Matter configurations, modeled as stable or metastable topological excitations of
  $\chi$, locally constrain the relaxation of the field.
  This leads to differential rates of effective proper-time evolution between
  neighboring regions.
  When expressed in geometric terms, these differences can be reinterpreted as an
  effective deformation of the spacetime metric.

  In the weak-field regime, this mechanism reproduces Newtonian gravity, while in the
  strong-field limit it yields Schwarzschild-like solutions in effective geometric
  descriptions.
  The resulting phenomenology is therefore consistent with the empirical successes
  of GR across its tested domain.

  From this perspective, gravitation is not introduced as a fundamental interaction,
  but emerges as a macroscopic manifestation of inhomogeneous $\chi$ relaxation.
  General Relativity is recovered as the appropriate effective theory describing
  this regime, rather than being supplanted or modified at the level of observable
  predictions.

  \subsection{Relation to Quantum Formalism}
  \label{subsec:relation-to-quantum-formalism}

  Quantum mechanics and quantum field theory (QFT) introduce probabilistic
  wavefunctions, operators, and quantization rules as foundational postulates~\cite{PeskinSchroeder1995QFT}.
  In contrast, Cosmochrony does not treat quantization or wave dynamics as fundamental.
  Instead, it describes a continuous pre-geometric relational substrate whose
  projected and thresholded effective descriptions give rise to the formal apparatus of quantum theory.

  Within this framework, particles correspond to localized, topologically stable configurations of the $\chi$ substrate.
  Discrete observables arise not from intrinsic microscopic discreteness, but from
  boundary conditions, topological constraints, and interaction-induced
  reprojection, which select a finite set of stable effective configurations.

  The Planck relation $E = h\nu$ is interpreted geometrically as a correspondence
  between the amount of relaxation potential redistributed during an interaction
  and the minimal projective resolution required for a coherent effective spacetime description.
  The parameter $\nu$ does not represent a fundamental oscillation of $\chi$, but a
  frequency characterizing the projective structure of the emergent spacetime description.

  Quantum correlations are described in purely relational terms.
  Entanglement corresponds to the persistence of a shared, non-factorizable $\chi$
  configuration across spatial separation, while decoherence reflects the
  irreversible loss of relational accessibility due to interaction with the environment.
  This interpretation reproduces standard quantum phenomenology, including
  nonlocal correlations, without invoking superluminal signaling, fundamental
  wavefunction collapse, or hidden variables.

  Importantly, within Cosmochrony such non-factorizable correlations are not generic:
  they persist only within a finite, critical regime of projection, and are suppressed
  when spectral rigidity, environmental coupling, or effective mass drive projected
  descriptions toward over-compression and effective factorization.

  \subsection{Analogy with Collective Phenomena in QCD}
  \label{subsec:analogy-with-collective-phenomena-in-qcd}

  A useful structural analogy may be drawn with quantum chromodynamics (QCD) in the
  low-energy regime, where the fundamental degrees of freedom introduced in the
  theory do not correspond directly to observable particles~\cite{Shifman2007QCDVacuum}.
  Quarks and gluons are not detected as isolated entities in spacetime; instead,
  hadronic properties, effective masses, and confinement phenomena emerge from a
  strongly interacting collective vacuum structure.

  In a similar conceptual spirit, the Cosmochrony framework does not attribute
  gravitational or quantum phenomena to fundamental fields propagating on a
  pre-existing spacetime.
  At the fundamental level, the theory is formulated solely in terms of the
  pre-geometric relational substrate $\chi$ and its intrinsic relaxation dynamics.
  Observable physical quantities arise only after projection, in regimes where
  $\chi$ admits a stable and sufficiently smooth effective description.

  In such regimes, it is convenient to introduce effective quantities, collectively
  denoted $\chi_{\mathrm{eff}}$, which summarize coarse-grained, relationally stable
  features of the underlying $\chi$ configurations.
  These effective quantities are not additional ontological layers, nor independent
  degrees of freedom.
  They function as regime-dependent descriptors, encoding how the relational
  structure of $\chi$ becomes expressible in terms of fields, observables, and
  geometric notions within emergent spacetime.

  This hierarchy of descriptions closely parallels the situation in QCD.
  While quarks constitute indispensable degrees of freedom at the level of the
  microscopic theory, their physical relevance is restricted to specific regimes,
  and they do not appear as freely propagating particles in the asymptotic spectrum.
  Their absence from direct observability does not signal incompleteness, but
  reflects the collective and confined nature of the underlying dynamics.

  Likewise, Cosmochrony does not require that all internal structures invoked in its
  description correspond to independently observable entities.
  What appear as elementary constituents at a given effective level may instead
  represent stable, regime-dependent invariants of the underlying $\chi$ dynamics.
  The absence of direct observability of such structures is therefore not a defect of
  the framework, but a natural consequence of its pre-geometric and relational
  character.

  As in QCD, the appropriate physical description in Cosmochrony depends critically
  on the scale and regime considered.
  While the fundamental dynamics of $\chi$ are simple in principle, the emergent
  macroscopic behavior is governed by nonlinear and collective effects that are most
  naturally captured by effective and phenomenological descriptions.
  This reinforces the view that geometry, gravitation, and quantum observables in
  Cosmochrony are emergent constructs, rather than fundamental ontological
  primitives.

  \subsection{Comparison with $\Lambda$CDM Cosmology}
  \label{subsec:comparison-with-lambdacdm-cosmology}

  The $\Lambda$CDM model provides a remarkably successful phenomenological
  description of large-scale cosmological observations by introducing cold dark
  matter, dark energy, and an early inflationary phase~\cite{peebles1993principles,planck2020results}.
  However, these components are postulated at the level of the effective model and
  are not derived from more fundamental principles.

  In Cosmochrony, cosmic expansion follows directly from the monotonic relaxation
  of the fundamental substrate $\chi$.
  The observed Hubble law emerges as a kinematic consequence of differential
  relaxation, without invoking a cosmological constant.
  When expressed in an effective geometric description, the expansion rate may be
  written as
  \begin{equation}
    H(t) = \frac{\dot{\chi}}{\chi},
  \end{equation}
  leading naturally to $H_0 \sim c / \chi(t_0)$ in the late-time regime.

  From this perspective, dark energy is not interpreted as an additional physical
  component, but as an effective description of the large-scale relaxation
  dynamics of $\chi$.
  Cosmic acceleration reflects the cumulative manifestation of this process over
  cosmological timescales.
  At the homogeneous and isotropic level, Cosmochrony reproduces the background
  expansion described by Friedmann--Lema\^{\i}tre cosmology, while offering an
  alternative interpretation of its underlying physical origin.

  Unlike $\Lambda$CDM, which introduces empirically fitted initial conditions
  and a persistent dark energy component, the Cosmochrony framework attributes the
  late-time acceleration to the intrinsic relaxation properties of the underlying
  field.
  In this view, the coincidence problem and the observed tension between local and
  global measurements of the Hubble parameter may admit an interpretation in terms of epoch-dependent relaxation
  dynamics rather than as indications of new fundamental constituents.

  At large angular scales, $\Lambda$CDM treats deviations from scale invariance in
  the cosmic microwave background (CMB) as statistical realizations around an
  ensemble-averaged spectrum, with individual low-$\ell$ modes subject to cosmic
  variance.
  Within Cosmochrony, constraints on the largest-scale configurations of the
  $\chi$ field allow for a scale-dependent attenuation of global modes.
  From this standpoint, the observed suppression of power at low multipoles may be
  may be explored as a possible structural consequence of the relaxation dynamics, rather than
  as a purely statistical fluctuation.

  Taken together, these considerations suggest that Cosmochrony offers an
  alternative interpretative framework for cosmological observations, while
  remaining compatible with the empirical successes of the standard model at the
  level of current observational precision.

  \subsection{Inflation, Horizon Problems, and Initial Conditions}
  \label{subsec:inflation-horizon-problems-and-initial-conditions}

  Standard inflationary theory addresses the horizon, flatness, and monopole
  problems by postulating a brief phase of accelerated expansion driven by an
  inflaton field.
  In the Cosmochrony framework, these issues are approached from a different
  conceptual standpoint.

  Because the fundamental field $\chi$ defines a global relaxation process rather
  than a metric expansion imposed externally, causal connectivity is preserved at
  the level of the underlying field dynamics.
  Large-scale coherence may therefore arise from the initial smoothness of $\chi$
  and its subsequent monotonic relaxation, potentially alleviating the need for a
  distinct inflationary epoch as a fundamental assumption.

  At this stage, this perspective should be regarded as an alternative
  interpretative framework rather than a complete replacement for inflationary
  cosmology.
  A detailed analysis of primordial perturbations, their spectrum, and their
  imprint on the cosmic microwave background (CMB) is required to determine the
  extent to which Cosmochrony reproduces, modifies, or departs from standard
  inflationary predictions.

  These questions define a clear direction for future work, in which the
  connection between early-time $\chi$ dynamics and observable cosmological
  signatures can be explored quantitatively.

  \subsection{Conceptual Implications and Open Challenges}
  \label{subsec:conceptual-implications-and-open-challenges}

  Cosmochrony proposes a unifying conceptual framework in which time, distance,
  energy, gravitation, and quantization emerge from the dynamics of a single
  pre-geometric relational substrate.
  This ontological economy constitutes a central strength of the framework, while
  also requiring a careful reassessment of notions traditionally treated as
  independent physical primitives.

  In particular, the framework suggests that time, energy, and irreversibility do
  not correspond to distinct fundamental entities.
  Temporal ordering arises from the monotonic relaxation of the $\chi$ substrate,
  while energy quantifies the residual capacity of localized configurations to
  resist this relaxation.
  Irreversibility then reflects the progressive exhaustion of such relaxation
  capacity.
  From this perspective, temporal flow and energetic processes are not independent
  axioms of nature, but complementary effective descriptions of the same underlying
  relational dynamics.

  At the level of effective physical descriptions, these relations are encoded in
  coarse-grained quantities such as $\chi_{\mathrm{eff}}$, which summarize how the
  relaxation structure of $\chi$ manifests in spacetime-based observables.
  These effective constructs carry no independent ontological status and remain
  valid only within regimes where a geometric interpretation is applicable.

  A concrete realization of this unification, including an explicit formulation of
  the relaxation operator and its spectral role in mass generation, is outlined in
  Appendix~\ref{subsec:perspectives_mass_spectrum}.
  While this reinterpretation addresses several long-standing conceptual tensions
  ---including the origin of the arrow of time and the status of energy conservation
  ---it also raises important open challenges.

  Among these challenges are:
  \begin{itemize}
    \item the quantitative reconstruction of cosmic microwave background anisotropies
    from early-time $\chi$ dynamics,
    \item the detailed treatment of non-equilibrium quantum measurements, decoherence,
    and reprojection processes,
    \item the emergence of gauge symmetries and interaction hierarchies from
    topological and relational features of $\chi$,
    \item and the long-term stability of solitonic particle configurations under
    extreme gravitational or radiative conditions.
  \end{itemize}

  Addressing these issues will require a combination of analytical, numerical, and
  experimental approaches, including:
  \begin{enumerate}
    \item large-scale numerical simulations of $\chi$ dynamics to quantify structure
    formation and cosmological signatures,
    \item the exploration of discretized, network-based, or lattice realizations of
    $\chi$ at microscopic scales,
    \item and targeted experimental tests of predicted $\chi$-dependent effects in
    quantum coherence, gravitation, and radiation processes.
  \end{enumerate}

  Progress along these directions may elevate Cosmochrony from a unifying
  interpretative framework to a quantitatively predictive theory, while preserving
  its minimal ontological foundation.

  \subsection{Ontological Parsimony and the Metric}
    \label{subsec:ontological-parsimony-and-the-metric}

    A potential criticism of Cosmochrony is that it merely replaces one geometric structure (the metric) with another
    (the $\chi$ field).
  This section addresses why this replacement constitutes genuine ontological progress rather than relabeling.

  \paragraph{Distinction from metric theories.}
    In General Relativity and its extensions:
    \begin{itemize}
      \item The metric $g_{\mu\nu}$ is a fundamental tensor field with 10 independent components.
      \item Spacetime curvature is a primitive geometric property.
      \item Matter and energy are conceptually distinct from geometry, coupled via the stress-energy tensor.
    \end{itemize}

    In Cosmochrony:
    \begin{itemize}
      \item Only the scalar field $\chi$ (1 component) is fundamental.
      \item The metric is a derived effective description, not an independent dynamical entity.
      \item Matter, energy, and geometry are unified as different manifestations of $\chi$ configurations.
    \end{itemize}

  \paragraph{Operational distinguishability.}
    The frameworks are operationally distinct:
    \begin{enumerate}
      \item \textbf{Degrees of freedom:} GR propagates 2 gravitational wave polarizations from 10 metric components.
      Cosmochrony propagates perturbations of 1 scalar field, with effective tensorial structure emerging only macroscopically.

      \item \textbf{Singularities:} GR singularities (where $g_{\mu\nu}$ diverges) are ontological.
      In Cosmochrony, apparent singularities mark the breakdown of the effective metric description, while $\chi$ remains well-defined.

      \item \textbf{Quantum regime:} Quantizing GR requires quantizing the metric (Wheeler-DeWitt equation).
      Quantizing Cosmochrony requires only quantizing $\chi$, with spacetime emerging from quantum $\chi$ configurations.
    \end{enumerate}

  \paragraph{The Occam's razor argument.}
    Cosmochrony achieves unification through reduction:
    \begin{align}
      \text{Traditional:} \quad & g_{\mu\nu} \,(\text{10 DOF}) + \psi \,(\text{matter}) + \Lambda \,(\text{dark energy}) \\
      \text{Cosmochrony:} \quad & \chi \,(\text{1 DOF}) \longrightarrow \{\text{spacetime}, \text{matter}, \text{expansion}\}
    \end{align}

    This represents genuine explanatory compression, not mere reformulation.

  \subsection{Relation to the Higgs Mechanism}
  \label{subsec:relation-to-the-higgs-mechanism}

  In the Standard Model, particle masses arise through spontaneous symmetry breaking of the electroweak sector, mediated
  by the Higgs field.
  In Cosmochrony, mass is not introduced as a fundamental parameter but emerges as a measure of resistance of localized
  $\chi$ configurations to the global relaxation flow.

  These two descriptions are not in contradiction.
  Rather, the Higgs field may be understood as an effective low-energy manifestation of the interaction between
  solitonic excitations and the surrounding $\chi$ background.
  In this view, the Higgs condensate encodes how localized field configurations acquire inertial properties within an
  already structured geometric substrate.

  Cosmochrony does not deny the empirical success of the Higgs mechanism, nor does it seek to modify its phenomenology
  at accessible energies.
  Instead, it suggests that the Higgs field is not fundamental, but emergent, much like the spacetime metric or quantum
  wavefunctions.
  The observed Higgs boson would then correspond to a collective excitation of the $\chi$ field associated with mass
  stabilization.
  In this framework, the Higgs vacuum expectation value (VEV) would be indirectly determined by
  the local properties of the $\chi$ background, effectively coupling the micro-physics of particle
  masses to the macro-dynamics of cosmic relaxation.

  A detailed derivation of the Higgs sector as an effective theory emerging from $\chi$ dynamics lies beyond the scope
  of the present work and is left for future investigation.

