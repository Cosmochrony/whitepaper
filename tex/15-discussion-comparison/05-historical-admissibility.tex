\subsection{Historical Admissibility of Projected Degrees of Freedom}
  \label{subsec:historical-admissibility}

  A key implication of the Cosmochrony framework is that the set of effective degrees
  of freedom accessible to physical description is not fixed once and for all, but
  depends on the admissibility conditions imposed by the relaxation state of the
  underlying substrate $\chi$.

  At the fundamental level, the dynamics of $\chi$ and the rules governing projection
  remain unchanged throughout cosmic history.
  However, the space of $\emph{admissible}$ projected configurations evolves
  irreversibly as relaxation proceeds.
  In the early Universe, strong relational constraints and high saturation levels
  severely restricted the class of configurations that could be projected into stable
  effective descriptions.
  Only highly coherent and low-complexity global configurations were admissible, while
  finer and more differentiated structures were not yet projectively stable.

  As relaxation progresses, these constraints are gradually relaxed, enlarging the
  space of admissible configurations.
  This progressive opening allows for the emergence of increasingly complex and
  localized invariants, which may be described in effective terms as particles,
  fields, and interactions.
  In this sense, the appearance of the particle spectrum is not an instantaneous or
  timeless feature of the Universe, but a historically conditioned outcome of the
  relaxation dynamics.

  Importantly, this historical admissibility does not imply any evolution of the
  fundamental laws or parameters.
  Rather, it reflects the regime-dependent stability of projected descriptions.
  Degrees of freedom that are fundamental at one effective level may be absent or
  ill-defined at another, not because they are forbidden, but because the conditions
  required for their stable projection are not satisfied.

  This perspective naturally reconciles the emergence of complexity with the
  monotonic increase of entropy.
  As the relaxation of $\chi$ enlarges the space of admissible configurations, entropy
  increases while simultaneously enabling the appearance of hierarchical and
  structured forms.
  The early Universe is thus characterized by both low entropy and low admissible
  complexity, whereas later epochs permit a richer spectrum of effective degrees of
  freedom.

  Within this framework, anomalies at the largest cosmological scales, such as the
  suppression of low-$\ell$ modes in the cosmic microwave background, may be interpreted
  as relics of this early regime of restricted admissibility, rather than as purely
  statistical fluctuations.
