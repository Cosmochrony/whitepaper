\subsection{Relation to Quantum Formalism}
  \label{subsec:relation-to-quantum-formalism}

  Quantum mechanics and quantum field theory (QFT) introduce probabilistic
  wavefunctions, operators, and quantization rules as foundational postulates~\cite{PeskinSchroeder1995QFT}.
  In contrast, Cosmochrony does not treat quantization or wave dynamics as fundamental.
  Instead, it describes a continuous pre-geometric relational substrate whose
  projected and thresholded effective descriptions give rise to the formal apparatus of quantum theory.

  Within this framework, particles correspond to localized, topologically stable configurations of the $\chi$ substrate.
  Discrete observables arise not from intrinsic microscopic discreteness, but from
  boundary conditions, topological constraints, and interaction-induced
  reprojection, which select a finite set of stable effective configurations.

  The Planck relation $E = h\nu$ is interpreted geometrically as a correspondence
  between the amount of relaxation potential redistributed during an interaction
  and the minimal projective resolution required for a coherent effective spacetime description.
  The parameter $\nu$ does not represent a fundamental oscillation of $\chi$, but a
  frequency characterizing the projective structure of the emergent spacetime description.

  Quantum correlations are described in purely relational terms.
  Entanglement corresponds to the persistence of a shared, non-factorizable $\chi$
  configuration across spatial separation, while decoherence reflects the
  irreversible loss of relational accessibility due to interaction with the environment.
  This interpretation reproduces standard quantum phenomenology, including
  nonlocal correlations, without invoking superluminal signaling, fundamental
  wavefunction collapse, or hidden variables.

  Importantly, within Cosmochrony such non-factorizable correlations are not generic:
  they persist only within a finite, critical regime of projection, and are suppressed
  when spectral rigidity, environmental coupling, or effective mass drive projected
  descriptions toward over-compression and effective factorization.
