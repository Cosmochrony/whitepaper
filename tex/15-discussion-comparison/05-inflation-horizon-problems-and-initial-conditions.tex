\subsection{Inflation, Horizon Problems, and Initial Conditions}
  \label{subsec:inflation-horizon-problems-and-initial-conditions}

  Standard inflationary theory addresses the horizon, flatness, and monopole
  problems by postulating a brief phase of accelerated metric expansion driven by
  an inflaton field.
  In the Cosmochrony framework, these issues are approached from a different
  conceptual standpoint, rooted in the pre-geometric nature of the underlying
  substrate.

  Because the fundamental quantity $\chi$ describes a global relaxation process
  rather than a metric expansion imposed on spacetime, causal connectivity is not
  defined in terms of spacetime lightcones at the most fundamental level.
  Instead, relational continuity is preserved within the $\chi$ substrate itself.
  As a consequence, large-scale coherence can arise from the initial relational
  smoothness of $\chi$ and its subsequent monotonic relaxation, without requiring
  a distinct inflationary phase as a fundamental dynamical ingredient.

  From this perspective, the horizon problem is not resolved by superluminal
  expansion within spacetime, but rendered inoperative by the absence of an
  initially fragmented causal structure at the pre-geometric level.
  Similarly, large-scale homogeneity and isotropy reflect global properties of the
  early $\chi$ configuration rather than the outcome of an inflationary smoothing
  mechanism acting on an already defined spacetime geometry.

  At the present stage, this proposal should be understood as an alternative
  interpretative framework rather than as a complete replacement for inflationary
  cosmology.
  In particular, a detailed quantitative treatment of primordial perturbations,
  their spectrum, and their imprint on the cosmic microwave background (CMB) is
  required to assess whether Cosmochrony reproduces, modifies, or departs from the
  successful predictions of standard inflationary scenarios.

  These open questions define a clear direction for future work, in which the
  connection between early-time $\chi$ dynamics, effective reprojection processes,
  and observable cosmological signatures can be explored in a systematic and
  quantitative manner.
