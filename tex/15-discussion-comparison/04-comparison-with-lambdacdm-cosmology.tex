\subsection{Comparison with $\Lambda$CDM Cosmology}
  \label{subsec:comparison-with-lambdacdm-cosmology}

  The $\Lambda$CDM model provides a remarkably successful phenomenological
  description of large-scale cosmological observations by introducing cold dark
  matter, dark energy, and an early inflationary phase~\cite{peebles1993principles,planck2020results}.
  However, these components are postulated at the level of the effective model and
  are not derived from more fundamental principles.

  In Cosmochrony, cosmic expansion follows directly from the monotonic relaxation
  of the fundamental substrate $\chi$.
  The observed Hubble law emerges as a kinematic consequence of differential
  relaxation, without invoking a cosmological constant.
  When expressed in an effective geometric description, the expansion rate may be
  written as
  \begin{equation}
    H(t) = \frac{\dot{\chi}}{\chi},
  \end{equation}
  leading naturally to $H_0 \sim c / \chi(t_0)$ in the late-time regime.

  From this perspective, dark energy is not interpreted as an additional physical
  component, but as an effective description of the large-scale relaxation
  dynamics of $\chi$.
  Cosmic acceleration reflects the cumulative manifestation of this process over
  cosmological timescales.
  At the homogeneous and isotropic level, Cosmochrony reproduces the background
  expansion described by Friedmann--Lema\^{\i}tre cosmology, while offering an
  alternative interpretation of its underlying physical origin.

  Unlike $\Lambda$CDM, which introduces empirically fitted initial conditions
  and a persistent dark energy component, the Cosmochrony framework attributes the
  late-time acceleration to the intrinsic relaxation properties of the underlying
  field.
  In this view, the coincidence problem and the observed tension between local and
  global measurements of the Hubble parameter may admit an interpretation in terms of epoch-dependent relaxation
  dynamics rather than as indications of new fundamental constituents.

  At large angular scales, $\Lambda$CDM treats deviations from scale invariance in
  the cosmic microwave background (CMB) as statistical realizations around an
  ensemble-averaged spectrum, with individual low-$\ell$ modes subject to cosmic
  variance.
  Within Cosmochrony, constraints on the largest-scale configurations of the
  $\chi$ field allow for a scale-dependent attenuation of global modes.
  From this standpoint, the observed suppression of power at low multipoles may be
  may be explored as a possible structural consequence of the relaxation dynamics, rather than
  as a purely statistical fluctuation.

  Taken together, these considerations suggest that Cosmochrony offers an
  alternative interpretative framework for cosmological observations, while
  remaining compatible with the empirical successes of the standard model at the
  level of current observational precision.
