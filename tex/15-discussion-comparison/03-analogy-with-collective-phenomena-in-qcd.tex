\subsection{Analogy with Collective Phenomena in QCD}
  \label{subsec:analogy-with-collective-phenomena-in-qcd}

  A useful structural analogy may be drawn with quantum chromodynamics (QCD) in the
  low-energy regime, where the fundamental degrees of freedom introduced in the
  theory do not correspond directly to observable particles~\cite{Shifman2007QCDVacuum}.
  Quarks and gluons are not detected as isolated entities in spacetime; instead,
  hadronic properties, effective masses, and confinement phenomena emerge from a
  strongly interacting collective vacuum structure.

  In a similar conceptual spirit, the Cosmochrony framework does not attribute
  gravitational or quantum phenomena to fundamental fields propagating on a
  pre-existing spacetime.
  At the fundamental level, the theory is formulated solely in terms of the
  pre-geometric relational substrate $\chi$ and its intrinsic relaxation dynamics.
  Observable physical quantities arise only after projection, in regimes where
  $\chi$ admits a stable and sufficiently smooth effective description.

  In such regimes, it is convenient to introduce effective quantities, collectively
  denoted $\chi_{\mathrm{eff}}$, which summarize coarse-grained, relationally stable
  features of the underlying $\chi$ configurations.
  These effective quantities are not additional ontological layers, nor independent
  degrees of freedom.
  They function as regime-dependent descriptors, encoding how the relational
  structure of $\chi$ becomes expressible in terms of fields, observables, and
  geometric notions within emergent spacetime.

  This hierarchy of descriptions closely parallels the situation in QCD.
  While quarks constitute indispensable degrees of freedom at the level of the
  microscopic theory, their physical relevance is restricted to specific regimes,
  and they do not appear as freely propagating particles in the asymptotic spectrum.
  Their absence from direct observability does not signal incompleteness, but
  reflects the collective and confined nature of the underlying dynamics.

  Likewise, Cosmochrony does not require that all internal structures invoked in its
  description correspond to independently observable entities.
  What appear as elementary constituents at a given effective level may instead
  represent stable, regime-dependent invariants of the underlying $\chi$ dynamics.
  The absence of direct observability of such structures is therefore not a defect of
  the framework, but a natural consequence of its pre-geometric and relational
  character.

  As in QCD, the appropriate physical description in Cosmochrony depends critically
  on the scale and regime considered.
  While the fundamental dynamics of $\chi$ are simple in principle, the emergent
  macroscopic behavior is governed by nonlinear and collective effects that are most
  naturally captured by effective and phenomenological descriptions.
  This reinforces the view that geometry, gravitation, and quantum observables in
  Cosmochrony are emergent constructs, rather than fundamental ontological
  primitives.
