\section{Conclusion and Outlook}
  \label{sec:conclusion-and-outlook}

  We have presented Cosmochrony, a minimalist framework in which a single fundamental entity, $\chi$, underlies the
  emergence of time, spacetime geometry, and a wide spectrum of physical phenomena.
  By identifying the irreversible relaxation of $\chi$ as the primary physical process, we have shown that familiar
  structures—from the metric tensor to the Standard Model—are not independent axioms but emergent harmonics of this relaxation.

  \textbf{A central result of this work is the \emph{ab initio} derivation of the dynamical laws.}
  Rather than postulating a convenient action, we have demonstrated that the Born--Infeld-like Lagrangian is the
  unique functional compatible with the causal saturation of relaxation fluxes at the speed $c_\chi$.
  By resolving the circularity between configuration and distance through spectral graph theory, we have
  established that spacetime geometry is the continuum encoding of microscopic connectivity, naturally recovering
  General Relativity as a thermodynamic limit.

  The framework provides a unified geometric origin for the Standard Model:
  \begin{itemize}
    \item \textbf{Gauge Interactions:}
    Reinterpreted as projection dynamics, where photons manifest as scalar transmittance and $W/Z$
    bosons as shear modes of the projection fiber, accounting for their mass without requiring a fundamental Higgs field
    as a primary ontological ingredient.
    \item \textbf{Matter and Mass:}
    Fermionic properties emerge from topological obstructions in the configuration space of $\chi$
    , while inertial mass is not postulated but arises from resistance to global relaxation, quantified by the
    \textit{spectral overlap} between localized excitations and the relaxation background.
    \item \textbf{The Dark Sector:}
    Within the same projection-based ontology, Dark Matter is identified as non-projected spectral density(sub-threshold
    inertia), contributing to gravitation without admitting a stable effective-field representation, while Dark Energy
    manifests as the global, irreversible relaxation flux $\Phi_\chi$.
  \end{itemize}

  At cosmological scales, expansion and the arrow of time follow directly from the diminishing tempo of relaxation as
  the substrate irreversibly approaches equilibrium.
  Within this framework, ``inflation'' and ``dark energy'' do not appear as fundamental ingredients, but as effective
  descriptions: their explanatory roles are replaced by pre-geometric connectivity in the early constrained regime and
  by epoch-dependent relaxation dynamics at late times.

  Beyond its conceptual unification, Cosmochrony offers a concrete program for validation.
  The transition from discrete relational constraints to effective field descriptions identifies clear numerical
  signatures for lattice simulations.
  While challenges remain—notably the precise numerical computation of the particle mass hierarchy—the framework now
  provides a complete, self-consistent theoretical bridge from the pre-geometric substrate to observable reality.

  By reducing the fundamental assumptions to a single dynamical origin, Cosmochrony offers a coherent foundation in
  which time, mass, and geometry arise as a unified whole.
  It provides a physically grounded starting point for further theoretical development, where the laws of physics are
  not postulated but derived from the structural requirements of a relaxing universe.

  \paragraph{Testable predictions and observational signatures.}
    While Cosmochrony does not aim at precision cosmology at its present stage, the framework generically allows for
    departures from standard predictions in regimes where relaxation effects become observationally relevant.
    These include large-scale cosmological correlations, strong-gravity wave propagation, and epoch-dependent effective
    expansion rates.
    The purpose of these signatures is to provide concrete criteria by which the Cosmochrony framework may be
    empirically scrutinized, as detailed in Section~\ref{sec:testable-predictions-and-observational-signatures}.
