\clearpage

\section{Conclusion and Outlook}
  \label{sec:conclusion-and-outlook}

  We have presented Cosmochrony, a minimalist framework in which a single fundamental
  entity, $\chi$, underlies the emergence of time, spacetime geometry, and a wide spectrum of physical phenomena.
  By identifying the irreversible relaxation of $\chi$ as the primary physical
  process, we have shown that familiar structures—from the metric tensor to the
  Standard Model—are not independent axioms but emergent harmonics of this relaxation.

  \textbf{A central result of this work is the \emph{ab initio} derivation of the effective dynamical laws.}
  Rather than postulating a convenient action, we have demonstrated that the
  Born--Infeld-like Lagrangian is the unique functional compatible with the causal
  saturation of relaxation fluxes at the speed $c_\chi$.
  By resolving the circularity between configuration and distance through spectral
  graph theory, we have established that spacetime geometry is the continuum encoding
  of microscopic connectivity, naturally recovering General Relativity as a
  thermodynamic limit of the underlying relational dynamics.

  The framework provides a unified geometric origin for the Standard Model:
  \begin{itemize}
    \item \textbf{Gauge interactions:}
    Reinterpreted as projection dynamics, where photons manifest as scalar
    transmittance and $W/Z$ bosons as shear modes of the projection fiber, accounting
    for their mass without requiring a fundamental Higgs field as a primary ontological ingredient.
    Electric charge and magnetic phenomena arise as oriented and chiral modes of the
    same bounded relaxation flux, rather than as independent conserved quantities or fundamental gauge fields.
    The inherently non-injective character of the projection from $\chi$ to effective
    observables provides a structural origin for quantum indeterminacy and discreteness.

    \item \textbf{Matter, mass, and stability:}
    Fermionic properties emerge from topological obstructions in the configuration
    space of $\chi$, while inertial mass is not postulated but arises as an invariant
    spectral property associated with topological frustration and inhibition of
    relaxation.

    Within this same spectral framework, particle stability and decay acquire a unified structural interpretation.
    Stable particles correspond to metastable projected $\chi$-configurations whose
    spectral invariants remain admissible under relaxation, while particle decay is
    reinterpreted as a structural instability arising when continued relaxation drives
    these configurations beyond the admissible domain of the projection.
    In this view, particle lifetimes are emergent geometric properties of the relaxation process, not stochastic inputs.

    \item \textbf{The dark sector:}
    Within the same projection-based ontology, dark matter is interpreted as non-projected spectral density
    (sub-threshold inertia), contributing to gravitation without admitting a stable effective-field representation,
    while dark energy manifests as the global, irreversible relaxation flux $\Phi_\chi$.
  \end{itemize}

  At cosmological scales, expansion and the arrow of time follow directly from the
  diminishing tempo of relaxation as the substrate irreversibly approaches
  equilibrium.
  Within this framework, ``inflation'' and ``dark energy'' do not appear as
  fundamental ingredients but as effective descriptions: their explanatory roles are
  replaced by pre-geometric connectivity in the early constrained regime and by
  vacuum-dominated relaxation regimes at late times.

  At galactic scales, the same relaxation-based ontology yields an effective gravitational phenomenology.
  Saturation of the $\chi$-relaxation constraint produces an emergent large-scale
  potential leading to asymptotically flat rotation curves in spiral galaxies, without invoking dark matter halos.
  The resulting predictions have been directly confronted with observed rotation
  curves across different morphological classes, providing a first nontrivial phenomenological confrontation of the
  framework with observations.

  A further consequence of this relaxation-based ontology is that the set of effective
  degrees of freedom accessible to physical description is not fixed once and for all.
  While the fundamental dynamics of $\chi$ and the rules governing projection remain unchanged throughout cosmic history,
  the space of \emph{admissible} projected configurations evolves as relaxation proceeds.
  In the early Universe, strong relational constraints and high saturation levels
  restricted admissible configurations to highly coherent and low-complexity global
  modes, precluding the stable projection of many structures that appear elementary at later epochs.
  As relaxation progresses, this admissible space progressively enlarges, allowing
  for the emergence of increasingly localized and differentiated invariants, which
  are described in effective terms as particles, fields, and interactions.
  In this sense, the particle content of the Universe is not a timeless input but a
  historically conditioned outcome of the relaxation dynamics, fully compatible with
  the monotonic increase of entropy and the emergence of complexity.

  In strong-gravity regimes, black hole evaporation is reinterpreted as a discrete
  reprojection process of $\chi$, governed by threshold crossings of the projection operator at the horizon.
  This mechanism ensures information preservation at the structural level, providing
  a resolution of the black hole information paradox without invoking non-unitary dynamics.

  Beyond its conceptual unification, Cosmochrony offers a concrete program for validation.
  The transition from discrete relational constraints to effective field descriptions
  identifies clear numerical signatures for lattice and spectral simulations.
  While challenges remain—notably the precise numerical computation of the particle
  mass hierarchy—the framework now provides a complete and self-consistent theoretical
  bridge from the pre-geometric substrate to observable reality.

  By reducing the fundamental assumptions to a single dynamical origin, Cosmochrony
  offers a coherent foundation in which time, mass, charge, and geometry arise as
  mutually constrained aspects of a single relaxation process.
  What evolves in Cosmochrony is not the laws of physics themselves, but the space of
  admissible forms through which a non-injective underlying reality can be consistently projected.

  \subsection*{Conceptual shift and outlook}
    Cosmochrony thus represents a conceptual shift from a traditional “matter-on-spacetime” paradigm to a
    “relaxation-of-substrate” ontology.
    By unifying gravitation, electromagnetism, and quantum phenomena within the dynamics of a single relational
    substrate $\chi$, the framework achieves a high degree of parsimony while providing finite resolutions to
    long-standing singularities.

    Beyond its unifying structure, the theory identifies a set of immediate and falsifiable research directions:

    \begin{itemize}
      \item \textbf{Cosmological signatures:}
      Deviations in the CMB power spectrum at large angular scales are expected to arise from the finite relaxation
      capacity of the primordial substrate, offering a structural explanation of low-$\ell$ anomalies.

      \item \textbf{Galactic phenomenology:}
      The effective gravitational anomaly commonly attributed to dark matter is predicted to correlate strictly with
      local relaxation gradients, providing a falsifiable alternative to both particle dark matter scenarios and
      MOND-like modifications, particularly in low-surface-brightness galaxies.

      \item \textbf{Fundamental-scale effects:}
      Environment-dependent variations in particle decay rates or effective couplings may occur in extreme gravitational
      or magnetic regimes, reflecting the local state of the $\chi$-substrate.
    \end{itemize}

    Future work will focus on the systematic derivation of the Standard Model spectrum as a hierarchy of topological
    frustration modes within the relaxation dynamics, further strengthening Cosmochrony as a finite, predictive, and
    structurally unified framework for fundamental physics.
