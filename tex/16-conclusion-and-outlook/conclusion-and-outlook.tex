\section{Conclusion and Outlook}
  \label{sec:conclusion-and-outlook}

  We have presented Cosmochrony, a minimalist framework in which a single fundamental entity, $\chi$, underlies the
  emergence of time, spacetime geometry, and a wide spectrum of physical phenomena.
  By identifying the irreversible relaxation of $\chi$ as the primary physical process, we have shown that familiar
  structures—from the metric tensor to the Standard Model—are not independent axioms but emergent harmonics of this
  relaxation.

  \textbf{A central result of this work is the \emph{ab initio} derivation of the dynamical laws.}
  Rather than postulating a convenient action, we have demonstrated that the Born--Infeld-like Lagrangian is the
  unique functional compatible with the causal saturation of relaxation fluxes at the speed $c_\chi$.
  By resolving the circularity between configuration and distance through spectral graph theory, we have
  established that spacetime geometry is the continuum encoding of microscopic connectivity, naturally recovering
  General Relativity as a thermodynamic limit.

  The framework provides a unified geometric origin for the Standard Model:
  \begin{itemize}
    \item \textbf{Gauge Interactions:}
    Reinterpreted as projection dynamics, where photons manifest as scalar transmittance and $W/Z$
    bosons as shear modes of the projection fiber, accounting for their mass without requiring a fundamental Higgs field
    as a primary ontological ingredient.
    The inherently non-injective character of the projection from $\chi$ to effective observables provides a
    structural origin for quantum indeterminacy and discreteness.
    \item \textbf{Matter and Mass:}
    Fermionic properties emerge from topological obstructions in the configuration space of $\chi$,
    while inertial mass is not postulated but arises as an invariant spectral property associated with
    topological frustration and inhibition of relaxation, rather than from fundamental coupling constants
    or symmetry-breaking mechanisms.
    \item \textbf{The Dark Sector:}
    Within the same projection-based ontology, Dark Matter is identified as non-projected spectral density
    (sub-threshold inertia), contributing to gravitation without admitting a stable effective-field
    representation, while Dark Energy manifests as the global, irreversible relaxation flux $\Phi_\chi$.
  \end{itemize}

  At cosmological scales, expansion and the arrow of time follow directly from the diminishing tempo of relaxation as
  the substrate irreversibly approaches equilibrium.
  Within this framework, ``inflation'' and ``dark energy'' do not appear as fundamental ingredients, but as effective
  descriptions: their explanatory roles are replaced by pre-geometric connectivity in the early constrained regime and
  by epoch-dependent relaxation dynamics at late times.

  A further consequence of this relaxation-based ontology is that the set of effective
  degrees of freedom accessible to physical description is not fixed once and for all.
  While the fundamental dynamics of $\chi$ and the rules governing projection remain
  unchanged throughout cosmic history, the space of \emph{admissible} projected
  configurations evolves irreversibly as relaxation proceeds.
  In the early Universe, strong relational constraints and high saturation levels
  restricted admissible configurations to highly coherent and low-complexity global
  modes, precluding the stable projection of many structures that appear elementary at
  later epochs.
  As relaxation progresses, this admissible space progressively enlarges, allowing for
  the emergence of increasingly localized and differentiated invariants, which are
  described in effective terms as particles, fields, and interactions.
  In this sense, the particle content of the Universe is not a timeless input, but a
  historically conditioned outcome of the relaxation dynamics, fully compatible with
  the monotonic increase of entropy and the emergence of complexity.

  In strong-gravity regimes, black hole evaporation is reinterpreted as a discrete reprojection process of $\chi$,
  governed by threshold crossings of the projection operator at the horizon.
  This mechanism ensures information preservation at the structural level, providing a resolution of the black hole
  information paradox without invoking non-unitary dynamics.

  Beyond its conceptual unification, Cosmochrony offers a concrete program for validation.
  The transition from discrete relational constraints to effective field descriptions identifies clear numerical
  signatures for lattice and spectral simulations.
  While challenges remain—notably the precise numerical computation of the particle mass hierarchy—the framework now
  provides a complete, self-consistent theoretical bridge from the pre-geometric substrate to observable reality.

  By reducing the fundamental assumptions to a single dynamical origin, Cosmochrony offers a coherent foundation in
  which time, mass, and geometry arise as a unified whole.
  It provides a physically grounded starting point for further theoretical development, where the laws of physics are
  not postulated but derived from the structural requirements of a relaxing universe.

  In this perspective, what evolves in Cosmochrony is not the laws of physics themselves,
  but the space of forms to which those laws can meaningfully apply.

  \paragraph{Testable predictions and observational signatures.}
    While Cosmochrony does not aim at precision cosmology at its present stage, the framework generically allows for
    departures from standard predictions in regimes where relaxation effects become observationally relevant.
    These include large-scale cosmological correlations, strong-gravity wave propagation, and epoch-dependent
    effective expansion rates.
    The purpose of these signatures is to provide concrete criteria by which the Cosmochrony framework may be
    empirically scrutinized, as detailed in
    Section~\ref{sec:testable-predictions-and-observational-signatures}.
