\subsection{Analytical Solutions of the \(\chi\)-Field Dynamics}
  \label{subsec:analytical_solutions_chi}

  To illustrate the behavior of the \(\chi\)
  field, we derive explicit analytical solutions of the dynamical equation
  \[
    \partial_t \chi = c \sqrt{1 - \frac{|\nabla \chi|^2}{c^2}},
  \]
  in two simple but physically relevant cases: \textbf{homogeneous configurations} and
  \textbf{spherically symmetric solutions}.

  \subsubsection{Homogeneous Solution}
    In a spatially homogeneous universe, \(\nabla \chi = 0\). The dynamical equation reduces to:
    \[
      \partial_t \chi = c.
    \]
    Integrating with respect to time, we obtain the trivial but fundamental solution:
    \[
      \chi(t) = \chi_0 + c t,
    \]
    where \(\chi_0\) is the initial value of \(\chi\). This solution describes the
    \textbf{background cosmological expansion} in Cosmochrony, where \(\chi\)
    grows linearly with time, directly yielding a Hubble-like law for the scale factor \(a(t) \propto \chi(t)\).

  \subsubsection{Spherically Symmetric Solution}
    Consider a spherically symmetric configuration, where \(\chi = \chi(r,t)\) and the gradient reduces to
    \(\nabla \chi = \partial_r \chi \, \hat{r}\). The dynamical equation becomes:
    \[
      \partial_t \chi = c \sqrt{1 - \frac{(\partial_r \chi)^2}{c^2}}.
    \]

    To find a stationary solution (\(\partial_t \chi = 0\)), we set:
    \[
      c \sqrt{1 - \frac{(\partial_r \chi)^2}{c^2}} = 0 \implies \partial_r \chi = \pm c.
    \]

    Integrating, we obtain:
    \[
      \chi(r) = \chi_0 \pm c r,
    \]
    where \(\chi_0\) is an integration constant. This solution represents a \textbf{conical profile} for \(\chi\)
    , with a gradient maximal (\(|\nabla \chi| = c\)
    ). While this solution is not physically realizable globally (as it violates the boundedness of \(\chi\)
    ), it illustrates the extreme case where the relaxation of \(\chi\)
    is maximally slowed by spatial gradients.

    For a more realistic, time-dependent solution, assume a separable ansatz \(\chi(r,t) = R(r)T(t)\)
    . Substituting into the dynamical equation and separating variables, we find:
    \[
      \frac{\dot{T}}{c \sqrt{1 - \frac{T^2 (R')^2}{c^2}}} = 1.
    \]
    This implies \(\dot{T} = c\), so \(T(t) = c t + T_0\). The spatial part \(R(r)\) must then satisfy:
    \[
      (R')^2 = \frac{c^2}{T^2} \left(1 - \frac{1}{c^2}\right).
    \]
    For \(T(t) = c t\), this simplifies to:
    \[
      R(r) = R_0 \pm r,
    \]
    yielding the time-dependent solution:
    \[
      \chi(r,t) = \chi_0 + c t \pm r.
    \]
    This solution describes a \textbf{propagating front} of \(\chi\), where the field grows linearly with time and varies linearly with radius.
    It is particularly relevant for modeling localized excitations, such as particle-like solitons, in a spherically symmetric geometry.

  \subsubsection{Planar Wave Solution}
    For a planar wave ansatz \(\chi(x,t) = \chi_0 + \delta \chi(x,t)\), where \(\delta \chi\)
    represents a small perturbation, we linearize the dynamical equation:
    \[
      \partial_t \delta \chi = -\frac{(\partial_x \delta \chi)^2}{2c}.
    \]
    Assuming a wave-like perturbation \(\delta \chi = \epsilon \sin(kx - \omega t)\),
    we substitute into the linearized equation and find the dispersion relation:
    \[
      \omega = \frac{c k^2}{2 \chi_0}.
    \]
    This shows that \textbf{high-wavenumber perturbations are strongly damped},
    confirming the stability of the homogeneous solution against small-scale fluctuations.
    The planar wave solution is particularly useful for modeling propagating disturbances in \(\chi\),
    such as gravitational waves or electromagnetic radiation in the Cosmochrony framework.

  \subsubsection{Conclusion}
    These analytical solutions illustrate the rich dynamical behavior of the \(\chi\)
    field in simple but physically meaningful configurations.
    The homogeneous solution underpins the cosmological expansion,
    while the spherically symmetric and planar wave solutions provide insights into localized excitations and propagating disturbances.
    Together, they confirm the consistency and versatility
    of the \(\chi\)-field dynamics as a unifying framework for spacetime, gravity, and quantum phenomena.
