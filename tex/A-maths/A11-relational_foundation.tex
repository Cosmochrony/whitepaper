\subsection{Relational Foundation and Emergent Geometry}
  \label{subsec:relational_foundation}

  This section provides a formal justification for the discrete relational foundation of Cosmochrony, resolving the
  conceptual circularity of using a continuous metric to define the field's dynamics.

  \subsubsection{The Cosmochrony Network}
    The universe is modeled as a graph $G = (V, E)$, where nodes $i \in V$ represent local states of the Cosmochron
    $\chi_i$, and edges $K_{ij} \in E$ represent their coupling strength.
    The evolution is governed by a discrete relaxation flow:
    \begin{equation}
      \frac{d\chi_i}{d\lambda} = c \sqrt{1 - \frac{1}{c^2} \sum_{j \sim i} K_{ij} (\chi_i - \chi_j)^2}
    \end{equation}
    In this framework, no background metric is required. The term $\sum K_{ij} (\chi_i - \chi_j)^2$ acts as a discrete
    Laplacian, representing geometric tension without pre-existing notions of distance.
    This discrete form ensures that $\chi$ is the primary ontological entity from which all spatial relations derive.

  \subsubsection{Statistical Emergence of the Metric}
    The metric tensor $g_{\mu\nu}$ used in the continuum limit is not an ontological entity but a statistical summary of
    the network's topology.
    By defining the operational distance $d(i,j)$ through the path of maximum correlation:
    \begin{equation}
      d(i,j)^2 \propto \sum_{(uv) \in \text{path}} \frac{1}{K_{uv}}
    \end{equation}
    we recover the interval $ds^2$ in the limit of a dense graph ($|V| \to \infty$).
    Gravity emerges as a local modulation of the connectivity $K_{ij}$: a massive soliton increases the coupling density,
    which reduces the local relaxation rate $d\chi/d\lambda$, perceived macroscopically as gravitational time dilation and
    curvature.

  \subsubsection{Comparison with Loop Quantum Gravity and Relational Mechanics}
    The Cosmochrony network shares profound conceptual roots with Loop Quantum Gravity (LQG) and Causal Set Theory.
    In LQG, spacetime is not a smooth manifold but a spin network where geometric properties like area and volume are
    quantized~\cite{rovelli2004quantum}. Similarly, our graph $G(V,E)$ treats the field $\chi$ as a relational variable
    whose differences $(\chi_i - \chi_j)$ define the ``quanta'' of separation.

    However, a key distinction lies in the role of the scalar field.
    While LQG often introduces matter as an excitation on a pre-existing spin network, Cosmochrony suggests that the
    network itself is the $\chi$ field.
    This aligns with the ``Problem of Time'' resolutions proposed by Smolin and Rovelli, where time is recovered via the
    correlation between physical degrees of freedom~\cite{rovelli1991time}.
    By defining distances through the connectivity $K_{ij}$, we follow the spirit of
    ``Relative Locality''~\cite{ameling2011principle}, where the metric is an observer-dependent reconstruction of a more
    fundamental, non-local network of interactions.
