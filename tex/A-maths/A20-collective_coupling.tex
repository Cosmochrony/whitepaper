\subsection{Collective Gravitational Coupling and Operational Geometry}
  \label{app:collective_coupling}

  \subsubsection{Status and Scope of the Construction}
    \label{subsubsec:status-and-scope-of-the-construction}

    This appendix provides the technical construction underlying the collective gravitational
    coupling introduced in Section~\ref{subsec:collective-gravitational-coupling-and-operational-geometry}.
    Its purpose is to demonstrate explicitly how an effective
    spatial geometry and a Newtonian gravitational interaction emerge from the local relaxation
    dynamics of the $\chi$ field, without assuming any pre-existing metric structure.

    The construction is intended as an effective and controlled realization of the discrete
    $\chi$ dynamics in the weak-gradient and quasi-static regime relevant for gravitational
    phenomena.
    No claim is made regarding uniqueness or fundamental completeness; rather, this
    appendix establishes the existence of a consistent mechanism supporting the physical
    interpretation presented in the main text.

  \subsubsection{Minimal Local Ansatz for the Collective Coupling $K_{ij}$}

    We consider a discrete network of sites labeled by indices $i,j$, each carrying a scalar
    field value $\chi_i$. The collective response of the $\chi$ field to local variations is
    encoded in a coupling coefficient $K_{ij}$ that governs the propagation of $\chi$ between
    neighboring sites.

    We adopt the minimal local constitutive ansatz
    \begin{equation}
      K_{ij} = K_0\, f(\Delta_{ij}),
      \qquad
      \Delta_{ij} \equiv \frac{|\chi_i - \chi_j|^2}{\bar{\chi}^2},
    \end{equation}
    where $K_0$ is a constant stiffness scale with dimensions $[{\rm length}]^{-2}$, $\bar{\chi}$
    denotes the spatially averaged background value of $\chi$, and $f(x)$ is a positive,
    dimensionless, monotonically decreasing regularization function satisfying $f(0)=1$.

    For concreteness, we adopt the illustrative choice
    \begin{equation}
      f(x) = \frac{1}{1+x},
    \end{equation}
    although the specific functional form is not essential for the emergence of the Newtonian
    limit. This ansatz ensures that the effective coupling weakens in regions of strong local
    $\chi$ variation, such as near solitonic excitations, while remaining purely local and
    symmetric ($K_{ij}=K_{ji}$).

  \subsubsection{Operational Definition of Distance on the Discrete Network}
    \label{subsubsec:operational-definition-of-distance-on-the-discrete-network}

    In the absence of a fundamental background geometry, spatial distance is defined
    operationally through the resistance to $\chi$ propagation across the network.

    For any discrete path $\gamma = \{i_0, i_1, \dots, i_n\}$ connecting two sites $i$ and $j$,
    we define the squared path length
    \begin{equation}
      d_\gamma^2 = \ell_0^2 \sum_{(i_k,i_{k+1})\in\gamma} \frac{K_0}{K_{i_k i_{k+1}}},
    \end{equation}
    where $\ell_0$ is a microscopic length scale characterizing the underlying network.

    The physical distance between sites $i$ and $j$ is then defined as the minimal path length
    over all admissible paths,
    \begin{equation}
      d(i,j) = \min_\gamma \sqrt{d_\gamma^2}.
    \end{equation}

    This definition is analogous to an electrical resistance network: strongly coupled sites
    (large $K_{ij}$) are operationally close, while weakly coupled sites are distant. Importantly,
    this notion of distance is not derived from a pre-existing geometry but emerges from the
    dynamics of $\chi$ itself.

  \subsubsection{Continuum Limit and Effective Spatial Line Element}
    \label{subsubsec:continuum-limit-and-effective-spatial-line-element}

    We now consider the continuum limit in which the lattice spacing $\varepsilon \to 0$ while
    the physical domain remains finite. For neighboring sites separated by $\varepsilon$ along
    direction $\hat{n}$, we expand
    \begin{equation}
      \chi_j - \chi_i \simeq \varepsilon\, \partial_{\hat{n}} \chi.
    \end{equation}

    In this limit, discrete sums over nearest neighbors reduce to integrals, and the collective
    interaction term becomes
    \begin{equation}
      \sum_j K_{ij}(\chi_i-\chi_j)^2
      \;\longrightarrow\;
      K_0\, \varepsilon^2 \int |\nabla \chi|^2\, d^3x,
    \end{equation}
    up to numerical factors of order unity arising from lattice geometry.

    In the weak-gradient regime, where $\chi=\bar{\chi}+\phi$ with $|\nabla\phi|\ll\bar{\chi}$,
    the operational distance induces an effective spatial line element of the form
    \begin{equation}
      ds^2 = \ell_0^2
      \left[
        \delta_{ij}
        + \mathcal{O}\!\left(\frac{\partial_i\phi\,\partial_j\phi}{\bar{\chi}^2}\right)
      \right]
      dx^i dx^j.
    \end{equation}

    This expression defines an effective spatial geometry to leading order in perturbations. The
    full spacetime structure, including temporal components, follows from the fundamental
    kinematic constraint governing the $\chi$ field.

  \subsubsection{From the Kinematic Constraint to the Field Equation}
    \label{subsubsec:from-the-kinematic-constraint-to-the-field-equation}

    The $\chi$ field obeys the fundamental kinematic constraint
    \begin{equation}
    (\partial_t \chi)^2 + |\nabla \chi|^2 = c^2,
    \end{equation}
    which enforces a bounded relaxation dynamics.

    In the discrete formulation, this constraint takes the form
    \begin{equation}
      \left(\frac{d\chi_i}{d\lambda}\right)^2
      =
      c^2\left[
           1 - \frac{1}{c^2}\sum_j K_{ij}(\chi_i-\chi_j)^2
      \right].
    \end{equation}

    Passing to the continuum and linearizing around a homogeneous background
    $\chi=\bar{\chi}(t)+\phi(\mathbf{x},t)$ with $\dot{\bar{\chi}}=c$, we obtain, to leading
    order,
    \begin{equation}
      \partial_t^2 \phi - c^2 \nabla^2 \phi = S[\rho],
    \end{equation}
    where $S[\rho]$ represents the source term associated with localized solitonic matter
    excitations.

  \subsubsection{Quasi-Static Regime and Newtonian Limit}
    \label{subsubsec:quasi-static-regime-and-newtonian-limit}

    For gravitational phenomena on astrophysical scales, characteristic timescales are much
    longer than the light-crossing time of the system. In this quasi-static regime,
    $\partial_t^2\phi$ is negligible compared to spatial gradients, and the field equation
    reduces to
    \begin{equation}
      \nabla^2 \phi = -\frac{1}{c^2} S[\rho].
    \end{equation}

    Modeling a localized soliton of rest mass $m$ as an effective source
    $S[\rho]=4\pi G c^2 \rho$, the equation becomes
    \begin{equation}
      \nabla^2 \phi = 4\pi G \rho,
    \end{equation}
    which is precisely the Poisson equation for the Newtonian gravitational potential.

    The physical gravitational potential is identified consistently with the main text as
    \begin{equation}
      \Phi = c^2 \ln\!\left(\frac{\partial_t \chi}{c}\right)
      \simeq \frac{c^2}{\bar{\chi}}\, \phi
      \qquad
      (|\phi|\ll\bar{\chi}).
    \end{equation}

  \subsubsection{Equivalence Principle}
    \label{subsubsec:equivalence-principle2}

    Within this construction, gravitational mass characterizes the strength with which a soliton
    acts as a source for $\chi$ perturbations, while inertial mass is determined by the energy
    stored in the soliton’s $\chi$ gradients,
    \begin{equation}
      M_{\rm inertial}
      =
      \frac{1}{c^2}
      \int |\nabla \chi_{\rm soliton}|^2\, d^3x.
    \end{equation}

    Because both masses originate from the same underlying $\chi$ configuration, they coincide
    within the present effective description,
    \begin{equation}
      M_{\rm grav} = M_{\rm inertial},
    \end{equation}
    establishing the equivalence principle as an emergent property rather than an independent
    postulate.

  \subsubsection{Summary}
    \label{subsubsec:summary8}

    This appendix has demonstrated explicitly how a purely local collective coupling of the
    $\chi$ field gives rise to an operational notion of distance, an effective spatial geometry,
    and a Newtonian gravitational interaction in the quasi-static regime.
    The construction supports the interpretation of gravitation presented in
    Section~\ref{subsec:collective-gravitational-coupling-and-operational-geometry} as a collective,
    emergent phenomenon rooted in the relaxation dynamics of the $\chi$ field.
