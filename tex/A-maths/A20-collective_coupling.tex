\section{Collective Gravitational Coupling and Operational Geometry}
  \label{sec:collective-coupling}

  \subsection*{A.20.1 Status and Scope of the Construction}
    This appendix provides the technical construction underlying the collective gravitational coupling introduced in Section~\ref{sec:gravity-emergence}. Its purpose is to demonstrate explicitly how an effective spatial geometry and a Newtonian gravitational interaction emerge from the local relaxation dynamics of the $\chi$ field, without assuming any pre-existing metric structure.

  \subsection*{A.20.2 Minimal Local Ansatz for the Collective Coupling $K_{ij}$}
    \label{subsec:Kij-ansatz}
    We adopt a \textbf{constitutive relation} for the connectivity matrix $K_{ij}$ that encodes the local resistance of the $\chi$ field to variations between neighboring nodes $i$ and $j$:
    \begin{equation}
      K_{ij} = K_0 \cdot f\left( \frac{|\chi_i - \chi_j|^2}{\chi_c^2} \right)
      \label{eq:Kij-ansatz}
    \end{equation}
    where:
    \begin{itemize}
      \item $K_0$ is a \textbf{fundamental stiffness scale} (with dimensions of $[\text{length}]^{-2}$), setting the maximal connectivity strength in the absence of excitations.
      \item $\chi_c$ is a \textbf{characteristic scale} of the $\chi$ field, naturally associated with the Planck length $\ell_P$ or the cosmological relaxation scale $c / H_0$.
      \item $f(x)$ is a \textbf{dimensionless regularization function} satisfying:
      \begin{itemize}
        \item $f(0) = 1$ (maximal connectivity for uniform $\chi$),
        \item $f(x) \to 0$ as $x \to \infty$ (vanishing connectivity for large gradients),
        \item $f(x)$ is monotonically decreasing (connectivity weakens with increasing gradients).
      \end{itemize}
    \end{itemize}

    For concreteness, we adopt the \textbf{illustrative choice}:
    \begin{equation}
      f(x) = \frac{1}{1 + x}
    \end{equation}
    This ansatz ensures that:
    \begin{enumerate}
      \item \textbf{Symmetry}: $K_{ij} = K_{ji}$, as required for a consistent relational network.
      \item \textbf{Locality}: $K_{ij}$ depends only on the \textbf{local difference} $|\chi_i - \chi_j|$, reflecting the pre-geometric nature of the theory.
      \item \textbf{Boundedness}: $0 < K_{ij} \leq K_0$, preventing unphysical divergences in the emergent metric.
      \item \textbf{Gradient sensitivity}: $K_{ij}$ decreases as $|\chi_i - \chi_j|$ increases, encoding the \textbf{resistance to relaxation} induced by localized excitations.
    \end{enumerate}

    \subsubsection*{Physical Interpretation}
      \begin{itemize}
        \item In regions where $\chi$ is uniform ($\chi_i \approx \chi_j$), $K_{ij} \approx K_0$, and the network behaves as a \textbf{flat spacetime} (minimal resistance to relaxation).
        \item Near localized excitations (e.g., particles or black holes), $|\chi_i - \chi_j|$ increases, reducing $K_{ij}$ and thus \textbf{slowing the relaxation of $\chi$}. This manifests as \textbf{gravitational time dilation} and \textbf{spacetime curvature}.
      \end{itemize}

    \subsubsection*{Relation to Fundamental Constants}
      The scales $K_0$ and $\chi_c$ are expected to be related to fundamental constants. For example:
      \begin{itemize}
        \item $\chi_c$ may be identified with the \textbf{Planck length} $\ell_P \approx 1.6 \times 10^{-35} \, \text{m}$, setting the scale at which quantum gravitational effects become significant.
        \item $K_0$ may be linked to the \textbf{cosmological relaxation rate} $H_0 \approx 70 \, \text{km/s/Mpc}$, via $K_0 \sim H_0^2 / c^2$.
      \end{itemize}

      The precise mapping between $(K_0, \chi_c)$ and observable constants (e.g., $G$, $\Lambda$) is derived in Section~\ref{subsec:newtonian-limit}.

  \subsection*{Operational Definition of Distance on the Discrete Network}
    In the absence of a fundamental background geometry, spatial distance is defined operationally through the resistance to $\chi$ propagation across the network. For any discrete path $\gamma = \{i_0, i_1, \ldots, i_n\}$ connecting two sites $i$ and $j$, we define the squared path length:
    \begin{equation}
      d_\gamma^2 = \ell_0^2 \sum_{(i_k, i_{k+1}) \in \gamma} \frac{K_0}{K_{i_k i_{k+1}}}
    \end{equation}
    where $\ell_0$ is a microscopic length scale characterizing the underlying network. The physical distance between sites $i$ and $j$ is then defined as the minimal path length over all admissible paths:
    \begin{equation}
      d(i, j) = \min_\gamma \sqrt{d_\gamma^2}
    \end{equation}
    This definition is analogous to an electrical resistance network: strongly coupled sites (large $K_{ij}$) are operationally close, while weakly coupled sites are distant. Importantly, this notion of distance is not derived from a pre-existing geometry but emerges from the dynamics of $\chi$ itself.

  \subsection*{Continuum Limit and Effective Spatial Line Element}
    We now consider the continuum limit in which the lattice spacing $\varepsilon \rightarrow 0$ while the physical domain remains finite. For neighboring sites separated by $\varepsilon$ along direction $\hat{n}$, we expand:
    \begin{equation}
      \chi_j - \chi_i \simeq \varepsilon \partial_{\hat{n}} \chi
    \end{equation}
    In this limit, discrete sums over nearest neighbors reduce to integrals, and the collective interaction term becomes:
    \begin{equation}
      \sum_j K_{ij} (\chi_i - \chi_j)^2 \longrightarrow K_0 \varepsilon^2 \int |\nabla \chi|^2 \, d^3 x
    \end{equation}
    up to numerical factors of order unity arising from lattice geometry. In the weak-gradient regime, where $\chi = \bar{\chi} + \phi$ with $|\nabla \phi| \ll \bar{\chi}$, the operational distance induces an effective spatial line element of the form:
    \begin{equation}
      ds^2 = \ell_0^2 \left[ \delta_{ij} + \mathcal{O}\left(\frac{\partial_i \phi}{\bar{\chi}^2}\right) \right] dx^i dx^j
    \end{equation}
    This expression defines an effective spatial geometry to leading order in perturbations. The full spacetime structure, including temporal components, follows from the fundamental kinematic constraint governing the $\chi$ field.

  \subsection*{From the Kinematic Constraint to the Field Equation}
    The $\chi$ field obeys the fundamental kinematic constraint:
    \begin{equation}
    (\partial_t \chi)^2 + |\nabla \chi|^2 = c^2
    \end{equation}
    which enforces a bounded relaxation dynamics. In the discrete formulation, this constraint takes the form:
    \begin{equation}
      \left( \frac{d \chi_i}{d \lambda} \right)^2 = c^2 \left[1 - \frac{1}{c^2} \sum_j K_{ij} (\chi_i - \chi_j)^2 \right]
    \end{equation}
    Passing to the continuum and linearizing around a homogeneous background $\chi = \bar{\chi}(t) + \phi(\mathbf{x}, t)$ with $\dot{\bar{\chi}} = c$, we obtain, to leading order:
    \begin{equation}
      \partial_t^2 \phi - c^2 \nabla^2 \phi = S[\rho]
    \end{equation}
    where $S[\rho]$ represents the source term associated with localized solitonic matter excitations.

  \subsection*{Quasi-Static Regime and Newtonian Limit}
    \label{subsec:newtonian-limit}
    For gravitational phenomena on astrophysical scales, characteristic timescales are much longer than the light-crossing time of the system. In this quasi-static regime, $\partial_t^2 \phi$ is negligible compared to spatial gradients, and the field equation reduces to:
    \begin{equation}
      \nabla^2 \phi = -\frac{1}{c^2} S[\rho]
    \end{equation}
    Modeling a localized soliton of rest mass $m$ as an effective source $S[\rho] = 4 \pi G c^2 \rho$, the equation becomes:
    \begin{equation}
      \nabla^2 \phi = 4 \pi G \rho
    \end{equation}
    which is precisely the Poisson equation for the Newtonian gravitational potential. The physical gravitational potential is identified consistently with the main text as:
    \begin{equation}
      \Phi = c^2 \ln \left( \frac{\partial_t \chi}{c} \right) \simeq \frac{c^2}{\bar{\chi}} \phi \quad (|\phi| \ll \bar{\chi})
    \end{equation}

    \subsubsection*{Emergence of the Gravitational Constant $G$}
    \label{sec:gravity-emergence}
      The gravitational constant $G$ emerges as an \textbf{effective coupling constant} derived from the microscopic parameters $K_0$ and $\chi_c$:
      \begin{equation}
        G = \frac{c^4}{16 \pi K_0 \chi_c^2}
        \label{eq:G-emergent}
      \end{equation}
      This relation shows that $G$ is not fundamental but \textbf{derived} from the underlying $\chi$-field dynamics. For example:
      \begin{itemize}
        \item If $\chi_c \approx \ell_P \approx 1.6 \times 10^{-35} \, \text{m}$, then $K_0 \approx 1.3 \times 10^{93} \, \text{m}^{-2}$ to match the observed $G$.
        \item If $\chi_c \approx c / H_0 \approx 1.4 \times 10^{26} \, \text{m}$, then $K_0 \approx 1.1 \times 10^{-52} \, \text{m}^{-2}$, reflecting a softer network with cosmological-scale effects.
      \end{itemize}

    \subsubsection*{Equivalence Principle}
      Within this construction, gravitational mass characterizes the strength with which a soliton acts as a source for $\chi$ perturbations, while inertial mass is determined by the energy stored in the soliton's $\chi$ gradients:
      \begin{equation}
        M_{\text{inertial}} = \frac{1}{c^2} \int |\nabla \chi_{\text{soliton}}|^2 \, d^3 x
      \end{equation}
      Because both masses originate from the same underlying $\chi$ configuration, they coincide within the present effective description:
      \begin{equation}
        M_{\text{grav}} = M_{\text{inertial}}
      \end{equation}
      establishing the equivalence principle as an \textbf{emergent property} rather than an independent postulate.

  \subsection*{A.20.8 Summary}
    This appendix has demonstrated explicitly how a purely local collective coupling of the $\chi$ field gives rise to an operational notion of distance, an effective spatial geometry, and a Newtonian gravitational interaction in the quasi-static regime. The construction supports the interpretation of gravitation presented in Section~\ref{sec:gravity-emergence} as a collective, emergent phenomenon rooted in the relaxation dynamics of the $\chi$ field.
