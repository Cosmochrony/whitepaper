\subsection{Resolution of the Horizon and Flatness Problems without Inflation in Cosmochrony}
  \label{subsec:cosmochrony_horizon_flatness}

  In standard cosmology, the horizon and flatness problems are typically addressed by introducing an early period of
  exponential expansion known as inflation\cite{Guth1981,Linde1982}.
  Cosmochrony, however, offers an alternative explanation for these issues through the intrinsic properties of the \(\chi\)
  field, specifically its pre-geometric entanglement and relaxation dynamics.
  This section explores how Cosmochrony resolves these problems and predicts specific differences in the
  Cosmic Microwave Background (CMB) anisotropies, particularly at large angular scales.

  The horizon problem arises because regions of the universe that are widely separated on the last scattering
  surface appear to be in thermal equilibrium, despite never having been in causal contact under standard
  Friedmann-Lema\^{\i}tre-Robertson-Walker expansion~\cite{Guth1981}.
  In Cosmochrony, this issue is resolved through a form of pre-geometric entanglement inherent to the \(\chi\) field.
  Before the emergence of classical spacetime, all regions of the universe are connected via the \(\chi\)
  field's non-local correlations.
  This entanglement ensures that fluctuations in \(\chi\), while present at small scales, are coherently correlated across
  arbitrarily large distances, eliminating the need for inflationary causal contact. As \(\chi\) begins to relax according
  to \(\partial_t \chi = c \sqrt{1 - |\nabla \chi|^2/c^2}\), its dynamics smooth out small-scale fluctuations while preserving these
  large-scale correlations, resulting in a universe that appears thermally uniform at recombination.

  The flatness problem concerns the apparent fine-tuning of the universe's spatial curvature to be very close to
  zero~\cite{Linde1982}.
  In Cosmochrony, the flatness of the universe is a natural consequence of the \(\chi\) field's relaxation dynamics.
  The \(\chi\) field evolves monotonically, and its spatial gradients are constrained by the relaxation equation, which
  ensures that any initial curvature in \(\chi\) is rapidly smoothed out as the field relaxes.
  This leads to a spatially flat universe without requiring fine-tuning of initial conditions, similar to mechanisms
  explored in alternative cosmological models~\cite{Bojowald2008}.

  Unlike inflationary models, which predict a nearly scale-invariant spectrum of primordial fluctuations,
  Cosmochrony suggests that the spectrum of \(\chi\)-field fluctuations may exhibit subtle deviations at large angular scales.
  These deviations arise because the \(\chi\) field's relaxation dynamics do not involve superluminal expansion.
  Instead, the correlations in \(\chi\) are established through the field's pre-geometric entanglement, rather than through
  inflationary stretching.
  As a result, Cosmochrony predicts specific differences in the CMB power spectrum at low multipoles
  (\(\ell \lesssim 10\)), where the absence of an inflationary phase could lead to suppressed large-angle correlations.

  \paragraph{Clarification on primordial B-modes.}
    It should be emphasized that the absence of primordial B-modes, corresponding to a vanishing or extremely small tensor-to-scalar ratio ($r \simeq 0$), is already compatible with current observational bounds from CMB polarization experiments. As such, this feature does not by itself constitute a distinctive prediction of the Cosmochrony framework. Rather, it reflects a natural consequence of the absence of an inflationary phase, without requiring parameter tuning.

    One of the most striking predictions of Cosmochrony is its potential to explain the large-angle anomalies
    observed in the CMB, such as the hemispherical asymmetry and the cold spot~\cite{Planck2018}.
    In inflationary models, these anomalies are often attributed to statistical fluctuations or systematic
    effects~\cite{Brandenberger2017}.
    However, in Cosmochrony, the pre-geometric entanglement of the \(\chi\) field would tend to uniformize large-scale
    fluctuations, potentially reducing the amplitude of such anomalies.
    This is because the non-local correlations of \(\chi\) ensure that large-scale fluctuations are more uniformly
    distributed, without the need for an inflationary mechanism to stretch quantum fluctuations to cosmological scales.

    Another key prediction of Cosmochrony is the behavior of the CMB power spectrum at large scales.
    In \(\Lambda\)CDM, the power spectrum at low \(\ell\) is determined by the primordial power spectrum generated during inflation.
    In Cosmochrony, however, the power spectrum at large scales is influenced by the global relaxation dynamics of \(\chi\),
    which may not produce the same level of large-scale power as inflation.
    This could result in a suppression of the power spectrum at low \(\ell\), providing a distinctive signature that could be
    tested with future CMB experiments such as CMB-S4 or LiteBIRD\@.

    Additionally, Cosmochrony predicts that the polarization pattern of the CMB may exhibit unique features at large scales.
    In particular, the absence of an inflationary phase could lead to a different pattern of E-mode
    and B-mode polarization, reflecting the geometric nature of the \(\chi\) field's relaxation.
    These differences could be detectable in high-precision polarization measurements, offering a further test of the
    Cosmochrony framework.

    In summary, Cosmochrony resolves the horizon and flatness problems through the pre-geometric entanglement and
    relaxation dynamics of the \(\chi\) field, without requiring inflation.
    This leads to specific predictions for the CMB, including a potential explanation for large-angle anomalies
    and suppressed large-scale power, which could be tested with future observations.
    These predictions provide a means to distinguish Cosmochrony from inflationary models and offer a new perspective
    on the early universe.
