\subsection{Topological Configurations of the \(\chi\) Field: Solitons as Particles}
  \label{subsec:topological_solitons}

  In Cosmochrony, particles are interpreted as \textbf{topologically stable solitons} of the \(\chi\) field, where
  their properties---such as \textbf{spin, charge, and mass}---emerge from the
  \textbf{local deformation of \(\chi\)} and
  its topological structure. Below, we classify these configurations and explicitly link them to particle properties
  , emphasizing how \textbf{charge arises from the modulation of \(\chi\)'s relaxation}.

  \subsubsection{Charge as Local Deformation of \(\chi\)}
    The \textbf{sign of a particle's charge} (positive or negative) is determined by how it deforms the \(\chi\)
    field:
    \begin{itemize}
      \item A \textbf{positive charge} corresponds to a \textbf{local extension of \(\chi\)} (a "peak"), which resists relaxation and repels other positive charges (as two peaks cannot merge).
      \item A \textbf{negative charge} corresponds to a \textbf{local contraction of \(\chi\)} (a "trough"), which attracts positive charges (as a peak and trough can annihilate or merge).
    \end{itemize}
    This geometric interpretation explains \textbf{Coulomb-like interactions}
    without invoking a fundamental electromagnetic field, but as a consequence of \(\chi\) dynamics.

  \subsubsection{Vortices (Charged Particles with Spin)}
    Vortices in the \(\chi\) field are characterized by a quantized winding number \(n\):
    \[
      n = \frac{1}{2\pi} \oint \nabla \arg(\chi) \cdot d\mathbf{l}.
    \]
    The \textbf{charge of the vortex} is determined by the \textbf{sign of its deformation}:
    \begin{itemize}
      \item For \(n > 0\), the vortex creates a \textbf{local extension of \(\chi\)} (positive charge).
      \item For \(n < 0\), the vortex creates a \textbf{local contraction of \(\chi\)} (negative charge).
    \end{itemize}
    The energy of the vortex scales with \(n^2\), reflecting the \textbf{mass of the particle}
    , while its winding determines the \textbf{spin} (e.g., \(n=1\) for spin-1 bosons).

  \subsubsection{Skyrmions (Fermions with Charge and Spin-1/2)}
    Skyrmions are 3D solitons with a topological charge \(Q\):
    \[
      Q = \frac{1}{4\pi} \int \mathbf{n} \cdot (\partial_x \mathbf{n} \times \partial_y \mathbf{n}) \, dx \, dy,
    \]
    where \(\mathbf{n} = \chi / |\chi|\). The \textbf{charge of the skyrmion} is linked to the
    \textbf{polarity of its \(\chi\) deformation}:
    \begin{itemize}
      \item A skyrmion with \(Q = +1\) and a \textbf{peak in \(\chi\)} represents a
      \textbf{positively charged fermion} (e.g., proton).
      \item A skyrmion with \(Q = -1\) and a \textbf{trough in \(\chi\)} represents a
      \textbf{negatively charged fermion} (e.g., electron).
    \end{itemize}
    The \(4\pi\)-periodicity of skyrmions under rotations explains their \textbf{spin-1/2}
    nature, while the deformation of \(\chi\) accounts for their charge.

  \subsubsection{Summary: Topology and Charge}
    The relationship between topology and charge in Cosmochrony is summarized in ~\ref{tab:solitons_charge}

    \begin{table}[htbp]
      \centering
      \caption{Topological Solitons, Charge, and \(\chi\) Deformation}
      \label{tab:solitons_charge}
      \begin{tabular}{|c|c|c|c|}
        \hline
        \textbf{Soliton Type} & \textbf{Topological Invariant} & \textbf{\(\chi\) Deformation} &
        \textbf{Particle Property} \\
        \hline
        Vortex & Winding number \(n\) & Peak (\(n>0\)) or trough (\(n<0\)) & Charge
        \(\propto n\), spin \(\propto |n|\) \\
        Skyrmion & Charge \(Q\) & Peak (\(Q>0\)) or trough (\(Q<0\)) & Charge
        \(\propto Q\), spin-1/2 \\
        \hline
      \end{tabular}
    \end{table}
