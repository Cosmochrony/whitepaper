\subsection{Cosmic Expansion as $\chi$ Relaxation}
  \label{subsec:cosmic-expansion-as-$chi$-relaxation}

  In cosmochrony, cosmic expansion is not driven by an initial impulse or by a cosmological constant.
  Instead, it results from the monotonic relaxation of the $\chi$ field toward larger characteristic wavelengths.

  As $\chi$ increases uniformly, spatial separations between comoving points grow proportionally.
  The recession velocity between distant objects thus arises as a cumulative effect of local $\chi$
  relaxation rather than as motion through space.

  \begin{figure}[h]
    \centering
    \begin{tikzpicture}[scale=1]

% Axes
      \draw[->] (0,0) -- (6,0) node[right]{Cosmic time};
      \draw[->] (0,0) -- (0,4) node[above]{Scale / Wavelength};

% LambdaCDM
      \draw[thick, gray, dashed]
      plot[smooth] coordinates {(0.5,0.7) (2,1.3) (4,2.5) (5.5,3.7)};
      \node[gray] at (4.5,3.2) {$\Lambda$CDM};

% Chi relaxation
      \draw[thick, blue]
      plot[smooth] coordinates {(0.5,0.8) (2,1.4) (4,2.2) (5.5,2.9)};
      \node[blue] at (4.7,2.5) {$\chi(t)$};

    \end{tikzpicture}
    \caption{Comparison between standard $\Lambda$CDM cosmological expansion and Cosmochrony. In the latter,
      the observed Hubble law emerges from the monotonic relaxation of the fundamental field $\chi$,
      without invoking dark energy.}
    \label{fig:cosmo_comparison}
  \end{figure}

  Primordial fluctuations in $\chi$ at the recombination epoch ($z \sim 1100$
  ) are imprinted as temperature anisotropies in the CMB. The near scale-invariance of these fluctuations
  reflects the universal relaxation dynamics of $\chi$
  , while their acoustic peaks arise from oscillatory coupling between $\chi$
  and matter excitations. Unlike inflationary models, no superluminal stretching is required: correlations
  extend across the observable universe because they originate from a single connected $\chi$
  -field configuration prior to relaxation.

\subsection{Emergent Hubble Law}
  \label{subsec:emergent-hubble-law}

  Let $\chi(t)$ denote the spatially averaged value of the field.
  The effective scale factor $a(t)$ scales proportionally to $\chi(t)$:
  \begin{equation}
    a(t) \propto \chi(t).
  \end{equation}

  The Hubble parameter follows directly:
  \begin{equation}
    H(t) = \frac{\dot{a}}{a} = \frac{\dot{\chi}}{\chi}.
  \end{equation}

  Assuming maximal relaxation speed $\dot{\chi} = c$, the present value of the Hubble constant becomes
  \begin{equation}
    H_0 \approx \frac{c}{\chi(t_0)},
  \end{equation}
  providing a natural scale for cosmic expansion without introducing dark energy.

\subsection{Cosmic Acceleration Without Dark Energy}
  \label{subsec:cosmic-acceleration-without-dark-energy}

  Because $\chi$ relaxation accumulates over time, recession velocities increase with distance.
  This leads to an apparent acceleration when interpreted through conventional cosmological models.

  In cosmochrony, this effect does not reflect a change in the expansion rate but the cumulative nature of $\chi$
  growth.
  Thus, accelerated expansion emerges without requiring a cosmological constant or exotic energy components.

\subsection{Cosmic Microwave Background}
  \label{subsec:cosmic-microwave-background}

  In this model, the Cosmic Microwave Background (CMB) reflects frozen fluctuations of the $\chi$
  field at the epoch when matter-radiation interactions decoupled.

  Primordial variations in $\chi$
  phase and amplitude imprint temperature anisotropies that persist as large-scale correlations.

  These fluctuations originate from stochastic variations in local $\chi$
  relaxation prior to large-scale structure formation.
  Their near scale invariance reflects the universal relaxation dynamics of the field.

  Unlike inflationary scenarios, no superluminal expansion is required to explain horizon-scale coherence:
  correlations originate from the pre-relaxation continuity of $\chi$.

  Further details on how $\chi$-field fluctuations reproduce the observed CMB anisotropies---including solutions to the
  horizon and flatness problems without inflation, as well as predicted deviations at large angular scales---are
  provided in Appendices~\ref{subsec:chi_cmb_spectrum} and~\ref{subsec:cosmochrony_horizon_flatness}.

\subsection{Hubble Tension}
  \label{subsec:hubble-tension}

  Measurements of the Hubble constant derived from early-universe observables~\cite{Planck2020,Riess2019}
  , such as the CMB, probe smaller values of $\chi$.
  In contrast, late-time measurements using local distance ladders correspond to larger accumulated $\chi$ values.

  This difference naturally produces a tension between inferred values of $H_0$
  without invoking systematic errors or new particles.

  The CMB anisotropy spectrum in the $\chi$-framework differs from inflationary predictions in the low-$\ell$
  regime, where the absence of a primordial inflationary phase would suppress large-angle correlations. This
  could be tested by future high-precision CMB experiments like CMB-S4 or LiteBIRD\@.

  The predicted decrease of $H_0$
  with cosmic time provides a natural explanation for the current Hubble tension.
  Early-universe probes (e.g., CMB-based measurements yielding $H_0 \approx 67 \, \text{km/s/Mpc}$) sample smaller $\chi$
  values than late-time distance ladder methods ($H_0 \approx 74 \, \text{km/s/Mpc}$), consistent with the observed $\sim 4.4\sigma$
  discrepancy.
  Although this numerical agreement should not be interpreted as a precision prediction, it is an indication that the
  framework naturally selects the observed cosmological scale. Future measurements of $H(z)$ across redshift ranges may
  distinguish between this geometric interpretation and dark energy models.

  This mechanism suggests that the ``tension'' is not a failure of the standard cosmological model but a
  manifestation of the non-linear coupling between matter density and the field’s relaxation rate.
  A formal derivation of this local correction, yielding an $8.4\%$ increase in $H_0$ within the KBC void, is detailed
  in Appendix~\ref{appendix:hubble_tension}.
\section{Cosmological Implications}
\label{sec:cosmology}

\subsection{Entropy and the Arrow of Time}
  \label{subsec:entropy-and-the-arrow-of-time}

  The monotonic increase of $\chi$ defines a preferred temporal direction that is fundamental within the
  Cosmochrony framework.
  This direction is not introduced through thermodynamic or statistical arguments but follows directly
  from the irreversible relaxation of the underlying field.

  As $\chi$ grows, localized excitations become increasingly separated and their configurations progressively
  lose the ability to reconcentrate relaxation potential.
  This leads to an effective irreversibility at the level of composite systems, even though the underlying
  dynamics of $\chi$ remains deterministic.

  Entropy increase is therefore interpreted as an emergent, coarse-grained description of this relaxation
  process, rather than as a fundamental driver of temporal ordering.
  In this view, thermodynamic entropy does not define the arrow of time but reflects it: the macroscopic
  growth of entropy mirrors the monotonic expenditure of the relaxation capacity of the $\chi$ field.

\subsection{Summary}
  \label{subsec:summary4}

  Cosmological expansion, apparent acceleration, the Hubble law, the CMB, and the arrow of time all emerge from the
  universal relaxation of the $\chi$ field.
  Cosmochrony reproduces key cosmological observations without introducing dark energy or modifying general
  relativity at large scales.
