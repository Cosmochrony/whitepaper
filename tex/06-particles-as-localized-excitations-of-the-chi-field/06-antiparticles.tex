\subsection{Antiparticles}
  \label{subsec:antiparticles}

  Within the Cosmochrony framework, antiparticles are interpreted as relationally
  conjugate counterparts of particle-like excitations.
  They correspond to configurations of the $\chi$ field that are topologically
  opposed to their particle partners within the internal configuration space of
  $\chi$.

  Annihilation processes occur when a particle and its conjugate excitation
  combine into a configuration that can relax continuously toward a more
  homogeneous state.
  In effective descriptions, this corresponds to the disappearance of localized
  constraints and the redistribution of previously trapped relaxation potential
  into delocalized, radiation-like excitations of the $\chi$ field.

  Throughout this process, no fundamental structure is destroyed.
  The total relational content of $\chi$ is conserved, while localized
  topological organization is converted into propagating fluctuations within the
  relaxation dynamics.
