\subsection{Spin as a Topological Property of Projected Configurations}
  \label{subsec:spin_topology}

  Within the Cosmochrony framework, spin is not introduced as an intrinsic kinematic
  degree of freedom, nor as a consequence of spacetime symmetries or fundamental
  rotation groups.
  It emerges instead, at the level of effective descriptions, as a purely topological
  property of admissible projected configurations.

  Certain stable particle-like projected configurations exhibit internal relational
  structures that cannot be continuously deformed into homogeneous effective
  descriptions.
  These configurations are characterized by non-trivial topology in their internal
  configuration space, independently of any background spatial geometry or primitive
  notion of rotation.

  In particular, a class of fermionic projected configurations requires a $4\pi$
  transformation in configuration space to return to an equivalent effective
  description.
  A $2\pi$ transformation corresponds to a non-contractible loop in the space of
  admissible projected configurations, while a $4\pi$ transformation is homotopic to
  the identity.
  Formally, this implies that the relevant configuration space admits a double covering,
  with fundamental group
  \begin{equation}
    \pi_1(\mathcal{C}_{\mathrm{eff}}) = \mathbb{Z}_2 ,
  \end{equation}
  where $\mathcal{C}_{\mathrm{eff}}$ denotes the space of admissible localized projected
  descriptions.

  When an effective quantum description becomes applicable, particle-like projected
  configurations are represented by complex wavefunctions encoding the phase structure
  associated with their topological class.
  For topologically non-trivial configurations, a $2\pi$ effective transformation
  induces a sign change of the associated wavefunction,
  \begin{equation}
    \psi \;\longrightarrow\; -\psi ,
  \end{equation}
  while a $4\pi$ transformation restores the original state.
  This $4\pi$-periodicity directly implies \textbf{fermionic antisymmetry} and the Pauli exclusion principle,
  as detailed in Section~\ref{app:relational_spin_statistics}.

  This behavior identifies such projected configurations as spin-$\tfrac{1}{2}$
  fermionic excitations.
  Importantly, the appearance of a spinorial phase does not rely on a fundamental
  representation of spatial rotation groups.
  It follows solely from the topological structure of the space of admissible projected
  descriptions.

  Fermionic statistics arises from the same topological origin.
  Two identical fermionic projected configurations belong to the same non-trivial
  topological class and therefore cannot be continuously merged into a single
  admissible configuration without violating the admissibility constraints of the
  effective description.

  Exchanging two identical fermionic excitations corresponds topologically to a
  $2\pi$ loop in the combined configuration space.
  As this operation induces a sign change of the effective wavefunction, symmetric
  configurations are dynamically excluded.
  This provides a geometric and topological origin of the Pauli exclusion principle
  within the Cosmochrony framework~\cite{Pauli1925}.

  In Cosmochrony, spin and fermionic statistics are therefore not postulated quantum
  properties.
  They are manifestations of topological obstructions in the space of admissible
  projected descriptions.
  The $4\pi$ periodicity, spin-$\tfrac{1}{2}$ behavior, and exclusion principle thus
  share a common geometric origin within the effective descriptive framework.
