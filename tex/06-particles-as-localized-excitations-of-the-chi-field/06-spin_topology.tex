\subsection{Spin as a Topological Property of $\chi$ Configurations}
  \label{subsec:spin_topology}

  Within the Cosmochrony framework, spin is not introduced as an intrinsic kinematic
  degree of freedom, nor as a consequence of spacetime symmetries.
  Instead, it emerges as a purely topological property of localized $\chi$ configurations.

  Certain stable solitonic excitations of the $\chi$ field possess an internal structure
  that cannot be continuously deformed to the vacuum configuration.
  These excitations are characterized by a non-trivial topology in their internal
  configuration space, independently of any background spatial geometry.

  In particular, a class of fermionic configurations requires a $4\pi$ internal rotation
  to return to an equivalent configuration.
  A $2\pi$ rotation corresponds to a non-contractible loop in the configuration space
  of $\chi$, while a $4\pi$ rotation is homotopic to the identity.
  Formally, this implies that the relevant configuration space admits a double covering,
  with fundamental group
  \begin{equation}
    \pi_1(\mathcal{C}_\chi) = \mathbb{Z}_2 ,
  \end{equation}
  where $\mathcal{C}_\chi$ denotes the space of admissible localized $\chi$ configurations.

  When an effective quantum description becomes applicable, localized $\chi$ excitations
  are represented by complex wavefunctions encoding the phase structure of underlying
  field fluctuations.
  For topologically non-trivial configurations, a $2\pi$ effective rotation induces
  a sign change of the associated wavefunction,
  \begin{equation}
    \psi \;\longrightarrow\; -\psi ,
  \end{equation}
  while a $4\pi$ rotation restores the original state.

  This behavior identifies such excitations as spin-$\tfrac{1}{2}$ fermions.
  Importantly, the appearance of a spinorial phase does not rely on a fundamental
  representation of the rotation group, but follows from the topological structure
  of the underlying $\chi$ configuration itself.

  The fermionic statistics of these excitations arises from the same topological origin.
  Two identical fermionic solitons correspond to configurations that share a common
  $\chi$-field topology and therefore cannot be continuously merged into a single
  configuration without violating field continuity.

  Exchanging two identical fermionic excitations corresponds topologically to a
  $2\pi$ rotation in the combined configuration space.
  As this operation induces a sign change of the effective wavefunction, symmetric
  configurations are dynamically forbidden.
  This provides a geometric and topological origin of the Pauli exclusion principle
  within the Cosmochrony framework~\cite{Pauli1925}.

  In Cosmochrony, spin and fermionic statistics are not postulated quantum properties,
  but manifestations of topological obstructions in the space of localized $\chi$
  configurations.
  The $4\pi$ periodicity, spin-$\tfrac{1}{2}$ behavior, and exclusion principle thus
  share a common geometric origin.
