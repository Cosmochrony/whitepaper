\subsection{Spectral Stability and the Lamb Shift}
  \label{subsec:lamb-shift}

  The stability of charged particle-like excitations in Cosmochrony is governed not
  only by their topological admissibility within the projection fiber, but also by the
  fine spectral structure induced by the coupling between localized modes and the
  global relaxation flow of the $\chi$ substrate.
  While the linear effective description predicts degenerate energy levels for certain
  atomic configurations, this degeneracy is generically lifted once non-linear
  saturation effects and projectability constraints are taken into account.

  A paradigmatic example is provided by the Lamb shift in atomic hydrogen, namely the
  lifting of the degeneracy between the $2S_{1/2}$ and $2P_{1/2}$ states.
  In standard quantum electrodynamics, this effect arises from radiative corrections
  associated with vacuum fluctuations and requires renormalization to remove
  ultraviolet divergences.
  In Cosmochrony, by contrast, no vacuum degrees of freedom are introduced, and no
  renormalization procedure is required.

  Within the present framework, atomic bound states correspond to admissible localized
  relaxation modes projected from the $\chi$ substrate.
  Although the $2S_{1/2}$ and $2P_{1/2}$ states carry the same global charge and share the
  same Dirac-level energy in the linear approximation, they differ in their degree of
  spatial and spectral localization.
  In particular, $S$-states ($\ell=0$) possess a non-vanishing probability density at
  the core of the charged excitation and therefore probe finer scales of the
  projection fiber, where saturation of the Born--Infeld-type dynamics and residual
  spectral frustration are maximal.
  By contrast, $P$-states ($\ell=1$) remain less sensitive to these inner-core
  constraints.

  As a result, the effective Born--Infeld dynamics governing charged excitations
  induces a small but finite upward shift of the $S$-state energy relative to the
  $P$-state.
  On dimensional grounds, this splitting may be estimated as
  \begin{equation}
    \Delta E_{\mathrm{Lamb}} \sim \kappa\, \alpha^5 m_e c^2 ,
  \end{equation}
  where $\alpha$ denotes the dimensionless ratio between the local relaxation flux
  associated with the charged excitation and the maximal admissible saturation flux
  $c_\chi$, and $\kappa$ is a numerical factor of order unity encoding the detailed
  projectability of the corresponding modes.
  Numerically, $\alpha^5 m_e c^2 \simeq 10^{-5}\,\mathrm{eV}$, corresponding to a
  frequency shift of order $1\,\mathrm{GHz}$, in agreement with the observed magnitude
  of the $2S_{1/2}$--$2P_{1/2}$ splitting in hydrogen.

  Crucially, this correction is intrinsically finite.
  The existence of a maximal relaxation speed $c_\chi$ bounds the admissible flux
  through the projection fiber and provides a natural ultraviolet cutoff.
  The Lamb shift thus emerges as a spectral signature of the coupling between a
  localized charged soliton and the global relaxation structure of the substrate,
  rather than as an effect of vacuum fluctuations.

  From this perspective, the Lamb shift is not an anomalous quantum correction but a
  generic consequence of spectral stability in a finite, non-linearly saturated
  relational substrate.
  It plays, in atomic physics, a role analogous to that of mild cosmological tensions
  at large scales: a precise indication that the underlying dynamics departs subtly
  but systematically from idealized linear laws.

  \paragraph{Spectral Probe Extension to Hyperfine Structure.}
    The same spectral-probe logic applies to hyperfine structure.
    When the electron occupies an $S$-state, its non-vanishing probability density at the
    nucleus brings the electronic and nuclear torsional cores into immediate spectral
    proximity within the projection fiber.
    The resulting interaction does not require a primitive magnetic coupling: it probes
    the local neighborhood structure of the substrate through the superposition of two
    chiral relaxation fluxes.

    Hyperfine splitting then reflects the relative alignment of these torsional fluxes.
    Parallel alignments accumulate local torsional frustration and oppose relaxation,
    leading to a higher projected energy, whereas antiparallel alignments partially
    compensate torsion and facilitate relaxation, lowering the energy of the effective
    mode.
    In this interpretation, the hyperfine transition—such as the $21\,\mathrm{cm}$ line
    of hydrogen—directly measures the short-distance torsional rigidity of the $\chi$
    substrate under near-contact conditions.

    Together with the Lamb shift, hyperfine structure thus emerges as a finite spectral
    correction governed by the same Born--Infeld saturation constraints acting at
    different relational scales.
