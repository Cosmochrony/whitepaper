\subsection{Charge as a Topological and Relaxational Property of \texorpdfstring{$\chi$}{χ}}
  \label{subsec:charge-as-a-topological-and-relaxational-property-of-chi}

  \subsubsection*{Unified relaxation budget for mass and charge}
    \label{subsec:unified_budget_mass_charge}

    In Cosmochrony, both inertial mass and electric charge draw on the same finite
    \emph{relaxation capacity} of the substrate \(\chi\).
    This follows from the fact that all effective observables are manifestations of
    admissible relaxation modes operating under the universal saturation constraint
    defined in Sec.~\ref{subsec:variational-formulation}.
    In particular, the projected dynamics admits a dimensionless saturation parameter
    \(S\) defined from local gradient density and constrained by \(S \le 1\)
    (Appendix~F).%
    \footnote{See the definition of the gradient saturation parameter \(S\) in the glossary.}

    \paragraph{Two uses of the same capacity.}
      A localized configuration supporting an effective particle excitation typically
      mobilizes relaxation capacity in (at least) two structurally distinct channels:
      (i) a \emph{scalar inhibition} channel associated with spectral frustration and inertial
      response (mass), and (ii) an \emph{oriented or chiral} channel associated with a
      net topological flux (charge). We denote their respective local contributions by
      \(\mathcal{B}_m\) and \(\mathcal{B}_e\), and postulate that admissible projected
      configurations satisfy a single combined bound
    \begin{equation}
      \label{eq:combined_budget_bound}
      \mathcal{B}_m[\chi] + \mathcal{B}_e[\chi] \le \mathcal{B}_{\max},
      \end{equation}
      where $\mathcal{B}_{\max}$ is fixed by the same universal relaxation capacity
      constraint introduced in Sec.~\ref{subsec:variational-formulation}.

    \paragraph{Structural origin of instability and decay.}
      Within this unified picture, particle stability is reinterpreted as the persistence of
      a projected \(\chi\)-configuration whose internal invariants remain \emph{admissible}
      under continued relaxation. Decay corresponds to a threshold crossing:
      when relaxation drives a configuration such that
      \(\mathcal{B}_m + \mathcal{B}_e\) approaches the saturation domain, the projected
      description can no longer maintain a stable representation and the excitation
      reconfigures into a set of lower-budget modes (reprojected products). This provides
      a non-stochastic structural origin for lifetimes and supports the possibility that
      instability rates depend weakly on the local relaxation environment.

    \paragraph{Implication for the particle spectrum.}
      Equation~\eqref{eq:combined_budget_bound} implies that effective mass and charge are
      not independent degrees of freedom but competing manifestations of a common
      substrate resource. The observed discreteness and hierarchy of particle masses may
      therefore be approached as a classification of admissible stable \(\chi\)-configurations
      under a single saturation constraint, rather than as a list of independent parameters.

  \subsubsection*{Charge as a Directed and Conjugate Relaxation Mode}
    \label{subsec:charge-as-directed-relaxation}

    In conventional quantum field theory, charge conjugation is introduced as an
    operation that reverses the sign of internal charges associated with a field.
    While this description is operationally effective, it presupposes the existence
    of fundamental charge degrees of freedom.
    Within the Cosmochrony framework, no such assumption is required.

    At the level of the pre-geometric $\chi$ substrate, there are no intrinsic charges,
    gauge fields, or conserved internal labels.
    What are described in effective theories as charges arise only after projection,
    as stable invariants characterizing admissible projected configurations.
    Charge is therefore not a fundamental attribute, but a relational property of projected descriptions.

    Charge conjugation is correspondingly reinterpreted as a transformation between
    \emph{relationally conjugate} projected configurations.
    A particle and its charge-conjugate counterpart belong to distinct but paired
    topological classes within the space of admissible projected descriptions.
    These classes are related by an internal reversal of relational organization,
    rather than by the inversion of a primitive scalar quantity.

    This relational reversal does not involve time reversal, energy inversion, or
    dynamical evolution at the level of the $\chi$ substrate.
    It is instead a symmetry of the admissible projection structure itself.
    The existence of such conjugate classes reflects the internal duality of the
    projection fiber, not an independent physical operation acting on spacetime fields.

    Within effective quantum descriptions, this relational reversal manifests as
    complex conjugation of wavefunctions and as the inversion of effective charge
    labels.
    These representations encode the topological distinction between conjugate
    projected configurations, without implying that charge is fundamental or that
    conjugation corresponds to a physical process occurring in time.

    Importantly, charge conjugation symmetry is not guaranteed within this framework.
    Because charge is a projective and relational invariant, the symmetry between
    conjugate classes may be broken by asymmetries in the projection structure itself.
    Violations of charge conjugation symmetry therefore admit a natural structural
    interpretation, without invoking explicit symmetry-breaking terms at the level
    of the $\chi$ substrate.

    In summary, charge conjugation in Cosmochrony is not the reversal of a fundamental
    charge, but a relational reversal between paired classes of admissible projected
    configurations.
    It reflects a symmetry of the effective descriptive structure, not an ontological
    operation acting on primitive physical entities.

    Within the Cosmochrony framework, electric charge is neither a fundamental scalar
    label nor a primitive phase degree of freedom.
    It is instead interpreted as a chiral--torsional invariant of the relaxation flux
    associated with a projected $\chi$ configuration.

    At the discrete level, the $\chi$ dynamics induces a bounded relaxation flux
    $\vec{J}_\chi \sim \nabla \chi$ defined on relational links.
    Charge corresponds to the non-integrability of this flux around localized excitations:
    the transport of its local orientation along a closed relational loop fails to return
    to its initial state.
    The resulting winding number defines an integer-valued topological invariant,
    whose sign encodes the chirality of the configuration.

    In this picture, phase emerges only at the level of projection as a descriptive residue
    of the underlying torsional structure.
    The bounded nature of $\vec{J}_\chi$, inherited from the universal saturation
    constraint on admissible relaxation fluxes (Sec.~\ref{subsec:variational-formulation}),
    implies a maximal admissible charge density and naturally removes short-distance
    singularities.
    Electric attraction and repulsion arise from the geometric compatibility or frustration
    between the torsional flux patterns of neighboring excitations.

  \paragraph{CP Symmetry as Projective Chirality}
      \label{subsec:cp-as-projective-chirality}

      In conventional particle physics, CP symmetry combines charge conjugation and
      spatial parity reversal.
      Its observed violation is typically introduced through complex phases in effective
      Lagrangians, without a deeper structural explanation.
      Within the Cosmochrony framework, CP symmetry admits a natural reinterpretation in
      terms of projective chirality.

      As established in the preceding sections, charge conjugation corresponds to a
      relational reversal between conjugate classes of admissible projected configurations.
      Parity, in turn, is not interpreted as a fundamental inversion of spatial coordinates,
      but as a reversal of orientation within the effective geometric description that
      emerges after projection.

      CP symmetry therefore corresponds to a combined transformation acting on the
      \emph{orientation of the projection itself}.
      In Cosmochrony, projected configurations may admit an intrinsic chirality: an
      orientation asymmetry in the mapping from the pre-geometric $\chi$ substrate to
      effective spacetime descriptions.
      This chirality is not imposed dynamically, nor encoded in a fundamental interaction,
      but arises from the geometry of the projection fiber.

      When the projection is achiral, conjugate configurations are mapped symmetrically,
      and CP symmetry is preserved at the effective level.
      When the projection is chiral, however, the relational reversal associated with charge
      conjugation is no longer equivalent to a parity-reversed description.
      CP symmetry is then violated as a direct consequence of projective asymmetry.

      Importantly, this violation does not require the introduction of explicit CP-violating
      terms at the level of the $\chi$ substrate.
      It reflects a geometric asymmetry in how relational structures are realized as
      effective spacetime configurations.
      CP violation is therefore reinterpreted as a manifestation of projective chirality,
      not as a fundamental breaking of symmetry in the underlying ontology.

      In this view, CP symmetry is effective, contingent, and emergent.
      Its violation signals an asymmetry of the projection itself, rather than a failure of
      fundamental physical laws.
