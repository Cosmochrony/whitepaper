\subsection{Summary}
  \label{subsec:summary-particles}

  Within the Cosmochrony framework, particles are not fundamental ontological
  constituents.
  They arise only at the level of effective descriptions, as stable and localized
  projected configurations that resist admissible relaxation ordering.
  Their physical properties are not postulated independently, but emerge as invariants
  of the structural and topological organization of admissible projected descriptions.

  Mass is identified with the degree of effective resistance to $\chi$ relaxation
  encoded in a localized projected configuration.
  It quantifies how strongly such a configuration constrains admissible relaxation
  ordering relative to a homogeneous effective background.
  In regimes where a relativistic description applies, this interpretation naturally
  leads to the relation $E = mc^2$, understood here as a kinematic identity expressing
  the equivalence between relaxation resistance and inertial response, rather than as
  a fundamental postulate.

  Spin and statistical behavior originate from topological obstructions in the space
  of admissible projected configurations.
  Fermionic configurations exhibit a $4\pi$ periodicity in configuration space, such
  that a $2\pi$ loop is non-contractible and induces a sign change of the effective
  wavefunction.
  This topological structure provides a unified origin for spin-$\tfrac{1}{2}$ behavior, fermionic antisymmetry,
  and the Pauli exclusion principle, without invoking additional quantum axioms or intrinsic spin degrees of
  freedom~\cite{Pauli1925,Dirac1928}.

  Within this perspective, different particle attributes correspond to distinct
  topological invariants of admissible projected descriptions.
  Spin is associated with non-trivial covering properties of configuration space,
  while electric charge may be interpreted, at the effective level, as an oriented
  topological defect or vortex-like structure within projected configurations.
  These attributes remain conceptually distinct, yet arise from a common relational
  substrate once a geometric interpretation becomes applicable.

  Beyond their role as defining invariants, these structural and topological properties
  also control the fine spectral stability of particle-like excitations.
  Residual spectral splittings—such as the Lamb shift and hyperfine structure—arise as
  finite corrections induced by non-linear saturation and projectability constraints,
  reflecting how localized configurations couple either to the global relaxation
  background or to the immediate torsional neighborhood within the projection fiber.
  These effects do not introduce new particle attributes, but probe the same underlying
  relational structure at higher spectral resolution.

  Taken together, these results provide a unified account of particle properties
  compatible with both relativistic and quantum phenomenology, without introducing
  particles or their attributes as fundamental entities.
  Particles appear instead as stable descriptive regimes of the underlying relational
  structure, whose observable properties reflect the topology of admissible projected
  configurations.
