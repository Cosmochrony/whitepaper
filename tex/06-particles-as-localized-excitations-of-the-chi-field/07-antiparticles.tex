\subsection{Antiparticles}
  \label{subsec:antiparticles}

  Within the Cosmochrony framework, antiparticles are not interpreted as independent
  fundamental entities or as excitations propagating backward in time.
  They arise, at the level of effective descriptions, as relationally conjugate
  counterparts of particle-like projected configurations.

  A particle and its antiparticle correspond to projected configurations belonging to
  distinct but conjugate topological classes within the space of admissible projected
  descriptions.
  These classes are related by an internal reversal of relational structure rather
  than by an inversion of a fundamental dynamical variable.

  Annihilation processes occur when a particle-like projected configuration and its
  conjugate combine into a composite projected description that no longer supports
  localized structural constraints.
  Such a configuration admits a continuous deformation toward a more homogeneous
  effective description, in which localized relaxation resistance disappears.

  In effective geometric and quantum descriptions, this transition manifests as the
  conversion of particle–antiparticle structure into delocalized radiation-like
  projected excitations.
  No fundamental structure is destroyed in this process.
  The underlying relational substrate remains intact, while localized topological
  organization is redistributed into admissible projected configurations with extended
  support.

  In this sense, particle–antiparticle annihilation does not represent the destruction
  of matter, but a reorganization of effective relational structure from localized to
  delocalized forms within the space of admissible descriptions.
