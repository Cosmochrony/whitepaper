\subsection{Particle Creation and Destruction}
  \label{subsec:particle-creation-and-destruction}

  Within the Cosmochrony framework, particle creation does not correspond to the
  appearance of new fundamental entities.
  It arises at the level of effective descriptions, when a projected configuration
  acquires sufficient structural organization to support a stable, localized
  topological class.
  Such configurations become identifiable as particle-like only once a spacetime
  interpretation becomes meaningful.

  Particle creation therefore reflects the emergence of a new admissible projected
  description with persistent localization and relaxation resistance.
  This process does not involve the generation of structure at the level of the
  $\chi$ substrate, but a reorganization of admissible projected configurations within
  the space of effective descriptions.

  Conversely, particle destruction does not represent the annihilation of a
  fundamental object.
  It occurs when a previously localized projected configuration loses its topological
  admissibility or stability class.
  In such cases, the configuration can no longer sustain localized relaxation
  constraints and admits a continuous deformation toward a more delocalized effective
  description.

  In effective geometric and quantum regimes, this transition manifests as the
  conversion of particle-like projected configurations into extended, radiation-like
  descriptions.
  Creation and destruction thus reflect changes in the organization and admissibility
  of projected descriptions, rather than the appearance or disappearance of
  fundamental entities.

  Within this perspective, particles are not primitive ontological constituents.
  They are stable descriptive regimes of the relational substrate, whose formation
  and dissolution correspond to transitions between distinct classes of admissible
  projected configurations.
