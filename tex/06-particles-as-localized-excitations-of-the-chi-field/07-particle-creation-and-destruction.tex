\subsection{Particle Creation and Destruction}
  \label{subsec:particle-creation-and-destruction}

  Within the Cosmochrony framework, particle creation corresponds to the formation
  of stable, localized configurations of the $\chi$ field.
  Such configurations may arise when dynamical fluctuations of $\chi$ interact or
  self-organize in a manner that leads to topological stabilization and resistance
  to relaxation.

  Conversely, particle destruction occurs when a localized excitation loses its
  topological stability, either through interactions with other excitations or
  through the progressive loss of internal coherence.
  In effective descriptions, this process corresponds to the conversion of
  localized structural constraints into delocalized, radiation-like fluctuations
  of the $\chi$ field.

  This perspective removes the need for particles as primitive ontological
  entities and replaces it with a purely dynamical and relational description, in
  which creation and annihilation reflect changes in the organization of $\chi$
  rather than the appearance or disappearance of fundamental objects.
