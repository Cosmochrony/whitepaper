\subsection{Fermions and Bosons}
  \label{subsec:fermions-and-bosons}

  Within the Cosmochrony framework, particle statistics arise from the internal
  topological structure of localized $\chi$ excitations rather than from
  postulated quantum rules.
  Distinct classes of excitations are characterized by how their internal
  configuration responds to continuous rotations in configuration space.

  Configurations that require a $4\pi$ internal phase rotation to return to an
  equivalent state exhibit fermion-like behavior, while configurations that are
  $2\pi$-periodic correspond to boson-like excitations.
  This distinction reflects a fundamental topological property of the underlying
  $\chi$ configuration, not a feature imposed by external symmetry principles.

  In effective descriptions, such $4\pi$-periodic configurations may be associated
  with non-orientable or twisted internal structures, while $2\pi$-periodic
  configurations correspond to orientable ones.
  This provides a natural qualitative explanation for the spin--statistics
  connection, without introducing additional quantum postulates at the
  fundamental level.

  As throughout this work, references to phase rotations or periodicity should be
  understood as properties of the internal configuration space of $\chi$.
  Geometric representations of these structures are effective and illustrative,
  and do not imply the existence of a fundamental spatial manifold.
