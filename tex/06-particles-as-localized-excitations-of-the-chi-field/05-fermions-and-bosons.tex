\subsection{Fermions and Bosons}
  \label{subsec:fermions-and-bosons}

  Within the Cosmochrony framework, particle statistics do not arise from fundamental
  quantization rules or postulated commutation relations.
  They emerge at the level of effective descriptions from the topological structure
  of admissible projected configurations.

  Distinct classes of particle-like projected configurations are characterized by how
  their internal configuration space responds to continuous rotations.
  Certain projected configurations require a $4\pi$ rotation in configuration space
  to return to an equivalent description, while others are $2\pi$-periodic.
  The former give rise to fermion-like behavior, whereas the latter correspond to
  boson-like excitations.

  This distinction reflects a topological obstruction in the space of admissible
  projected descriptions rather than a symmetry principle imposed at the fundamental
  level.
  In effective geometric regimes, $4\pi$-periodic configurations may be associated
  with twisted or non-orientable internal structures, while $2\pi$-periodic
  configurations correspond to orientable ones.
  Such associations are descriptive and do not imply the existence of a fundamental
  spatial manifold or intrinsic spin variables.

  Within this perspective, the spin--statistics connection admits a natural qualitative
  interpretation.
  Fermionic and bosonic behavior reflects the topological classification of admissible
  projected configurations under continuous transformations, without introducing
  additional quantum postulates at the level of the $\chi$ substrate.

  As throughout this work, references to phase rotations, periodicity, or internal
  configuration space should be understood strictly within the effective descriptive
  framework.
  They characterize properties of projected descriptions and their topological
  invariants, not intrinsic attributes of the pre-geometric $\chi$ substrate.
