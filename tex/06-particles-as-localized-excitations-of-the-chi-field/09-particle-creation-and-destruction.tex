\subsection{Particle Creation and Destruction}
  \label{subsec:particle-creation-and-destruction}

  Within the Cosmochrony framework, particle creation does not correspond to the
  appearance of new fundamental entities.
  It arises at the level of effective descriptions, when a projected configuration
  acquires sufficient structural organization to support a stable, localized
  topological class.
  Such configurations become identifiable as particle-like only once a spacetime
  interpretation becomes meaningful.

  Particle creation therefore reflects the emergence of a new admissible projected
  description with persistent localization and relaxation resistance.
  This process does not involve the generation of structure at the level of the
  $\chi$ substrate, but a reorganization of admissible projected configurations within
  the space of effective descriptions.

  Conversely, particle destruction does not represent the annihilation of a
  fundamental object.
  It occurs when a previously localized projected configuration loses its topological
  admissibility or stability class.
  In such cases, the configuration can no longer sustain localized relaxation
  constraints and admits a continuous deformation toward a more delocalized effective
  description.

  In effective geometric and quantum regimes, this transition manifests as the
  conversion of particle-like projected configurations into extended, radiation-like
  descriptions.
  Creation and destruction thus reflect changes in the organization and admissibility
  of projected descriptions, rather than the appearance or disappearance of
  fundamental entities.

  Within this perspective, particles are not primitive ontological constituents.
  They are stable descriptive regimes of the relational substrate, whose formation
  and dissolution correspond to transitions between distinct classes of admissible
  projected configurations.

\paragraph{CPT as a Global Projective Property}
  \label{subsec:cpt-as-global-projective-property}

  In conventional quantum field theory, CPT symmetry is elevated to the status of a
  fundamental theorem, derived under assumptions of locality, Lorentz invariance,
  unitarity, and a fixed spacetime background.
  Within the Cosmochrony framework, none of these structures are fundamental.
  CPT symmetry therefore cannot be treated as a primitive axiom, but must be
  reinterpreted in relational and projective terms.

  As established in the preceding sections, charge conjugation (C) corresponds to a
  relational reversal between conjugate classes of admissible projected configurations,
  parity (P) reflects an orientation reversal within the effective geometric
  description, and time reversal (T) has no fundamental meaning at the level of the
  $\chi$ substrate.
  Each of these operations is individually effective and representation-dependent.

  Nevertheless, when projected descriptions admit a stable spacetime interpretation,
  they must satisfy a global consistency requirement.
  The projection from the pre-geometric $\chi$ substrate to effective spacetime cannot
  generate observable descriptions that are mutually incompatible representations of
  the same underlying relational structure.
  This requirement imposes a global constraint on admissible projected descriptions.

  CPT symmetry emerges in Cosmochrony as precisely this constraint.
  While C, P, and T may each be violated individually at the effective level due to
  asymmetries in projection, their combined action corresponds to a full relational
  conjugation of projected descriptions.
  This combined transformation maps any admissible effective configuration to another
  admissible configuration representing the same underlying $\chi$ structure.

  In this sense, CPT invariance is not a microscopic symmetry acting on fundamental
  degrees of freedom.
  It is a global projective consistency condition ensuring that the space of admissible
  projected descriptions is closed under full relational conjugation.
  Violations of CPT would therefore signal not new dynamics, but a breakdown of
  projectability or internal inconsistency in the effective description.

  Importantly, this interpretation explains why CPT symmetry is observed to be
  extraordinarily robust in effective physical theories, even when C, P, and CP are
  violated.
  CPT invariance reflects the structural coherence of the projection itself, not a
  fundamental invariance of spacetime or quantum fields.

  In summary, CPT symmetry in Cosmochrony is reinterpreted as a global property of
  admissible projection.
  It expresses the requirement that all effective descriptions remain faithful,
  mutually consistent realizations of a single underlying relational substrate.
  CPT is therefore preserved not because it is postulated, but because any violation
  would correspond to a failure of the projection to represent a coherent physical
  universe.

\paragraph{Why CPT Survives Quantum Gravity}
\label{subsec:why-cpt-survives-quantum-gravity}

Approaches to quantum gravity often raise the possibility that CPT symmetry might
fail once spacetime locality, Lorentz invariance, or unitarity are no longer
fundamental.
In many frameworks, CPT invariance is tied to properties of a fixed spacetime
background and to the axioms of relativistic quantum field theory.
From this perspective, its survival in a pre-geometric regime may appear
non-trivial.

In the Cosmochrony framework, this concern is resolved by a shift in perspective.
CPT symmetry is not treated as a microscopic invariance acting on fundamental
degrees of freedom, but as a global consistency property of admissible projections
from the pre-temporal $\chi$ substrate to effective spacetime descriptions.
As such, its validity does not depend on the existence of a fundamental spacetime,
local fields, or canonical quantization rules.

At the level of the $\chi$ substrate, there is no notion of time reversal, spatial
inversion, or charge as an intrinsic attribute.
Consequently, none of the individual operations C, P, or T have fundamental
meaning.
What does exist is a relational structure admitting conjugate realizations under
full relational reversal.
The requirement that these conjugate realizations correspond to mutually consistent
effective descriptions imposes a global constraint on projection.

This constraint is precisely what manifests as CPT invariance at the effective
level.
Even in regimes where spacetime geometry becomes highly curved, fluctuating, or
partially non-projectable, admissible effective descriptions must remain closed
under full relational conjugation.
Any failure of CPT would correspond not to a novel quantum-gravitational effect,
but to a breakdown of projectability itself, signaling that the effective
description has exceeded its domain of validity.

Importantly, quantum-gravitational phenomena such as strong curvature, horizon
formation, or near-deprojection regimes do not invalidate this constraint.
They modify the range and resolution of admissible effective descriptions, but do
not alter the underlying relational structure of $\chi$.
As long as a projected description exists at all, its global consistency under
relational conjugation is preserved.

This provides a structural explanation for the remarkable empirical robustness of
CPT symmetry.
CPT survives quantum gravity not because it is protected by spacetime symmetries,
but because it expresses the minimal requirement for a coherent physical projection
of the underlying relational substrate.
Quantum gravity may challenge locality, geometry, and even the notion of time, but
it cannot violate CPT without undermining the possibility of a consistent emergent
universe.

In summary, CPT invariance in Cosmochrony is not threatened by quantum gravity.
It survives precisely because it is not a dynamical symmetry, but a global
projective property ensuring the coherence of all admissible effective descriptions
derived from a single pre-geometric relational structure.
