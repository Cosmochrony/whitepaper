\subsection{Mass as Resistance to $\chi$ Relaxation}
  \label{subsec:mass_as_resistance}

  In Cosmochrony, mass is not introduced as an intrinsic or fundamental property of
  matter.
  It emerges only at the level of effective descriptions, as a quantitative measure
  of how strongly a localized projected configuration resists admissible relaxation
  ordering.

  A particle-like excitation is described, within effective regimes, as a stable and
  localized projected configuration, denoted $\chi_{\mathrm{eff},s}$, characterized
  by persistent internal structure.
  Such configurations locally constrain the admissible relaxation ordering relative
  to a homogeneous effective background.
  When a geometric description applies, this constraint manifests phenomenologically
  as inertial persistence and gravitational time dilation.

  We define the effective structural energy associated with a projected solitonic
  configuration $\chi_{\mathrm{eff},s}$ as a measure of the excess resistance to
  relaxation encoded in its internal structure:
  \begin{equation}
    E[\chi_{\mathrm{eff},s}] \;\equiv\;
    \int_{\Sigma}
    \left(
      \frac{1}{\sqrt{1 - |\nabla \chi_{\mathrm{eff},s}|^2 / c^2}} - 1
    \right)
    \, d\Sigma ,
    \label{eq:chi_soliton_energy}
  \end{equation}
  where $\Sigma$ denotes a hypersurface of constant effective ordering parameter,
  and $|\nabla \chi_{\mathrm{eff},s}|$ quantifies effective structural deformation
  within the projected description.
  This expression does not represent a fundamental energy stored in the $\chi$
  substrate, but a descriptive measure of how strongly a given projected
  configuration departs from homogeneous relaxation ordering.

  The inertial mass associated with such a configuration is then defined
  operationally as
  \begin{equation}
    m \;\equiv\; \frac{E[\chi_{\mathrm{eff},s}]}{c^2}.
    \label{eq:mass_definition}
  \end{equation}

  This relation is not postulated as a fundamental axiom.
  It follows directly from the role of $E[\chi_{\mathrm{eff},s}]$ as a measure of
  resistance to effective relaxation ordering.
  The universal constant $c$ appears as the maximal admissible relaxation rate and
  therefore provides the unique conversion factor between structural resistance and
  inertial response.

  Within this framework, the relation $E = mc^2$ is interpreted as a kinematic
  identity.
  Mass quantifies the amount of effective relaxation resistance locally encoded in a
  persistent projected configuration, while energy represents the same quantity
  expressed in relaxation units.

  In this sense, mass is not an independent attribute of matter.
  It is a derived invariant characterizing how strongly a localized projected
  configuration resists the irreversible ordering that, in effective descriptions,
  defines physical time.

  The question of how different particle masses arise from distinct classes of
  projected configurations is addressed in
  Appendix~\ref{app:topological_solitons}, where a spectral characterization of the
  stability properties of admissible projected descriptions is proposed as the
  geometric origin of mass hierarchies.
