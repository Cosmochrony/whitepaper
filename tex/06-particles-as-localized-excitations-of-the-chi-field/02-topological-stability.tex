\subsection{Topological Stability}
  \label{subsec:topological-stability}

  The stability of particle-like excitations in Cosmochrony does not rely on
  fundamental conserved charges postulated a priori.
  It arises instead at the level of effective descriptions, from intrinsic structural
  constraints on admissible projected $\chi$ configurations.
  Certain projected configurations exhibit non-trivial internal organization that
  prevents them from being continuously deformed into homogeneous effective
  descriptions.

  This form of stability is topological in nature.
  It reflects the existence of inequivalent classes of projected $\chi$
  configurations that cannot be smoothly related without violating the admissibility
  constraints imposed by effective relaxation ordering.
  As a result, particle-like excitations appear discrete and robust under
  perturbations, without requiring externally imposed symmetries or fundamental
  conservation laws.

  Importantly, these topological distinctions are not defined with respect to a
  pre-existing spacetime geometry.
  They are properties of the configuration space of admissible projected
  descriptions and remain well-defined even in the absence of an effective geometric
  interpretation.
  Geometric representations, when employed, serve only as descriptive tools valid in
  regimes where a spacetime language has emerged.

  The long-lived character of solitonic structures therefore follows from the
  incompatibility between distinct classes of admissible projected configurations,
  rather than from a dynamical balance of forces or nonlinear self-interactions.
  This mechanism provides a natural foundation for particle stability within a
  purely relational and pre-geometric framework.
