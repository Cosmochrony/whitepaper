\subsection{Topological Stability}
  \label{subsec:topological-stability}

  The stability of particle-like excitations in Cosmochrony does not rely on
  conserved charges postulated a priori, but arises from intrinsic structural
  constraints of the $\chi$ field.
  Certain localized configurations of $\chi$ possess non-trivial internal
  organization that prevents their continuous relaxation into the homogeneous
  vacuum state.

  This form of stability is topological in nature: it reflects the existence of
  inequivalent classes of $\chi$ configurations that cannot be smoothly
  transformed into one another without crossing a high-relaxation barrier.
  As a result, particle-like excitations are both discrete and robust under
  perturbations, without requiring externally imposed symmetries or conservation
  laws.

  Importantly, these topological constraints are not defined with respect to a
  pre-existing spacetime geometry.
  They are intrinsic to the internal configuration space of $\chi$ itself and
  remain well-defined even in the absence of effective spatial notions.
  Geometric representations of such configurations, when employed, should be
  understood as descriptive tools valid only in regimes where a spacetime
  interpretation has emerged.

  The long-lived character of solitonic excitations thus follows from a balance
  between localization tendencies driven by nonlinear self-interactions and
  structural constraints encoded in the configuration of $\chi$.
  This mechanism provides a natural foundation for particle stability within a
  purely relational and pre-geometric framework.
