\subsection{Energy--Frequency Relation}
  \label{subsec:energy-frequency-solitons}

  Within effective descriptions of Cosmochrony, the energy associated with a
  particle-like excitation is linked to a characteristic internal spectral scale of
  the corresponding projected configuration.
  This scale quantifies how strongly the configuration resists admissible relaxation
  ordering: configurations with higher characteristic frequencies correspond to more
  tightly constrained projected structures and a greater effective capacity to encode
  relaxation resistance.

  This provides an effective interpretation of the relation
  \begin{equation}
    E \propto \nu ,
  \end{equation}
  in which energy measures the degree of effective relaxation resistance associated
  with a projected configuration, while the frequency $\nu$ characterizes the
  associated spectral scale of its internal structure.
  The frequency should not be interpreted as an oscillation with respect to a
  fundamental time parameter.
  It acquires a temporal interpretation only within effective geometric regimes,
  where a notion of time becomes meaningful.

  Within this perspective, Planck's constant emerges as an effective proportionality
  factor relating energy and frequency.
  Its apparent universality reflects the robustness of the spectral scales governing
  admissible projected configurations, rather than the postulation of a fundamental
  quantization constant at the level of the $\chi$ substrate.

  In this sense, the energy--frequency relation expresses a kinematic correspondence
  between effective structural resistance and spectral organization within projected
  descriptions.
  A more explicit realization of this correspondence, in the context of radiation and
  photon-like projected excitations, is presented in
  Sec.~\ref{subsec:energy-frequency-radiation}.
