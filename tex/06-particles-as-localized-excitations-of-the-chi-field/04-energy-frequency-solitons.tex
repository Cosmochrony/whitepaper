\subsection{Energy--Frequency Relation}
  \label{subsec:energy-frequency-solitons}

  The energy associated with a particle-like excitation is linked to the internal
  oscillation rate of its $\chi$ configuration.
  Within the Cosmochrony framework, this rate characterizes how strongly a localized
  structure resists relaxation: configurations with more rapid internal
  reorganization correspond to tighter localization and a greater capacity to
  store relaxation potential.

  This provides an effective interpretation of the relation
  \begin{equation}
    E \propto \nu ,
  \end{equation}
  in which energy measures the amount of relaxation potential trapped in a given
  configuration, while the frequency $\nu$ quantifies the characteristic rate at
  which this potential is internally redistributed.
  The frequency should not be interpreted as oscillation with respect to a
  fundamental time parameter, but as an intrinsic property of the excitation,
  which admits a temporal interpretation only at the effective geometric level.

  Within this perspective, Planck's constant emerges as an effective proportionality
  factor relating energy and frequency, determined by the intrinsic scales and
  coupling properties of the $\chi$ field.
  Its apparent universality reflects the robustness of these underlying scales
  across stable configurations, rather than the postulation of a fundamental
  quantization constant.

  A more explicit derivation of this relation, in the context of radiation and
  photon-like excitations of the $\chi$ field, is presented in
  Sec.~\ref{subsec:energy-frequency-radiation}.
