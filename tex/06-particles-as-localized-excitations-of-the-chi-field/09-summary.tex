\subsection{Summary}
  \label{subsec:summary7}

  Within the Cosmochrony framework, particles are not fundamental entities.
  They emerge only at the level of effective descriptions, as stable and localized
  projected configurations that resist admissible relaxation ordering.
  Their physical properties are not postulated but arise as invariants of the
  structural and topological organization of admissible projected descriptions.

  Mass is identified with the degree of effective relaxation resistance encoded in a
  localized projected configuration.
  It quantifies how strongly such a configuration constrains admissible relaxation
  ordering relative to a homogeneous effective background.
  In regimes where a relativistic description applies, this interpretation naturally
  leads to the relation $E = mc^2$, understood as a kinematic identity rather than a
  fundamental postulate.

  Spin and statistical behavior originate from topological obstructions in the space
  of admissible projected configurations.
  Fermionic configurations exhibit a $4\pi$ periodicity in configuration space, such
  that a $2\pi$ rotation corresponds to a non-contractible loop and induces a sign
  change of the effective wavefunction.
  This topological structure provides a common origin for spin-$\tfrac{1}{2}$
  behavior, fermionic antisymmetry, and the Pauli exclusion principle
  without invoking additional quantum axioms.

  Within this perspective, different particle attributes correspond to distinct
  topological invariants of admissible projected descriptions.
  Spin is associated with non-trivial covering properties of configuration space,
  while electric charge may be interpreted, at an effective level, as an oriented
  topological defect or vortex-like structure within projected descriptions.
  These attributes remain conceptually distinct but arise from a common relational
  substrate once a geometric interpretation becomes meaningful.

  Taken together, these results provide a unified account of particle properties
  compatible with both relativistic and quantum phenomenology, without introducing
  particles or their attributes as fundamental ontological constituents.
  Particles appear instead as stable descriptive regimes of the underlying relational
  structure, whose properties reflect the topology of admissible projected
  configurations.
