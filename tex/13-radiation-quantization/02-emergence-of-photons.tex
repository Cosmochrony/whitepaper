\subsection{Emergence of Photons}
  \label{subsec:emergence-of-photons}

  In the Cosmochrony framework, photons are not fundamental entities, nor are they
  identified with propagating disturbances of the $\chi$ substrate.
  They arise as effective descriptions associated with transitions between localized
  and delocalized regimes of projectability.

  Prior to emission or detection, no photon exists as an independent object.
  What exists is a reconfiguration of relational structure within $\chi$ that ceases
  to admit a localized particle-like projection and becomes expressible only through
  extended, delocalized effective modes.

  In effective spacetime descriptions, these delocalized regimes are represented as
  electromagnetic waves.
  However, this wave character does not correspond to a physical oscillation of $\chi$,
  but to a continuous descriptive projection of relational structure compatible with
  field-like representation.

  Photon-like events emerge only at interaction.
  When a delocalized projective mode becomes locally constrained by interaction with a
  localized excitation (such as an atom or detector), the projection collapses into a
  discrete transfer of relaxation capacity.
  Quantization is therefore not a property of propagation, but of interaction and
  local reprojection.

  In this sense, wave--particle duality reflects a duality of description rather than a
  duality of underlying ontology.
  Interference phenomena, such as those observed in double-slit experiments, arise from
  the coherence of delocalized projective modes, while individual detection events
  correspond to localized reprojections.
  No fundamental wavefunction collapse or stochastic emission process is required.
